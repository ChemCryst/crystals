\subsection{CrystalSystem}
{\footnotesize CrystalSystem.h modified  30/04/2003 \\ CrystalSystem.cpp modified 30/04/2003}

\subsubsection{Heading}
\begin{table}[h]
\begin{tabular}{|l|}\hline
$<<$MyObject$>>$\\
\textbf{Heading}\\ \hline
-iMatrix:Matrix$<$float$>$\\
-iName:char*\\
-iID:int\\
\hline
+Heading(char* pLines):\\
+\til Heading():\\
+getName():char*\\
+getMatrix():Matrix$<$float$>$*\\
+getID():int\\
+output(std::ostream\ands\xspace pStream):std::ostream\ands\\
\hline
\end{tabular}
\end{table}

\subsubsection{Headings}
\begin{table}[h]
\begin{tabular}{|l|}\hline
$<<$MyObject$>>$\\
\textbf{Headings}\\ \hline
-iHeadings:ArrayList$<$Heading$>$*\\
\hline
+Headings():\\
+\til Headings():\\
+getMatrix(int pIndex):Matrix$<$float$>$*\\
+getName(int pIndex):char*\\
+getID(int pIndex):int\\
+output(std::ostream\ands\xspace pStream):std::ostream\ands\\
+addHeading(char* pLine):char*\\
+readFrom(filebuf\ands\xspace pFile):void\\
+length():int\\
\hline
\end{tabular}
\end{table}

This class should just extends the arraylist$<$Heading$>$ class and not have it as one of it's member variables. This is to be done.

\subsubsection{Condition}
\begin{table}[h]
\begin{tabular}{|l|}\hline
$<<$MyObject$>>$\\
\textbf{Condition}\\ \hline
-iMatrix:Matrix$<$float$>$\\
-iName:char*\\
-iID:int\\
-iMult:float\\
\hline
+ Condition(char* pLines):\\
+\til Condition():\\
+getName():char*\\
+getMatrix():Matrix$<$float$>$*\\
+getMult():float\\
+getID():int\\
+output(std::ostream\ands\xspace pStream):std::ostream\ands\\
\hline
\end{tabular}
\end{table}

\subsubsection{Conditions}
\begin{table}[h]
\begin{tabular}{|l|}\hline
$<<$MyObject$>>$\\
\textbf{Conditions}\\ \hline
-Conditions:ArrayList$<$Condition$>$*\\
\hline
+ Conditions():\\
+\til Conditions():\\
+getName(int pIndex):char*\\
+getMult(int pIndex):float\\
+getMatrix(int pIndex):Matrix$<$float$>$*\\
+output(std::ostream\ands\xspace pStream):std::ostream\ands\\
+addCondition(char* pLine):char*\\
+readFrom(filebuf\ands\xspace pFile):void
+length():int\\
\hline
\end{tabular}
\end{table}

This class should just extends the arraylist$<$Condition$>$ class and not have it as one of it's member variables. This is to be done.

\subsubsection{Index}
\begin{table}[h]
\begin{tabular}{|l|}\hline
$<<$MyObject$>>$\\
\textbf{Index}\\ \hline
-iValue:int\\
\hline
+Index(signed char pValue):\\
+Index(Index\ands pObject):\\
+get():signed char\\
+set(signed int pValue):void\\
+operator$=$(Index\ands\xspace pObject):Index\ands\\
+operator$<$(Index\ands\xspace pObject):bool\\
+operator$>$(Index\ands\xspace pObject):bool\\
+operator$==$(Index\ands\xspace pObject):bool\\
output(std::ostream\ands\xspace pStream):std::ostream\ands\\
\hline
\end{tabular}
\end{table}

\subsubsection{Indexs}
\begin{table}[h]
\begin{tabular}{|l|}\hline
$<<$MyObject$>>$\\
\textbf{Indexs}\\ \hline
-iIndexs:ArrayList$<$Index$>$\\
\hline
+Indexs(signed char pIndex):\\
+\til Index():\\
+addIndex(singed char pIndex):void\\
+number():int\\
+getIndex(int pIndex):Index*\\
+getValue(int pIndex):int\\
+output(std::ostream\ands\xspace pStream):std::ostream\ands\\
\hline
\end{tabular}
\end{table}

\subsubsection{Column}
\begin{table}[h]
\begin{tabular}{|l|}\hline
$<<$MyObject$>>$\\
\textbf{Column}\\ \hline
\\
\hline
+setHeading(char* pHeading):void\\
\hline
\end{tabular}
\end{table}

\subsubsection{ConditionColumn}
\begin{table}[h]
\begin{tabular}{|l|}\hline
$<<$Virtually Column$>>$\\
\textbf{ConditionColumn}\\ \hline
-iHeadingConditions:ArrayList$<$Index$>$*\\
-iConditions:ArrayList$<$Indexs$>$*\\
\hline
+ConditionColumn():\\
+\til ConditionColumn():\\
+addHeading(signed char pIndex):void\\
+setHeading(char* pHeading):void\\
+getHeading(const int pIndex):int\\
+getHeadings():ArrayList$<$Index$>$*\\
+addCondition(signed char pIndex, int pRow):void\\
+addEmptyCondition(int pRow):void\\
+getConditions():Indexs*\\
+countCondition():int\\
+countHeadings():int\\
+output(std::ostream\ands\xspace pStream, Headings* pHeadings, Conditions* pConditions):std::ostream\ands\\
\hline
\end{tabular}
\end{table}

\subsubsection{SpaceGroup}
\begin{table}[h]
\begin{tabular}{|l|}\hline
$<<$MyObject$>>$\\
\textbf{SpaceGroup}\\ \hline
-iSymbol:char*\\
\hline
+SpaceGroup():\\
+\til SpaceGroup():\\
+getSymbol():char*\\
+output(stD::ostream\ands\xspace pStream):std::ostream\ands\\
\hline
\end{tabular}
\end{table}

\subsubsection{SpaceGroups}
\begin{table}[h]
\begin{tabular}{|l|}\hline
$<<$Virtually Column$>>$\\
\textbf{SpaceGroups}\\ \hline
-iPointGroup:char*\\
-iSpaceGroups:ArrayList$<$SpaceGroup$>$*\\
\hline
+SpaceGroups():\\
+\til SpaceGroups():\\
+add(char* pSpaceGroup, int pRow):void\\
+get(int pIndex):SpaceGroup*\\
+length():int\\
+setHeading(char* pHeading):void\\
+getPointGroup():char*\\
\hline
\end{tabular}
\end{table}

\subsubsection{Table}
\begin{table}[h]
\begin{tabular}{|l|}\hline
$<<$MyObject$>>$\\
\textbf{Table}\\ \hline
-iName:char*\\
-iColumns:ArrayList$<$ConditionColumn$>$*\\
-iSpaceGroups:ArrayList$<$SpaceGroups$>$*\\
-iHeadings:Headings*\\
-iConditions:Conditions*\\
\hline
-columnHeadings(char* pHeadings, int pColumn):void\\
-addLine(char* pLine, int pColumn):void\\
-addCondition(char* pCondition, CondtionColumn* pColumn, int pRow):void\\
-addSpaceGroup(char* pSpaceGroup, SpaceGroups* pSpaceGroups, int pRow):void\\
+Table(char* pName, Headings* pHeadings, Conditions* pConditions, int pNumColumns, int pNumPointGroups):\\
+\til Table():\\
+addLine(char* pLine):void\\
+readColumnHeadings(char* pHeadings):void\\
+readFrom(filebuf\ands\xspace pFile):void\\
+getName():char*\\
+getHeadings(int pI):ArrayList$<$Index$>$\\
+output(std::ostream\ands\xspace pStream):std::ostream\ands\\
+output(int pLineNumstd::ostream\ands\xspace pStream):std::ostream\ands\\
+getNumPointGroups):int\\
+getConditions(int pRow, int pColumn):Indexs*\\
+numberOfColumns():int\\
+numberOfRows():int\\
\hline
\end{tabular}
\end{table}

\subsubsection{Tables}
\begin{table}[h]
\begin{tabular}{|l|}\hline
\textbf{Tables}\\ \hline
$<<$MyObject$>>$\\
-iTables:ArrayList<Table>*\\
-iHeadings:Headings*\\
-iConditons:Conditons*\\
\hline
+Tables(char* pFileName):\\
+\til Tables():\\
+getHeadings():Headings*\\
+getConditions():Conditions*\\
+addTable(Table* pTable):void\\
+readFrom(filebuf\ands\xspace pFile):void\\
+findTable(char* pName):Table*\\
+output(std::ostream\ands\xspace pStream):std::pstream\ands\xspace\\
\hline
\end{tabular}
\end{table}

\subsubsection{RowRating}
\begin{table}[h]
\begin{tabular}{|l|}\hline
\textbf{RowRating}\\ \hline
iRowNum:int\\
iTotNumVal:int\\
iSumRat1:float\\
iSumRat2:float\\
iSumSqrRat1:float\\
iSumSqrRat2:float\\
iMean:float\\
\hline
\end{tabular}
\end{table}

\subsubsection{RankedSpaceGroups}
\begin{table}[h]
\begin{tabular}{|l|}\hline
$<<$MyObject$>>$\\
\textbf{RankedSpaceGroups}\\ \hline
iRatingList:LList$<$RowRating$>$\\
iRatings:RowRating*\\
iTable:Table*\\
\hline
+RankedSpaceGroups(Table\ands\xspace pTable, Stats\ands\xspace pStats, bool pChiral):\\
+\til RankedSpaceGroups():
+output(std::ostream\ands\xspace pStream):std::ostream\ands\\
-static calcRowRating(RowRating* pRating, int pRow, Table\ands\xspace pTable, Stats\ands\xspace pStats):void\\
-static addConditionRatings(RowRating pRating, Stats\ands\xspace pStats,\\
\xspace Index* tIndexs, Index* pHeadingIndex):void\\
-static void addRating(RowRating* pRating, float pRating1, float pRating2):void\\
-addToList(RowRating* pRating):void\\
\hline
\end{tabular}
\end{table}

