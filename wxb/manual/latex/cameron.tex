\documentclass[10pt,a4paper]{report}
\usepackage[centertags]{amsmath}
\usepackage{amsfonts}
\usepackage{amssymb}
\usepackage{amsthm}
\usepackage{makeidx}
\usepackage{newlfont}
\usepackage{fancyhdr}
\usepackage[bookmarks,colorlinks,plainpages,backref]{hyperref}
\pagestyle{fancy}
\hfuzz2pt % Don't bother to report over-full boxes if over-edge is < 2pt
\makeindex
\addtolength{\headwidth}{80pt}
\addtolength{\textwidth}{80pt}
\addtolength{\oddsidemargin}{-30pt}
\begin{document}
\lhead{\slshape \rightmark}
\chead{}
\rhead{\thepage}
\lfoot{Fri Jul  5  2024}
\cfoot{}
\rfoot{}
\renewcommand{\headrulewidth}{0pt}
\renewcommand{\footrulewidth}{0pt}
\fancypagestyle{plain}{%
\fancyhf{}
\fancyfoot[L]{Fri Jul  5  2024}
\fancyfoot[C]{\thepage}
\renewcommand{\headrulewidth}{0pt}
\renewcommand{\footrulewidth}{0pt}}
\newcommand{\Instruction}[1]{{\bf #1}}
\newcommand{\Directive}[1]{{\bf \emph{#1}}}
\newcommand{\Keyword}[1]{\emph{#1}}
\sloppy
\title{Cameron Manual}
\author{Chemical Crystallography Laboratory, Oxford}
\date{Fri Jul  5  2024}
\maketitle
\tableofcontents
\chapter{Introduction}


CAMERON was designed and built in the Chemical Crystallography Laboratory, Oxford, by Lisa Pearce, Keith Prout and David Watkin. Manypeople have contributed ides for improvements, and Louis Ferrugiapioneerd a completely WINDOWS based version.\section{HOW TO CONTROL THE PROGRAM}


The program is controlled by typing commands at the keyboard,
by use of a mouse to pick items from a menu, or by using the mouse
to manipulate the diagram.
\section{Command Groups}


The commands have been collected into groups (twenty-three at present).
Each group contains commands and sub-commands which perform related
functions or which will act as qualifiers to each other. The command
processor will spot invalid combinations of commands and sub-commands
and enter error mode. Once in this mode the error can be corrected.


Command groups are used to decide when the user has finished entering
all the qualifiers for a particular operation. The qualifiers are
assumed to be finished when a command of another group is entered.
For example :-
\small\begin{verbatim}
XROT 20 YROT 30 VIEW LINE ALL
\end{verbatim}\normalsize




{\bf XROT,} {\bf YROT}  and {\bf VIEW}  are all in the same group but {\bf LINE}
 (a command that
sets the drawing style) is not. When the command processor reaches the
{\bf LINE} command it will then execute the previous commands ie {\bf XROT}
to {\bf VIEW.}
This is useful because it means that the user does not need to input all
the qualifiers on one line, and also because the user does not need to
tell the computer when it has finished with an operation - the computer
is able to work this out for itself. The input lines :-

\small\begin{verbatim}
XROT 20 YROT 30 VIEW LINE
\end{verbatim}\normalsize


and
\small\begin{verbatim}
XROT 20
YROT
20 VIEW
LINE
\end{verbatim}\normalsize


will produce exactly the same results.


If the user wishes to execute a group of commands without entering a
command of another group this can be done by sending a blank line to the
command processor (ie press the RETURN key twice).
\small\begin{verbatim}
XROT 20 YROT 30 VIEW <RETURN> <RETURN>
\end{verbatim}\normalsize


will execute the three commands without the need for {\bf LINE.}
\section{ERROR HANDLING}


The command processor is able to check for some errors in the commands
input by the user. The computer checks for two types of error:



\small\begin{verbatim}
1. Whether a command / sub-command combination is valid.
2. Whether the arguments supplied after a command are valid and also if there is
the correct number of arguments.
\end{verbatim}\normalsize




The action taken after an error depends upon the setting of the
{\bf EDIT} flag. If EDIT is OFF, only an error message is displayed.
If EDIT is ON, the faulty input can be edited.
Once an error has been detected the
user can take action and modify the input line so that as many commands
as possible can be processed. (If error checking is not carried out at
this level, the detection of an error while executing a command will
cause that command - and possibly its sub commands - to be lost.)


The user is given three options once an error has been detected.


\bigskip\Instruction{Abandon}


Do not execute any more commands not yet executed.


\bigskip\Instruction{Edit}


The action taken by choosing the second option will depend on the nature of
the error. If a word
has been input that the computer has not recognised the user will be
asked to enter an alternative word. If $<$return$>$ is used then the
word is replaced by a blank space - it is effectively deleted from the
input line. If too many arguments are supplied the user is given the
opportunity to delete the
excess arguments. Alternatively if arguments are missing the user will
be asked to input them.


\bigskip\Instruction{Help}




Help information is supplied if the user requests it. This will say
what arguments, if any,
are required by the current command. Also, if any sub commands are valid
in the context of the error these are listed.


For example, the input line
\small\begin{verbatim}
XROT 10 YROT ZROT 15 VIEW
\end{verbatim}\normalsize


contains an error - an argument is missing after {\bf YROT.} The program will
give the user three options :-
\small\begin{verbatim}
Abandon      Don't execute any of the commands
Edit         The user can supply the needed argument
             and the new line is then processed.
Help         This will display the help information

     This command requires one numeric argument - the angle of rotation in degrees.
\end{verbatim}\normalsize


\section{The HELP facility}


The CAMERON {\bf HELP} facility can be entered in one of two ways, either
once an error has occurred (as described above) or by requesting it
directly. This second way of accessing HELP is achieved by entering {\bf 'HELP'}
or '?' at any point in the input line. If a command follows {\bf 'HELP'} / '?' then
information is given on this command - otherwise information is given on
the previous command (if any). If no HELP is available (because the
command is not recognised) then the user is supplied with general HELP
information. The only difference between 'HELP' and '?' is in the detail of
the information supplied. Each command comes with a help information
line. Entering '?' will provide the user with the information line for the
command on which you have requested HELP and a list of the sub-commands
that are valid at this point. Entering 'HELP' gives the same information
except that the help line for each of the sub-commands shown is also
listed. 'HELP' will therefore provide the user with more detailed
information than '?' will.


Once the HELP information has been read the user may 'Continue' or
'Abort'.
The later two options are the same as for Error Handling and
are described above. 'Continue' simply removes the word 'HELP' or '?' (and any
following words) from the input line and the user can carry on as
before.
\section{Mouse Activation}


The Mouse can be used to pick out atoms (and elements if a KEY is
present). In some cases this is much quicker than typing on the keyboard
- especially if the user does not know the names of the atoms concerned.
The mouse cursor (an arrow) is present once an appropriate output device has
been chosen.


For some commands the mouse is used for functions other than
atom/element picking. In such cases - eg text positioning and labelling -
the mouse is activated by typing MOUSE as a separate command.



\section{General Input Syntax}


When the atoms are in their initial state (ie no symmetry operators
have been applied) they are referred to by their normal atom or element
names. With commands that allow elements to be affected eg
COLOUR, BALL you may use a '*' to refer to ALL atoms. For example CONNECT 
O * 0.0 3.0 will find all bonds between oxygen and any other atom within
3 angstroms. A '*' can also be used to refer to GROUP names (see DEFGROUP 
later).


Once pack operations have been carried out the atoms are
identified differently. During PACK and ENCLOSURE the user is provided
with information similar to :-
\small\begin{verbatim}
2 additional symmetry generated units.
\end{verbatim}\normalsize


The atoms generated will be referred to by suffixing their packnumber 
eg O1\_2. These numbers 
can be shown by using LABEL GENERATED. Therefore, consider an asymmetric
unit which contains N1 and O2. The following will be available after
packing:-

\small\begin{verbatim}
N1_1      the single atom
N1        all N1 atoms in the packed structure
N         all the nitrogen atoms
N_2       all the nitrogen atoms with pack number 2
*_1       all the atoms with pack number 2
\end{verbatim}\normalsize


These pack numbered atom identifiers are produced by mouse clicking on
the atoms concerned. The symmetry operation used to generate the atoms 
can be found from INFO PACKNUMBER.  The operators are also written to
the CRYSTALS listing file





When CAMERON is parsing the commands input it checks to see whether
the next work on the line is a command or not. Only the first 4
characters of each word are significant. Therefore, the processor will
not know the difference between the command COLOUR and the file
COLOUR.PST. If you require the later then it must be enclosed in quotes
- eg COPY "COLOUR.PST".
\chapter{How To Get Started}
\index{Cameron - How to get started}

There is a basic 'startup' procedure that can be followed in order toget a picture on to the screen.The following steps are required :-\small\begin{verbatim}1)      Load in a spacegroup if required2)      Load in the unit cell dimensions3)      Load in the atomic coordinates4)      Select an output device5)      Set the atom drawing style to BALL6)      Draw a picture\end{verbatim}\normalsize

This can be accomplished with the following commands, which areillustrated for a compound which has a spacegroup P 21, unit cellparameters a=6.0, b=7.0, c=8.0, alpha=90, beta=115, gamma=90. The atomiccoordinates are held in a "list5" type file NIGEL.L5.\small\begin{verbatim}SPACEGROUP P 21INPUTCELL 6 7 8 90 115 90LIST5 NIGEL.L5DEVICE VGABALL ALLFILLVIEW\end{verbatim}\normalsize



From here the view direction can be controlled via XROT, YROT, ZROT orCURSOR. If you require a picture of all the atoms in the unit cell, and the unit cell also displayed,the commands are :-\small\begin{verbatim}PACK CELLINCLUDE CELLVIEW\end{verbatim}\normalsize

\chapter{Data Input}
\index{Cameron - Data Input}

At present data is read from CRYSTALS generated LIST5 files and CSSRfiles (see the CRYSTALS manual).\section{Input of CRYSTALS List 5 files}


These files do not contain information on the dimensions of the unit
cell. It is therefore necessary to use CELL to enter this information
prior to the reading of the list 5 file.


\bigskip\Instruction{INPUT}




\bigskip\Directive{CELL}
The syntax of this sub-command is :-
\small\begin{verbatim}
INPUT CELL a b c alpha beta gamma
\end{verbatim}\normalsize


ie we require six arguments - cell dimensions in angstroms and angles in
degrees. Their values are assigned as shown above.


\bigskip\Keyword{LIST5}
This command requires as its argument the name of the file that contains
the atomic coordinates. The syntax is :-
\small\begin{verbatim}
LIST5 filename.L5
\end{verbatim}\normalsize


NB The LIST5 command will not be accepted without the previous command
CELL.


\bigskip\Directive{CSSR}


CSSR files are input by using INPUT CSSR. As with LIST5 files above
it is advisable to input the symmetry information before inputting the
atomic coordinates.


Related commands : OBEY, OUTPUT
\chapter{Outputting Data}
\index{Cameron - Data output}

\bigskip\Instruction{OUTPUT}



CAMERON outputs data in CRYSTALS LIST5 and CSSR format.

\bigskip\Directive{LIST5}This outputs a crystals list5. It must be followed by a filename. Thelist5 will contain LIST1 unit cell information at the end - such a filecan be re-read by CAMERON with no problem.

\bigskip\Directive{CSSR}

\bigskip\Keyword{FRACT}

\bigskip\Keyword{ORTH}These subcommands are needed to specify the coodinate type for the CSSRfile ie orthogonal or fractional. The command is followed by thefilename of the output file.\small\begin{verbatim}Example:OUPUT CSR FRACT mydata.css\end{verbatim}\normalsize



Related commands : INPUT\chapter{Editing The Atom List}
\index{Cameron - Editing the atom list}

The user is able to edit the names and elemental types of the atoms inthe current list. The following commands are available -

\bigskip\Instruction{RENAME}

The syntax is:\small\begin{verbatim} RENAME at1 at2    or: RENAME el1 el2.\end{verbatim}\normalsize



Arguments are entered inpairs - the first is renamed to the second - and must be like with like.i.e. you cannot rename an atom to an element or vice versa.

\bigskip\Instruction{RETYPE}

This allows the user to change the element of a list of atoms.

\bigskip\Directive{ATOMS}The syntax is -\small\begin{verbatim}RETYPE element ATOM at1 at2 at3\end{verbatim}\normalsize

\chapter{Obeying Files}
\index{Cameron - OBEY files}

\bigskip\Instruction{OBEY}



If required the user is able to hand over control of CAMERON to anexternal file. This is particulary useful if a set of operations is tobe carried out on a number of pictures because the file need only begenerated once and then it can be used over and over again. At thepresent time CAMERON is able to read files is its own format - ie inwith the same syntax as the commands input at the keyboard - or it canread a SNOOPI.INI file as output by CRYSTALS. The syntax is :-\small\begin{verbatim}OBEY nnn.nn\end{verbatim}\normalsize

Note that any commands entered after OBEY (on the same line) areignored. This is to allow for possible errors in the OBEY file - errorswill cause the OBEYed file to be closed. OBEY files may themselvescontain the names of other files to be OBEYed.



Related commands : LOG\chapter{Archiving And Retrieving Views}
\index{Cameron - Archiving and retrieving views}

The user can archive the present structure view on disk or retrieve apreviously archived view from disk. At present only structure views residingin the same directory as the crfilev2.dsc (the working directory) can beretrieved. Equally well, views can only be archived in the working directory.As an alternative to the line commands, the user may choose the commandsthat are available in the pull-down menu 'File'.The following commands are available -

\bigskip\Instruction{ARCHIVE}



The syntax is:\small\begin{verbatim} ARCHIVE "foo.foo"\end{verbatim}\normalsize



Note that the quotes (") are necessary to avoid confusion with non-existingsub-commands to this command. The filename must not exceed 12 characters.



\bigskip\Instruction{RETRIEVE}



The syntax is:\small\begin{verbatim} RETRIEVE "foo.foo"\end{verbatim}\normalsize



Note that the quotes (") are necessary to avoid confusion with non-existingsub-commands to this command. The filename must not exceed 12 characters.\chapter{Graphical Output Devices}
\index{Cameron - Output devies}

The user has to specify a graphical output device before the commandVIEW can be used. This is done with the command SCREEN. Note that if thetitle screen file CAMERON.SRT is present the screen is automatically setto VGA.

\bigskip\Instruction{SCREEN}

The following devices are currently supported:-

\bigskip\Directive{VGA}This sends the output to a VGA monitor eg to a PC graphics screen.

\bigskip\Directive{EGA}This sends the output to an EGA monitor.

\bigskip\Directive{SIGMA}This sends the output to a SIGMA terminal.

\bigskip\Directive{TEK}This is used for VAX terminals with Tektronics 4010 emulation.



\bigskip\Instruction{HARDCOPY}



If hardcopy output is required the user must specify a separatedevice which is then drawn to using COPY. Those available are :-

\bigskip\Directive{POSTSCRIPT}The graphical output is sent to a file in POSTSCRIPT format so that itcan be printed on a postscript printer.

\bigskip\Directive{ENCAPPOST}The graphical output will be sent to a file in the same way as withPOSTSCRIPT, the onlydifference is that the resulting file will be in Encapsulated form sothat it can be incorporated into other packages eg Word Perfect (using the Graphics range of commands).

\bigskip\Directive{CPOST}This generates a file in Colour Postscript (Level 1) format.



\bigskip\Directive{CENCAP}This generates an colour encapsulated postscript file.

Related commands : VIEW, COPY\chapter{View Direction Control}
\index{Cameron - View direction control}

One of the most important features of a graphics package is theeasewith which a required molecular view can be obtained. CAMERONcontains alarge number of options which allow the user to control the viewdirection.\section{BASIC CONTROL}


Five commands are available to control the view direction by
applying
rotations to the molecule. These rotations are applied
cumulatively -
rotating about the x axis by 20 degrees then by -10 degrees will
result
in an overall rotation of 10 degrees \emph{relative} \emph{to} \emph{the} \emph{starting} \emph{point.}


\bigskip\Instruction{XROT}




This command is used to apply a rotation of n degrees about the

x-axis which lies horizontally across the screen. The syntax is:
\small\begin{verbatim}
XROT n
\end{verbatim}\normalsize




\bigskip\Instruction{YROT}




\bigskip\Instruction{ZROT}




These two commands are identical to XROT except for the axis
about
which the rotation is carried out. The y-axis lies vertically up
the
screen and the z-axis is perpendicular to the screen.


\bigskip\Instruction{ROT}




This command is a 'shorthand' for the previous three commands
as it
allows you to apply three successive rotations by entering only
one
command. The syntax is :-
\small\begin{verbatim}
ROT x y z
\end{verbatim}\normalsize


Note that the order of rotation is: rotate about x then y then
z. The
order of rotation is important - YROT 10 XROT 5 ZROT 15 will NOT
produce
the same result as ROT 5 10 15 as the rotation matrices are
non-commutative.


\bigskip\Instruction{CURSOR}




This command allows the user to control rotation with the
cursor keys.
Each time a key is struck a rotation of 5 degrees is applied
about the
relevant axes. After each key stroke the molecule is rotated and
then
re-drawn. To speed up this process the molecule is drawn in line style
during the
CURSOR rotation and is not scaled. The set-up prior to CURSOR is
restored
once rotation is terminated. The keys used are:

\small\begin{verbatim}
Rotation about the x-axis

Positive      up arrow
Negative      down arrow

Rotation about the y-axis

Positive      left arrow
Negative      right arrow

Rotation about the z-axis

Positive      Delete key (PC)
Negative      End key (PC)
\end{verbatim}\normalsize


Any other key stroke will terminate the CURSOR control and draw
the
resultant picture.
\section{ORIENTATION COMMANDS}


More specific orientation of the molecule can be achieved by
relating
the view direction to the position of certain atoms within the
picture.


\bigskip\Instruction{PLANE}




In some cases it is useful to be able to define a PLANE and
then set
the view direction to be perpendicular to it. The syntax of this
command is :-
\small\begin{verbatim}
PLANE at1 at2 at3 ...
\end{verbatim}\normalsize


At least 3 atoms must be used to define the plane (element names
will
not be accepted). If more than 3 are used then the program
calculates
the best plane through the atoms and projects onto this.


\bigskip\Directive{ALL}
PLANE ALL will generated a 'best view' of the current structure.


\bigskip\Instruction{FACE}




Alternatively you may want to view onto a particular
crystallographic
face of the unit cell. The syntax for this command is:
\small\begin{verbatim}
FACE h k l
\end{verbatim}\normalsize


where h k and l are the Miller indices of the FACE in question.


\bigskip\Instruction{ALONG}




It is possible to define the view direction as that looking
down an
interatomic direction. The direction is specified by inputting
:-
\small\begin{verbatim}
ALONG at1 at2
\end{verbatim}\normalsize


As with PLANE atoms not elements must be used as the arguments.
Note
that the view direction is calculated as that looking down the
at1 - at2
direction ie at1 is the closest of the two. The reverse view will
be
obtained by requesting ALONG at2 at1. ALONG also has a
sub-command
associated with it:


\bigskip\Directive{AXIS}


It is possible to define the view direction as looking along
a
particular unit cell axis. The syntax of the command is:
\small\begin{verbatim}
ALONG AXIS x
\end{verbatim}\normalsize


where x is A, B or C.


\bigskip\Instruction{VERTICAL}




\bigskip\Instruction{HORIZONTAL}




These two commands are similar to ALONG except that the at1 -
at2
direction is oriented up (VERTICAL) or across (HORIZONTAL) the
screen
as required.


\bigskip\Directive{AXIS}


The AXIS sub command may be applied to the HORIZONTAL and VERTICAL 
commands. This rotates the diagram around the z axis so that the 
required axis is HORIZONTAL or VERTICAL as specified. This is 
particularly useful when generating hard copy output.


\bigskip\Instruction{BISECT}




This command requires three arguments, at1 at2 and at3. It will
generate a view direction looking along the bisector of the
at1-at2-at3
angle. The syntax is:
\small\begin{verbatim}
BISECT at1 at2 at3
\end{verbatim}\normalsize


where at2 is the apex atom.
\section{Other related commands}


IVIEW


The 'title' command of this group of commands is VIEW. IT IS A VERY IMPORTANT COMMAND.  This
requires
no arguments as causes a picture to be output to the current
output
device (see SCREEN). The picture is generated according to all
of the
parameters that are set up prior to its use. For example:
\small\begin{verbatim}
XROT 10 VIEW
\end{verbatim}\normalsize


will rotate the molecule by 10 degrees about x and then draw a
picture.
\small\begin{verbatim}
VIEW XROT 10
\end{verbatim}\normalsize


would draw the picture first and then carry out the rotation.
This
rotation would not be observed until the next VIEW is entered.





\bigskip\Instruction{COPY}




COPY is the equivalent command for hardcopy output. COPY must be
followed by a filename to which the output will be send. HARDCOPY must
have been used prior to this to specify the output type. The filename
may be one that has already been used in the current run of CAMERON in
which case the user is given an option to append the information.





\bigskip\Instruction{MATRIX}


This command allows the user to save the current view matrix.


\bigskip\Directive{FILE}
This is followed by a filename. If the file exists (and it is of the
correct format) then any matrices contained within it are available and
any matrices stored are appended to the file.


\bigskip\Directive{STORE}
This stores the current matrix. It is followed by a piece of text (in
quotes "text") to act as a description for the matrix. On storage a number
is assigned to the matrix which can be used to retrieve it later.


\bigskip\Directive{RETRIEVE}
This command obtains a view matrix from the file set up with MATRIX
FILE. This command must be followed by the number of the matrix
required. Note that these matrices depend on the unit cell parameters and
are specific to a particular structure. They may be used for other
structures but this will have very strange results. IF this happens an
operation such as ALONG AXIS C will recalculate a correct matrix.


\bigskip\Directive{LIST}
This outputs a list of the descriptions and numbers of the matrices
currently stored.





\bigskip\Instruction{STEREO}


This generates output in the form of stereo pairs.


\bigskip\Directive{DEGREE}
The angle of rotation between the stereo pairs is given by using the 
DEGREE command.


\bigskip\Instruction{NOSTEREO}


Reverts output back to normal.
\section{PHOTOGRAPHS}


\bigskip\Instruction{PHOTO}




\bigskip\Directive{ON}


\bigskip\Directive{OFF}
These commands are used to control the PHOTO facility. VIEW displays the
user with information about the current scale and displays the mouse
cursor once it has finished. PHOTO ON turns off this and waits for a key
press before proceeding. This allows the user to obtain a 'clean' screen
for photographs to be taken.


Related commands : SCREEN, HARDCOPY
\chapter{Include And Exclude}
\index{Cameron - Include and Exclude}

The commands INCLUDE and EXCLUDE are very useful as theycontrol whichatoms are drawn at any one time. The syntax for both of thecommands isvery similar as they are effectively the reverse of each other.

\bigskip\Instruction{INCLUDE}



\bigskip\Instruction{EXCLUDE}



These commands may be entered on their own and be follow byatom orelement names. The syntax is :-\small\begin{verbatim}nnCLUDE at1 at2 el1 el2\end{verbatim}\normalsize

Once an atom has been EXCLUDED from the picture it is not drawnand itis not included in any calculations eg PACK or ENCLOSURE.\index{Cameron - Include and Exclude All}

\bigskip\Directive{ALL}The ALL sub-command is a 'blanket' command that applies theIN/EXCLUDEoperation to ALL of the atoms in the current view. Therefore ifwe wantto exclude more atoms than we want to include we can use EXCLUDEALL toget rid of all of the atoms and then add in those we want byusingINCLUDE.\index{Cameron - Include and Exclude Group}

\bigskip\Directive{GROUP}The command DEFGROUP is available to define groups of atoms thatare tobe referred to as a whole. The IN/EXCLUDE command can be usedwith GROUPto remove or atoms all of the atoms in the group in the pictureasrequired.\index{Cameron - Include and Exclude Fragment}

\bigskip\Directive{FRAGMENT}The user can specify atoms to be IN/EXCLUDEd by defining groups of atomsas fragments. A fragment is defined by a single atom, and consists ofall the atoms linked to it by the current CONNECTIVITY TABLE. That is,if atoms are excluded from the picture the bonds ARE NOT broken andFRAGMENT will use bonds involving excluded atoms in its calculations. The usercan select this atom by typing or by clicking on it with the mouse.ie. EXCLUDE FRAGMENT C1 will exclude C1, any atoms joined to C1, anyatoms joined to those atoms etc.\index{Cameron - Include and Exclude Cell}

\bigskip\Directive{CELL}It is often useful to see how the atoms being drawn relate to theunitcell. IN/EXCLUDE CELL is used to control the inclusion of theunit cell(in the LINE style) on the picture.\index{Cameron - Include and Exclude Area}

\bigskip\Directive{AREA}Choosing INCLUDE or EXCLUDE AREA in the PC version allows the user to draw a polygonal area with the mouse. The polygon is created by clickingwith the left button on the position required for the vertices. The polygon is closed by clicking close to the initial position. The mouse cursor is changed to a cross during polygon creation and to an arrow when you are close enough to the initial point for closure to occur. Hitting the $<$return$>$ key or clicking with the right mouse button will abort the operation.\index{Cameron - Include and Exclude Select}

\bigskip\Directive{SELECT}The select option is equivalent to EXCLUDE ALL followed by INCLUDE AREA.The polygonal area is chosen in the same way as described above and all atoms not within that area are excluded.\section{Further related commands}


For pictures with more than one 'type' of constituent atom three more
commands will prove to be useful.
\index{Cameron - Mask and Unmask}


\bigskip\Instruction{MASK}




\bigskip\Instruction{UNMASK}




The MASK command has the same syntax as EXCLUDE ie.
\small\begin{verbatim}
(UN)MASK at1 at2 el1 el2 ...
\end{verbatim}\normalsize


It is used to remove atoms from current INCLUDE operations - ie atoms
that have been MASKed will not be included when using INCLUDE ALL. For
example the user could MASK out solvent molecules from a diagram leaving
the 'basic' asymmetric unit to be worked on. UNMASK can be used to
re-include the required atoms. The sub commands ALL and GROUP are
available for use with both MASK and UNMASK.
\index{Cameron - Include and Exclude Switch}


\bigskip\Instruction{SWITCH}




The SWITCH command causes any atoms that are EXCLUDEd from the picture
to be INCLUDEd and vice versa. This command can be used in conjunction
with MASK as any MASKed atoms WILL NOT be included in the picture after
a SWITCH operation.


Consider an asymmetric unit containing an anion, a cation and some
solvent. The user could MASK out the solvent atoms. EXCLUDE the anion
and obtain pictures of the cation atoms. The SWITCH command then
allows the user to examine the anion without having to INCLUDE and
EXCLUDE atoms, and the solvent atoms remain excluded throughout.
\section{Generation of Dummy atoms}
\index{Cameron - Include and Exclude Dummy}


\bigskip\Instruction{DUMMY}




In certain circumstances it may be useful for the user to be
able to
generate new 'dummy' atoms from the initial atomic coordinates.
One
example of this is a compound that contains a cyclopentadienyl ring. The
bonding in
such a system can either be represented as five M - C bonds or
as a
single bond to the ring's centre. The syntax for the command is:
\small\begin{verbatim}
INCLUDE DUMMY d
\end{verbatim}\normalsize


where d is the name of the created atom. This must then be
followed by a
sub-command which defines the atoms position.


\bigskip\Directive{COORDS}
This specifies the atoms coordinates in unit cell fractions. The
syntax
is:
\small\begin{verbatim}
COORDS x y z
\end{verbatim}\normalsize


\index{Cameron - Include and Exclude Centroid}


\bigskip\Directive{CENTROID}
Alternatively, the atoms position can be defined relative to
others
already in the molecule. At least TWO atoms must be entered to
define
new atoms position and it is placed at the CENTROID of their
coordinates. The syntax is:
\small\begin{verbatim}
CENTROID at1 at2 at3 ...
\end{verbatim}\normalsize




\bigskip\Keyword{ALL}
A further sub-command is available so that the new atom is placed
at the
centroid of all of the atoms in the molecule. This will prove
particulary useful when trying to alter the spacegroup of a
structure eg
to convert from P 1 to P -1 as the centre of inversion will lie
at the
newly generated dummy atom whose coordinates are output once they
have
been calculated.



\section{Example}


To create a dummy atom in the centre of a cyclopentadienyl ring you
would use:
\small\begin{verbatim}
INCLUDE
DUMMY d1
CENTROID C1 C2 C3 C4 C5
\end{verbatim}\normalsize


\chapter{Drawing Style Control}
\index{Cameron - Style control}

The user is able to choose from three different drawing styles.Theseare:

\bigskip\Instruction{LINE}



\bigskip\Instruction{BALL}



\bigskip\Instruction{ELLIPSE}





The basic syntax for style control is:\small\begin{verbatim}XXXX at1 at2 el1 el2\end{verbatim}\normalsize

where XXXX is LINE, BALL or ELLIPSE as required. The user mayspecifyatoms eg C1, C2 and/or elements eg C O to be drawn in the XXXXstyle.The user may therefore have all three styles present at any onetimewithin the picture. Alternatively the ALL may be usedtoconvert the drawing style of all the atoms present to XXXX.The initial drawing style used is LINE.



The ELLIPSE drawing style is only available if the user hasinput datathat contains information on the temperature factors (isotropicoranisotropic) of the atoms concerned. The ellipse drawing stylerepresents atoms by their thermal ellipsoid. Note that negative eigen values are reset at input to .001* the next largest.If the informationavailable is U[iso] only, a circle is plotted whose radiusis scaled according to its value. 

\bigskip\Instruction{ALL}



\bigskip\Instruction{FRAGMENT}



\bigskip\Instruction{GROUP}



These subcommands can be used to specify which atoms are to be affected by the drawing style command.

Several other commands are available in addition to the basicstylecommands.\section{LINE commands.}


There are no extra commands following LINE.
\section{BALL commands.}


\bigskip\Directive{RADII}
The user can specify the drawing radius (in angstroms) of a
specified
atom or element. The syntax is:
\small\begin{verbatim}
BALL RADII C 0.8 N1 1.1
\end{verbatim}\normalsize


which will draw all C atoms with a radius of 0.8 angstroms, and
the N1
atom with radius of 1.1 angstroms. 


\bigskip\Keyword{DEFAULT}
This sets all the radii to their initial covalent values.
The syntax is:
\small\begin{verbatim}
BALL RADII DEFAULT.
\end{verbatim}\normalsize




\bigskip\Keyword{COVALENT}


\bigskip\Keyword{IONIC}


\bigskip\Keyword{VANDERWAALS}
This sets the radii of the specified atoms/elements to the appropriate 
values. The syntax is:
\small\begin{verbatim} BALL RADII COVALENT N
\end{verbatim}\normalsize




The further subcommand ALL
can be used - BALL RADII IONIC ALL - to set all atoms if required.


\bigskip\Directive{FILL}


\bigskip\Directive{UNFILL}
These are header commands and are used to specify whether the
circles
drawn in the BALL style are to be filled with colour or not. No
arguments are required by these commands.
\section{ELLIPSE commands}


\bigskip\Directive{TYPE}


ELLIPSE may be followed by the sub command TYPE which enables
the user
to control the type of ellipse used to represent the atom.
The syntax is:
\small\begin{verbatim}
 ELLIPSE TYPE at/els/all n.
\end{verbatim}\normalsize


Type 'n' can take any of four values:
\small\begin{verbatim}
 1  - bounding ellipse only.
 2  - bounding ellipse and principal ellipses.
      This is the default representation.
 3  - as 2 but excluding the principle axes.
 4  - as 2 but with shading.
\end{verbatim}\normalsize




\bigskip\Keyword{ALL}
ALL can be used followed by a number to set the ellipse type of all 
atoms in the drawing. 


{\bf Note} that any use of ELLIPSE TYPE will set all the
atoms referred to ellipse type even if they were previously in LINE 
or BALL.
Hence,
\small\begin{verbatim}
ELLIPSE TYPE C 2  will have all carbon atoms drawn in type 2 form.
ELLIPSE TYPE ALL 4 will have all atoms drawn in type 4 form
\end{verbatim}\normalsize




\bigskip\Directive{NEGATIVE}


The NEGATIVE sub command requires one argument. If atoms are
input
with negative temperature factors the atoms temperature factor
is reset
using the value specified by ELLIPSE NEGATIVE u. The default
value of u
is 0.01.


\bigskip\Directive{PROBABILITY}


This is used to specify the size of the ellipsoid probability envelope
displayed. It is followed by a percentage value.
\chapter{Connectivity Control}
\index{Cameron - Connectivity control}

It is important that the user has complete control over thebonds displayed in a drawing. The CONNECT group of options is acomplicated one but has been designed so as to provide to userwith afully flexible set of commands.\section{CONNECT}


CONNECT is used in one of two ways, either on its own or with
a
modifying sub-command.


\bigskip\Instruction{CONNECT}


The syntax for this command is:
\small\begin{verbatim}
CONNECT X Y dmin dmax
\end{verbatim}\normalsize




This will create two lists of atoms (one from X and one from Y
which may
be atoms, elements or * as required) and uses these lists
when calculating connectivity. dmin and dmax are in angstroms. The
distances between all combinations of the atoms in the two lists
are
determined. If distances any lie within
the specified dmin -$>$ dmax range (and the bonds do not already
exist)
they will be added into the connectivity lists.  To connect up
CU1 O bonds for example you could
use:
\small\begin{verbatim}
CONNECT CU1 O 0.0 2.0
\end{verbatim}\normalsize


which would create one list - containing Cu1 only - and another
list -
containing all of the oxygen atoms - and then search for
inter-list
bonds. In this way, if any O - O bonds exist within the given
range they
will NOT be found. Note that if you require
connections to be calculated between atoms of the same type you
use:
\small\begin{verbatim}CONNECT X X dmin dmax.
\end{verbatim}\normalsize

\bigskip\Directive{ALL}
CONNECT ALL requires two arguments dmin and dmax. The result is
identical to CONNECT above except that ALL of the atoms are
included in
the calculation.
\small\begin{verbatim}
CONNECT ALL 0.0 2.0
\end{verbatim}\normalsize


will draw all bonds that fall within the 0.0 - 2.0 angstrom range.


\bigskip\Directive{DEFAULT}
CONNECT DEFAULT requires no arguments. This command does two
things:
\small\begin{verbatim}
1)      Reset all of the connectivity radii
        to their initial values
2)      Calculate the connectivity according
        to these radii.
\end{verbatim}\normalsize


It is effectively a 'start again' option as it removes any
changes in
bonding that have been introduced with the JOIN, REMOVE etc
options.


\bigskip\Directive{RADII}
This is a sub-command of CONNECT and comes immediately after it.
It requires the following arguments:
\small\begin{verbatim}
CONNECT RADII X r
\end{verbatim}\normalsize


where X is the name of an atom or element and r is its new
connectivity
radius. The connectivity of this atom/element is then
redetermined.


\bigskip\Keyword{COVALENT}


\bigskip\Keyword{IONIC}


\bigskip\Keyword{VANDERWAALS}
These subcommands set the specified atoms to appropriate connectivity 
radii eg CONNECT RADII IONIC N. The further subcommand ALL can be used 
to set all atoms.


\bigskip\Directive{HBONDS}
This command enables the user to search for hydrogen bonds within
the
structure. The syntax is :-
\small\begin{verbatim}
CONNECT HBONDS dmin dmax X Y
\end{verbatim}\normalsize


dmin/dmax is the range for searching (in angstroms), X/Y are atoms/elements to
be
included in the search. For a H-bond to be valid it must be
bonded to
one of the atoms in the list X,Y etc AND be within the range of
another
of the atoms. For example, if we are searching for H-bonds
involving oxygen atoms:
\small\begin{verbatim}
CONNECT HBONDS 0.0 2.5 O
\end{verbatim}\normalsize


will achieve this. There may well be eg C - H - O linkages within
the range
but only O - H - O ones will be registered. In this way we can
filter
the search and set a large value for dmax so that all the H-bonds
are
discovered without obtaining spurious information. The bond style
used
for H-bonds is dotted.


\bigskip\Directive{INTER}


This option is intended to be used after a PACK or ENCLOSURE 
operation. This will find connections between atoms which have DIFFERENT
packnumbers - ie. between assymetric units or between different GROUPS 
if PACK or ENCLOSURE GROUP has been used. The bonds are in the DOTTED 
style when generated. The syntax is:
\small\begin{verbatim}
CONNECT INTER O C 0.0 3.0
\end{verbatim}\normalsize


as for CONNECT itself.


\bigskip\Directive{FULL}


\bigskip\Directive{DOTTED}


The CONNECT command can also be used if bonds already exist to
alter the
style. There are three options:
\small\begin{verbatim}
CONNECT DOTTED at1 at2
CONNECT DOTTED at1 el1
CONNECT DOTTED el1 el2
\end{verbatim}\normalsize


These change the style of the single bond at1-at2, the
style
of any el1-at1 bonds and the style of any el1-el2 bonds respectively. Note
that in
the latter case, el1 and el2 may be the same element ie you can
make all
C-C bonds dotted.





\bigskip\Instruction{JOIN}




\bigskip\Directive{DOTTED}


\bigskip\Directive{FULL}


Another connectivity header command is JOIN. This is a more
specific
command which is used to make a new bond. The syntax is:
\small\begin{verbatim}
JOIN DOTTED at1 at2
JOIN FULL al1 at2
\end{verbatim}\normalsize


This will create a new bond between at1 and at2 of the specified type.


\bigskip\Instruction{REMOVE}


This is the reverse command to JOIN. It will break the bond between 
the specified atoms. The syntax is:
\small\begin{verbatim}
REMOVE at1 at2
\end{verbatim}\normalsize




\bigskip\Instruction{DISCONNECT}




This command is the reverse of CONNECT.
\small\begin{verbatim}
DISCONNECT C O 0.5 1.5
\end{verbatim}\normalsize


will get rid of any C-O bonds that are OUTSIDE the range 0.5 - 1.5
angstroms.


\bigskip\Directive{ALL}
DISCONNECT ALL removes all of the connectivity information.


\bigskip\Directive{GROUP}
DISCONNECT GROUP is intended primarily for use with disordered
structures although other uses can be envisaged. For example a
substituent
may be 'flipping' between two sites generating very short
distances between
atoms in different groups. It is not possible to DISCONNECT these
bonds
since some of the distances may appear to be 'normal'. To get
over this,
each disordered part is defined as a group (See DEFGROUP) and
DISCONNECT
GROUP is used to remove any bonds that exist between them. The
syntax is:
\small\begin{verbatim}
DISCONNECT GROUP group1 group2
\end{verbatim}\normalsize




\bigskip\Directive{ATOM}


This command deletes ALL bonds involving the specified atom. The 
syntax is:
\small\begin{verbatim}
DISCONNECT ATOM at1 
\end{verbatim}\normalsize


\section{Miscellaneous CONNECT commands}


There are three other commands which either relate to
connectivity or
to how the bond is represented on the output device.


\bigskip\Instruction{TAPER}


This controls the bond tapering. It has an initial value of 2.0. Increasing this increases the tapering of the bonds. This
tapering is
useful as it introduces a 3-D effect into the drawing. If
tapering is
not required then entering:
\small\begin{verbatim}
CONNECT TAPER 0.0
\end{verbatim}\normalsize


will achieve this.


\bigskip\Instruction{THICKNESS}




This command controls the thickness (radius) of the bonds in
angstroms. It
requires one argument, whose default value is 0.02.
\small\begin{verbatim}
CONNECT THICKNESS 0.04
\end{verbatim}\normalsize


will double the radius of the bonds as drawn.


\bigskip\Instruction{TOLERANCE}




This sets the tolerance used when determining whether or not
a bond
exists. The formula used is:
\small\begin{verbatim}
If  dist < ( C1 + C2 ) * tol  then a bond exists
\end{verbatim}\normalsize


where C1 and C2 are the connectivity radii of the atoms in
question.
The initial value of 'tol' is 1.1 ie the interatomic distance has
to be no
more than 10\% greater than the sum of the two connectivity radii
for a
bond to be found. The syntax for the command is:
\small\begin{verbatim}
CONNECT TOLERANCE n
\end{verbatim}\normalsize


\chapter{Control Of Colour}
\index{Cameron - Colour control}

Within CAMERON it is possible to control the colour of eachindividualatom and bond and also the colour of the labels.

\bigskip\Instruction{COLOUR}

This is the header command for colour control - it may befollowed byatom/element names if required. The syntax is:\small\begin{verbatim}COLOUR C BLUE N YELLOW ...\end{verbatim}\normalsize

The atom/elements are entered in pairs together with the colour name.A list of the current colours can be obtained by using 'INFO COLOUR'. There are two setsof colournames - those for normal colour and those for greyscale eithermay be used interchangeably as once the colour type is altered thecolours are translated accordingly.\small\begin{verbatim}COLOUR ALL colour\end{verbatim}\normalsize

Colours all the atoms the given colour

\bigskip\Directive{GROUP}

This sets the colour of all of the atoms in a given group. The syntaxis:\small\begin{verbatim}COLOUR GROUP groupname colourname\end{verbatim}\normalsize



\bigskip\Directive{FRAGMENT}COLOUR FRAGMENT n col will set the colour of all atoms in the fragment attached to atom n.

\bigskip\Directive{BACKGROUND}Sets the background colour.

\bigskip\Directive{TEXT}Sets the colour of the title and other annotation text.

\bigskip\Directive{MENUTEXT}

\bigskip\Directive{BUTTON}Selects the colours of the text and buttons in menu mode.

\bigskip\Directive{LABCOLOUR}

This sets the colour of the labels and requires a singleargument - thename of the new colour.

\bigskip\Directive{BONDS}

Altering the colour of bonds is a more complicated procedure.It isfollowed by atoms/elements that define the bond and a colourname.These arguments are therefore entered in threes. For example, tomakeall carbon carbon bonds colour blue and all CU1 to oxygen bondscolour yellowyou would use:\small\begin{verbatim}COLOUR BONDS C C BLUE CU1 O YELLOW\end{verbatim}\normalsize



\bigskip\Keyword{ALL}There is a further sub-command ALL which changes the colour ofALL ofthe bonds eg:\small\begin{verbatim}COLOUR BONDS ALL PINK\end{verbatim}\normalsize

makes all bonds colour PINK.

\bigskip\Directive{GROUP}

This sub-command requires arguments in pairs - the group nameand thenew bond colour:\small\begin{verbatim}COLOUR BONDS GROUP g1 LGREY\end{verbatim}\normalsize

colours all bonds BETWEEN atoms in group g1 (both atoms in a bondmust be inthe group for its colour to be altered) to colour LGREY.

\section{Example}


To have a picture with the following requirements:

\small\begin{verbatim}
All C atoms colour BLUE
All N atoms colour PINK
Cu1 colour YELLOW
All Cu1 - O bonds colour LGREY
All bonds between atoms in group g1 colour RED
All other bonds colour GREEN
All labels colour PURPLE
\end{verbatim}\normalsize




the commands would be:

\small\begin{verbatim}
COLOUR C BLUE N PINK CU1 YELLOW
BONDS ALL RED
(do this first)
BONDS CU1 O LGREY
GROUP g1 GREEN
LABCOLOUR PURPLE
\end{verbatim}\normalsize




\bigskip\Directive{BACKGROUND}


The default background colour is WHITE but this can be changed as 
required by the COLOUR BACKGROUND colour command.


\bigskip\Directive{DEFAULT}


The default colours for the elements are as defined in the PROP.CMN 
file -
\small\begin{verbatim}
B - ORANGE
Br,Cl,F,I - LGREEN
C - GREEN
D,H - LGREY
N - BLUE
O - RED
P - PURPLE
S - YELLOW
SI - DGREY
\end{verbatim}\normalsize


These colours can be recovered if altered by using COLOUR DEFAULT which 
returns ALL atoms to their original colours.


\bigskip\Directive{NORMAL}


\bigskip\Directive{GSCALE}


These sub-commands allow the user to see how the hardcopy greyscale
picture will look. COLOUR GSCALE changes the screen colours to the
equivalent greyscale colours. The colour names GREYn (n=1,14) can be
used to specify colour changes if required as COLOUR N PURPLE makes
little sense on a greyscale picture.

\small\begin{verbatim}
      BLACK      BLACK
      BLUE       GREY1
      GREEN      GREY2
      ORANGE     GREY3
      RED        GREY4
      CYAN       GREY5
      MAGENTA    GREY6
      LGREY      GREY7
      GREY       GREY8
      LGREEN     GREY9
      LBLUE      GREY10
      LRED       GREY11
      PINK       GREY12
      PURPLE     GREY13
      YELLOW     GREY14
      WHITE      WHITE
\end{verbatim}\normalsize


\chapter{Atom Labelling}
\index{Cameron - Labelling atoms}

\bigskip\Instruction{LABEL}



\bigskip\Instruction{NOLABEL}



The LABEL and NOLABEL commands control the atom labelling. Theyareset up in an identical way to INCLUDE and EXCLUDE and the syntaxisidentical:\small\begin{verbatim}LABEL C1 C2 ONOLABEL C H1LABEL ALLNOLABEL ALLLABEL GROUP g1 ....\end{verbatim}\normalsize

are all valid. Note that atoms will not be labelled if they arenot included in the picture.

\bigskip\Directive{MOUSE}

All atoms will be labelled if the LABEL command has been used forthem. The label positions can be altered using the mouse if required.Note that label positions are recalculated if atoms are included orexcluded or a change in the view direction has occurred. Therefore, it isadvisable that the 'final' view is obtained before labels are positionedwith the mouse.

Mouse labelling is controlled as follows. The user clicks on aposition on the screen (once the message Mouse Labelling activated hasbeen seen). If this position is over a label then the label is replacedby a box of the same size as the label. A red cross is drawn over theatom that the label refers to to aid identification. The mouse is thenused to position the TOP LEFT HAND CORNER of the label by a secondclick. Alternatively, hitting the N key (for Nolabel) will remove thelabel altogether.

If the mouse is clicked on an atom which is NOT labelled then a labelwill appear at the atom centre, this label can then be moved asdescribed above.

If at any time the user wishes to view the current picture without releasing the mouse, this can be done by hitting the 'V' key.

\bigskip\Directive{GENERATED}

\bigskip\Directive{INITIAL}

These commands relate to the pack number display while labelling.Atoms are assigned a pack number after PACK or ENCLOSURE. This number isthen displayed as eg N1\_5. These numbers are used to refer to atoms andelements as required. INITIAL (the default) will just display N1 whileGENERATED shows N1\_5.

\bigskip\Directive{CELL}Controls the cell labelling.

\bigskip\Directive{FRAGMENT}Allows the user to set labels for a given fragment.

\bigskip\Instruction{FONT}

This sets the point size of the font to be used in hardcopy output. It requires one argument.

\bigskip\Directive{DEFAULT}This resets the point size to its default value.

\bigskip\Instruction{TEXT}



\bigskip\Directive{POSITION}This allows the user to annotate a picture. The syntax is:\small\begin{verbatim}TEXT "text string" POSITION x y\end{verbatim}\normalsize

The text must be in quotes. x and y are the position of the text in percentages from the top left hand corner of thediagram. After the text is processed it is assigned a number.

\bigskip\Directive{NUMBER}The NUMBER command can be used to move a text item after it has been created by using TEXT NUMBER n POSITION x y.  \chapter{Other Picture Controlling Commands}
\index{Cameron - Misc picture control}

There are a few other commands in CAMERON which deal with control ofthe type of picture being used.

\bigskip\Instruction{MAXIMISE}



This command is related to the VIEW commands PLANE, ALONG, VERTICALetc. In these commands the view direction does not define all of thedegrees of freedom of the molecule. It is possible to rotate themolecule further so that as much of is as possible is shown on thescreen.

\bigskip\Directive{ON}

\bigskip\Directive{OFF}

These two commands switch maximisation calculations ON and OFF. Thesecalculations are slow - especially for large molecules so they areincluded as an option.

\bigskip\Instruction{SCALE}



DFIX

DUNFIX

These two commands allow the user to control the calculation ofpicture scale. SCALE FIX sets the scale to its current value and doesnot recalculate it, irrespective of changes in the number and positionof atoms in the picture. SCALE UNFIX reverts to scale calculations.

\bigskip\Directive{SET}

SCALE SET is followed by a number, the value of the scale to be usedfor all subsequent pictures. This can be altered with another SCALE SETcommand or by issuing SCALE UNFIX.

\chapter{Symmetry Input}
\index{Cameron - Symmetry Input}\section{Input of symmetry operators}


CAMERON is a crystallographically oriented program and hence
many of
its functions require the use of symmetry operators. There are
two
methods of inputting the symmetry operators, inputting the
Spacegroup
symbol and inputting the individual operators themselves.


\bigskip\Instruction{SPACEGROUP}




The SPACEGROUP command is followed by the symbol, which must
contain
all UPPER CASE letters. If the symbol is in a non-standard
setting then
the full symbol must be entered. For example  P 21 will be
interpreted
as P 1 21 1 which is the standard setting. If P 21 1 1 is
required then
the "1's" must be entered to force the choice of unique axis.


The syntax of the command is:
\small\begin{verbatim}
SPACEGROUP X X X X
\end{verbatim}\normalsize


where X are "fields" of the symbol, with spaces between fields
ie
entered as P 21 and NOT P21. The program will then output the
operators
that have been calculated for the symbol . Note that if a centre
of
inversion is present the inverted operators will NOT be shown.
Also the
operators generated from the centring vectors - eg 1/2 1/2 1/2
for body
centring - will NOT be shown. The complete list of operators used
for
packing etc can be found by entering INFO SYMMETRY.


\bigskip\Instruction{SYMMETRY}




The input of symmetry operators can be done "by hand" if
required.
There are several steps and sub-commands available to do this.


\bigskip\Directive{OPERATORS}


This sub-command MUST be entered even if only the x y z
operator is to
be included. The syntax is :-
\small\begin{verbatim}
SYMMETRY OPERATORS x y z -x y+1/2 -z ...
\end{verbatim}\normalsize


with the operators being entered with the translational part in
fractional form ( 1/2 , 1/3 , 1/4 , 2/3 , 1/6 , 5/6 are
recognised )
which must come after the x/y/z as required. The
fractional
part must be linked to the x/y/z part with a + or - sign. Note
that NO
spaces are allowed within each part of the operator. This is so
that
ambiguities cannot arise.


\bigskip\Directive{CENTRE}


This command is used to introduce a centre of inversion into
the
symmetry information it must be followed by a centring letter 
(P,A,B,C,I,F,R) or the command VECTORS.


\bigskip\Directive{NOCENTRE}


This command is used to specify that there is no centre of
inversion - the centring vectors are the specified by letter or by using
VECTORS.


\bigskip\Keyword{VECTORS}


This command enables the user to introduce centring vectors
into the
symmetry operators eg for body centring use :-
\small\begin{verbatim}
VECTORS 0 0 0 1/2 1/2 1/2
\end{verbatim}\normalsize




\bigskip\Directive{USE}


\bigskip\Directive{NOUSE}
These commands allow the user to omit certain symmetry operators from
the packing calculations. The command is followed by the operator
numbers
(found using INFO SYMMETRY) of the operators needed.


\bigskip\Keyword{ALL}
This may be used to USE/NOUSE all of the operators. Note that using
NOUSE ALL without following it with USE n will result in no atoms being
generated after a pack operation!
\section{Example of Space Group Input}


In order to end up with the Spacegroup F m m m we require :-

\small\begin{verbatim}
SPACEGROUP F M M M

or

SYMMETRY OPERATORS
X Y Z  -X -Y Z  -X Y -Z  X -Y -Z
CENTRE
VECTORS
0 0 0
0 1/2 1/2
1/2 1/2 0
1/2 0 1/2
\end{verbatim}\normalsize


\section{Other symmetry related operations}


\bigskip\Instruction{SETUNIT}




This command causes the atoms that are to be included in the
current
picture to be set as the asymmetric unit. Their data then
replaces that
of the initially input atoms and they are used as the asymmetric
unit
from now on. This is useful if, for example only half of a
molecule is
present in the asymmetric unit. The other half can be generated
with ADD
and then SETUNIT is used to treat all of the atoms as the basic
"building block". SETUNIT is a DANGEROUS command in that it
cannot be
undone - UNPACK will NOT reverse the operation. Therefore the
user is
prompted for confirmation before the command is executed. The
syntax is
simply SETUNIT.


\bigskip\Instruction{ENANTIO}


This command inverts the hand of the atoms in the asymmetric unit. Note
that the operator applied is :-

\small\begin{verbatim}
-1  0  0
 0 -1  0
 0  0 -1
\end{verbatim}\normalsize


and no spacegroup changes that may be required are made.


\bigskip\Instruction{SPECIAL}




\bigskip\Directive{ON}


\bigskip\Directive{OFF}
Controls the special position calculations that occur during packing to 
eliminate duplicate atoms. By default this is ON.



\chapter{Crystal Packing Commands}
\index{Cameron - Packing commands}

There are two methods of crystal packing available, PACK andENCLOSURE. Both use ALL of the symmetry operators in theSpacegroup togenerate all of the atoms that lie within a user-defined volume.Thecommands ADD and MOVE are available if single symmetry operatorsarerequired. The difference between the two commands lies in themethod ofvolume definition.\section{PACK}


The PACK command allows the user to define a volume to be
filled
relative to the unit cell. One of two-sub commands is required
to define
this volume.


\bigskip\Instruction{CELL}


PACK CELL will cause the program to generate all the atoms that
lie
within the unit cell. This is the default option if no range for
packing
is input.


\bigskip\Instruction{WINDOW}


PACK WINDOW allows the user to define the volume in terms of the
unit
cell axes. It is followed by three pairs of numbers.
\small\begin{verbatim}
PACK WINDOW xmin xmax ymin ymax zmin zmax
\end{verbatim}\normalsize


The values of xmin etc are relative to the unit cell origin.
Therefore
to define a volume of which contained all of the x axis, all of
the y axis and
the first half of the z axis we would use:
\small\begin{verbatim}
PACK WINDOW 0.0 1.0 0.0 1.0 0.0 0.5
\end{verbatim}\normalsize


If more than one unit cell is required negative numbers and
numbers
exceeding one may be used.


One more sub-command may be entered after the WINDOW or CELL
commands.
If this command is omitted then the option chosen in the last
PACK
command will be used. If this is the first time PACK is used then
CUT is
the default option.
Three sub-commands are available:


\bigskip\Directive{CUT}
CUT will generate all the atoms that lie within the defined
volume. ie.
the generation is "cut" at the boundary.


\bigskip\Directive{COMPLETE}
COMPLETE is most useful for molecular crystallographers. It will
generate all the asymmetric units that have ANY ATOMS lying with
the
defined volume.


\bigskip\Directive{CENTROID}
CENTROID is similar to COMPLETE except that it calculates the
centroid
of the asymmetric unit (as though all of the atoms have equal
weight) and
includes only those which have their centroid within the defined
volume.
This is particulary useful for molecular crystallographers as it
creates
a picture without the "odd atoms" at the edge of the unit cell.
\section{Dealing with connectivity}


There are three other qualifying commands that apply to PACK:


\bigskip\Instruction{INTRA}




\bigskip\Instruction{INTER}




\bigskip\Instruction{KEEP}


These deal with the treatment of connectivity calculations once
the PACK
has been carried out.


\bigskip\Instruction{INTRA}




This qualifiers means that the connectivity will be calculated
within
each newly generated asymmetric unit only. Any changes to the
bonding -
eg with JOIN, REMOVE, CONNECT, COLOUR BONDS - will be undone.
This is
the fastest option.


\bigskip\Instruction{INTER}




In this case connectivity is calculated once all of the atoms
have
been generated - therefore if any intra-unit bonds exist they
will be
found.


\bigskip\Instruction{KEEP}




This is the default option. Connectivity is copied from the
unit used
to do the packing - this includes colour and style changes if any
- into
the bond info of the new atoms. This is done for each asymmetric
unit as
it is generated and is slower than INTRA as it requires more
comparisons
to be carried out.
\section{EXAMPLES}


Therefore, to create a picture containing all of the complete
molecules within a cube of side equal to 2 unit cells we need :-

\small\begin{verbatim}
PACK
WINDOW 0.0 2.0 0.0 2.0 0.0 2.0
COMPLETE
\end{verbatim}\normalsize


Or alternatively,
\small\begin{verbatim}
PACK
WINDOW -1.0 1.0 -1.0 1.0 -1.0 1.0
COMPLETE
\end{verbatim}\normalsize


\section{ENCLOSURE}


The enclosure command is more flexible than PACK as it enables
the
user to choose the "origin" for the atom generation. The first
task is
to specify this origin which is either a point in the unit cell
or an
atom.
\section{Choice of enclosure "origin"}


\bigskip\Instruction{ATOM}




If we wish to generate the atoms around Cu1 in order to examine
the
coordination environment for example we can use:
\small\begin{verbatim}
ENCLOSURE ATOM CU1
\end{verbatim}\normalsize




\bigskip\Instruction{POINT}




Alternatively we can choose the centre of the unit cell:
\small\begin{verbatim}
ENCLOSURE POINT 0.5 0.5 0.5
\end{verbatim}\normalsize


The POINT sub-command can be used if an atom does not lie at the
point in
question eg if we are examining a "hole" within a structure.
\section{Type of volume to be used}


There are three different ways of defining the volume of
enclosure:


\bigskip\Instruction{SPHERE}


This will generate a sphere of enclosure about the origin. The
syntax is:
\small\begin{verbatim}
SPHERE r
\end{verbatim}\normalsize


where r is the radius in angstroms of the sphere.
\small\begin{verbatim}
e.g. ENCLOSURE ATOM C1 SPHERE 4.0 CUT VIEW
\end{verbatim}\normalsize




\bigskip\Instruction{ANORTHIC}


This is used to generate an ANORTHIC box ie a box whose sides are parallel to the unit cell axes. As this box is directly related
to the
unit cell its dimensions are given in fractional coordinates. The
syntax
is:
\small\begin{verbatim}
ANORTHIC -x +x -y +y -z +z
\end{verbatim}\normalsize


Therefore, to generate an anorthic box with sides extending one
quarter of a unit
cell in all six directions from the defined origin we use:
\small\begin{verbatim}
ANORTHIC 0.25 0.25 0.25 0.25 0.25 0.25
\end{verbatim}\normalsize




\bigskip\Instruction{ORTHOGONAL}


This is used to generate a box whose sides are perpendicular to
each
other. The z axis lies along the current view direction and the
x and y
axes lie across and vertically up the screen respectively. The
dimensions of this box is defined in a similar way to the
ANORTHIC box
except that they are given in angstroms. For example, to
generate a box that is 4.0 angstroms wide in x, 1 in y and 0.5
in z we
would use:
\small\begin{verbatim}
ORTHOGONAL 2.0 2.0 1.0 1.0 0.25 0.25
\end{verbatim}\normalsize


It is important to note that this volume is related to the
CURRENT view
direction. The VIEW ALONG AXIS command can be used to orient the
picture
prior to carrying out this command if required.


\bigskip\Directive{CUT}


\bigskip\Directive{COMPLETE}


\bigskip\Directive{CENTROID}
As with PACK a further sub-command can be used if desired to
define the
type of boundary handling used. CUT, COMPLETE and CENTROID have
the same
meanings as described above for PACK.


\bigskip\Keyword{INTRA}


\bigskip\Keyword{INTER}


\bigskip\Keyword{KEEP}
These sub commands have an identical meaning to those described
for PACK
above.
\section{EXAMPLE}


To generate all the atoms that lie within a sphere of radius
5.0
angstroms about a CU1 atom we use:
\small\begin{verbatim}
ENCLOSURE
ATOM CU1
SPHERE 5.0
CUT
\end{verbatim}\normalsize




To generate all the asymmetric units that have any atoms within
a box
of side 0.5 units around the unit cell centre we use:-
\small\begin{verbatim}
ENCLOSURE
POINT 0.5 0.5 0.5
ANORTHIC 0.25 0.25 0.25 0.25 0.25 0.25
COMPLETE
\end{verbatim}\normalsize




And to generate all the atoms that lie inside a box centred on
the
point 0.25 0.25 0.25 and of sides x=1.0 y=2.0 and z=3.0 angstroms
:-
\small\begin{verbatim}
ENCLOSURE
POINT 0.25 0.25 0.25
ORTHOGONAL 0.5 0.5 1.0 1.0 1.5 1.5
CUT
\end{verbatim}\normalsize


\section{PACKING MORE COMPLICATED STRUCTURES}


The PACK and ENCLOSURE commands always work on the initial data-
unless a SETUNIT command has been issued. In some cases however, it is
more useful for the user to be able to deal with certain sections of the
structure separately. This is most likely to occur where there is more
than one distinct unit in the asymmetric unit. This is dealt with by the
command GROUP.


\bigskip\Instruction{GROUP}




GROUP can be used directly after both PACK and ENCLOSURE. The syntax
is:-

\small\begin{verbatim}
PACK GROUP groupname CELL ...
or
PACK GROUP groupname WINDOW ...
etc
\end{verbatim}\normalsize


This causes the groups to be packed individually, the CENTROID,
COMPLETE and CUT commands are applied to the group and not to the
asymmetric unit as a whole. Packs are cumulative, unless a
PACK/ENCLOSURE is done without the GROUP sub command in which case the
previously generated atoms are overwritten. Groups are defined with the
command DEFGROUP.


If required, more than one groupname can be packed at once - they are
all treated separately. All groups can be packed in turn if PACK GROUP *
is used.



\section{UNPACK}


This command causes all atoms generated via PACK or ENCLOSURE
to be
removed from the calculations, drawings etc. It has no
sub-commands.
It also works with ADD and MOVE generated data.



\chapter{Add And Move - Further Symmetry Related Commands}
\index{Cameron - Add and Move}

The PACK and ENCLOSURE commands already detailed allow the usertoapply all of the symmetry operators in the spacegroup to theinitialset of atoms in order to get a fully 'packed' result. In somecaseshowever, the user may wish to apply only one symmetry operatoror toapply ones that are not present in the spacegroup. The ADD andMOVEcommands allow this.\section{ADD}


The ADD command allows the user complete control over the
symmetry
operator used to generate new atoms. The first task of the user
is to
generate a list of those atoms to be used in the symmetry
generation
later. One of the following sub-commands is required :-


\bigskip\Instruction{ATOMS}


The names of atoms to be included in the pack list are specified
here.
Element names can also be used if required.


\bigskip\Instruction{ALL}


ALL refers to the atoms that are in the current list. If any
atoms
have been generated by previous PACK , ENCLOSURE or ADD commands
then
these will all go into the list.


\bigskip\Instruction{INITIAL}


ADD INITIAL means that the only atoms to go into the ADD list are
those
that were initially input.


\bigskip\Instruction{GROUP}


This is followed by a group name. The group must have been
previously been defined by the command DEFGROUP.





Once the ADD list has been created the user must then supply
the
symmetry operators which will act on the atoms in this list to
generate
the new atoms. The symmetry input is in two parts.


\bigskip\Directive{OPERATOR}
The symmetry operator may be input in decimal or fractional form eg
\small\begin{verbatim}
x y+1/2 z

-0.333-x -y -z
ETC
\end{verbatim}\normalsize


Decimal translations may  come before or after the axis symbol.
The fractions 1/2, 1/3, 1/4, 2/3, 1/6 and 5/6 are accepted by the

program, but {\bf must} appear {\bf after} the
x/y/z
character. There must be spaces between the three parts but NO
SPACES
within the operator ie
\small\begin{verbatim}
X + 1/2 -Y  Z
\end{verbatim}\normalsize


will produce an error as it is not possible to tell whether you
mean
X+1/2, -Y, Z or X, +1/2-Y, Z. This strict input syntax is used
to eliminate
any ambiguities.


\bigskip\Directive{TRANS}
Translations can also be applied if required. The translations
are
applied in unit cell fractions. The syntax is :-
\small\begin{verbatim}
TRANS x y z
\end{verbatim}\normalsize


\section{EXAMPLES}
To generate an atom at x+1/2 y z from an atom at x y z we can use
\small\begin{verbatim}
ADD
ATOMS C1
OPERATOR X+1/2 Y Z
\end{verbatim}\normalsize


or we could use
\small\begin{verbatim}
ADD
ATOMS C1
TRANS 0.5 0 0
\end{verbatim}\normalsize


The OPERATOR and TRANS commands can be used together if required.
We can
apply a symmetry operator followed by a translation. This reduces
the
errors that may occur when trying to combine the two things into
one
symmetry operator.



The use of INITIAL versus ALL is illustrated  below. Start with
:-
\small\begin{verbatim}
ADD
ALL
TRANS 1 0 0
\end{verbatim}\normalsize


which gives us a molecule at x y z and another at x+1 y z.
Follow this with :-
\small\begin{verbatim}
ADD
ALL
TRANS 0 1 0
\end{verbatim}\normalsize


and we get four molecules, x y z , x+1 y z , x+1 y+1 z and x y+1
z.
Following it with :-
\small\begin{verbatim}
ADD
INITIAL
TRANS 0 1 0
\end{verbatim}\normalsize


Gives us three molecules, x y z, x+1 y z and x y+1 z.



\section{MOVE}


The syntax for this command is identical to that for ADD.
Therefore
the commands available are :-


\bigskip\Instruction{ATOMS}




\bigskip\Instruction{ALL}




\bigskip\Instruction{GROUP}




\bigskip\Instruction{INITIAL}


Which must be followed one (or both) of :-


\bigskip\Directive{OPERATOR}


\bigskip\Directive{TRANS}


The MOVE command applies a symmetry operator and/or a
translation to
all of the atoms held in the list defined by the
ATOMS/ALL/GROUP/INITIAL
commands. Unlike ADD therefore, the same number of atoms are
present at
the beginning and end of the operation.



\chapter{Distance And Angle Calculations}
\index{Cameron - Distance and Angle Calculations}

CAMERON allows the user to calculate distances , angles andtorsion angles.

\bigskip\Instruction{DISTANCE}



In order to perform a distance calculation two atoms lists must begenerated. The first list is used as a 'starting atom' and the secondlist is for 'target atoms'. Distances will be calculated between atomsin separate lists BUT NOT within the lists themselves. The list may begenerated in two ways:-\small\begin{verbatim}DISTANCE N O\end{verbatim}\normalsize

generates two lists - both of which can all the N and all the O atoms.In this case N-O, N-N and O-O distances will be found.

\bigskip\Directive{FROM}

\bigskip\Directive{TO}These subcommands allow to specify different starting and target atoms.\small\begin{verbatim}DISTANCE FROM N TO O\end{verbatim}\normalsize

will only calculated N-O distances. Note that the distance arguments maybe either atoms or elements as required.

\bigskip\Directive{RANGE}This sets the minimum and maximum ranges for displayed distances. Thesyntax is :-\small\begin{verbatim}DISTANCE RANGE dmin dmax\end{verbatim}\normalsize

distances are given in angstroms. If only TWO atoms are present in theatom list then the distance will be outputted irrespective of the range.However, as the full calculation will be carried out first there may bea time delay while the calculation proceeds.\section{Method of Calculation}


The distances output make use of the symmetry operators in order to
find distances within the given range. The starting atom coordinates are
NOT ALTERED but those of the target atoms are. The symmetry operators
(and any suitable translations) are used to move the target atoms
around. The output produced is:-

\small\begin{verbatim}
N1_0      O2      2.323
Operator x y z Translations 0 0 1
N1_0      O2_4    1.114
Operator -x y+1/2 -z Translations 0 0 -1
\end{verbatim}\normalsize


These show the N1 to O2 distances. The first distance relates to an O2
atom which does not currently exist. The operator and translations shown
can be used with ADD to generate the atom if required. The second
distance is to an atom (O2\_4) which does exist. ie. if no pack number is
given for the second atom it is not present in the current list.


\bigskip\Instruction{ANGLE}




\bigskip\Instruction{TORSION}




These commands are used to find angles and torsion angles between
atoms that are in the current list. They are entered in sets of three
(or four) as required.



\chapter{Information On Data Held Within The Program}
\index{Cameron - Information on data (meta data)}

CAMERON holds a number of pieces of information while it is runningand in some cases it is useful for the user to be able to access thisinformation.

\bigskip\Instruction{INFORMATION}



This command is followed by a sub-command which specifies the type ofinformation wanted.

\bigskip\Directive{CELL}Outputs the unit cell parameters.

\bigskip\Directive{ATOMS}Outputs the names of atoms stored within the program. It produces twolists - one of atoms currently included and one of atoms currentlyexcluded from the picture.

\bigskip\Directive{COLOUR}This outputs the colour names that are available.

\bigskip\Directive{SYMMETRY}This command outputs the symmetry operators currently stored.

\bigskip\Directive{GROUP}Outputs a list of the currently defined groups and their members.

\bigskip\Directive{PACKNUMBERS}Outputs the symmetry operator and translation associated with a given packnumber. The syntax is INFO PACKNUMBER n1 n2 n3.

\chapter{Group Definitions}
\index{Cameron - Group definitions}

For complicated structures it is sometimes useful to defineGROUPS ofatoms which can be referred to as a whole later. EXCLUDE,INCLUDE,COLOUR BONDS etc can all be used with the GROUP sub-command.

\bigskip\Instruction{DEFGROUP}



This is the main header command and is followed by the name ofthegroup. Note that it is not possible to have group names thatbegin withGROUP itself - g1, g2 are valid names but group1, group2 are not.Up totwelve characters are allowed to define the group name.

\bigskip\Directive{ATOMS}

This is followed by a list of atoms/elements to be included inthegroup.

\bigskip\Directive{GROUP}

It is possible to have an atom as a member of up to threegroups atonce. You can therefore add groups into other groups (see Example).

\bigskip\Directive{FRAGMENT}

The user can include atoms in a group by defining a fragment. Thefragment definition requires a single atom. Any atoms joined to it, andany atoms to those etc are made part of the group.

\bigskip\Directive{DELETE}

You can also remove atoms from groups if required.\section{Example}


Consider an molecule that contains a tri-phenyl phosphine. A
use of
the DEFGROUP command would be :-

\small\begin{verbatim}
DEFGROUP PHENYL1
ATOMS C1 C2 C3 C4 C5 C6
DEFGROUP PHENYL2
ATOMS C11 C12 C13 C14 C15 C16
DEFGROUP PHENYL3
ATOMS C21 C22 C23 C24 C35 C36
DEFGROUP PPH3
ATOMS P
GROUP PHENYL1 PHENYL2 PHENYL3
\end{verbatim}\normalsize


If you then realise that there are two phosphorus atoms in the
molecule
P1 and P2 you can use :-
\small\begin{verbatim}
DEFGROUP PPH3 DELETE P2
\end{verbatim}\normalsize


to remove P2 as it is not a member of the tri-phenyl phosphine
group.



\chapter{Miscellaneous Commands}
\index{Cameron - Miscellaneous commands}

\bigskip\Instruction{RESET}

This command allows the user to begin again without having to exit and reload CAMERON. All current flags will be set to their initial states. Note that if a CAMERON.INI file is present it will not be obeyed after aRESET command. The screen will be set to VGA automatically (as on startup) if a CAMERON.SRT file is present.

\bigskip\Instruction{LOG}

The LOG command is used to generate a log of the CAMERON session, thismay be OBEYed later if required.LOG must be followed by a filename. If a file is already in use for alog file then the log information will be transferred to the new file.

\bigskip\Instruction{EDIT}



\bigskip\Directive{ON}

\bigskip\Directive{OFF}This sets the edit option which is used when an error is found in an input line. If edit is on (the default) then the user is prompted as described in Chapter 1. If not the computer will bleep and put up the message "Error: Automatic Abandon", ignore any unexecuted commands and wait for a new input line.

\bigskip\Instruction{MENUS}



\bigskip\Directive{ON}

\bigskip\Directive{OFF}The user is able to control the program via push button menus at the side of the screen (PC only). These can be turned on or off as required.When the menus are active the user can input commands as normal by typing - they will appear in a box at the bottom of the screen.

\bigskip\Instruction{ISSUE}

This allows the user to execute a single system command. This command should be enclosed in quotes if it is longer than one word or if it corresponds to a CAMERON command.

\bigskip\Instruction{PRINT}

This allows the user to print a hardcopy file. The file will be closed if it is currently attached to CAMERON via a COPY command. {\bf NOTE} that the print job will be very slow - it may be quicker to close down CAMERON first.\chapter{How To Stop The Program}
\index{Cameron - How to stop the program}

\bigskip\Instruction{END}



The command END will cause the program to finish.\chapter{Menu Definition File}
The labels on the menu buttons and the actions they invoke are in the file cameron$\backslash$button.cam.

The first line gives the number of columns and the number of lines in the menu.After this, adjacent pairs of lines give the text to appear on the button, and the CAMERON commands to be obeyed by the button. The pairs of lines are given first for column 1, then column 2 etc.

\chapter{Some Useful Ideas}
\index{Cameron - Some useful ideas}

To produce black and white illustrations for papers,try\small\begin{verbatim}colour c blackcolour h blackcolour n blackcolour o blackcolour bond all whiteball all unfilball h fillcolour back greyELLIPCNOELLIPTYPEALL4CONNTAPER3VIEW\end{verbatim}\normalsize



To hack up a packing diagram, cutting out most and just keeping a bit,try (all on one line)\small\begin{verbatim}EXCLUDE ALL INCLUDE AREA\end{verbatim}\normalsize

Then use the mouse to draw round the bit you want to keep.



\chapter{Cameron Manual}


 \small\begin{verbatim}\end{verbatim}\normalsize



\printindex
\end{document}
