\documentclass[10pt,a4paper]{report}
\usepackage[centertags]{amsmath}
\usepackage{amsfonts}
\usepackage{amssymb}
\usepackage{amsthm}
\usepackage{makeidx}
\usepackage{newlfont}
\usepackage{fancyhdr}
\usepackage[bookmarks,colorlinks,plainpages,backref]{hyperref}
\pagestyle{fancy}
\hfuzz2pt % Don't bother to report over-full boxes if over-edge is < 2pt
\makeindex
\addtolength{\headwidth}{80pt}
\addtolength{\textwidth}{80pt}
\addtolength{\oddsidemargin}{-30pt}
\begin{document}
\lhead{\slshape \rightmark}
\chead{}
\rhead{\thepage}
\lfoot{Fri Jul  5  2024}
\cfoot{}
\rfoot{}
\renewcommand{\headrulewidth}{0pt}
\renewcommand{\footrulewidth}{0pt}
\fancypagestyle{plain}{%
\fancyhf{}
\fancyfoot[L]{Fri Jul  5  2024}
\fancyfoot[C]{\thepage}
\renewcommand{\headrulewidth}{0pt}
\renewcommand{\footrulewidth}{0pt}}
\newcommand{\Instruction}[1]{{\bf #1}}
\newcommand{\Directive}[1]{{\bf \emph{#1}}}
\newcommand{\Keyword}[1]{\emph{#1}}
\sloppy
\title{Crystals Manual}
\author{Chemical Crystallography Laboratory, Oxford}
\date{Fri Jul  5  2024}
\maketitle
\tableofcontents
\chapter{Introduction To The System}


\section{Layout of this manual}



This manual is organised as follows


 1.0      Introduction


 \ref{defandconv}      Definitions and Conventions


 \ref{immediate}      Immediate commands - tailoring the program


 \ref{crysdatabase}      The CRYSTALS database


 \ref{initialdinput}      Initial Data input


 \ref{reflectdata}      Reflection data


 \ref{atomparams}      Atoms and Parameters


 \ref{sfandls}      Structure factors and least squares


 \ref{fouandpat}      Fourier and Patterson maps


 \ref{geomandres}     Geometry and Results


 \ref{grcameron}     Graphics - CAMERON


 \ref{twinning}     Twinned Data


 \ref{matrixcalc}     The Basic Matrix Calculator




\section{Status}


Version 14 of CRYSTALS continues to be based on a
 version (Issue 2) written by J.R. Carruthers in collaboration with
 J.S.Rollett during 1977-78, which was a rewrite of the 1975 CRYSTALS 
 system implemented  
 on the ICL 1900 series of computers. It contains significant
 contributions by Paul Betteridge, David Kinna, Lisa Pearce, Allen Larson,
 and Eric Gabe and many students and visitors to the Chemical
 Crystallography Laboratory, Oxford. The graphical user interface (GUI)
 was written by Richard Cooper as part of a Part II and D
 Phil project (supported by a CCDC studentship) in collaboration with
 Ludwig Macko and Markus Neuburger in Basel, who were working on a
 parallel Macintosh interface.


 A suitable citation is:


 Betteridge, P. W.,  Carruthers, J. R.,  Cooper, R. I.,
 Prout, K., Watkin, D. J. (2003). J. Appl. Cryst. 36, 1487.





While CRYSTALS can still be executed in 'batch mode' (ASCII file in,
ASCII file out), the major demand is now for the version running under
Windows on Intel processors, with some small demand for the
LINUX version. The GUI permits the user to continually see the structure
as it develops, and to interact with it and the analysis through
conventional windows features. The 'command line' and 'use file' modes
have been retained for experienced users, or users wishing to explore
new ideas. The 'SCRIPTing' language has been extended to enable full
control and design of the user interface to be handled from ASCII files.






\chapter{Definitions And Conventions}
\index{Command syntax}\index{Syntax}\label{defandconv} \section{Syntax of Commands}

 Commands are given to CRYSTALS as small packages, rather like
 sentences. This enables the program to recognise when the user thinks
 that a piece of input
 is complete and then, after inserting any default values and checking
 for errors, perform the task.





\bigskip\Instruction{Format of Commands}




All command packages follow the same general format. The command is
 introduced
 with a backslash (or alternatively, a hash symbol or ampersand \&) and
 ends with the
 word 'END'.

\small\begin{verbatim}
      \COMMAND ([keyword=]value ) ...
      (DIRECTIVE ([keyword=]value ) ...)
      END
\end{verbatim}\normalsize


 Items in
 round brackets '()' may be absent, items in square brackets '[]' are
 optional. Ellipsis '...' means the preceding item may be repeated.


Actual data on a COMMAND or DIRECTIVE line is input in free format,
 with at least one space (or sometimes an optional comma) terminating an
 item. Data items may either be preceded with the keyword and its '=' sign,
 or if the order given in the definition is strictly followed, just by
 the data values. COMMANDS, DIRECTIVES and KEYWORDS can be abbreviated to
 the minimum string which resolves ambiguity.
 Both types of identification can be intermingled.


The following examples are all identical to the program. They use
the command $\backslash$DISTANCES to compute all interatomic distances in the range
0.0-1.9 Angstrom, and all interatomic angles involving bonds in the range
1.6-2.1 Angstrom.

\small\begin{verbatim}
      \DISTANCE
      SELECT RANGE=LIMITS
      LIMITS DMIN=0.0 DMAX=1.9 AMIN=1.6 AMAX=2.1
      END
\end{verbatim}\normalsize



\small\begin{verbatim}
      \DIS
      SE RA=LI
      LI 0.0 1.9 1.6 2.1
      EN
\end{verbatim}\normalsize



\small\begin{verbatim}
      \DIST
      SEL RANGE=LIM
      LIMI DMAX=1.9 1.6 2.1
      END
\end{verbatim}\normalsize




 Note that in the last example, the value for DMIN is omitted (its
default value turns out to be 0.0) and the list of parameters starts
with a DMAX: CRYSTALS knows that the next parameter is AMIN so it 
need not be specified.



\section{Syntax of the manual}


Those parts of the manual describing data and command input will
 generally be in the following format:
\small\begin{verbatim}
      1. A summary of the complete command, with all directives and
         keywords. Ellipsis (...) may be printed to represent omitted
         but similar parameters.
      2. An typical example. This may not demonstrate all the available
         options
      3. A description of all the directives and keywords.
\end{verbatim}\normalsize


For a simple example, see LIST 1, section \ref{LIST01}.
\index{Command types}
\section{Types of Commands:}



\bigskip\Instruction{Lists}




These contain related data items, grouped together so that CRYSTALS can
 check
 that the data is complete. These lists are stored in the CRYSTALS
 database. Usually, input of a new list of a given type over-writes an
 existing list of the same type. In general, LISTS do not 'do' anything.


There are two types of syntax for LISTS:


\bigskip\Directive{Keyed LISTS}


In these, CRYSTALS can know in advance how much and what kind of
 input to expect. Each element of data is identified by an optional
 keyword. See, for example, LIST 1 (section \ref{LIST01}).


\bigskip\Directive{Lexical LISTS}


In these, CRYSTALS cannot know in advance what kind of data the user
 may wish to input. Each line of input is processed by a lexical scanner,
 and parsed to determine the action needed. See, for example, LIST 12 
 (section \ref{LIST12}).


For a list of all the lists, see the Lists overview (section \ref{LISTS}).





\bigskip\Instruction{Commands}




These cause CRYSTALS to 'do' something, for example, compute a Fourier
 map. There are two type of syntax for commands, similar to those used
 for LISTS:


\bigskip\Directive{Keyed Commands}


\bigskip\Directive{Lexical Commands}
\label{immediate} 


\bigskip\Instruction{Immediate Commands}




These are special commands which are acted upon immediately they are
 issued. They are never more than one line long, and do not require an
 'END'. They can be issued whenever the cursor is at the beginning of a
 line, even inside a COMMAND or LIST. They are not usually involved with
 the crystallographic calculation, but control some aspect of the way
 CRYSTALS works, such as hooking in an external data file, or changing
 the amount of output produced.





\bigskip\Instruction{Comments}


 Any data line starting with a backslash or hash followed immediately with
 a space is ignored, and may thus be used as a comment, or to deactivate
 the line without deleting it from the file.








\bigskip\Instruction{Continuation Lines}


 The directive CONTINUE indicates that the data on the current line is a
 continuation of the previous line.



\index{Immediate commands}
\section{Immediate commands}



\bigskip\Instruction{$\backslash$FINISH}




This command closes down CRYSTALS neatly.



\bigskip\Instruction{$\backslash$ ..... COMMENTS ....}




This is a comment line. Column 1 contains the  $\backslash$  character and
 column 2 {\bf must} be left blank.
 The remaining columns (3-80) may be used for a descriptive comment.
 Such a comment line may appear at any point in the input.



\bigskip\Instruction{$\backslash$TITLE ..... A title to be printed .....}




This command changes the title that appears at the start of every
 operation. The characters  $\backslash$TITLE  are terminated by a space in
 column 7 and the remainder of the line contains the title.



\bigskip\Instruction{$\backslash$TYPE  'filename'}




The file 'filename' is typed on the users terminal without its contents
 being interpreted by CRYSTALS. Thus $\backslash$ commands in this file are NOT
 acted on, giving the user a method of previewing a USE file.



\bigskip\Instruction{$\backslash$USE source}




This command controls the source of commands for CRYSTALS.
 If 'source' is a filename then commands are read from that file.
 If 'source' is LAST ,the current USE file is closed and
 commands are read from the previous level USE file.
 If  'source' is CONTROL  , all USE files open are closed and
 commands are read from the main control
 stream for the job , for example the terminal in an interactive job.


One USE
 file may contain other USE commands. The maximum depth of USE
 files  allowed will be installation dependent.
 Note that the USEd file need not be a complete list - it can be as little
 as only one line.
 An indirect file should end with '$\backslash$USE LAST' , '$\backslash$USE CONTROL' , or '$\backslash$FINISH'
 .
 This command is only available in some implementations.


\bigskip\Instruction{$\backslash$SCRIPT filename}




This command is only available in interactive mode, and passes control
 to the 'script' file, which tries to assist the user by prompting him
 for data and information. A separate manual describes the writing of
 user define scripts.


\bigskip\Instruction{$\backslash$SPAWN      'shell command'}




A 'shell command' can be issued from inside CRYSTALS with this command.
 $\backslash$SPAWN without a command spawns a sub-process and passes control of the
 sub-process to the terminal. Return to CRYSTALS by closing the
 sub-process.


\bigskip\Instruction{\$      'shell command'}




A 'shell command' can be issued from inside CRYSTALS with this command.
 Typical examples are: \$dir, \$notepad, \$edit afile, \$netscape something.html





\bigskip\Instruction{$\backslash$COMMANDS   command}




This command, which takes other command-names as a parameter 
 (without the $\backslash$ ), produces a listing of the available parameters,
 keywords and defaults for those commands. The listing is
 derived directly from the 'command file', and is thus
 completely up to date for the program being run. This command will
 not operate correctly if the preceding command ended in error. Clear
 the error flag by performing some successful operation.
 The facility
 is an aide-memoire, and not intended to replace the manual.
 The full significance of the output is detailed in section
 \ref{crysdatabase} on the CRYSTALS database.


\bigskip\Instruction{? text}




This facility allows the user to make brief inquiries from the command
 file on the commands, directives, and parameters available at the
 current point in the job. If a command is not being processed, and
 '?' is entered alone, a list of the commands is displayed. If
 '? command'
 is entered, a list of the directives available with that command is
 displayed. If '? command directive' is entered, a list of parameters
 for
 the given command and directive is displayed, and so on.


If a command is currently being processed, the behaviour is similar,
 but no command name is allowed. Then '?' alone gives a list of 
 directives, while '? directive' gives a list of parameters, and so 
 on. In this case care should be taken: After a '?', CRYSTALS loses track
 of the last parameter that was input so using CONTINUE
 will have unpredictable results. To work around this, specify the parameter
 explicitly on the command line, for example:
\small\begin{verbatim}
  \EDIT
  \  Start entering a new atom to be added to the model:
  ATOM U 1.0
  \  Refresh your memory as to the rest of the syntax:
  ?atom
  \  Carry on entering the atom, but give the 
  \ parameter name, X, explicitly.
  CONTINUE X=0.2 0.4 0.5
  END
\end{verbatim}\normalsize




\bigskip\Instruction{$\backslash$MANUAL  'name'}




The 'name' parameter is the name of the volume whose index is required.
 The
 special name 'INDEX' gives a list of subjects for each volume. The special
 name 'LISTS' gives a list of the function of each LIST.



\bigskip\Instruction{$\backslash$HELP   'topic'}




The topic 'HELP' contains a list of topics for which help is given. This
 is likely to be very site-specific.






\bigskip\Instruction{$\backslash$RELEASE  devicename filename}




The file currently open on 'devicename' is closed, and a new file opened
 on that device if necessary. The file just closed can be examined using
 the $\backslash$TYPE command. The filename parameter is optional. If it is
 specified, the new file will be opened with that name.


Useful devices currently available include PRINTER, PUNCH, LOG and MONITOR.







\bigskip\Instruction{$\backslash$OPEN devicename filename}




This command is similar to RELEASE, except that a wider range of
 device
 names may be specified, and different messages are produced. An important
 facility available with this command is to open a named HKLI file, for
 the reflection input
 commands HKLI and LIST 6, using the device name HKLI. Similarly permanent
 files may be used in data reduction work by using the device names M32
 and M33
 overriding the default scratch files.
\small\begin{verbatim}
      e.g.
      \OPEN HKLI reflection.hkl

\end{verbatim}\normalsize








\bigskip\Instruction{$\backslash$CLOSE devicename }




Any file on specified device is closed.






\bigskip\Instruction{$\backslash$APPEND devicename filename}




Output to the specified device is appended to any output already in
the specified file.










\bigskip\Instruction{$\backslash$FLUSH devicename }




Buffered output on the specified device is flushed to the connected file
 so that it can be seen with an editor.
\small\begin{verbatim}
      e.g.
      \FLUSH LOG
      \FLUSH PRINT

\end{verbatim}\normalsize





\bigskip\Instruction{Device names}




Devices are names used inside CRYSTALS to refer
to files that it has opened. Deviecs recognised by CRYSTALS are:





 DISCFILE - the database containing everything.


 HKLI     - an hkl file during input using $\backslash$LIST 6 or $\backslash$HKLI.


 CONTROL  - commands being input (can be a file if you type
$\backslash$USE filename, but is usually the command input line)


 PRINTER  - the listing file (bfilenn.lis by default)


 PUNCH    - the punch file (bfilenn.pch by default)


 LOG      - record of all commands issued (bfile.log by default)


 MONITOR  - Copy of text that appears on the screen


 SPY      - obsolete, used to collect usage data (locally!).


 NEWDISC  - a second database, open during $\backslash$PURGE commands.


 COMMANDS - the file commands.dsc defines the syntax and data 
structures of all the commands and lists.


 M32 M33 MT1 MT2 MT3 MTE - scratch files


 SRQ      - system request queue. Some high-level commands 
may issue other CRYSTALS commands using this file.


 FORN1, FORN2 - output of data for 'foreign' (ie. external) 
programs.


 SCPDATA - scripts can read directly from any file opened on
this device (using EXTRACT script command)


 SCP2 - scripts can read directly from any file opened on
this device (using EXTRACT2 script command)s


 SCPQUEUE - stores commands that scripts are building up.



Not all may be opened/closed by the user from a command prompt. Some are
only available inside the initialisation file CRYSTSLS.SRC (See STORE
below).



\bigskip\Instruction{$\backslash$PAUSE interval}




This command causes the program to wait for 'interval' seconds before
 proceding. The maximum value of 'interval' is 200 seconds. It might be
 useful in a 'USE' file.



\bigskip\Instruction{$\backslash$BENCH nparam nref}




This simulates sfls (structure factor least squares, 
i.e. a cycle of refinement, see \ref{SFLS}).
to enable processor speeds to be compared. No real
refinement is done, and the structure is not modified.
\small\begin{verbatim}
      nparam defaults to 500
      nref   defaults to 5000
\end{verbatim}\normalsize\small\begin{verbatim}
Times for a Microvax 3800 (circa 1995) are printed for comparison.
\end{verbatim}\normalsize





\bigskip\Instruction{$\backslash$SET LIST   state}




This command allows the user to control the monitoring level
 of transfer of lists to and from the database
 in conjunction with the SET WATCH command. There are four states
 available.


If state is OFF, no list logging information is produced.


If state is READ or WRITE, list logging information is only produced
 when lists  are
 read or written respectively.


If state is BOTH, both reading and writing
 operations may be monitored. Note that the logging operation may be qualified
 by a list type specified by SET WATCH. The initial state is WRITE, with
 the
 specific watch set on list 5 (the model parameters), so that only operations 
 creating or modifying list 5 will be monitored.


\bigskip\Instruction{$\backslash$SET WATCH number}




This command is used in conjunction with SET LIST to control monitoring
 of list operations. If number is 0, operations on all list types may be
 monitored, according to the state set with SET LIST. If number is a positive
 integer, representing a list type, only operations on that list type may
 be
 monitored. The initial value for the list watch is 5, which in conjunction
 with the initial monitoring state means that operations creating or modifying
 list 5 will be monitored.


\bigskip\Instruction{$\backslash$SET FILE type}




This command is used to control the case of file names generated
 by CRYSTALS. Possible values are:

\small\begin{verbatim}
 LOWER        Filenames are converted to all lower case.
 UPPER        Filenames are converted to all upper case.
 MIXED        Filenames are left as input or defined.
\end{verbatim}\normalsize





\bigskip\Instruction{$\backslash$SET GENERATE state}




This command is used to control the generation of output file names
 and pseudo-generation numbers on
 non-VMS systems.  By default, CRYSTALS modifies the root of filenames
 for files which should not be overwritten (normally .LIS, .MON, .LOG).


 OFF suppresses automatic name generation.



\bigskip\Instruction{$\backslash$SET EXPORT state}




If 'state' is 'on' then LISTS 5 (atoms), 12 (constraints) and 16 
 (restraints) are copied to the PUNCH file
 when CRYSTALS closes down. These can be archived for safety, or exported
 to another computer.



\bigskip\Instruction{$\backslash$SET UEQUIV state}


 This controls the calculation of Uequiv. Both definitions are acceptable
 to Acta. The arithmetic mean of the principle axes is often similar
 to the refined value of Uiso. The geometric mean is more sensitive to
 long or short axes, and so is more useful in publications.
\small\begin{verbatim}
      ARITH = (U1+U2+U3)/3
      GEOM  = (U1*U2*U3)**1/3
\end{verbatim}\normalsize





\bigskip\Instruction{$\backslash$SET PAUSE value}




This command sets a time, in seconds, for which the program will pause
 at
 the end of each screen full of output. It is only effective on DOS machines,
 and enables the user to use the 'pause' key to hold a selected screen.
 The maximum value of 'interval' is 200 seconds.


\bigskip\Instruction{$\backslash$SET LOG state}




If state = ON  then all user input commands are written
 to a log file.



If  state = OFF  then subsequent commands are not written
 to the log. Any change made to
 the log state applies only to the current level of USE file and any
 USE file called by it.
 Because the log file is a direct copy of the users commands, it may
 subsequently be used (probably after modification) as a control file.


\bigskip\Instruction{$\backslash$SET MIRROR state}




If 'state' = ON, then all input is reflected in the monitor or list file.
 If 'state' = OFF, monitoring.
 is suppressed. Any change made to
 the monitoring state applies only to the current level of USE file and
 any
 USE file called by it.


\bigskip\Instruction{$\backslash$SET PAGE length}




This command is used to change the length of the assumed 'page' when
 displaying files on the monitor channel, using the commands 'HELP', 'MANUAL',
 and 'TYPE'. The initial length is 20 lines. After the number of lines
 specified have been typed, the listing stops and a message indicates the
 program is waiting. A blank line or carriage return
 at this point will cause the listing to
 continue. Any other input is executed normally. If the length is set to
 zero,
 or a negative number, the feature is disabled.


\bigskip\Instruction{$\backslash$SET TERMINAL device}




This command controls the display of SCRIPT menus on some
 terminals. Possible device types are

\small\begin{verbatim}
 UNKNOWN This is the default, and requires no special terminal.
 VT52    For use on terminals with limited screen management facilities.
 VT100   For use on advanced terminals.
 VGA     For use on PC VGA terminals
 WIN     For use under Win32 and X-windows.
\end{verbatim}\normalsize




\bigskip\Instruction{$\backslash$SET COMMUNICATION speed}




This command is used to indicate to the program the speed of the
 communication line or terminal
 on which it is being run. This indication is used by some facilities to
 determine how much output to produce on the monitor channel. The possible
 values for the speed are "SLOW" and "FAST". These keywords are not associated
 with any particular terminal speed, and the appropriate value will depend
 on
 the user's patience. The initial value is "FAST"


\bigskip\Instruction{$\backslash$SET TIME state}




This command is used to indicate to the program whether the timing
 messages usually produced at the end of each facility are produced. If
 'state'
 is "OFF" the messages are not displayed. If 'state' is "ON" the messages
 are
 displayed.


\bigskip\Instruction{$\backslash$SET PRINTER state}




This command is used to control output to the 'printer' file. The
 state
 OFF suppresses printer output.


\bigskip\Instruction{$\backslash$SET MONITOR state}




This command is used to control output to the 'monitor' file. 
\small\begin{verbatim} 
 OFF  suppresses all output 
 ON   copy screen output to monitor
 SLOW copy all screen output PLUS output for graphics to monitor.
 (Mainly used for debuging graphics)
\end{verbatim}\normalsize\small\begin{verbatim}
\end{verbatim}\normalsize

\bigskip\Instruction{$\backslash$SET OPENMESSAGE state}




This command is used to enable or disable file handling messages.
 OFF suppresses messages.


\bigskip\Instruction{$\backslash$SET MESSAGE state}




This command is used to indicate to the program whether the command
 messages usually produced at the end of each facility are produced. If
 'state'
 is "OFF" the messages are not displayed. If 'state' is "ON" the messages
 are
 displayed. Error messages are always displayed.


\bigskip\Instruction{$\backslash$SET SRQ state}




This command is used to control mirroring of CRYSTALS internal commands.
 The normal state OFF suppresses the mirroring.





\bigskip\Instruction{$\backslash$SET RATIO state}




This command controls whether Io or Ic is compared with Sigma(Io) 
during filtering. Arguments are Io or Ic, default is Io.
See "The use of partial observations, partial models and partial residuals
to improve the least-squares refinement of crystal structures."
A. David Rae, Crystallography Reviews, 2013. Vol. 19, No. 4, 165.


Io is appropriate during initial stages of an analysis, but once the R factor
drops below 10-15\%, Ic gives an unbiassed filtering of the weak reflections. A
threshold ratio of 1 or 2 is useful if there are a very large number of very
weak reflections.



\section{Files}


When CRYSTALS runs it stores all crystallographic data in a single file,
by default named crfilev2.dsc. This is a binary file and should not be
opened with any other program.


CRYSTALS outputs results and analysis to several places:


\bigskip\Keyword{The listing file: }
this will be called bfilenn.lis where nn is a number
from 00-99, incremented each time the program is run. When nn reaches 99
this file will be continuously overwritten every time CRYSTALS is run.



The listing file contains verbose output about all the calculations that
take place.


\bigskip\Keyword{The punch file: }
called bfile.pch by default. The idea of 'punching' data
is a throwback to the days of punched cards. This file is used to write out
specific bits of data \emph{as} \emph{commands} that can then be read back into the 
program. For example, the parameters defining the crystallographic model 
are stored in CRYSTALS as a 'List 5'. Then 'Punching List 5':
\small\begin{verbatim}
  \PUNCH 5
  END
\end{verbatim}\normalsize


will result in a full $\backslash$LIST 5 command being written to the file bfile.pch.
You can then recover that version of the model parameters by typing
\small\begin{verbatim}
  \USE bfile.pch
\end{verbatim}\normalsize


which causes the commands written in that file to be executed by CRYSTALS.


\bigskip\Keyword{The monitor file: }
called bfilenn.mon this is obsolete. If there is any
output in it, it is from a bit of the program that hasn't been looked at
recently.


\bigskip\Keyword{The log file: }
called bfilenn.log, this contains everything that you typed into CRYSTALS, 
and commands that were issued on your behalf by the menu system or script
processor.



\index{Errors}
\index{Warnings}
\section{Errors and Warnings}



CRYSTALS recognises the following run time error categories,
 in addition
 to any detected by the operating system.


\bigskip\Instruction{Warnings}




These will occur only for tasks which produce user-readable
 output, and do not write to the database.
 The current task is abandoned if necessary,
 and the next task fetched from the input stream.


\bigskip\Instruction{Errors}




The error is such that the current task must be abandoned. In batch or
 online modes, the job will be terminated as well. In interactive mode
 the current task is abandoned , and control is returned to the user at
 terminal. The processing of any 'USE' file will be abandoned.


\bigskip\Instruction{Severe Errors}




The error detected is such that it is not possible for the job to
 continue. These errors are usually caused by database management failures.


\bigskip\Instruction{Catastrophic Errors.}




The job is completely terminated.
 This is usualy the consequence of
 errors in reading or writing to the database.


\bigskip\Instruction{Programming Errors}




The program has detected an inconsistency either in the code, or in
 the command file. A dump and error report will be generated if
 possible. The error and the conditions that cause it should be reported
 to Oxford.





\bigskip\Instruction{Errors Detected During the Creation of a LIST}




During all operations that create new versions of a list, either by
 input or internally, errors may be found that cause the new list not to
 be written to the database. To prevent the system using an old version of
 a list
 when the creation of the latest version has failed, the relevant list
 type
 is marked as an 'error list' in the 'list control table' (see below).
 This error flag is cleared when a new version of the list is created or
 by user action.


If a list which is marked as an error list is accessed, a message will
 be output, and the job terminated. However, for the printing of most lists,
 the error flags are not checked.



\chapter{The Crystals Database}
\label{crysdatabase}  It is quite possible to use CRYSTALS so that all the data is read from text files at the start of every job (as in SHELXL). However, for interactive working, it is preferable to establsh a database of crystallographic information which can be used whenever it is needed. This database is called the CRYSTALS 'disk' file. It is usualy a permanent file on the hard disk, and usually has the extension '.DSC'.\small\begin{verbatim} It is a direct-access binary file. DO NOT try to print or edit it.\end{verbatim}\normalsize

The data is organised in this file as lists, corresponding to the external user-input lists. In modern terminology, these 'lists'  would be called 'objects' or 'data structures'. Each list groups together related information, e.g. the cell parameters, the atoms,  the reflections etc.

 While in general input of a list will overwrite a previous version, for the atomic coordinate list (LIST 5,  see section \ref{LIST05}) a new version is added to the database. In the event that a refinement goes wrong, the user can usually recover an earlier version of the structure.

In the  CRYSTALS  system, most of the data required to refine a crystal structure must be input to the computer as ASCII, translated into an internal format and stored on a random access disk file. On the disk, different types of data are recorded separately, in what are called  LISTS. Each list holds only one type of crystallographic information and is identified by a number called the 'list type number'.

Normally, each structure uses a separate disk file, which is preserved between different jobs and updated whenever a program generates some new data. This means that several different versions of a given list may be produced during the course of a structure analysis. In order that each list may be uniquely identified, every list has associated with it a second number, called the 'list serial number'. To specify unambiguously a list that is stored in the database, it is necessary to know both the relevant list type number and the list serial number.

In most cases, however, the version of a given list that is required is the latest list of that type to be created. Accordingly, an index called the 'current list index' is maintained, which contains an entry for the latest version of each list. When a program requires information about the current version of a list, it accesses the current list index.

As well as an index of the current version of each list, a second index is kept, called the 'file index'. For each disk file, this latter index contains the information for every list that is present in the database. This index is always checked when a list is written to the database, to ensure that its list type number and list serial number refer to only one list.

For a user, the major advantage of splitting the input data into logical units in the database is that, for any run, only those lists which need to be changed have to be re-input to the database. New versions of each list that are generated by programs are automatically output to the database so that, provided the database file is not destroyed or erased, it is probable that each run will contain only control commands. For example, during a structure factor least squares calculation a new set of  Fcalcs  and a new normal matrix will be generated and stored in the database ready for further calculations.\label{richtest} 

Old versions of a list may be reused, provided that they have not been overwritten or deleted, by finding the relevant entry in the file index and copying it into the current list index.

 The database contains two indices which control access to it.

\bigskip\Instruction{Current List Index}



This index contains the serial numbers of each list to be used in calculations. This index is updated whenever a list is written to database.

\bigskip\Instruction{Disk Index}



This is an index of all the lists contained in the database. When a new list is added to the database, its internal address is added to this index, and also inserted into the current disk index. It is possible to copy address from this index to the current disk index, thus changing the currently active version of a list.



\index{$\backslash$Disk}\index{Retreiving old data lists}\section{General List-control Directives  -  $\backslash$DISK}



The list control table can be marked and used in various ways with this
 command.
\small\begin{verbatim}
 \DISK
 PRINT INDEX= PUNCH= LIST=
 MARKERROR LIST= SERIAL= RELATIVE= ACTION=
 RETAIN LIST= SERIAL= RELATIVE= ACTION=
 DELETE LIST= SERIAL= RELATIVE= ACTION=
 RESET LIST= SERIAL= RELATIVE=
 USAGE LIST= SERIAL= RELATIVE= FLAG=
 EXTEND RECORDS= FREE= TRIES= SIZE=
 CHECK
 END
\end{verbatim}\normalsize


For example:
\small\begin{verbatim}
 \DISK
 \  Print the current list index
 PRINT INDEX=CURRENT
 \ Print the index containing all the lists stored
 \ on the disk
 PRINT INDEX=DISK
 \ Reset LIST 5 (the model parameters) to the one with serial number 4
 RESET 5 4
 \ Reset LIST 10 (Fourier peaks) to the 'current serial number - 1'
 RESET 10 0 -1
 \ Retain LIST 5 with serial number 6 when the disk
 \ is purged
 RETAIN 5 6
 \ Delete current LIST 11 (normal/inverted least squares matrix)
 DELETE 11
 END
\end{verbatim}\normalsize




\bigskip\Instruction{$\backslash$DISK}




This is the command which initiates the
 routines to modify the list control table.
 The directives MARKERROR,
 DELETE, RETAIN, USAGE, and RESET all accept the parameters
 SERIAL and RELATIVE.
 The parameters  SERIAL  and  RELATIVE  should not both
 be changed from their default settings on the same directive.


All directives are executed immediately after they have been entered.


\bigskip\Directive{PRINT INDEX=}


This directive causes one of the indexes to be printed.


\bigskip\Keyword{INDEX=}
\small\begin{verbatim}
      CURRENT  -  default value
      DISK
      SUMMARY
\end{verbatim}\normalsize


 If  'INDEX'  is  'CURRENT', the current list index is
 printed (i.e. the index containing the current version of each
 list stored).
 If  'INDEX'  is  'DISK', the index containing information about all
 the lists stored on the disk is printed.
 If 'INDEX' is 'SUMMARY', a summary of the usage of the disk file is printed.


\bigskip\Keyword{PUNCH=}
\small\begin{verbatim}
      NO  -  default value
      YES
\end{verbatim}\normalsize


 Writes a machine readable summary of whatever is selected by INDEX to
 the currently open PUNCH file.


\bigskip\Keyword{LIST=}
\small\begin{verbatim}
      0  -  default value
      n
\end{verbatim}\normalsize


By default (0) all lists are printed or punched. If you specify an
alternative value for the LIST parameter, then only information about
that list number is output.
\label{MARKERROR} 


\bigskip\Directive{MARKERROR LIST= SERIAL= RELATIVE= ACTION=}


This directive can either mark a specified type of list as an 'error
 list', or alternatively, such a mark can be removed from the list control
 table.


\bigskip\Keyword{LIST=n}
 'n' is the list type number (there is no default value).


\bigskip\Keyword{SERIAL=n}
 'n' is the list serial number. The default value is zero, which
 represents the serial number of the current list of this type.


\bigskip\Keyword{RELATIVE=n}
 'n' is the offset applied to the serial. The default value
 is 0. If both SERIAL and RELATIVE are zero (the defaults) 
 the current list of the specified type will be 
 the one that is marked.


\bigskip\Keyword{ACTION=}
\small\begin{verbatim}
      NO
      YES  -  default value
\end{verbatim}\normalsize


 If  ACTION is  NO , the entry in the list control table for the list type
 specified is altered so that it is not marked as an 'error list'.
 If  ACTION is  YES, the entry in the list control
 table for the list type specified is altered so that it is marked as an
 'error list'. If a program attempts to use such a list, an error is
 reported, and the job terminated.





\bigskip\Directive{RETAIN LIST= SERIAL= RELATIVE= ACTION=}


With this directive, certain old versions of specified lists can be
 retained when the disk file is  PURGED  (see \ref{PURGE}).


\bigskip\Keyword{LIST=n}
 'n' is the list type number (there is no default value).


\bigskip\Keyword{SERIAL=n}
 See markerror definition above (\ref{MARKERROR}).


\bigskip\Keyword{RELATIVE=}
 See markerror definition above (\ref{MARKERROR}).


\bigskip\Keyword{ACTION=}
\small\begin{verbatim}
      NO
      YES  -  default value
\end{verbatim}\normalsize


 If  'ACTION'  is  'YES', the list with the specified
 type and serial number will be retained when the disk file is  PURGED.
 If  'ACTION'  is  'NO' , the particular list will be deleted when the
 disk file is  PURGEd (see section \ref{PURGE}).


\bigskip\Directive{DELETE LIST= SERIAL= RELATIVE= ACTION=}


With this directive certain lists can be deleted from the file index.


\bigskip\Keyword{LIST=}
 See markerror definition above (\ref{MARKERROR}).


\bigskip\Keyword{SERIAL=}
 See markerror definition above (\ref{MARKERROR}).


\bigskip\Keyword{RELATIVE=}
 See markerror definition above (\ref{MARKERROR}).


\bigskip\Keyword{ACTION=}
\small\begin{verbatim}
      NO
      YES  -  default value
\end{verbatim}\normalsize


 If  'ACTION'  is  'YES', the list with the specified
 serial number will be marked to be deleted from the file when the
 file is PURGED.
 If  ACTION  is  'NO' , the specified list will not be marked as one to
 be deleted when the disk file is  PURGEd (see section \ref{PURGE}).
\label{RESET} 


\bigskip\Directive{RESET LIST= SERIAL= RELATIVE=}
 This directive alters the entry for the list type in the
 current list index.


\bigskip\Keyword{LIST=}
 This parameter is the list type number,
 for which there is no default value.


\bigskip\Keyword{SERIAL=}
 See markerror definition above (\ref{MARKERROR}).


\bigskip\Keyword{RELATIVE=}
 See markerror definition above (\ref{MARKERROR}).


\bigskip\Directive{USAGE LIST= SERIAL= RELATIVE= FLAG=}
 This directive alters the list write/overwrite flag.


\bigskip\Keyword{LIST=n}
 'n' is the list type number
 (there is no default value).


\bigskip\Keyword{SERIAL=}
 See markerror definition above (\ref{MARKERROR}).


\bigskip\Keyword{RELATIVE=}
 See markerror definition above (\ref{MARKERROR}).


\bigskip\Keyword{FLAG=}
\small\begin{verbatim}
      OVERWRITE
      READY
      UPDATE  -  default value.
\end{verbatim}\normalsize




\bigskip\Directive{EXTEND RECORDS= FREE= TRIES= SIZE=}
 This directive 
 allows the user to extend the database by a specified number
 of records, and to control the auto-extension. On modern
 platforms the extension of the database is automatic by
 default.


\bigskip\Keyword{RECORDS=}
 This parameter specifies a number of records to be added to the
 file. The default value is zero i.e. no records are added.


\bigskip\Keyword{FREE=}
 This parameter specifies a number of records that must be available
 for use in the file. The file will be extended until there are at
 least 'FREE' records unused after the next free disk address.


\bigskip\Keyword{TRIES=}
 This is the number of times the system may try adding 'SIZE' records to
 the disk file to achieve enough space for the current operation. Usual
      default is 1.


\bigskip\Keyword{SIZE=}
 This is the size, in records, that the system will increase the disk by
 to try to accommodate the current operation. The usual default is 5.
 If the write still fails after TRIES x SIZE records have been added it 
 produces an error. Setting SIZE to zero enables uncontrolled
 extension of the disk file. This is the default, but if the
 user gets carried away doing crystallography, they may fill their 
 disk.





\bigskip\Directive{CHECK}
 This directive checks the integrity of the disk file. It takes
 no parameters.


\bigskip\Keyword{Errors in the DISK file.}
 If CRYSTALS
 reports errors in the disk file, run this utility to get a list of those
 LISTS corrupted. Use $\backslash$PUNCH 5, $\backslash$PUNCH 12 and $\backslash$PUNCH 16 to make ASCII
 copies of these important lists (by default output to the 'punch' 
 file, bfile.pch - rename it after punching to e.g. 'savedlists.pch'). 
 If the final version 
 of a list is corrupt, use $\backslash$DISK RESET (see above \ref{RESET}) to point 
 the program 
 to an earlier version and punch that instead.
 If the disk file becomes totally
 unusable, delete it, read in the reflections again, and then read in
 these 'punched' lists ($\backslash$USE savedlists.pch).



 If only certain lists are unusable,
 use the DELETE directive in $\backslash$DISK to mark defective LISTS for
 deletion, and then use $\backslash$PURGE NEW to create a new disk file (by 
 default new.dsc). Rename new.dsc to crfilev2.dsc to make CRYSTALS 
 open it by default when it starts in that directory.

\small\begin{verbatim}
 The following lists can always safely be deleted since CRYSTALS creates
 new ones automatically.
\end{verbatim}\normalsize

 10 Fourier peaks (section \ref{LIST10}).


 11 Normal matrix (section \ref{LIST11}).


 20 Saved geometrical transformation matrices (section \ref{LIST20}).


 22 Constraints in internal format (section \ref{LIST22}).


 24 Least squares shift list (section \ref{LIST24}).


 26 Restraints in internal format (section \ref{LIST26}).


 33 SFLS command in internal format (section \ref{SFLS}).


 36 List dependency tracking (not documented).





 See section \ref{LISTS} for all the list definitions.





Example 1

\small\begin{verbatim}
Imagine that the current versions of LIST 5 (the model parameters) 
and LIST 29 (element properties) have beome corrupt.
We will also delete any recreatable lists.

      \DISK
      \  Check the index
      CHECK
      \  Delete current version of atomic params
      DEL 5
      \  Point index to the previous version
      RESET 5 0 -1
      \  Remove corrupt and safe-to-delete lists:
      DEL 29
      DEL 10
      DEL 11
      DEL 20
      DEL 22
      DEL 24
      DEL 26
      DEL 33
      DEL 36
      END
      \  Open a new file save.dat on the 'PUNCH device'.
      \  Release is equivalent to CLOSE followed by OPEN.
      \RELEASE PUNCH SAVE.DAT
      \  Output the important data lists:
      \PUNCH 5
      END
      \PUNCH 12
      END
      \PUNCH 16
      END
      \  Write a new database called rescue.dsc with only the 
      \  current index of lists in it:
      \PURGE RESCUE
      \  Close CRYSTALS
      \FINISH
\end{verbatim}\normalsize


 At this point it would be useful on a stand-alone computer to
 run a system utility (e.g. SCANDISK on a PC) to check the integrity of
 the computers hard disk, since CRYSTALS rarely corrupts the disk itself.



 One would then open a command prompt in the working folder and do this:
\small\begin{verbatim}
      del crfilev2.dsc
      ren rescue.dsc crfilev2.dsc
\end{verbatim}\normalsize


 Now restart CRYSTALS in the same folder. Recover the important saved 
 data by typing:
\small\begin{verbatim}
      \USE save.dat
\end{verbatim}\normalsize





\index{Purge}
\index{Shrinking the contents of a dsc file}
\section{Elimination of Old Versions of LISTS  -  $\backslash$PURGE}
\label{PURGE} 

\small\begin{verbatim}
 \PURGE FILE= INITIALSIZE= LOG= LIST=
 ACTION REOPEN=
 END
\end{verbatim}\normalsize




\bigskip\Instruction{$\backslash$PURGE}




This command deletes all but the current
 version of each list, and any lists explicitly marked to be deleted,
 except for lists marked to be retained,
 and then rewrites the disk file
 so that the space occupied by the old deleted lists is overwritten.
 This operation does not normally shorten the file physically (see parameter
 FILE below for a method of doing this) , but frees
 a lot of space that can be reused.
 The optional parameters 'FILE' and 'INITIALSIZE' (available
 in some implementations
 only), allow the user to create a new file with only the current
 version of each list in it. This file may be smaller than the
 original disk file.


\bigskip\Keyword{FILE=}
\small\begin{verbatim}
      OLD - default value
      name - The root to be used form the new database - name.DSC
      DATE - The root name is generated by CRYSTALS from the date and
             time.
\end{verbatim}\normalsize




This parameter controls whether a new file is created. If the value
 specified is not OLD, a new file will be created containing only the
 current versions of each list, and any which are marked to be retained.
 The program automatically extends the file to the size required for all
 the lists to be retained. This new file has the extension .DSC,
 and may be used in future CRYSTALS tasks.


\bigskip\Keyword{INITIALSIZE=}


This parameter is used to specify an initial size for a new file. If
 a new file is not to be created, the value is ignored. The default
 value of zero causes the new file to have the smallest size necessary
 to contain all the copied lists.


\bigskip\Keyword{LOG=}
\small\begin{verbatim}
      OFF - default value
      ON
\end{verbatim}\normalsize




If the value 'ON' is given for this parameter, the types and serial
 numbers of all lists deleted because they were old versions are listed.
 The types and serials of all lists not copied for other reasons, e.g.
 deleted lists, and lists which are marked as being updated, are always
 listed


\bigskip\Keyword{LIST = n}


'n' is the type of list to be purged from the database, all old versions 
of all other lists are preserved. If 'n' is 0 (the default value)
lists of all types are purged.





\bigskip\Keyword{ACTION REOPEN=}
\small\begin{verbatim}
    YES
    NO
\end{verbatim}\normalsize




YES attaches the newly constructed .dsc file on NCDFU after the purge and
 closes the old one. It does it at the point in xpurge that refreshes all
 the indexes after a normal purge



 This will allow us to use a 'Save As...' type of behaviour that users will
 expect, but without having to restart the current CRYSTALS process.


NO Leaves the user accessing the current dsc file.




\index{Lists overview}
\section{LISTS in Current use}
\label{LISTS} 

Lists marked * cannot be directly input by the User



   List Number     Type of data





     1    Cell parameters (section \ref{LIST01})


     2    Unit cell symmetry (section \ref{LIST02})


     3    Atomic scattering factors (section \ref{LIST03})


     4    Weighting parameters (section \ref{LIST04})


     5    The model parameters (section \ref{LIST05})


     6    Reflection data (section \ref{LIST06})


     7    Reflection data not used for refinement (section \ref{LIST07})


     9*   ESDs of model parameters (section \ref{LIST9})


    10    Peak coordinates from Fourier (section \ref{LIST10})


    11    Least squares matrix (section \ref{LIST11})


    12    Refinement directives (section \ref{LIST12})


    13    Crystal and collection data (section \ref{LIST13})


    14    Fourier directives (section \ref{LIST14})


    16    General Restraints (section \ref{LIST16})


    17    Special Restraints (section \ref{LIST17})


    18    SMILEs string representation (section \ref{LIST18})


    20*   Transformation matrices saved by $\backslash$GEOMETRY (section \ref{LIST20})


    22*   Refinement directives in internal format (section \ref{LIST22})


    23    Structure factor control list (section \ref{LIST23})


    24*   Least squares shift list (section \ref{LIST24})


    25    Twin component operators (section \ref{LIST25})


    26*   Constraints in internal format (section \ref{LIST26})


    27    Diffractometer scales (section \ref{LIST27})


    28    Reflection condition/filter list (section \ref{LIST28})


    29    Contents of asymmetric unit and elemental properties (section \ref{LIST29})


    30    General information (section \ref{LIST30})


    31    Cell parameter E.S.D.'s (section \ref{LIST31})


    33*   Internal - Refinement control (last SFLS command, see \ref{SFLS})


    36*   Tracking interdependencies of parameters, normal matrix, weights etc.


    40    Bond forming/breaking directives (section \ref{LIST40})


    41*   Bonds between atoms (section \ref{LIST41})



\index{$\backslash$Summary}
\section{Printed Summary of Data lists}

\small\begin{verbatim}
   \SUMMARY OF= TYPE= LEVEL=
   END
\end{verbatim}\normalsize


For example:
\small\begin{verbatim}
   \  Detailed list of model parameters:
   \SUMMARY OF=LIST TYPE=5 LEVEL=HIGH
   END

   \  Again, but less typing:
   \SUM L 5 HI
   END

   \  Summary of reflection data:
   \SUM L 6
   END
\end{verbatim}\normalsize




\bigskip\Instruction{$\backslash$SUMMARY OF= TYPE= LEVEL=}




\bigskip\Directive{ }


\bigskip\Keyword{OF=}


This parameter determines the extent of the data summary.
\small\begin{verbatim}
      LIST
      EVERYTHING
\end{verbatim}\normalsize




If 'LIST' is specified, a summary of a specific list whose type is
 given by the 'TYPE' parameter is produced. If 'EVERYTHING' is specified,
 a summary of all current lists for which summaries are available is
 produced. If 'OF = EVERYTHING' is
 specified, the value of 'TYPE' is ignored.


\bigskip\Keyword{TYPE=}


This parameter indicates the
 list type for which a summary is required. If a summary of this list
 type is not available, a warning message will be printed.





\bigskip\Keyword{LEVEL=}
\small\begin{verbatim}
      LOW
      MEDIUM
      HIGH
\end{verbatim}\normalsize


This parameter has three values which indicate the level of detail
 required in the data summary. The parameter is ignored unless the list
 type is 5 or 10.

\small\begin{verbatim}
 List      Level   Data printed
 ----      -----   ---- -------
 5 or 10  LOW     Atom names
 5 or 10  MEDIUM  Atom names and positional parameters
 5 or 10  HIGH    All atomic and overall parameters
\end{verbatim}\normalsize


\index{Atom name syntax}
\section{Element and Atom names}
\label{ATOMSYNTAX} 






\bigskip\Instruction{Element Names}




Elements are specified by a name, called the atom {\bf TYPE.} The
 'TYPE' is used to associate the refined atoms with atomic properties,
 such as form factors, radii, etc. The TYPE {\bf must} start with a letter,
 and is not case sensitive.
 REMEMBER that 'blank'
 is used as a delimiter in CRYSTALS and so must
 not appear in an atom TYPE. Elements in the form factor list (LIST 3 - see
 section \ref{LIST03})
 and the atomic properties (LIST 29 - section \ref{LIST29}) are associated by 
 their TYPE with atoms in the parameter list (LIST 5). Numeric properties of the
 elements are pre-prepared in the files SCATT, SCATCU, SCATMO and
 PROPERTI(.dat) in the SCRIPT directory (Often CRYSTALS$\backslash$SCRIPT). The
 'elements' C', C", H' and H" are in these files with the properties of C and H.
 You can define your own element names, but may not use +,- or *.





\bigskip\Instruction{Atom names}


 Individual atoms are specified by combining the TYPE with a SERIAL
 number. This is an integer in the range 1-9999. When combined with the
 TYPE, it must be enclosed in parentheses ().
\small\begin{verbatim}
   e.g.   C(2) CL(123)
\end{verbatim}\normalsize







\bigskip\Instruction{Symmetry equivalent atoms}





 Atoms specified as above correspond exactly to those in the refinable
 atom list (LIST 5). If a symmetry equivalent atom is required, up to 5
 additional items are included inside the parentheses. The full
 specification of an atom is:
\small\begin{verbatim}
       TYPE(SERIAL,S,L,Tx,Ty,Tz)
\end{verbatim}\normalsize







\bigskip\Directive{S }
 is the serial number of the symmetry operator in the stored list of
 space group operators. If negative, it indicates that the atom
 coordinates are first negated before being operated upon. The default is
 1





\bigskip\Directive{L }
 is the serial number of the non-centring lattice translation to be
 used. The default is 1.





\bigskip\Directive{Tx,Ty,Yz }
 are whole cell translations parallel to the cell axes. The
 default is 0.



 If the value of an item is its default, it may be omitted altogether, though
 its place must be marked by its associated comma. A series of trailing
 commas may also be omitted. For example:

\small\begin{verbatim}
       C(99,-1,1,0,0,0) - an atom related by inversion at the origin
                          (assuming x,y,z is the first operator).
       C(99,-1,1,,,0)   - same atom as above, omitting defaults.
       C(99,-1)         - same atom as above, omitting 
                          defaults and trailing commas.
\end{verbatim}\normalsize




For more detailed information, see Atomic and
 Structural parameters (section \ref{atomparams})



\index{External program links}
\section{Foreign Program Links}



\bigskip\Instruction{$\backslash$FOREIGN}




This command provides links to 'FOREIGN' programs, that is,
 programs which are not part of CRYSTALS, but which provide useful
 functions in providing a complete system. These programs often come
 from other laboratories, and are only distributed with the authors
 permission unless they are public domain. Where practical, we make no
 changes to the original code. We offer little or no support in
 connection with these programs, though usually they are in frequent
 use in Oxford. The linking routine prepares data files for the foreign
 program, and may initiate a subprocess to execute the rogram.


\bigskip\Keyword{PROGRAM=}
\small\begin{verbatim}
      SNOOPI  - A plotting program using Tektronix 4010 devices.
      CAMERON - An interactive graohics program.
      MULTAN  - Prepares files for MULTAN 84.
      SHELXS  - Prepares files for SHELXS.
      SIRxx   - Prepares files for the SIR direct methods system
\end{verbatim}\normalsize




\bigskip\Keyword{MODE=}


This keyword controls the mode of operation of the foreign program.
 It currently only applies to SHELXS and SIR.
\small\begin{verbatim}
      NORMAL      - DEFAULT, prepares a default data file with recommended
                    settings.
      DIFFICULT   - Prepares a file with non-default settings.
      PATTERSON   - Prepares a default Patterson calculation (SHELXS only).
      SPECIAL     - Prepares a full SHELX data file (SHELX only)
\end{verbatim}\normalsize


\chapter{Initial Data Input}
\label{initialdinput} 

\section{Scope of the Initial Data Input section.}






The areas covered are:
\small\begin{verbatim}
 Abbreviated startup command                      QUICKSTART
 Input of the cell parameters                     LIST 1
 Input of the unit cell parameter errors          LIST 31
 Input of the space group symmetry information    SPACEGROUP
 Alternative input of the symmetry information    LIST 2
 Input of molecular contents                      COMPOSITION
 Input of the atomic scattering factors           LIST 3
 Input the structural formula as a SMILES string  LIST 18
 Input of the contents of the unit cell           LIST 29
 Input of the crystal and data collection details LIST 13
 Input of general crystallographic data           LIST 30
\end{verbatim}\normalsize





\index{QUICKSTART}
\index{Getting started}
\section{Abbreviated startup command  -  QUICKSTART}



The command QUICKSTART is provided to assist in migration from other
 systems to CRYSTALS. It requires that data reduction (section \ref{DATAREDUC}) 
 has already been
 done or that a simple 4-circle Lp correction be suitable,
  and that the reflection data are available in a fixed format file with
 one reflection per line. This command expands the given data into
 standard CRYSTALS lists, as described elsewhere in the manuals. The user
 is free to overwrite LISTS created by QUICKSTART by entering new LISTS
 manually.

\small\begin{verbatim}
 \QUICKSTART
 SPACEGROUP SYMBOL=
 CONTENTS FORMULA=
 FILE NAME=
 FORMAT EXPRESSION=
 DATA WAVELENGTH= REFLECTIONS= RATIO=
 CELL  A= B= C= ALPHA= BETA= GAMMA=
 END
\end{verbatim}\normalsize



For example:
\small\begin{verbatim}
 \QUICKSTART
 SPACEGROUP P 21/n
 CONTENT C 6 H 4 N O 2 CL
 FILE CRDIR:REFLECT.DAT
 FORMAT (3F3.0, 2X, 2F8.2)
 DATA 1.5418
 CELL 10.2 12.56 4.1 BETA=113.7
 END
\end{verbatim}\normalsize







\bigskip\Instruction{$\backslash$QUICKSTART}


 None of the directives may be omitted, though some parameters do have
 default values. {\bf CONTINUE} {\bf directives} {\bf may} {\bf not} {\bf be} {\bf used.}





\bigskip\Directive{SPACEGROUP SYMBOL=}


This directive generates symmetry information from the spacegroup symbol.
 The syntax is exactly as describe for the command SPACEGROUP, given
 in section \ref{SPACEGROUP}.


\bigskip\Keyword{SYMBOL=}


There is no default for the symbol, it should be a valid H-M space group
symbol, e.g. 'P 21 21 21' or 'P 21/c' or 'I -4 3 m'. Use spaces to 
separate each of the operators.





\bigskip\Directive{CONTENTS FORMULA=}


This directive takes the contents of the UNIT CELL 
 (cf LIST 29 - section \ref{LIST29}) and generates scattering factors 
 (LIST 3 - section \ref{LIST03}) and elemental properties (LIST 29 - section
 \ref{LIST29}).


\bigskip\Keyword{FORMULA=}
 The formula for the UNIT CELL contents
 {\bf (NOT} {\bf ASYMMETRIC} {\bf UNIT} - for compatibility with SIR92)
 is given as a list with entries of the type
\small\begin{verbatim}
       'element name' 'number of atoms'

 e.g. CONTENT FORMULA = C  24  H  36  O  8  N  4
\end{verbatim}\normalsize




The items in the list {\bf must} be separated by at least one space. The
 number of atoms may be fractional or, if omitted, they 
 default to 1.0.





\bigskip\Directive{FILE NAME=}


This directive associates the file containing the reflections with
 the program. The special name \emph{'COMMANDS'} causes reflection data to be read
 from the command stream. The reflections MUST then be terminated with an
 'h' value of -512, otherwise the end-of-file is sufficient.


\bigskip\Keyword{NAME=}
 The name of the file containing the reflections. The
 syntax of the name must conform to the computers operating system. See
 the {\bf IMMEDIATE} command $\backslash$SET FILE for case sensitive systems.





\bigskip\Directive{FORMAT EXPRESSION=}


This directive controls the reading of the reflection list. The reflection
 file must contain the following items in the order given. Only one
 reflection is permitted per line.
 See $\backslash$LIST 6 for more flexible input (section \ref{LIST06})
\small\begin{verbatim}
          h k l F and optionally sigma(F)
\end{verbatim}\normalsize





 F and sigma(F) may be replaced by I or F-squared.


\bigskip\Keyword{EXPRESSION=}
 The expression is a normal FORTRAN format expression, {\bf including} {\bf the}
 {\bf open} {\bf and} {\bf close} {\bf parentheses.}
 The descriptor 'nX' may be used to skip unwanted
 columns. The indices may be I or F format.
 There is no default expression.








\bigskip\Directive{DATA WAVELENGTH= REFLECTIONS= RATIO=}





\bigskip\Keyword{WAVELENGTH=}
 The wavelength, in Angstroms, used in selecting elemental properties. The
 default is 0.7107 (Molybdenum K-alpha radiation).


\bigskip\Keyword{REFLECTIONS=}
 A keyword to indicate whether the input data is F, F-squared or I.
\small\begin{verbatim}
      FOBS     -  Default, indicating F values being input.
      FSQUARED -  Indicating F squared values being input.
      I        -  Indicating intensity values being input.
\end{verbatim}\normalsize




If REFLECTIONS equals I, then an Lp correction is done assuming four circle
 geometry. Note that the reflections from modern diffractometers are 
 unlikely to be stored as FOBS. Some old X-ray data and neutron data may 
 still be given as FOBS.


\bigskip\Keyword{RATIO=}
 The minimum ratio of I/sigma(I) to be used in selecting reflections.
 Default is 3.0





\bigskip\Directive{CELL  A= B= C= ALPHA= BETA= GAMMA=}
 The real cell parameters. The angles default to 90.0 degrees.






\index{LIST 1}
\label{LIST01} 
\index{Cell parameters}
\section{Input of the cell parameters  -  LIST 1}



Either the real cell parameters or the reciprocal cell
 parameters may be input and the three angles be given in degrees or
 as their cosines.
 A mixed form, containing both angles and cosines is not allowed.

\small\begin{verbatim}
 \LIST 1
 REAL A= B= C= ALPHA= BETA= GAMMA=
 END
\end{verbatim}\normalsize



For example
\small\begin{verbatim}
 \LIST 1
 REAL 14.6 14.6 23.7 GAMMA=120
 END
\end{verbatim}\normalsize







\bigskip\Instruction{$\backslash$LIST 1}







\bigskip\Directive{REAL A= B= C= ALPHA= BETA= GAMMA=}


This directive introduces the real cell parameters.
 If this directive is present, the directive  RECIPROCAL
 will lead to an input error, and no new LIST 1 will be generated.


\bigskip\Keyword{A=, B=, C=}
 These parameter are the real cell lengths along the  A, B and C axes.
 There are no default values.


\bigskip\Keyword{ALPHA=, BETA=, GAMMA=}
 These parameters give the real cell angles
 or their cosines. The default value is 90 degrees.





\bigskip\Directive{RECIPROCAL A*= B*= C*= ALPHA*= BETA*= GAMMA*=}
 This directive introduces the reciprocal cell parameters.
 If this directive is present, the directive  REAL
 will lead to an input error, and no new LIST 1 will be generated.


\bigskip\Keyword{A*=, B*=, C*=}
 These parameters are the reciprocal cell lengths.


\bigskip\Keyword{ALPHA*=, BETA*=, GAMMA*=}
 These parameters give the reciprocal cell angles or their cosines.
 The default value is 90 degrees.



\section{Printing the cell parameters}






\bigskip\Instruction{$\backslash$PRINT 1}




This command lists the cell parameters, and all the other information
 derived from them which is stored in LIST 1.
 The inter-axial angles are stored in radians in LIST 1, and printed as
 such.





\bigskip\Instruction{$\backslash$PUNCH 1}




Punches the real cell parameters from  LIST 1.









\index{Cell errors}
\index{List 31}
\section{Input of the unit cell parameter errors  -  LIST 31}
\label{LIST31} 



This list contains the variance-covariance matrix of the unit
 cell parameters. The input consists of a multiplier which is
 applied to all input parameters,
 followed by the upper
 triangle of the variance-covariance matrix (21 Numbers).
 The units for the angles {\bf MUST} be
 radians and those for the cell lengths are Angstroms.

\small\begin{verbatim}
 \LIST 31
 AMULT VALUE=
 MATRIX V(11)= V(12)= .. V(16)= .. V(22)= .. V(26)= .. V(66)=
 END
\end{verbatim}\normalsize



For example
\small\begin{verbatim}
 \LIST 31
 \ the values of the input matrix are to be multiplied
 \ by 0.000001
 AMULT 0.000001
 \ the cell is trigonal,
 \ with errors of 0.002 along 'a' and 'b', and 0.004 along 'c'
 MATRIX 4 4 1 0 0 0
 CONT     4 1 0 0 0
 CONT      16 0 0 0
 CONT         0 0 0
 CONT           0 0
 CONT             0
 END
\end{verbatim}\normalsize







\bigskip\Instruction{$\backslash$LIST 31}







\bigskip\Directive{AMULT VALUE=}


This directive gives the value by which all the subsequent terms
 are to be multiplied, and has a default of 1.0.


\bigskip\Keyword{VALUE=}





\bigskip\Directive{MATRIX V(11)= V(12)= . . V(16)= V(22)= . . V(66)=}


This directive is used to read in the variance-covariance matrix.


If you only have the parameter e.s.d's, input the square of these for
 V(11), V(22) etc.


\bigskip\Keyword{V(11)= V(12)= . . V(16)= V(22)= . .V(66)=}
  V(11) is the variance of  A ,  V(12)  is the covariance of  A
 and  B ,  V(16)  is the covariance of  A  and  GAMMA ,
  V(22)  is the variance of  B , and  V(66)  is the variance of
  GAMMA . The default values for V(11), V(22) and V(33) correspond to axis
 e.s.d's of .001 A, V(44), V(55) and V(66) to angle e.s.d's of .01 degree.



\section{Printing the cell variance-covariance matrix}






\bigskip\Instruction{$\backslash$PRINT 31}




This prints list 31. There is no command for punching LIST 31.






\index{Space group input}
\index{SPACEGROUP}
\section{Space Group input - $\backslash$SPACEGROUP}
\label{SPACEGROUP} 

 The spacegroup symbol interpretation routines in CRYSTALS are derived
 from subroutines developed by Allen C. Larson and Eric Gabe.
 It is distributed with their permission. Standard CRYSTALS
 command input, error handling, data storage, and output has been
 added to the basic routines. In addition a more flexible method
 of  specifying the unique axis in a monoclinic spacegroup is
 used. The routine generates a LIST 2 (symmetry information - section
 \ref{LIST02}), and a
 LIST 14 (Fourier and Patterson asymmetric unit limits - 
 section \ref{LIST14}).

\small\begin{verbatim}
 \SPACEGROUP
 SYMBOL EXPRESSION=
 AXIS UNIQUE=
 END
\end{verbatim}\normalsize



For example
\small\begin{verbatim}
 \ Input the symbol for a cubic spacegroup
 \SPACEGROUP
 SYMBOL F d 3 m
 END

 \ Input the symbol for a common monoclinic spacegroup
 \SPACEGROUP
 SYMBOL P 21/c
 END

 \ Input the symbol for a triclinic spacegroup
 \SPACEGROUP
 SYMBOL P -1
 END
\end{verbatim}\normalsize







\bigskip\Instruction{$\backslash$SPACEGROUP}







\bigskip\Directive{SYMBOL EXPRESSION=}


This directive is used to specify the space group symbol.


\bigskip\Keyword{EXPRESSION=}
 The value of this parameter is the text making up the spacegroup
 symbol.  At least one
 space character should appear between each of the axis symbols
 in the spacegroup symbol. e.g.
\small\begin{verbatim}
 Use  P 21 3 rather than P 213, P2 1 3, or P2 13
\end{verbatim}\normalsize




Failure to put spaces in the correct place in the symbol will
 lead to misinterpretation.


Rhombohedral cells are always assumed to be on hexagonal indexing.





\bigskip\Directive{AXIS UNIQUE=}


This directive specifies the unique
 axis orientation for
 monoclinic spacegroups where the symbol specified
 contains only one axis symbol (short symbol). In other cases any information
 specified with this directive is ignored.


\bigskip\Keyword{UNIQUE=}
\small\begin{verbatim}
      A
      B
      C
      GENERATE - the default value.
\end{verbatim}\normalsize




When UNIQUE has the value A, B, or C the program uses the 'a',
 'b', or 'c' axis respectively as the unique axis.
 When UNIQUE
 has the value GENERATE, the program will attempt to select the
 unique axis on the basis of the cell parameters currently stored in
 LIST 1. If this is not possible, because the angles in LIST 1
 are all close too 90 degrees or there is no valid cell parameter
 information, the program will assume that the unique axis is
 'b'.





Further examples.

\small\begin{verbatim}
 \LIST 1
 REAL 10.2 11.3 14.1 88.3 90 90
 END
 \ Input symmetry - the program will  automatically select 'a' as the
 \ unique axis based on the cell parameters.
 \SPACEGROUP
 SYMBOL P 21/M
 END
\end{verbatim}\normalsize



\small\begin{verbatim}
 \ Explicitly specify 'c' unique by giving the full symbol.
 \SPACEGROUP
 SYMBOL P 1 1 21/M
 END
 \
 \ Explicitly specify 'c' unique by using the UNIQUE parameter.
 \SPACEGROUP
 SYMBOL P 21/M
 AXIS UNIQUE=C
 END
\end{verbatim}\normalsize





\index{LIST 2}
\index{Symmetry data}
\section{Input of the symmetry data  -  LIST 2}
\label{LIST02} 



The result of inputting a $\backslash$SPACEGROUP command (section \ref{SPACEGROUP}) 
is the automatic generation of a 'LIST 2' containing the explicit 
symmetry operators and other information that defines the spacegroup.


Direct input of this list enables the user to specify explicitly 
 the symmetry operators to be used. The advantage of this is that 
 they need not comply to any standard convention - the only
 check made by the program is to ensure that the determinant is 
 not zero. For example, this technique may be used to enter a 
 set of symmetry operators that contains a translation of a half along
 an axis - normally that cell length would be halved instead, but it may
 be useful in order to work consistently with a structure that undergoes
 a cell-doubling phase transition.

\small\begin{verbatim}
 \LIST 2
 CELL NSYMMETRIES=  LATTICE=  CENTRIC=
 SYMMETRY  X=  Y=  Z=
 SPACEGROUP LATTICE= A-AXIS= B-AXIS= C-AXIS=
 CLASS NAME=
 END
\end{verbatim}\normalsize


For example:

\small\begin{verbatim}
 \ the space group is B2/b
 \LIST 2
 CELL NSYM= 2, LATTICE = B
 SYM X, Y, Z
 SYM X, Y + 1/2,  - Z
 SPACEGROUP B 1 1 2/B
 CLASS MONOCLINIC
 END
\end{verbatim}\normalsize




The CELL directive defines the Bravais lattice type,
 the number of equivalent positions to be input, and whether the
 cell is centric or acentric.
 The equivalent positions are defined by  SYMMETRY  directives, which contain
 one equivalent position each, and must follow the  CELL  directive.
 The equivalent positions input should not include those related
 by a centre of symmetry if the lattice is defined as centric, and should
 not include those related by non-primitive lattice translations if
 the correct Bravais lattice type is given.
 Positions generated by the last two operations are computed by the
 system.
 The unit matrix, defining x, y, z, {\bf MUST} {\bf ALWAYS} be input.
 If a centric cell is used in a setting which does not place the centre
 at the origin, then ALL the operators must be given and the cell be
 treated as non-centric. This will of course increase the time for
 structure factor calculations.


Rhombohedral cells can be treated in two ways. If used with
 rhombohedral indexing (a=b=c, alpha=beta=gamma), the lattice type is P,
 primitive.
 If used with hexagonal indexing, the lattice type is R.





\bigskip\Instruction{$\backslash$LIST 2}







\bigskip\Directive{CELL NSYMMETRIES=  LATTICE=  CENTRIC=}


\bigskip\Keyword{NSYMMETRIES=}
 This defines the number of SYMMETRY directives that are to follow.
 There is no default.


\bigskip\Keyword{LATTICE=}
 This defines the Bravais lattice type, and must take
 one of the following values :



\small\begin{verbatim}
      P  -  Default value.
      I
      R
      F
      A
      B
      C
\end{verbatim}\normalsize




\bigskip\Keyword{CENTRIC=}
 This parameter defines whether the cell is centric or acentric, and must
 take one of the values :
\small\begin{verbatim}
      NO
      YES  -  The default value.
\end{verbatim}\normalsize




\bigskip\Directive{SYMMETRY  X=  Y=  Z=}
 This directive is repeated  NSYMMETRIES  times, and each separate occurrence
 defines one equivalent position in the unit cell.
 The parameter keywords  X ,  Y  and  Z  are normally omitted on this
 directive, and the equivalent position typed up exactly
 as given in international tables.
 The expressions may contain any of the following :
\small\begin{verbatim}
      +X or -X
      +Y or -Y
      +Z or -Z
      + or - a fractional shift.
\end{verbatim}\normalsize


 The fractional shift may be represented by one number divided by another
 (e.g. 1/2 or 1/3) or by a true fraction (0.5 or 0.33333...).
 Apart from terminating text, spaces are optional and ignored.
 The terms for the new x, y and z must be separated by a  comma (,) , and the
 whole expression may be terminated by  ;  if required.





\bigskip\Directive{SPACEGROUP LATTICE= A-AXIS= B-AXIS= C-AXIS=}
 This directive inputs the space group symbol, and is optional for the
 correct working of CRYSTALS. However, some foreign programs need
 the symbol as input data, and they will extract it from this record.
 The keywords LATTICE, A-AXIS etc are normally omitted, and the full
 space group symbol given with spaces between the operators, e.g.
\small\begin{verbatim}
        SPACEGROUP P 1 21/C 1
\end{verbatim}\normalsize







\bigskip\Directive{CLASS NAME=}
 This directive inputs the crystal class. It is not used by CRYSTALS, but is
 required for cif files.






\section{Printing the symmetry information}






\bigskip\Instruction{$\backslash$PRINT 2}




This prints LIST 2. There is no command for punching LIST 2.





Further examples.

\small\begin{verbatim}
 \ THE SPACE GROUP IS P1-BAR.
 \LIST 2
 CELL NSYM= 1
 SYM X, Y, Z
 SPACEGROUP P -1
 END
\end{verbatim}\normalsize



\small\begin{verbatim}
 \ THE SPACE GROUP IS P 321
 \LIST 2
 CELL CENTRIC= NO, NSYM= 6
 SYM X, Y, Z
 SYM -Y, X-Y, Z
 SYM Y-X, -X, Z
 SYM Y, X, -Z
 SYM -X, Y-X, -Z
 SYM X-Y, -Y, -Z
 END
\end{verbatim}\normalsize



\small\begin{verbatim}
 \ THE SPACE GROUP IS P 6122 (note alternative notation for fractions)
 \LIST 2
 CELL NSYM= 12, CENTRIC= NO
 SYM X,Y,Z
 SYM -X    ,   -Y  ,Z+.5
 SYM +Y, +X,1/3-Z
 SYM -Y,-X,5/6-Z
 SYM -Y, X-Y, .333333333+Z
 SYM Y, Y-X, Z+10/12
 SYM -X, Y-X, 4/6-Z
 SYM X, X-Y, 1/6-Z
 SYM Y-X, -X, Z+4/6
 SYM  X-Y, X, Z+1/6
 SYM X-Y, -Y, -Z
 SYM Y-X, Y ,  -Z+.5
 SPACEGROUP P 61 2 2
 END
\end{verbatim}\normalsize








\index{Molecular composition}
\index{COMPOSITION}
\section{Input of molecular composition $\backslash$COMPOSITION}



This command takes the contents of the asymmetric unit, searches the
 specified data files for required values, and then internally creates normal
 scattering factors (LIST 3 - section \ref{LIST03}) and elemental 
 properties (LIST 29 - section \ref{LIST29}). {\bf NOTE}
 {\bf LISTS} {\bf 1} (see ) {\bf and} {\bf 13} (see  \ref{LIST01}RLIST13

) 
 {\bf must} {\bf have} {\bf been} {\bf input} {\bf beforehand.}

\small\begin{verbatim}
 \COMPOSITION
 CONTENTS FORMULA=
 SCATTERING FILE=
 PROPERTIES FILE=
 END
\end{verbatim}\normalsize


For example:

\small\begin{verbatim}
 \COMPOSITION
 CONTENT C 6 H 5 N O 2.5 CL
 SCATTERING CRSCP:SCATT.DAT
 PROPERTIES CRSCP:PROPERTIES.DAT
 END
\end{verbatim}\normalsize







\bigskip\Instruction{$\backslash$COMPOSITION}




There are three directives, none of which have default values.





\bigskip\Directive{CONTENTS FORMULA=}


\bigskip\Keyword{FORMULA=}
 The formula for the UNIT CELL
 (NOT ASYMMETRIC UNIT) is given as a list with entries
\small\begin{verbatim}
 'element TYPE' 'number of atoms'.
\end{verbatim}\normalsize


 The items in the list MUST be separated by at least one space. The number
 of atoms may be omitted, when they default to 1.0, and may be fractional.



 The element TYPE must conform to the TYPE conventions described in the
 atom syntax, section \ref{ATOMSYNTAX}.





\bigskip\Directive{SCATTERING FILE=}
 This directive gives the name of the
 file to be searched for scattering factors, and must conform to the syntax
 of the computing system. A file CRSCP:SCATT.DAT is provided for some
 implementations, and contains all the scattering factors listed in
 Volume IV, International Tables.





\bigskip\Directive{PROPERTIES FILE=}
 This directive gives the name of the
 file to be searched for elemental properties, and must conform to the syntax
 of the computing system. A file CRSCP:PROPERTIES.DAT is provided for
 some
 implementations, and contains values gleaned from various sources. The
 file contains references.






\index{LIST 3}
\index{Scattering factors}
\section{Input of the atomic scattering factors  -  $\backslash$LIST 3}
\label{LIST03} 



This list contains the scattering factors that
 are to be used for each atomic species that may appear in the
 atomic parameter list (LIST 5)
  - see the section of the user guide on Atom and Element names).

\small\begin{verbatim}
 \LIST 3
 READ  NSCATTERERS=
 SCATTERING TYPE= F'= F''= A(1)= B(1)= A(2)= . . . B(4)= C=
 END
\end{verbatim}\normalsize





For example

\small\begin{verbatim}
 \LIST 3
 READ 2
 SCATT C    0    0
 CONT  1.93019  12.7188  1.87812  28.6498  1.57415  0.59645
 CONT  0.37108  65.0337  0.24637
 SCATT S 0.35 0.86  7.18742  1.43280  5.88671  0.02865
 CONT               5.15858  22.1101  1.64403  55.4561
 CONT              -3.87732
 END
\end{verbatim}\normalsize


 The scattering factor of an atom in LIST 5 (the model parameters) 
 is determined by its  TYPE, an entry for which must exist in LIST 3.



 The form factor is calculated analytically at each value
 of sin(theta)/lambda,  s , from the relationship :
\small\begin{verbatim}
 f = sum[a(i)*exp(-b(i)*s*s)] + c       i=1 to 4.
\end{verbatim}\normalsize


 The coefficients a(1) to a(4), b(1) to b(4) and c and the real and
 imaginary parts of the anomalous dispersion correction
 are input for each element TYPE.





\bigskip\Instruction{$\backslash$LIST 3}




This is the normal calling command for the input of LIST 3.





\bigskip\Directive{READ  NSCATTERERS=}


\bigskip\Keyword{NSCATTERERS=}
 This must be set to the number of atomic species to be stored
 in LIST 3, and thus the number of  SCATTERING  directives to follow.
 There is no default value.





\bigskip\Directive{SCATTERING TYPE= F'= F''= A(1)= B(1)= A(2)= . . . B(4)= C=}
 This directive provides the form factor details for one atomic species.
 This directive must be repeated  NSCATTERERS  times.


\bigskip\Keyword{TYPE=}
 The element TYPE must conform to the TYPE conventions described in the
 General Introduction.
 The values used for  TYPE  in LIST 3 will have their counterparts
 in the  TYPEs stored for atoms in LIST 5 (the model parameters), 
 and in the  TYPEs
 stored for atomic species in LIST 29 (see section \ref{LIST29}).
 There is no default for this parameter.


\bigskip\Keyword{F'= F''=}
 These define the real and imaginary parts of the anomalous dispersion
 correction for this atomic species at the appropriate wavelength.
 A default value of zero is assumed if these parameters are omitted.


\bigskip\Keyword{A(1)= B(1=) A(2=) B(2)= A(3)= B(3)= A(4)= B(4)= C=}
 These define the coefficients used to compute the
 scattering factor for this atomic species. There are
 no default values.

\small\begin{verbatim}
  For neutrons, all the A(i) and B(i) are set to zero, and C is set to
 the scattering length.
\end{verbatim}\normalsize





\section{Printing the scattering factors}






\bigskip\Instruction{$\backslash$PRINT 3}




This prints LIST 3. There is no command for punching LIST 3.



\index{LIST 13}
\index{Experimental details}
\section{Input of the crystal and data collection details  -  LIST 13}
\label{LIST13} 



LIST 13 contains information about those experimental details
 which may be needed during structure analysis.
 Information only required for
 the generation of a cif are held in LIST 30 (section \ref{LIST30}).


If no LIST 13 has been input and one is 
 required, a default list is generated.

\small\begin{verbatim}
 \LIST 13
 CRYSTAL FRIEDELPAIRS= TWINNED= SPREAD=
 DIFFRACTION GEOMETRY= RADIATION=
 CONDITIONS WAVELENGTH= THETA(1)= THETA(2)= CONSTANTS . .
 MATRIX R(1)= R(2)= R(3)= . . . R(9)=
 TWO H= K= L= THETA= OMEGA= CHI= PHI= KAPPA= PSI=
 THREE H= K= L= THETA= OMEGA= CHI= PHI= KAPPA= PSI=
 REAL COMPONENTS= H= K= L= ANGLES=
 RECIPROCAL COMPONENTS= H= K= L= ANGLES=
 AXIS H= K= L=
\end{verbatim}\normalsize


For example:

\small\begin{verbatim}
 \LIST 13
 DIFF GEOM= CAD4
 COND WAVE= .7107
 MATRIX
 END
\end{verbatim}\normalsize







\bigskip\Instruction{$\backslash$LIST 13}




This directive describes properties that relate to the whole
 crystal.








\bigskip\Directive{CRYSTAL FRIEDELPAIRS= TWINNED= SPREAD=}





\bigskip\Keyword{FRIEDELPAIRS=}
  This parameter defines whether Friedel's law should be used during
 $\backslash$SYSTEMATIC in
 data reduction. It should be set to NO for high accuracy or absolute
 structure determinations. If omitted, Friedel's law will be used.
\small\begin{verbatim}
      YES  -  default value.
      NO
\end{verbatim}\normalsize




\bigskip\Keyword{TWINNED=}
 This parameter is used during refinement to indicate
  whether the twin laws should be used. It is automatically updated
if twinned reflection data is input.
\small\begin{verbatim}
      NO  -  Default value.
      YES
\end{verbatim}\normalsize




\bigskip\Keyword{SPREAD=}
 This parameter defines the type of mosaic spread in the crystal.
 This information is used during the calculation of an extinction
 correction.
\small\begin{verbatim}
      GAUSSIAN  -  Default value. Suitable for X-rays
      LORENTZIAN - Suitable for Neutrons
\end{verbatim}\normalsize







\bigskip\Directive{DIFFRACTION GEOMETRY= RADIATION=}


This directive defines the experimental conditions used to
 collect the data.


\bigskip\Keyword{GEOMETRY=}
 This defines the type of data collection method used
 to measure the raw intensities, and determines the type of Lp
 correction.

\small\begin{verbatim}
      NORMAL  -  Normal beam Weissenberg geometry.
      EQUI    -  Equi-inclination Weissenberg geometry.
      ANTI    -  Anti-equi-inclination Weissenberg geometry.
      PRECESSION
      CAD4    -  Nonius CAD4 diffractometer, Eulerian angles.
      KAPPA   -  Nonius CAD4 in kappa geometry.
      ROLLETT -  Abstract machine, see page 28 , Computing Methods
                 in Crystallography.
      Y290    -  Hilger-Watts Y290 4-Circle diffractometer.
      NONE    -  Default.
\end{verbatim}\normalsize




\bigskip\Keyword{RADIATION=}
 This parameter defines the type of radiation used to collect the
 data.
\small\begin{verbatim}
      XRAYS  -  Default value
      NEUTRONS
\end{verbatim}\normalsize




\bigskip\Directive{CONDITIONS WAVELENGTH= THETA(1)= THETA(2)= CONSTANTS . .}


This directive describes the conditions that were used when the
 data were collected.
  CONSTANTS  is short for four constants.
\small\begin{verbatim}
      CONSTANT(1)= CONSTANT(2)= CONSTANT(3)= CONSTANT(4)=
\end{verbatim}\normalsize




\bigskip\Keyword{WAVELENGTH=}
 This defines the wavelength of the radiation used to collect
 the data.
 If omitted, a default value of 0.71073 is assumed,(Mo k-alpha).


\bigskip\Keyword{THETA(1)=}
 This defines the Bragg angle of the monochromator.
 If omitted, a default of 6.05 is assumed, indicating
 that a monochromator was used with Mo radiation


\bigskip\Keyword{THETA(2)=}
 This defines the angle between the plane of the
 monochromator and the diffracting planes of the crystal.
 If this parameter is omitted, a default value of 90 is assumed.
 This value is not used if  THETA(1)  is zero.
 Since the angle  THETA(2)  is fixed, the Lp correction computed
 using these constants is correct only for experiments where  THETA(2)
 is a constant.
 This is true for equatorial geometry experiments, but is not true
 for equipment that uses Weissenberg or precession geometry.  It is not 
 true for area detector instruments.


\bigskip\Keyword{CONSTANT(1)= CONSTANT(2)= CONSTANT(3)= CONSTANT(4)=}
 These four parameters are used to input fundamental constants
 for the diffractometer used to collect the data.
 How many of the constants, and what values they should have are
 determined by the equipment and its setting.
 To determine the values required, consult your local diffractometer
 expert.
 The default values for c(1), c(2) and c(3) are the Nonius CAD4 GONCON
 constants, and c(4) is the theta value for the change from
 bisecting to fixed chi mode (and has a value of 90 degrees).
 These constants are important when machine geometry dependent
 calculations are made - for example, absorption corrections.
 The defaults in the program were correct for the Nonius CAD4 
 in the Oxford Chemical Crystallography lab on 13 October 1980.





\bigskip\Directive{MATRIX R(1)= R(2)= R(3)= . . . R(9)=}


This directive is used to input the orientation matrix directly.
 If this directive is input, the directives  TWO ,  THREE ,  REAL ,
 and  RECIPROCAL (detailed below) may not be used.
 This directive is normally used for diffractometer collected data.


\bigskip\Keyword{R(1)= R(2)= R(3)= . . . R(9)=}
 The elements of the matrix must be input in the order
 (1,1), (1,2), (1,3), (2,1), etc.
 The default is a unit diagonal matrix.





\bigskip\Directive{TWO H= K= L= THETA= OMEGA= CHI= PHI= KAPPA= PSI=}


This directive is used to input the setting details
 required to define a diffractometer orientation matrix from
 two reflections.
 The details for the two reflections must be input on separate
 directives, so that this directive must be repeated twice.
 This directive may only be input when the  GEOMETRY parameter 
 on the DIFFRACTION directive is  Y290  or  CAD4 .
 If this directive is input, the directives  THREE ,  REAL ,
  RECIPROCAL , and  MATRIX  may not be used.
 The reflections should be given in the same order as
 in the original experiment.


\bigskip\Keyword{H= K= L=}
 These three parameters define the indices of the reflection
 that is to be used to calculate the orientation matrix.


\bigskip\Keyword{THETA= OMEGA= CHI= PHI= KAPPA= PSI=}
 These parameters define the setting angles for the reflection
 whose indices are given by  H ,  K  and  L .
 There are no default values for  THETA ,  OMEGA  and  PHI , and
 one of  CHI or KAPPA must be input.
 The default values for CHI , KAPPA and PSI are zero.





\bigskip\Directive{THREE H= K= L= THETA= OMEGA= CHI= PHI= KAPPA= PSI=}


This directive is used to input the setting details
 required to define a diffractometer orientation matrix from
 three reflections.
 The details for the three reflections must be input on separate
 directives, so that this directive must be repeated three times.
 This directive may only be input when the  GEOMETRY  parameter 
 on the DIFFRACTION  directive is  Y290  or  CAD4 .
 If this directive is input, the directives  TWO ,  REAL ,
  RECIPROCAL , and  MATRIX  may not be used.


\bigskip\Keyword{H= K= L= THETA= OMEGA= CHI= PHI= KAPPA= PSI=}
 These parameters are defined as for TWO above.





\bigskip\Directive{REAL COMPONENTS= H= K= L= ANGLES=}
 This directive is used to define the orientation matrix for
 the Nonius CAD4 diffractometer from the components of the real
 vector along the phi axis and the setting angles of one reflection.
 The  items COMPONENTS  and  ANGLES  are short for:





COMPONENT(1)= COMPONENT(2)= COMPONENT(3)=


and


THETA= OMEGA= CHI= PHI= KAPPA= PSI=



 If this directive is input, the directives  TWO ,  THREE ,
  RECIPROCAL , and  MATRIX  may not be used.
 This directive may only be input when the  GEOMETRY  parameter
 on the DIFFRACTION  directive is  CAD4 .


\bigskip\Keyword{COMPONENT(1)= COMPONENT(2)= COMPONENT(3)=}
 These three parameters provide the components of the real cell
 vector that is parallel to the phi axis.


\bigskip\Keyword{H= K= L= THETA= OMEGA= CHI= PHI= KAPPA= PSI=}
 These parameters are defined as in TWO above





\bigskip\Directive{RECIPROCAL COMPONENTS= H= K= L= ANGLES=}
 This directive is used to define the orientation matrix for
 the Nonius CAD4 diffractometer from the components of the reciprocal
 vector along the phi axis and the setting angles of one reflection.
 The  items COMPONENTS  and  ANGLES  are short for:





COMPONENT(1)= COMPONENT(2)= COMPONENT(3)=


and


THETA= OMEGA= CHI= PHI= KAPPA= PSI=



 If this directive is input, the directives  TWO ,  THREE ,
  REAL , and  MATRIX  may not be used.
 This directive may only be input when the  GEOMETRY  parameter
 on the DIFFRACTION  directive is  CAD4 .


\bigskip\Keyword{COMPONENT(1)= COMPONENT(2)= COMPONENT(3)=}
 These three parameters provide the components of the reciprocal cell
 vector that is parallel to the phi axis.


\bigskip\Keyword{H K L THETA OMEGA CHI PHI KAPPA PSI}
 These parameters are defined as in TWO above





\bigskip\Directive{AXIS H= K= L=}


This directive is used to define the axis about which data were
 collected in Weissenberg geometry.
 This directive may only be given when the  GEOMETRY  parameter 
 on the  DIFFRACTION  directive is one of  NORMAL ,  EQUI  or  ANTI .


\bigskip\Keyword{H= K= L=}
 These three parameters define the zone axis [hkl] about which the
 crystal was rotated during data collection.
 If any of these parameters is omitted, a default value of zero
 is assumed.



\section{Printing the experimental conditions, LIST 13}






\bigskip\Instruction{$\backslash$PRINT 13}




This prints list 13. There is no command for punching LIST 13.









\index{LIST 18}
\index{Structural Formula as a SMILES string}
\section{Input of Structural Formula as a SMILES string - LIST 18}
\label{LIST18} 



This list holds the structural formula as a SMILES string

\small\begin{verbatim}
 \LIST 18
 SMILES    string.
 END
\end{verbatim}\normalsize


For example

\small\begin{verbatim}
\end{verbatim}\normalsize

list 18
smile CC(C1=CC=CC=[N]1[Ga]2345[N](N=C(S5)Nc6ccccc6)=C(C)
cont C7=CC=CC=[N]73)=[N]2N=C(S4)Nc8ccccc8
smile n(o)(o)(o)
end








\bigskip\Instruction{$\backslash$LIST 18}







DSMILE smiles text. Remeber there is an 80 character line length. Break
 the SMILES at a suitable point and use a CONTINUATION card.


 If there is more than one discrete moiety in the cell, enter each on 
its own SMILES card






\section{Printing the SMILES string, LIST 18}






\bigskip\Instruction{$\backslash$PRINT 18}




This prints list 18. 



\section{Punching the SMILES string, LIST 18}






\bigskip\Instruction{$\backslash$PUNCH 18}




This punches list 18. 







\index{LIST 29}
\index{Element properties}
\section{Input of the contents of the asymmetric unit  -  LIST 29}
\label{LIST29} 



To perform calculations based on elemental properties, such as Sim
 weighting for Fourier maps (section \ref{FOURIER}), connectivity 
 calculations, absorption
 and density calculations, it is necessary to input the numbers and
 properties of the elements in the cell.
 This information is stored in LIST 29.

\small\begin{verbatim}
 \LIST 29
 READ  NELEMENT=
 ELEMENT TYPE= COVALENT= VANDERWAALS= IONIC= NUMBER= MUA= WEIGHT= COLOUR=
 END
\end{verbatim}\normalsize


For example:

\small\begin{verbatim}
 \LIST 29
 READ NELEMENT=4
 ELEMENT MO NUM=0 .5
 ELEMENT S NUM=2
 ELEMENT O NUM=3
 ELEMENT C NUM=10
 END
\end{verbatim}\normalsize







\bigskip\Instruction{$\backslash$LIST 29}







\bigskip\Directive{READ  NELEMENT=}


\bigskip\Keyword{NELEMENT}
 This must be set to the number of atomic species in the asymmetric
 unit, and
 consequently the number of  ELEMENT directives that are about to 
 follow this directive.
 If this directive is omitted, a default value of one is assumed for
 NELEMENT.





\bigskip\Directive{ELEMENT TYPE= COVALENT=  VANDERWAALS= IONIC= NUMBER= MUA= WEIGHT=}
 Each  ELEMENT directive provides the information about that atomic
 species in the asymmetric unit.


\bigskip\Keyword{TYPE=}
 The element TYPE must conform to the TYPE conventions described in the
 section on atom syntax, \ref{ATOMSYNTAX}.
 The default value for this parameter is taken from the COMMAND file.
 When LIST 29 is used for Simm weighting,
 the  TYPE  is compared with the  TYPEs stored
 in LIST 3 (section \ref{LIST03}) to determine the scattering 
 factor of the given species.


\bigskip\Keyword{COVALENT=}


\bigskip\Keyword{VANDERWAALS=}


\bigskip\Keyword{IONIC=}
 The radii used during geometry calculations,
 with a default values set in the COMMAND file. The covalent radius is
 incremented by 0.1 A for distance contacts, and
 used for defining restraint targets (see $\backslash$DISTANCES).
 The van der Waals radius is incremented by .25A for finding non-bonded
 contacts, and used for defining energy restraints
 The ionic radius may be used during geometry calculations.


\bigskip\Keyword{NUMBER=}
 This parameter gives the number of atoms of the given type in the
 asymmetric unit.
 This number can be fractional, depending
 on the number of atoms in the cell and whether they occupy special
 positions, and whether they are disordered.


\bigskip\Keyword{MUA=}
 This is the atomic absorption coefficient x10**(-23)  /cm as in INT TAB
 VOL III.  Note that in Vol IV the units are x10**(-24).
 Take care to ensure that the coefficients are appropriate
 for the wavelength used.


\bigskip\Keyword{WEIGHT}
 This is the Atomic weight





\bigskip\Keyword{COLOUR}
 This is the colour to be used for each atom in CAMERON. The available 
colours are:
\small\begin{verbatim}
 BLACK BLUE    CYAN   GREEN GREY   LBLUE LGREEN LGREY  
 LRED  MAGENTA ORANGE PINK  PURPLE RED   WHITE  YELLOW 
\end{verbatim}\normalsize

	



\section{Printing the contents of the asymmetric unit, LIST 29}






\bigskip\Instruction{$\backslash$PRINT 29}




This prints list 29. There is no command for punching LIST 29.



\index{LIST 30}
\index{General crystallographic data}
\section{Input of General Crystallographic Data - LIST 30}
\label{LIST30} 



This list holds general crystallographic information for later
 inclusion in the cif file. CRYSTALS contains no COMMAND for editing
 this list - inputting a new LIST 30 over writes any existing version.
 However, some CRYSTALS commands update LIST 30 as an analysis proceeds, 
and there is a SCRIPT which enables some details to be changed.

\small\begin{verbatim}
 \LIST 30
 DATRED     NREFMES= NREFMERG= RMERGE= NREFFRIED= RMERGFRIED= REDUCTION=
 CONDITIONS MINSIZE= MEDSIZE= MAXSIZE= NORIENT=
 CONTINUE   THORIENTMIN= THORIENTMAX= TEMPERATURE= STANDARDS= DECAY= SCANMODE=
 CONTINUE   INTERVAL= COUNT= INSTRUMENT=
 REFINEMENT R= RW= NPARAM= SIGMACUT= GOF= DELRHOMIN= DELRHOMAX=
 CONTINUE   RMSSHIFT= NREFUSED= FMINFUNC= RESTMINFUN= TOTALMINFUN= COEFFICIENT=
 INDEXRANGE HMIN= HMAX= KMIN= KMAX= LMIN= LMAX= THETAMIN= THETAMAX=
 ABSORPTION ANALMIN= ANALMAX= THETAMIN= THETAMAX= EMPMIN= EMPMAX=
 CONTINUE   DIFABSMIN= DIFABSMAX= ABSTYPE=
 GENERAL    DOBS= DCALC= F000= MU= MOLWT= FLACK= ESD= ANALYSE-CUT= 
 CONTINUE   ANALYSE-NREF= ANALYSE-R= ANALYSE-RW= SOLUTION=
 COLOUR
 SHAPE
 CIFEXTRA   CALC-SIGMA= CALC-NREF= CALC-R= CALC-RW= 
 CONTINUE   ALL-SIGMA= ALL-NREF= ALL-R= ALL-RW=
 END
\end{verbatim}\normalsize


For example

\small\begin{verbatim}
 \LIST 30
 CONDITIONS MINSIZE=.1 MEDSIZE=.3 MAXSIZE=.8 NORIENT=25
 CONTINUE   THORIENTMIN=15.0 THORIENTMAX=25.0
 CONTINUE   TEMPERATURE=293 STANDARDS=3 DECAY=.05 SCANMODE=2THETA/OMEGA
 CONTINUE   INSTRUMENT=MACH3
 INDEXRANGE HMIN=-12 HMAX=12 KMIN=-13 KMAX=13 LMIN=-1 LMAX=19
 COLOUR RED
 SHAPE PRISM
 END
\end{verbatim}\normalsize







\bigskip\Instruction{$\backslash$LIST 30}







\bigskip\Directive{DATRED NREFMES= NREFMERG= RMERGE= NREFFRIED= RMERGFRIED= REDUCTION=}
 Information about the data reduction process.


\bigskip\Keyword{NREFMES=}
 The number of reflections actually measured in the diffraction
 experiment


\bigskip\Keyword{NREFMERG=}
 Number of unique reflections remaining after merging equivalents
 applying Friedel's Law


\bigskip\Keyword{RMERGE=}
 Merging R factor (R int) applying Friedel's Law (as decimal not \%)


\bigskip\Keyword{NREFFRIED=}
 Number of unique reflections remaining after merging equivalents
 without applying Friedel's Law


\bigskip\Keyword{RMERGFRIED=}
 Merging R factor (R int) without applying Friedel's Law (as decimal not \%)








\bigskip\Directive{CONDITIONS MINSIZE= MEDSIZE= MAXSIZE= NORIENT= THORIENTMIN= THORIENTMAX= }


\bigskip\Directive{CONDITIONS (continued) TEMPERATURE= STANDARDS= DECAY= SCANMODE=}


\bigskip\Directive{CONDITIONS (continued) INTERVAL= COUNT= INSTRUMENT=}
 Information about data collection.


\bigskip\Keyword{MINSIZE=}


\bigskip\Keyword{MEDSIZE=}


\bigskip\Keyword{MAXSIZE=}
 The crystal dimensions, in mm.


\bigskip\Keyword{NORIENT=}
 Number of orientation checking reflections.


\bigskip\Keyword{THORIENTMIN=}
 Minimum theta value for orientating reflections.


\bigskip\Keyword{THORIENTMAX=}
 Maximum theta value for orientating reflections.


\bigskip\Keyword{TEMPERATURE=}
 Data collection temperature, Kelvin.


\bigskip\Keyword{STANDARDS=}
 Number of intensity control reflections.


\bigskip\Keyword{DECAY=}
 Average decay in intensity, \%.


\bigskip\Keyword{SCANMODE=}
 Data collection scan method. Options are

\small\begin{verbatim}
      2THETA/OMEGA (Default)
      OMEGA
      UNKNOWN
\end{verbatim}\normalsize




\bigskip\Instruction{INTERVAL=}


 Intensity control reflection interval time, minutes. Used if standards
 are measured at a fixed time interval


\bigskip\Keyword{COUNT=}
 Intensity control reflection interval count. Used if standards are
 measured after a fixed number (count) of general reflections.


\bigskip\Keyword{INSTRUMENT}
 Instrument used for data collection. Known instruments are:

\small\begin{verbatim}
      UNKNOWN (default)
      CAD4
      MACH3
      KAPPACCD
      DIP
      SMART
      IPDS
\end{verbatim}\normalsize




\bigskip\Directive{REFINEMENT R= RW= NPARAM= SIGMACUT= GOF= DELRHOMIN= DELRHOMAX= RMSSHIFT=}


\bigskip\Directive{REFINE (cont) NREFUSED= FMINFUNC= RESTMINFUNC= TOTALMINFUNC= COEFFICIENT=}
 Information about the refinement procedure.


\bigskip\Keyword{R=}
 Conventional R-factor.


\bigskip\Keyword{RW=}
 Hamilton weighted R-factor.





The weighted R-factor stored in  LIST 6 (section \ref{LIST06}) and LIST 30 
is that computed
during a structure factor calculation. The conventional R-factor is
updated by either an SFLS calculation (section \ref{SFLS})
or a SUMMARY of LIST 6.


\bigskip\Keyword{NPARAM=}
 Number of parameters refined in last cycle.


\bigskip\Keyword{SIGMACUT=}
 The I/sigma(I) threshold used during refinement.


\bigskip\Keyword{GOF=}
 GOF, Goodness-of-Fit, S.


\bigskip\Keyword{DELRHOMIN=}


\bigskip\Keyword{DELRHOMAX=}
 Minimum and maximum electron density in last difference synthesis.


\bigskip\Keyword{RMSSHIFT=}
 R.m.s (shift/e.s.d) in last cycle of refinement.


\bigskip\Keyword{NREFUSED=}
 Number of reflections used in last cycle of refinement.


\bigskip\Keyword{FMINFUNC=}
 Minimisation function for diffraction observations.


\bigskip\Keyword{RESTMINFUNC=}
 Minimisation function for restraints.


\bigskip\Keyword{TOTALMINFUNC=}
 Total minimisation function.


\bigskip\Keyword{COEFFICIENT=}
 Coefficient for refinement. Alternatives are:

\small\begin{verbatim}
      F (Default)
      F**2
      UNKNOWN
\end{verbatim}\normalsize




\bigskip\Directive{INDEXRANGE HMIN= HMAX= KMIN= KMAX= LMIN= LMAX= THETAMIN= THETAMAX=}
 Range of reflection limits during data collection.


\bigskip\Keyword{HMIN= HMAX= KMIN= KMAX= LMIN= LMAX=}
 Minimum and maximum values of h,k and l.


\bigskip\Keyword{THETAMIN= THETAMAX=}
 Minimum and maximum values of theta.





\bigskip\Directive{ABSORPTION ANALMIN= ANALMAX= THETAMIN= THETAMAX= EMPMIN= EMPMAX=}


\bigskip\Directive{ABSORPTION (continued) DIFABSMIN= DIFABSMAX= ABSTYPE=}
 Information about absorption corrections.



 {\bf NOTE} the keywords PSIMIN and PSIMAX have been removed.  Store values 
 as EMPMIN  and EMPMAX


\bigskip\Keyword{ANALMIN= ANALMAX=}
 Minimum and maximum analytical corrections


\bigskip\Keyword{THETAMIN= THETAMAX=}
 Minimum and maximum theta dependant corrections


\bigskip\Keyword{EMPMIN= EMPMAX=}
 Minimum and maximum empirical corrections (usually combination of theta
 and psi or multi-scan for area detectors).


\bigskip\Keyword{DIFABSMIN= DIFABSMAX=}
 Minimum and maximum DIFABS type correction, i.e. based on residue 
 between Fo anf Fc (see section \ref{DIFABS}). In the cif it is called
 a refdelf correction.


\bigskip\Keyword{ABSTYPE=}
 Type of absorption correction. Alternatives are:

\small\begin{verbatim}
      NONE (default) EMPIRICAL    GAUSSIAN     SPHERICAL
      DIFABS         MULTI-SCAN   ANALYTICAL   CYLINDRICAL
      SHELXA         SADABS       NUMERICAL
                     SORTAV       INTEGRATION
                     PSI-SCAN             

\end{verbatim}\normalsize




\bigskip\Directive{GENERAL DOBS= DCALC= F000= MU= MOLWT= FLACK= ESD=}


\bigskip\Directive{GENERAL (continued) ANALYSE-CUT= ANALYSE-NREF=}


\bigskip\Directive{GENERAL (continue)  ANALYSE-R= ANALYSE-RW= SOLUTION=}
 General information, usually provided by CRYSTALS.


\bigskip\Keyword{DOBS= DCALC=}
 Observed density and that calculated by CRYSTALS.


\bigskip\Keyword{F000=}
 Sum of scattering factors at theta = zero.


\bigskip\Keyword{MU=}
 Absorption coefficient, calculated by CRYSTALS.


\bigskip\Keyword{MOLWT=}
 Molecular weight, calculated bt CRYSTALS.


\bigskip\Keyword{FLACK=}


\bigskip\Keyword{ESD=}
 The Flack parameter and its esd, if refined.


\bigskip\Keyword{ANALYSE-CUT=}


\bigskip\Keyword{ANALYSE-NREF=}


\bigskip\Keyword{ANALYSE-R=}


\bigskip\Keyword{ANALYSE-RW=}
      These values are updated when ever $\backslash$ANALYSE is run, and can be used 
 to record the effect of different LIST 28 schemes. {\bf Remenber} that if 
 the LIST 28 conditions are modified to include more reflections than 
 were used in the last $\backslash$SFLS calculation (section \ref{SFLS}), the values 
 of Fc for the 
 additional reflections will be incorrect. A $\backslash$SFLS calculation sets 
these to the same values as in REFINEMENT above.



\bigskip\Keyword{SOLUTION=}
      The program/procedure used for structure solution

\small\begin{verbatim}
      UNKNOWN (Default)
      SHELXS
      SIR88
      SIR92
      PATTERSON
      SIR97
      DIRDIF
\end{verbatim}\normalsize




\bigskip\Directive{COLOUR}
 The crystal colour.


\bigskip\Directive{SHAPE}
 The crystal shape.



\bigskip\Directive{CIFEXTRA}
 These are filled in by the $\backslash$SFLS CALC operation (section \ref{SFLS}).  
 Structure factors are  computed for ALL reflections along with R and Rw - 
 LIST 28 is ignored (LIST 28, reflection filtering, see section \ref{}LIST28). 
 R and Rw are also computed for reflections above a given 
 threshold.



\section{Printing the general information, LIST 30}






\bigskip\Instruction{$\backslash$PRINT 30}




This prints list 30. There is no command for punching LIST 30.







\chapter{Reflection Data Input}


\label{reflectdata} \section{Scope of the Reflection Data Input section of the Reference Manual}






The areas covered are:

\small\begin{verbatim}
 Reflection Data
 Simple input of F or Fsq data               - \LIST 6
 Advanced input of F or Fsq data             - \LIST 6
 Reflection Parameter Coefficients
 Storage of reflection data
 Compressed reflection files
 Intensity data                              - \HKLI
 Standard Decay Curves                       - \LIST 27
 Data Reduction                              - \LP
 Systematic absence removal                  - \SYSTEMATIC
 Sorting data                                - \SORT
 Merging equivalent reflections              - \MERGE
 Theta-dependent absorption correction       - \THETABS
 Analysis of data                            - \THLIM, WILSON, SIGMADIST
\end{verbatim}\normalsize





\index{Reflection data format}
\section{Reflection Data}









\bigskip\Directive{Format of reflection data}
      The reflection data may be embedded into the control data, but it
 is more normal to hold it in a separate file, the HKLI file. This file
 may have one of more reflections per line, or a reflection may span
 several lines. The parameters for each reflection may be in fixed
 format, \emph{i.e.} right adjusted columns, or be in free-format, with
 at least a single space separating items.


If fixed-format input is used, the user must supply a FORTRAN format
 statement. This specifies the width of the input fields, where the
 decimal points are, and any fields to be skipped.  Even though the indices
 are usually integer values, CRYSTALS read them as floating point numbers.
 A FORTRAN 'I' format is automatically changed to an 'F' format.
 Note that if the input figures contain decimal points, these will
 over-ride values given in the format statement.

\small\begin{verbatim}
      Examples - ^ represents a space.

      FORMAT (3F4.0, 2F8.2)      ^^^1^^12^^^3^^^47.23^^^^9.32
      FORMAT (3I4, 2F8.2)        ^^^1^^12^^^3^^^47.23^^^^9.32

      FORMAT (3F4.0, 2F8.0)      ^^^1^^12^^^3^123456.^^312.16

      FORMAT (3F4.0, 3X,2F8.0)   ^^^1^^12^^^3ABC^123456.^^312.16
\end{verbatim}\normalsize




\bigskip\Directive{Termination of reflection data}
 The reflection data themselves should be terminated with  a
 value less than or equal to -512 for the first value on the 
 final input line.



 If the reflections are embedded into the control data, then correct
 termination is {\bf vital.} Incorrect termination may lead to the program
 trying to read commands as reflections, producing massive error files.
 If the reflections are in the HKLI file, most
 implementations will detect the end-of-file and terminate input.








\bigskip\Directive{F or Fsq?}
 CRYSTALS will accept either F or Fsq observations, signed or unsigned.
 Either quantity is referred to by the name 'Fo'. If sigma values
 are given, they must refer directly to the signed input F or Fsq values.
 reflections are stored as Fo, and standard deviations are
 transformed or approximated so that Least-Squares refinement
 can be performed with either F or Fsq independent of input type. Raw
 intensities, I, can be input with the HKLI command. The reflection
 input routines (LIST 6 or HKLI) are the only routine able to take the
 square root of the observation. See the chapter on refinement for a
 brief discussion of the merits of F and FSQ refinements.





\bigskip\Directive{Merged or unmerged data?}
 CRYSTALS supports two levels of merging (averaging) simultaneously.  
 For Fourier syntheses it is important that all symmetry operations of 
 the Laue Group are applied, including Friedel's Law. For refinement it 
 is permitted to used un-merged data, though in general some merging is 
 performed. For non-centrosymmetric structures containing strong
 anomalous scatterers 
 Friedel pairs should be kept separate, but other symmetry operations 
 should be applied. 


The reflection list with the minimal amount of merging is the 
 principal reflection list, LIST 6 (section \ref{LIST06}). This can 
 be used to create 
 a full-merged list for Fourier (or other) calculations, LIST 7 
 (section \ref{LIST07}).
 The user can indicate to most commands which use reflections whether 
 to use LIST 6 or LIST 7, but by default all use LIST 6 for backwards 
 compatibility. 


When working from the menus, CRYSTALS tries to determine whether a LIST 
 6 or a LIST 7 would be most appropriate.  If a LIST 7 is required, the 
 SCRIPT COPY67 is activated to output a LIST 6 as a temporary file, and 
 re-input it as a LIST 7.  This can then be manupulated independently of 
 the original LIST 6.


LIST 7s are currenlty automatically created for:
\small\begin{verbatim}
      Fourier Maps
      Slant Fourier Maps
      Superflip Structure Solution
      Absolute Structure Determination
\end{verbatim}\normalsize




 The experienced or adventurous user can of course use 
 LIST 6 and LIST 7 quite independently for different purposes.



In LIST 7 the ARCHIVE keyword is set to NO, to prevent the original 
archived data from being over-written






\index{LIST 6}
\index{F or Fsq data}
\section{Simple input of F or Fsq data - LIST 6}
\label{LIST06} 



LIST 6 will accept reflection data either as F or Fsq.
 For routine work, a pre-specified set of coefficients

\small\begin{verbatim}
      h k l Fobs sigma(Fobs)
\end{verbatim}\normalsize


 are input and
 stored
 for each reflection.



 {\bf NOTE} that 'Fobs' will refer either to F or Fsq, depending on the
 value of F's.



 The input coefficient list may be expanded for
 non-routine work - see section \ref{ADVANCEDL6} below.

\small\begin{verbatim}
 \LIST 6
 READ F'S=
 FORMAT EXPRESSION=
 END
\end{verbatim}\normalsize



\small\begin{verbatim}
 \ The OPEN command connects the reflection file
 \OPEN HKLI REFLECT.DAT
 \LIST 6
 READ F'S=FSQ
 FORMAT (3F4.0, 2F8.0)
 END
 \ Close the reflection file
 \CLOSE HKLI
\end{verbatim}\normalsize







\bigskip\Directive{READ F'S=}


\bigskip\Keyword{F'S=}
 This parameter is used to indicate whether Fo or Fo**2
 type coefficients are being read in, and must take one of the following
 values :

\small\begin{verbatim}
      FSQ
      FO   -  Default
\end{verbatim}\normalsize


 The default value of  'FO'  indicates that coefficients corresponding
 to Fo are being read in.



 By default, the reflections are assumed to come in fixed format from the HKLI
 channel, and may be terminated either by the end-of-file, or with -512.





\index{COPY 6 7}
\index{Copy List 6 to 7}
\section{Creation of LIST 7 from LIST 6 - COPY 6 7}
\label{LIST07} 


This command creates a LIST 7 as an exact copy of LIST 6 (see \ref{LIST06}).
 The LIST 7 can then be merged using Friedel's Law to create a 
reflection list suitable for Fourier syntheses

\small\begin{verbatim}
 \COPY INPUT= OUTPUT=
 END
\end{verbatim}\normalsize


Example
\small\begin{verbatim}
 \COPY 6 7
 END
\end{verbatim}\normalsize









\section{Printing LIST 6}



The reflections can be output to listing file as follows :


\bigskip\Instruction{$\backslash$PRINT 6 mode}


 Mode controls the type of output.
\small\begin{verbatim}
     A - Default - The reflections are in compressed 
                   format, on the scale of Fo.
     B -           The reflections are in compressed 
                   format, on the scale of Fc.
     C -           A general print of all the data 
                   stored for each reflection.
\end{verbatim}\normalsize


See also $\backslash$REFLECTIONS (section \ref{REFLECTIONS}), which produces tables 
for publication.



\index{Output of reflection data}
\section{Punching LIST 6}



LIST 6 can be punched as an ASCII file in several formats.


\bigskip\Instruction{$\backslash$PUNCH 6 mode}


 Mode controls the format of the output.
\small\begin{verbatim}
      A - Output the reflections in a compressed format - Default.
      B - Output the reflections in 'cif' format. F's
      C - Output stored information in tabulated format.
      D - Output original information in tabulated format.
      E - Output the reflections in 'cif' format. F^s, scale of Fo
      F - Output Fo, sigma information in SHELX format.
      G - SHELX output with statistically generated sigmas
      H - Output the reflections in 'cif' format. F^s, scale of Fc
      I - Outputs all set items as a LIST 6.
      J - Used to convert pre-2016 esds to more precise ones. 

     AF - JF as above with F indicating that LIST 28 filters will be applied
\end{verbatim}\normalsize




LIST 6 is also output by the links to the direct methods programs. In
 these files, the magnitudes of Fo or Fsq are scaled so that the largest
 fits the format statement. The SHELX file contains Fsq, the SIR file
 contains Fo.



\index{Reflection data input - advanced}
\section{Advanced input of F or Fsq data - LIST 6}
\label{ADVANCEDL6} 



LIST 6 will accept reflection data either as F or Fsq.
 The data may be
 in free or fixed format. For routine work, a pre-specified set of
 parameters is stored for each reflection. This may be expanded for
 non-routine work by {\bf INPUT} and {\bf OUTPUT} coefficients.

\small\begin{verbatim}
 \LIST 6
 READ NCOEFFICIENT= TYPE= F'S= NGROUP= UNIT= CHECK= ARCHIVE=
 INPUT COEFFICIENT(1)= COEFFICIENT(2)= .  .
 STORE NCOEFFICIENT= MEDIUM= APPEND=
 OUTPUT COEFFICIENT(1)= COEFFICIENT(2)=  .  .
 FORMAT EXPRESSION=
 MULTIPLIERS VALUE=
 MATRIX M11=  M12=  ... M33= TOLER= TWINTOLER=
 END
\end{verbatim}\normalsize



\small\begin{verbatim}
 \ The OPEN command connects the reflection file
 \OPEN HKLI REFLECT.DAT
 \LIST 6
 READ NCOEF=5 TYPE=FIXED UNIT=HKLI F'S=FSQ
 FORMAT (3F4.0, 2F8.2)
 INPUT H K L /FO/ SIGMA(/FO/)
 STORE NCOEF=7
 OUTPUT INDICES /FO/ SQRTW /FC/ BATCH/PHASE RATIO/JCODE SIGMA(/FO/)
 END
 \CLOSE HKLI
\end{verbatim}\normalsize







\bigskip\Directive{READ NCOEFFICIENT= TYPE= F'S= NGROUP= UNIT= CHECK= ARCHIVE=}





\bigskip\Keyword{NCOEFFICIENT= }
 Specifies the number of coefficients to be input per reflection. A list
 of permitted coefficients is given below. If this directive is omitted,
 the default is 5.



 The default input coefficients are

\small\begin{verbatim}
      H K L FOBS SIGMA(F)
\end{verbatim}\normalsize




\bigskip\Keyword{TYPE=}
 This parameter determines the form of the reflections as they are read
 in, and must take one of the following values :

\small\begin{verbatim}
      FIXED     -  Fixed format data
      FREE      -  Free format text  -  default value
      COMPRESSED-  See 'Compressed Reflection Data' below
      COPY      -  LIST 6 is copied from the current input device to the
                   output device designated on the STORE directive with
                   the number of coefficients given on the OUTPUT and
                   COEFFICIENT directives.
\end{verbatim}\normalsize




\bigskip\Keyword{F'S=}
 This parameter is used to indicate whether Fo or Fo**2
 type coefficients are being read in, and must take one of the following
 values :

\small\begin{verbatim}
      FSQ
      FO   -  Default value
\end{verbatim}\normalsize


 The default value of  'FO'  indicates that coefficients corresponding
 to Fo are being read in.


\bigskip\Keyword{NGROUP=}
 This parameter defines the number of reflections per line for fixed
 format input. (For free format input, the system
 can work out this information).
 NGROUP will be less than unity if the reflection spans several lines.


\bigskip\Keyword{UNIT=}
 This parameter defines the source of the reflection data that are
 to be input.

\small\begin{verbatim}
      HKLI   -  Default value.
      DATAFILE
\end{verbatim}\normalsize





 HKLI  indicates that the reflection data are in a separate file from
 the main input data.
 The local implementation may set up default names for this file, or
 the $\backslash$OPEN directive can be used to connect the file to CRYSTALS.



  DATAFILE  indicates that the reflections follow the directives for
  '$\backslash$LIST 6' in the normal data input stream.
 If this is the case, the directives for  $\backslash$LIST6  {\bf must} be terminated
 by the directive  END, otherwise the reflection lines will be
 processed as normal directives associated with the  $\backslash$LIST6
 command, and generate a very large number of input errors.



 By default, the data are assumed to come from the alternative HKLI
 channel.


\bigskip\Keyword{CHECK}
 This parameter determines whether reflections are rejected on input
 if they have a zero or negative value for Fo.

\small\begin{verbatim}
      YES
      NO -  Default value.
\end{verbatim}\normalsize


 By default checking is disabled so that negative reflections are
 accepted on input.





\bigskip\Keyword{ARCHIVE}
This parameter controls the creation of a file ARCHIVE-HKL.CIF, an  
image of the data in 3i4,2f8.2 (SHELX) format. 

\small\begin{verbatim}
      YES -  Default value.
      NO 
\end{verbatim}\normalsize


The default value is YES except when a COPY is being performed. The 
ARCHIVE-HKL.CIF can be embedded in the final publish.cif file.








\bigskip\Directive{INPUT COEFFICIENT(1)= COEFFICIENT(2)= .  .}


This directive defines the coefficients that are to be read in.
 The number of coefficients is given by the  NCOEFFICIENT  parameter
 above, or its default value.


\bigskip\Keyword{COEFFICIENT(1)= COEFFICIENT(2)= }
 Defines the coefficients and their input order. The coefficients must be
 selected from the following list

\small\begin{verbatim}
      H             K             L             /FO/
      SQRTW         FCALC         PHASE         A-PART
      B-PART        TBAR          FOT           ELEMENTS
      SIGMA(F)      BATCH         INDICES       BATCH/PHASE
      SINTH/L**2    FO/FC         JCODE         SERIAL
      RATIO         THETA         OMEGA         CHI
      PHI           KAPPA         PSI           CORRECTIONS
      FACTOR1       FACTOR2       FACTOR3       RATIO/JCODE
\end{verbatim}\normalsize


 For the meaning of these coefficients, see section \ref{REFPARCOEF} - 
 'Reflection Parameter Coefficients'



 {\bf NOTE} that 'Fobs' will refer either to F or Fsq, depending on the
 value of F's. Reflections are available during refinement as either signed
 Fsq or signed Fo independent of the type of input values.








\bigskip\Directive{STORE NCOEFFICIENT= MEDIUM= APPEND=}


\bigskip\Keyword{NCOEFFICIENT= }
 Specifies the number of coefficients to be stored per reflection. A list
 of permitted coefficients is given above. If this directive is omitted,
 the default is 9.



 The default output coefficients are

\small\begin{verbatim}
      INDICES /FO/ SQRTW /FC/ BATCH/PHASE RATIO/JCODE SIGMA(/FO/)
      CORRECTIONS ELEMENTS
\end{verbatim}\normalsize




\bigskip\Keyword{MEDIUM}
 This parameter sets the output reflection storage device. This can be a
 text file, but more normally it is the database, the '.dsc' file. 
 See section \ref{STOREREF} - 'Storage of Reflection Data'.

\small\begin{verbatim}
      FILE        A named or scratch ASCII serial file
      INPUT       A file of the same type as the input reflection source
      DISK   -    Default - The current structure database
\end{verbatim}\normalsize




\bigskip\Keyword{APPEND=}
 This parameter determines whether the input reflections are to replace
 or be appended to existing reflections.
\small\begin{verbatim}
      YES      The input reflections are appended to existing reflections
      NO   -   Default - The input reflections replace any existing reflections
\end{verbatim}\normalsize







\bigskip\Directive{OUTPUT COEFFICIENT(1)= COEFFICIENT(2)= .  .}


This directive defines the coefficients that are to be stored.
 The number of coefficients is given by the  NCOEFFICIENT  parameter
 above, or its default value, and the coefficients selected from the list
 above.



 If the  OUTPUT  directive is omitted, as many of the default
 coefficients as are required by NCOEFFICIENT  are used as output
 coefficients :



 If the  OUTPUT  directive is omitted and  NCOEFFICIENT  is greater than 9,
 it is reset to 9 so that the coefficients above can be used.








\bigskip\Directive{FORMAT EXPRESSION=}


This directive allows the user to define a format statement if
 fixed format input is being used.
 This directive is only valid if the  TYPE  parameter on the  READ  
 directive is  FIXED .


\bigskip\Keyword{EXPRESSION=}
 This parameter defines the format to be used.
 Normally this keyword is omitted, so that the directive looks
 like a FORTRAN format statement, except that there must be at least
 one space between the  'FORMAT' and the expression, to terminate the
 directive.
 Since all the data are read as real numbers, the format expression
 can only contain F , E , and X field definitions - either find a good
 Fortran reference book for examples, or ask someone who did 
 crystallography before 1990.








\bigskip\Directive{MULTIPLIERS VALUE=}


This directive allows the user to define the multipliers to
 be applied to the data if they are being read in compressed format.
 This directive is only valid if the  TYPE parameter on the  READ  directive 
 is COMPRESSED .


\bigskip\Keyword{VALUE=}
 This parameter, whose default value is unity,
 is repeated the number of times specified by the  NCOEFFICIENT  parameter
 on the  READ  directive.
 The order is the same as the  INPUT  coefficients.





\bigskip\Directive{MATRIX M11= M12= ...M33= TOLER= TWINTOLER=}


This directive inputs a matrix to be applied to the reflection indices
as they are read in. If any component of the index differs by more than
TOLER from an integer, the reflection is rejected. TWINTOLER is a value,
in A-2, for overlap of potentially twinned reflections. See the chapter
on twinning (\ref{twinning}).


\bigskip\Keyword{Mij=}
 The 9 elements (by row) of an index transformation matrix. The default
is a unit matrix


\bigskip\Keyword{TOLER=}
 The reflection is rejected if any transformed index differs from an
integer by more than TOLER. The default is 0.1.


\bigskip\Keyword{TWINTOLER=}
 The twin element tag is updated if the generated reciprocal lattice
point differs from a base lattice point by less than TWINTOLER
reciprocal Angstrom. The default is 0.001, but an ideal value will depend
upon the integration method, the mosaicity, and the lengths of the cell
edges.









\section{Reflection Parameter Coefficients}
\label{REFPARCOEF} 



CRYSTALS has a very flexible procedure for storing reflection
 information, enabling the user to optimise disk space use. The user
 must indicate to the program what information is available in the input
 data, and what information is to be stored. Storage space may also be
 reserved for data yet to be computed.


During data reduction (section \ref{DATAREDUC}), space is reserved for 
 relevant coefficients.
 These coefficients (\emph{e.g.} setting angles) may not be needed during
 structure analysis, so they are not normally preserved beyond reduction.





\bigskip\Instruction{Special Reflection storage}




The user might need to arrange special reflection storage under the
 following conditions:





\bigskip\Directive{Refinement using a partial model}


If the user is experiencing difficulties with a small part of an otherwise
 well behaved large structure, the real and imaginary parts of the
 structure factors due to the well behaved part can be precomputed and
 stored and these atoms removed from the atom list (LIST 5).
 The user then only needs recompute the contributions
 from the varying fragment. The total Fo, Fc, real and imaginary parts are
 stored with the keys
\small\begin{verbatim}
      /FO/      /FC/      APART      BPART
\end{verbatim}\normalsize







\bigskip\Directive{Twinned structures}


See chapter \ref{twinning} on handling twinned data.






\bigskip\Instruction{Recognised reflection coefficients}





\small\begin{verbatim}
 Coefficients recognised are:
 H            Reflection index h
 K            Reflection index k
 L            Reflection index l
 INDICES      Packed reflection indices
 /FO/         The observed intensity, Fsq or Fo value
 /FOT/        The observed intensity, Fsq or Fo value for a twinned crystal
 /FC/         The calculated structure factor
 SIGMA(/FO/)  Standard deviation of the input observation
 SQRTW        Sqrt of weight to be given a reflection during least squares
 A-PART       Real part of structure factor
 B-PART       Imaginary part of structure factor
 PHASE        Phase angle, radians
 BATCH        An integer associated with reflections measured in batches
 BATCH/PHASE  Packed (compressed into one word) Batch and Phase
 SINTH/L**2   (Sintheta/lambda)**2
 FO/FC        Fo/Fc
 ELEMENTS     Integers corresponding to twin elements
 SERIAL       Serial number of reflection
 JCODE        reflection quality code. See below
 RATIO        Ratio Fo**2/sigma(Fo**2)
 RATIO/JCODE  Packed ratio and jcode
 TBAR         Absorption weighted X-ray path length
 THETA        Bragg angle
 OMEGA        Setting angle
 CHI          Setting angle
 PHI          Setting angle
 KAPPA        Setting angle
 PSI          Setting angle
 CORRECTIONS  Composite correction factor for Fo
 FACTOR1      Individual correction factor for Fo
 FACTOR2      Individual correction factor for Fo
 FACTOR3      Individual correction factor for Fo
 NOTHING      A spare location for programmers use
\end{verbatim}\normalsize




If an output coefficient is specified without the corresponding input
 coefficient, it value is set to zero except for BATCH (default is 1.0)
 and SINTH/L**2  (value computed from cell parameters). Packed INDICES are
 restricted to +/- 127, packed RATIO to range 0.0 - 999.0, JCODE to range
 0 - 9. 


JCODE valuse assigned by RC93 for MACH3 data are

\small\begin{verbatim}
            1     normal reflection
            9     weak reflection
            7     flagged strong S but not flagged D
            2     deviates from expected position/peak shape, but not W
            3     failed non-equal test at least once
            6     flagged weak
            4     reflection is bad
            8     flagged strong T but not flagged D
     The order of comparisons corresponds to the order of likelihood of
     having a particular code.
\end{verbatim}\normalsize





\section{Storage of reflection data}
\label{STOREREF} 



Reflections may be stored either in the structure database (the DSC
 file), or as external binary serial files. The latter is used mainly
 during data reduction (section \ref{DATAREDUC}).


When a change is made to most other
 data lists, they are either completely overwritten (LIST1, cell
 parameters), or a new list created in addition to the old list (LIST 5,
 atom parameters). Because the reflections are special, they are handled
 differently. A small piece of information (called the LIST 6 Header) is
 created to hold information about the rest of the reflection list,
 and new headers are stored each time the main body is updated. The main
 body of the reflection list is modified in-situ if the only changes are
 ones which can easily be recomputed ( \emph{e.g.} Fc, phase, sqrtw), thus
 reducing the disk activity. If an error occurs during the updating of
 the body, the list becomes inaccessible to other processes, and the
 failing process must be re-run correctly.
 If the changes involve a change in size of
 the list, then a new body is created.


During raw data processing (Data reduction, section \ref{DATAREDUC}) the 
 size of the reflection
 list can change a lot (coefficients being added or removed, reflections
 being merged or rejected). To prevent the .DSC file growing too large,
 binary serial files are used to hold the body of the reflection list.
 One is used for input and one for output at each stage, the roles being
 reversed after each stage. The header is kept in the .DSC file, and
 keeps track of the bodies. When data reduction is complete, the body
 must be copied to the .DSC file as follows:

\small\begin{verbatim}
 \  After data reduction, make a final copy of the reflections
 \  and STORE THEM IN THE .DSC FILE:
 \LIST 6
 READ TYPE=COPY
 END
\end{verbatim}\normalsize





\section{Compressed reflection data}



CRYSTALS can produce files containing reflections is a 'compressed'
 format. This might be useful for archiving data. The compressed data is
 headed by the correct information for its reinput.


The file contains information for  h, k, l, /FO/ or /FOT/, RATIO/JCODE
 and elements.
 For each  KL  pair, the  K  value is given for this group of
 reflections, then the  L  value for the group, followed by
 the  H  and /FO/ and other values for the first reflection, the  H   /FO/
 and other values for
 the second reflection, and so on, finishing with 512, which is the
 terminator for this  KL  pair.
 This pattern is repeated for all the  KL  pairs, the terminator
 for the last  KL  pair being -512, and indicates the end of the reflection
 list. Take care if you try to edit these files, and
 note that  K  and  L  are the two constant indices for each group,
 while  H  changes most rapidly.






\section{Intensity Data - HKLI}



Raw intensity data require more processing than F or Fsq
 values. The instruction '$\backslash$HKLI' is related to '$\backslash$LIST 6', but has
 different default coefficients and additional directives for geometrical
 corrections.


\small\begin{verbatim}
 \HKLI
 READ NCOEFFICIENT= TYPE= F'S= NGROUP= UNIT= CHECK= ARCHIVE=
 INPUT COEFFICIENT(1)= COEFFICIENT(2)= .  .
 STORE NCOEFFICIENT= MEDIUM= APPEND=
 OUTPUT COEFFICIENT(1)= COEFFICIENT(2)=  .  .
 FORMAT EXPRESSION=
 CORRECTIONS NSCALE NFACTOR
 FACTORS COEFFICIENT(1)= COEFFICIENT(2)=  .  .
 ABSORPTION PRINT= PHI= THETA= TUBE= PLATE=
 PHI NPHIVALUES= NPHICURVES=
 PHIVALUES PHI= .........
 PHIHKLI H= K= L= I[MAX]=
 PHICURVE I= .........
 THETA NTHETAVALUES=
 THETAVALUES THETA=
 THETACURVE CORRECTION= ........
 TUBE NOTHING OMEGA= CHI= PHI= KAPPA= MU=A[MAX]=
 PLATE NOTHING OMEGA= CHI= PHI= KAPPA= MU=A[MAX]=
 END
\end{verbatim}\normalsize


For example

\small\begin{verbatim}
 \  The OPEN command connects the reflection file:
 \OPEN HKLI REFLECT.DAT
 \  The HKLI instruction reads the data in:
 \HKLI
 \  There are 12 items to read:
 READ NCOEF=12 FORMAT=FIXED UNIT=HKLI F'S=FSQ CHECK=NO
 \  This is what they are:
 INPUT H K L /FO/ SIGMA(/FO/) JCODE SERIAL BATCH THETA PHI OMEGA KAPPA
 \  And this is their format:
 FORMAT (5X,3F4.0,F9.0,F7.0,F4.0,F9.0,F4.0,4F7.2)
 \  We only want to store six of them:
 STORE NCOEF=6
 \  Specifically, these ones:
 OUTPUT INDICES /FO/ BATCH RATIO/JCODE SIGMA(/FO/) CORRECTIONS SERIAL
 \  Some absorption corrections have been measured:
 ABSORPTION PHI=YES  THETA=YES PRINT=NONE
 \  Here is the theta dependent absorption curve:
 THETA 16
 THETAVALUES
 CONT 0  5 10 15 20 25 30 35 40 45 50 55 60 65 70 75
 THETACURVE
 CONT 3.61  3.60  3.58  3.54  3.50  3.44  3.37  3.30
 CONT 3.23  3.16  3.09  3.02  2.96  2.91  2.86  2.82
 \  And here is one azimuthal absorption curve containing 26 points:
 PHI 26  1
 PHIVALUES
 CONT   6  16  21  26  31  36  41  61  66  76
 CONT  81  86  91  96 111 121 131 136 141 146
 CONT 151 156 161 166 171 176
 \  This is the reflection we used for the scan:
 PHIHKLI    -3   -1    0    28392
 PHICURVE
 CONT    26887   25377   24608   23990   23445   23049
 CONT    22867   22801   22782   22937   23104   23368
 CONT    23713   24129   25669   26836   27892   28250
 CONT    28291   28256   28101   28009   28204   28373
 CONT    28392   28203
 END
 \  All done. Close the hkl file.
 \CLOSE HKLI
\end{verbatim}\normalsize


 In the following description,
 for items defined under LIST 6 above only the default value will be given.








\bigskip\Instruction{$\backslash$HKLI}







\bigskip\Directive{READ NCOEFFICIENT= TYPE= F'S= NGROUP= UNIT= CHECK= ARCHIVE=}
 This directive is the same as the READ directive in $\backslash$LIST 6 above, 
 except that the following parameters have
 different default values:

\small\begin{verbatim}
 NCOEFFICIENT= default value is 12
 TYPE= default value is FIXED
 F'S= default value is FSQ
 NGROUP= default value is 1
 UNIT= default value is HKLI
\end{verbatim}\normalsize








\bigskip\Directive{INPUT COEFFICIENT(1)= COEFFICIENT(2)= .  .}


This directive defines the coefficients that are to be read in.
 The number of coefficients is given by the  NCOEFFICIENT  parameter
 above, or its default value.



 The default input coefficients are (i.e. for RC93 output):

\small\begin{verbatim}
      H K L /FO/ SIGMA(/FO/) JCODE SERIAL BATCH THETA PHI OMEGA KAPPA
\end{verbatim}\normalsize







\bigskip\Directive{STORE NCOEFFICIENT= MEDIUM= APPEND=}


\bigskip\Keyword{NCOEFFICIENT=}
The number of coefficients that will appear on the OUTPUT directive.
The default is 9.


\bigskip\Keyword{MEDIUM=}
 The default value is 'FILE'. Since the reflections will be much 
 changed during data reduction 
 (section \ref{DATAREDUC}), the
 intermediate storage is usually a scratch serial file.


\bigskip\Keyword{APPEND=}
The default value is 'NO'.





\bigskip\Directive{OUTPUT COEFFICIENT(1)= COEFFICIENT(2)= .  .}
 The default coefficients are:
\small\begin{verbatim}
      INDICES /FO/ SQRTW /FC/ BATCH/PHASE RATIO/JCODE SIGMA(/FO/)
      CORRECTIONS ELEMENTS
\end{verbatim}\normalsize


Note that H, K, and L are compressed into one key: 'INDICES'.





\bigskip\Directive{FORMAT EXPRESSION=}


This directive is only valid if the  TYPE  parameter on the  
 READ  directive is  FIXED.


\bigskip\Keyword{EXPRESSION=}
 If the diffractometer type indicated in LIST 13 (section \ref{LIST13})
 is CAD4, the default
 corresponds to RC93 or RC85 output, otherwise an expression must be
 given.
\small\begin{verbatim}
          e.g. (5X,3F4.0,F9.0,F7.0,F4.0,F9.0,F4.0,4F7.2)
\end{verbatim}\normalsize











Directives found in HKLI commands, but not in LIST 6 commands are:





\bigskip\Directive{CORRECTIONS NSCALE= NFACTOR=}



\bigskip\Keyword{NSCALE=}
 Set to 1 or 2 to select the first or second scale factor in LIST 27
 (see section \ref{LIST27}).



 The default is 2.


\bigskip\Keyword{NFACTOR=}
 Up to three correction per reflection to be applied to the input
 observations can be included in the input file. This keyword specifies
 how many to use.



 The default is 0.





\bigskip\Directive{FACTORS COEFFICIENT(1)= COEFFICIENT(2)= .  .}
 The permitted coefficients are FACTOR1, FACTOR2 and FACTOR3. These are
 applied to the input observation before any other action (including
 square rooting if requested) is performed.





\bigskip\Directive{ABSORPTION PRINT= PHI= THETA= TUBE= PLATE=}
 This directive controls approximate absorption corrections to be applied
 during input. They are only suitable if the diffractometer used is one
 of those permitted in LIST 13 (section \ref{LIST13}).


\bigskip\Keyword{PRINT=}
 Permitted levels are
\small\begin{verbatim}
      FULL                  Two lines of information per reflection
      NONE    - Default     No output is produced
      PARTIAL -             Summary for each reflection
\end{verbatim}\normalsize




\bigskip\Keyword{PHI=}
\small\begin{verbatim}
      NO      - Default
      YES
\end{verbatim}\normalsize


 If YES, then phi (azimuthal scan) data must follow.


\bigskip\Keyword{THETA=}
\small\begin{verbatim}
      NO      - Default
      YES
\end{verbatim}\normalsize


 If YES, then a theta dependent correction curve must follow.


\bigskip\Keyword{TUBE=}
\small\begin{verbatim}
      NO      - Default
      YES
\end{verbatim}\normalsize


 If YES, then orientation angles for the tube must follow.


\bigskip\Keyword{PLATE=}
\small\begin{verbatim}
      NO      - Default
      YES
\end{verbatim}\normalsize


 If YES, then orientation angles for the plate must follow.








\bigskip\Directive{PHI NPHIVALUES= NPHICURVES=}
 If phi has been set to 'YES' above,
 this directive sets up input and computation of azimuthal scan
 absorption corrections, by the method of North, Phillips and Mathews,
 Acta Cryst., {\bf A24,} 351 (1968).


\bigskip\Keyword{NPHIVALUES=}
 Number of sampling points on the phi curve. These need not be equally
 spaced


\bigskip\Keyword{NPHICURVES=}
      Number of phi curves that will be entered after this directive.





\bigskip\Directive{PHIVALUES PHI= .....}
 The 'Nphivalue' phi angles  of the points on the absorption curve.





\bigskip\Directive{PHIHKLI H= K= L= I[MAX]=}
 The h,k,l and Imax values for the following 'Nphicurve' phi profiles,
 in the same order as the profiles.





\bigskip\Directive{PHICURVE I= .....}
 The 'Nphivalue' intensity values for the profile  at the phi values
 given on the Phivalues directive. There is a Phicurve corresponding to
 each PHIHKLI directive.





\bigskip\Directive{THETA NTHETAVALUES=}
 If theta has been set to 'YES' above this directive sets up the input
 for and computation of a theta dependent absorption correction. Except
 when the data has been corrected by a proper analytical correction,
 a theta dependent correction is {\bf ALWAYS} recommended, since neither
 a phi scan, multi-scan nor DIFABS (section \ref{DIFABS}) will make a 
 good theta 
 approximation. See Int Tab,
 Vol II, p295 and 303 for suitable profiles.


\bigskip\Keyword{NTHETAVALUES=}
 The number of sampling points on the theta curve.





\bigskip\Directive{THETAVALUES THETA= .....}
 The Nthetavalues at which the curve is sampled





\bigskip\Directive{THETACURVE CORRECTION= ......}
 The Nthetavalue values of the correction factor profile.





\bigskip\Directive{TUBE NOTHING OMEGA= CHI= PHI= KAPPA= MU A[MAX]}
 If TUBE has been set to 'YES' above, this directive sets up the
 correction for a sample in a tube, or for an acicular crystal steeply
 inclined to the phi axis. See J. Appl. Cryst, 8. 491, 1975. 'NOTHING' is
 a place-holder for internal workings.


\bigskip\Keyword{OMEGA= CHI= PHI= KAPPA=}
 These are the settings needed to bring the tube axis into the
 equatorial plane and perpendicular to the incident X-ray beam. Only one
 of Chi and Kappa may be given.


\bigskip\Keyword{MU=}
 The product of Mu and the thickness of the tube wall.


\bigskip\Keyword{A[MAX]}
 The maximum permitted correction. Values greater than A[max] generate a
 warning.





\bigskip\Directive{PLATE NOTHING OMEGA= CHI= PHI= KAPPA= MU A[MAX]}
 If PLATE has been set to 'YES' above, this directive sets up the
 correction for an extended plate-like sample. See J. Appl. Cryst, 8. 491,
 1975. 'NOTHING' is
 a place-holder for internal workings.


\bigskip\Keyword{OMEGA= CHI= PHI= KAPPA=}
 These are the settings needed to bring the plate normal into the
 equatorial plane and perpendicular to the incident X-ray beam. Only one
 of Chi and Kappa may be given.


\bigskip\Keyword{MU=}
 The product of Mu and the plate thickness.


\bigskip\Keyword{A[MAX]}
 The maximum permitted correction. Values greater than A[max] generate a
 warning.






\index{LIST 27}
\index{Intensity decay curves}
\section{Intensity Decay Curves  $\backslash$LIST 27}
\label{LIST27} 


\small\begin{verbatim}
 \LIST 27
 READ NSCALE=
 SCALE SCALENUMBER= RAWSCALE= SMOOTHSCALE= SERIAL=
 END
\end{verbatim}\normalsize




If each reflection has been assigned a serial number (or some other
 incrementing value, such as total X-ray exposure time) then CRYSTALS can
 apply a correction which is linked to this value. The corrections, on the
 scale of Fsq, are held in LIST 27. Two correction factors can be stored,
 but only one used. For example, these can be the actual corrections computed
 from the decay of the standard reflections, and those obtained from
 a 3-point smoothing of the same correction data.
 The applied scale factor is obtained by interpolating between those given
 scale factors with serial numbers above and below the serial number of
 the current reflection.
 If there is a dramatic change in scale (for example due to remeasurement
 of some very strong reflections with attenuated X-rays), it is important
 not to interpolate over this discontinuity. To achieve this, a dummy
 scale factor is inserted at this point with scale values the same as the
 current scales, but with the same serial number as the first scales
 after the discontinuity - for example:

\small\begin{verbatim}
 \LIST 27
 READ NSCALE=16
 SCALE   1      1.000  1.000    1
 SCALE   2      1.066  1.066    4
 SCALE   3      1.074  1.053   57
 SCALE   4      0.997  1.018   83
 SCALE   5      1.003  1.003  564
 SCALE   6      0.370  0.370  564
 SCALE   7      0.372  0.371  617
 END
\end{verbatim}\normalsize







\bigskip\Instruction{$\backslash$LIST 27}







\bigskip\Directive{READ NSCALE=}


\bigskip\Keyword{NSCALE=}
 The number of  SCALE directives to follow.
 There is no default value for this parameter.





\bigskip\Directive{SCALE SCALENUMBER= RAWSCALE= SMOOTHSCALE= SERIALNUMBER=}


This directive is repeated once for each scale factor that is
 to be read in.


\bigskip\Keyword{SCALENUMBER=}
 This parameter indicates the number of the scale factor, starting
 from one.  There is no default for this parameter, which currently is
 not used.


\bigskip\Keyword{RAWSCALE=}
 This parameter gives the initial scale factor, computed directly
 from the intensities of the standard reflections.



 There is no default.


\bigskip\Keyword{SMOOTHSCALE=}
 This parameter gives the scale factor after the raw scale factors have been
 smoothed, so that a continuous curve is fitted to all the data.



 There is no default.


\bigskip\Keyword{SERIALNUMBER=}
 This parameter gives the serial number of the first standard reflection
 contributing to this scale.
 The data reduction programs use the  SERIAL to locate the correct
 scales to use for a given reflection.



 There is no default.






\section{Printing the decay curve}






\bigskip\Instruction{$\backslash$PRINT 27}




This command prints the decay curve.
 There is no command to punch LIST 27.






\index{LP}
\index{Data reduction}
\section{Data Reduction - Lp}
\label{DATAREDUC} 



This command causes the Lp correction to be calculated for each
 reflection.


The diffraction geometry, wavelength, etc. are taken from
 LIST 13 (section \ref{LIST13}).
 If LIST 13 is input incorrectly, or has to be generated by the system,
 the message 'illegal diffraction geometry flag' will be output
 and the job terminated. If the user has forced the storage of Fsq
 values in $\backslash$HKLI, it is necessary to indicate this to the Lp correction.

\small\begin{verbatim}
 \LP
 STORE MEDIUM= F'S=
 END
\end{verbatim}\normalsize


For example
\small\begin{verbatim}
 \  Apply an LP correction for the geometry stored
 \  in List 13.
 \LP
 END
\end{verbatim}\normalsize







\bigskip\Instruction{$\backslash$LP}







\bigskip\Directive{STORE MEDIUM= F'S=}


\bigskip\Keyword{MEDIUM=}
      Determines the output medium.

\small\begin{verbatim}
      FILE              A serial file
      INPUT - Default   The same as the input medium
      DISC              The .DSC file.
\end{verbatim}\normalsize


 The default output medium is the same as te input medium -  usually a
 serial file.


\bigskip\Keyword{F'S=}

\small\begin{verbatim}
      FO      -      Default
      FSQ            Indicating that square roots were not taken at
                     input time.
\end{verbatim}\normalsize





\index{SYSTEMATIC}
\index{Removing systematic absences}
\section{Systematic absence removal - $\backslash$SYSTEMATIC}
\label{SYSTEMATIC} 


\small\begin{verbatim}
 \SYSTEMATIC INPUTLIST=
 STORE MEDIUM= F'S= NEWINDICES=  FRIEDEL=
 END
\end{verbatim}\normalsize


For example:
\small\begin{verbatim}
 \  Remove systematic absences and move each hkl index
 \  by symmetry so that they all lie in the same part of
 \  the reciprocal lattice:
 \SYST
 END
\end{verbatim}\normalsize




This routine uses the symmetry operators in LIST 2 (section \ref{LIST02})
 to identify
 systematic absences, which are listed and rejected. It can also
 use the symmetry operators to transform indices to that the reflections
 fall into a unique part of the reciprocal lattice. The unique set is
 bounded by the maximum range in 'l', maximum range of 'k' given the 'l'
 range, and maximum range of 'h', given the 'k,l' range.



Friedel's Law may be invoked, depending on the flag in 
 LIST 13 (section \ref{LIST13}).
 It is important {\bf NOT} to use Friedel's Law for structures which have
 strong anomalous scatterers,
 since reflections related by Friedel's law are not
 equivalent in this case and should not be merged together.
 Similarly, if orientation dependent corrections are to be made (\emph{e.g.}
 DIFABS),original indices should be preserved. Note that in this case,
 only exactly equivalent reflections will be merged, and care must be
 taken when computing Fourier maps. See the sections on Fourier maps, 
  and DIFABS  \ref{FOURIER}RDIFABS

.  


If FRIEDEL is set to "YES", Friedel's law is applied whatever the 
 space group.  A flag is set in the 'JCODE' slot for each reflection 
 to indicate if the law was invoked or not. 
 DO NOT USE excluded JCODES IN $\backslash$MERGE. If the data is then
 sorted but not merged, Friedel pairs will be adjacent and flagged. Used
 by the function "$\backslash$TON" for evaluating absolute configuration.








\bigskip\Instruction{$\backslash$SYSTEMATIC}







\bigskip\Directive{SYSTEMATIC INPUTLIST=}





\bigskip\Keyword{INPUTLIST=}
     6 OR 7





\bigskip\Directive{STORE MEDIUM= F'S= NEWINDICES=}


\bigskip\Keyword{MEDIUM=}
      Determines the output medium.

\small\begin{verbatim}
      FILE              A serial file
      INPUT - Default   The same as the input medium
      DISC              The .DSC file.
\end{verbatim}\normalsize


 The default output medium is the same as the input medium -  usually a
 serial file.





\bigskip\Keyword{F'S=}

\small\begin{verbatim}
      FO      -      Default
      FSQ            Indicating that square roots were not taken at
                     input time.
\end{verbatim}\normalsize







\bigskip\Keyword{FRIEDEL=}

\small\begin{verbatim}
      No      -      Default
      YES            Indicating that Friedels law should be applied 
                     whatever the space group.
\end{verbatim}\normalsize




\bigskip\Keyword{NEWINDICES=}
 Determines whether new indices are computed.

\small\begin{verbatim}
      YES  -  Default - Permits transformation of indices.
      NO
\end{verbatim}\normalsize





\index{Sorting reflection data}
\section{Sorting of the reflection data  -  $\backslash$SORT}
\label{SORT} 


\small\begin{verbatim}
 \SORT INPUTLIST=
 STORE MEDIUM=
 END
\end{verbatim}\normalsize


For example:
\small\begin{verbatim}
 \  Sort reflections into order by L, then K, then H:
 \SORT
 END
\end{verbatim}\normalsize




This routine sorts the data so that the reflections are placed
 in a predetermined order, in which reflections with the same indices
 are adjacent in the list. Upon output, the reflections are arranged
 so that they are in groups of constant  L, starting with the group
 with the smallest  L  value. Within any  L  group, the reflections
 are ordered in groups of constant  K, starting with the group with the
 smallest  K  value. Within each  group of constant K and L, the 
 reflections are arranged
 with the smallest  H  value first and the largest last in ascending
 order.


The method of sorting is a multi-pass tree sort, in which as
 many reflections as possible are held in memory during each pass.
 If all the reflections with a given value of  L  cannot be in memory
 at the same time, the program will terminate in error.





\bigskip\Instruction{$\backslash$SORT}







\bigskip\Directive{STORE MEDIUM}


\bigskip\Keyword{MEDIUM=}
      Determines the output medium.

\small\begin{verbatim}
      FILE              A serial file
      INPUT - Default   The same as the input medium
      DISC              The .DSC file.
\end{verbatim}\normalsize


 The default output medium is the same as the input medium -  usually a
 serial file.






\index{Merging reflection data}
\index{MERGE}
\section{Merging equivalent reflections  -  $\backslash$MERGE}


\small\begin{verbatim}
 \MERGE INPUT= TWINNED=
 STORE MEDIUM=
 REFLECTIONS NJCODE= LIST= LEVEL= F'S=
 JCODE NUMBER= VALUE=
 REJECT RATIO= SIGMA=
 WEIGHT SCHEME= NPARAMETERS= NCYCLE=
 PARAMETERS P .....
 END
\end{verbatim}\normalsize


For example:

\small\begin{verbatim}
 \MERGE
 WEIGHT SCHEME=2 NPARAM=6
 PARAMETERS .5  3.0  1.0  2.0  .01 .00001
 END
\end{verbatim}\normalsize




The merge routine takes a list of
 reflections and combines groups of adjacent reflections with
 \emph{exactly} the same indices to
 produce a single mean structure amplitude.


$\backslash$SYST (section ) and $\backslash$SORT (section  \ref{SYSTEMATIC}RSORT

) produce 
 a suitable list, and if
 either of them have been omitted, it is extremely
 likely that the list of reflections produced by the merge process
 will contain duplicated entries for certain reflections.


It is possible to combine equivalent
 reflections in several different ways, depending upon how each
 individual contributor is weighted when the mean is computed.
 Several different weighting schemes are provided, and these are described
 in the next section (the  WEIGHT  directive).


The JCODE key in the list of reflections may be input from some
 diffractometers (e.g. a CAD4) to indicate that the value may be
 inaccurate. Reflections which have JCODES that differ from unity 
 are thought
 to be inaccurate and can be down-weighted or eliminated during the merge
 process (the  JCODE  directive). Note that JCODES MUST be positive
 and less than 10.


Although under normal circumstances LIST 6 (reflections) contains /Fo/
 data rather than /Fo/**2 data, the calculations performed during
 the merge are done on the scale of /Fo/**2.
 This means that r-values are computed which refer to /Fo/**2,
 and that reflections can be rejected on the basis of the
 ratio of /Fo/**2 to its standard deviation.
 If for some reason the LIST 6 contains /Fo/**2 data rather
 than the normal /Fo/ data, it is necessary to use the "F's"
 parameter of the "REFLECTIONS" directive to inform the system 
 of this fact.


During the merge process, the system calculates and then prints a
 set of merging r-values, which are defined as follows :

\small\begin{verbatim}
 R = 100*SUM[ Sd(i) ]/SUM[ M(i) ],
             where 'i' runs over all reflections.


 Sd(i) = SUM[ <Fsq(i)> - Fsq(j) ],
             summed over 'j' contributors.

 and

 M(i) = SUM[ <Fsq(i)> ],
            summed 'j' times for 'j' contributors.
\end{verbatim}\normalsize


 The sum variable  'i'  runs over all the reflections produced by the merge
 process which have more than one contributor.
 The sum variable  'j'  runs over all the contributors for each reflection
 produced by the merge process.
  $<$Fsq(I)$>$  is the mean value for the reflection  'i' , while  Fsq(j)  is
 the observed value of Fsq for the contributor  'j'.


If the crystal is twinned, this will affect the merge. See chapter 
\label{twinning}  on twinned crystals


If the data is in Batches with different BATCH scale factors, this will
 affect the merge.





\bigskip\Instruction{WEIGHTING SCHEMES FOR THE DATA MERGE}




At present there are three different weighting schemes available
 for merging equivalent reflections.
 These are :

\small\begin{verbatim}
      1.      Each reflection is given equal weight (unit weights).
      2.      Weights based on a Gaussian distribution.
      3.      W(i) = 1.0/Sigma(i)**2  for each reflection.
\end{verbatim}\normalsize




Unit and statistical weights (schemes 1 and 3) are more or less
 equivalent unless some reflections have been remeasured under very
 different regimes ( \emph{e.g.} with an attenuator set, mA turned down,
 different crystal)


Scheme 2 is designed to discriminate against outliers, \emph{i.e.}
 reflections lying farther from the mean than might be expected.

 For this scheme, a weighted mean value of Fsq is determined
 iteratively, starting from unit weights.
 At each iteration, the weights are recomputed to discriminate against
 outliers and  the contributing reflections are given a new
 weight w(i) given by :

\small\begin{verbatim}
      w(i) = exp [ (-log(a) * q(i)**2)/(b**2 * e(i)**2) ]

 Where

     q(i)  is the deviation of the particular Fsq(i) from the current average.
     e(i)  is a predicted mean deviation of the reflection 'i' from the
           current mean and is given by a function similar to that used
           in Least Squares :
\end{verbatim}\normalsize


 e = c + d*Sig(Fsq) + g*Sig(Fsq)*/Fo/ + h*Sig(Fsq)*Fsq



 a,b,c,d,g,h  are 6 input parameters provided by the user



 'a'  and  'b' define the Gaussian distribution.



 'a'  is the weight to be given to a reflection which has a
 deviation given by 'q(i) = b*e(i)'.



 Suggested values of  'a'  and  'b'  are  0.5 and 3.0 respectively,
 so that if for example, 'e(i) = 3*Sig(Fsq)' (d=3, c=g=h=0),
 a deviation  q(i)  of  6*Sig*fsq  will assign a reflection a
 weight of 0.5.



  'c'  Provides the bias necessary to allow for
 failures in the counting statistics at low count
 rates.



  'd'  is a scaling constant.



  'g'  and  'h'  allow for the increased dispersion of strong reflections.

\small\begin{verbatim}
 For a conventional diffractometer, suggested values for the parameters are :

      a = .5      b=3.0      c=1.0      d=2.0      g=.01      h=.00001
\end{verbatim}\normalsize




It is recommended that the Gaussian scheme be used, as it discriminates
 against zero or widely dispersed intensities very efficiently.





\bigskip\Instruction{Standard deviations produced by the merge}




After the equivalent reflections have been merged
 two different standard deviations
 are computed and can be output :

\small\begin{verbatim}
 SIGMA1 = Sqrt (sum [ w(i)*q(i)**2 ] / n * sum [ w(i) ])
             that is, the weighted r.m.s. deviation.

 SIGMA2 = Sqrt (sum [ w(i)*s(i)**2 ] / n * sum [ w(i) ])
             that is, the weighted average standard deviation.
\end{verbatim}\normalsize


 Either of these two standard deviations can be selected as an estimate
 of Sigma(Fsq), and perhaps be converted to a Least Squares weight. If a
 reflection is measured very many times, SIGMA1 should be similar to
 SIGMA2. It is almost always much greater.





\bigskip\Instruction{$\backslash$MERGE}







\bigskip\Directive{MERGE=}


\bigskip\Keyword{INPUT=}
  Either 6 or 7.  Default is 6.





\bigskip\Keyword{TWINNED=}
\small\begin{verbatim}
  NO      Treat data as un-twinned
  LIST13  Treat data according to list 13 
  YES     Treat data as twinned
\end{verbatim}\normalsize







\bigskip\Directive{STORE MEDIUM=}


\bigskip\Keyword{MEDIUM=}
      Determines the output medium.

\small\begin{verbatim}
      FILE              A serial file
      INPUT - Default   The same as the input medium
      DISC              The .DSC file.
\end{verbatim}\normalsize


 The default output medium is the same as the input medium -  usually a
 serial file.








\bigskip\Directive{REFLECTIONS NJCODE= LIST= LEVEL= F'S=}


\bigskip\Keyword{NJCODE=}
 Specifies the number of JCODE directives to follow - default zero.


\bigskip\Keyword{LIST=}
 Determines the amount of information printed
 during the merge process.

\small\begin{verbatim}
      OFF
      MEDIUM  -  Default value
      HIGH
\end{verbatim}\normalsize


 If  LIST  is 'HIGH' , all Fsq are listed with their
 contributors and their deviations from the computed mean.
 The default value of  MEDIUM  indicates that the merged Fsq are listed
 with the contributors and their deviations
 from the computed mean if the r.m.s. deviation exceeds
 LEVEL*(mean standard deviation).
  HIGH  is equivalent to  MEDIUM  with  LEVEL  set at zero.


\bigskip\Keyword{LEVEL=}
 This parameter specifies the r.m.s. deviation level above which
 contributors are printed if  LIST  is equal to  MEDIUM .

 They are printed if sigma1 exceeds level*sigma2.
 The default value for this parameter is 3.


\bigskip\Keyword{F'S=}

\small\begin{verbatim}
      FO      -      Default
      FSQ            Indicating that square roots were not taken at
                     input time.
\end{verbatim}\normalsize







\bigskip\Directive{JCODE NUMBER= VALUE=}


This directive allows reflections whose JCODE key differs from
 unity to be down-weighted or eliminated from the merge.
 It is repeated once for each JCODE that is read in.


\bigskip\Keyword{NUMBER=}
 The number of the JCODE must be given.
 There is no default value for this parameter.


\bigskip\Keyword{VALUE=}
 This is the absolute weight, associated with the JCODE number,
 that is given to the reflection.
 If this parameter is omitted a default value of zero is assumed,
 indicating that the reflection is to be eliminated
 and not included in the merge at all.





\bigskip\Directive{REJECT RATIO= SIGMA=}


This directive causes reflections
 whose mean intensity is less than product of the ratio and sigma to 
 be eliminated.


\bigskip\Keyword{RATIO=}
 The default value for this parameter is -10. Use LIST 28 (section 
\ref{LIST28}) to suppress the
 use of reflections with RATIOs below a suitable threshold.


\bigskip\Keyword{SIGMA=}




\small\begin{verbatim}
      1
      2  -  Default value
\end{verbatim}\normalsize


 If sigma is equal to 1 the e.s.d. is the weighted r.m.s. deviation.
 If sigma is equal to 2 the e.s.d. is the weighted standard deviation.





\bigskip\Directive{WEIGHT SCHEME= NPARAMETERS= NCYCLE=}


This directive determines the weighting scheme to be used in
 merging equivalent reflections.


\bigskip\Keyword{SCHEME=}
 This parameter determines which of the weighting schemes defined above
 is to be used in the merging of equivalent reflections,
 and must take one of the following values:




\small\begin{verbatim}
      1  -  Default value (unit weights)
      2                   (modified Gaussian)
      3                   (statistical)
\end{verbatim}\normalsize


 If this parameter is omitted, unit weights are applied (scheme=1).


\bigskip\Keyword{NPARAMETERS=}
 This must be set to the number of parameters required to define the
 weighting scheme, and thus the number of values on the
  PARAMETERS  directive to follow.
 The default value for this parameter is zero,
 as schemes 1 and 3 require no parameters.


\bigskip\Keyword{NCYCLE=}
 This parameter has a default value of 5 and is the number of cycles of
 refinement of the weighted mean if scheme 2 is being used in the merge.





\bigskip\Directive{PARAMETERS P .....}
 This directive contains  NPARAMETERS  values.


\bigskip\Keyword{P=}
 For weighting scheme 2, these parameters give the values  'a'  to  'h'
 defined above, and describe the form of the Gaussian distribution.






\index{Theta-dependent Absorption Correction}
\index{THETABS}
\section{Theta-dependent Absorption Correction  -  $\backslash$THETABS}


\small\begin{verbatim}
 \THETABS
 THETA NTHETAVALUES=
 THETAVALUES THETA=
 THETACURVE CORRECTION= ........
 END
\end{verbatim}\normalsize





For example

\small\begin{verbatim}
 \THETABS
 THETA 16
 THETAVALUES
 CONT 0  5 10 15 20 25 30 35 40 45 50 55 60 65 70 75
 THETACURVE
 CONT 3.61  3.60  3.58  3.54  3.50  3.44  3.37  3.30
 CONT 3.23  3.16  3.09  3.02  2.96  2.91  2.86  2.82
 END
\end{verbatim}\normalsize




 Except
 when the data has been corrected by a proper analytical correction,
 a theta dependent correction is {\bf ALWAYS} recommended, since neither
 a phi scan multi-scan nor DIFABS (section \ref{DIFABS}) will make a 
 good theta 
 approximation. See Int Tab,
 Vol II, p295 and 303 for suitable profiles.








\bigskip\Directive{THETA NTHETAVALUES=}
 

\bigskip\Keyword{NTHETAVALUES=}
 The number of sampling points on the theta curve.





\bigskip\Directive{THETAVALUES THETA= .....}
 The Nthetavalues at which the curve is sampled





\bigskip\Directive{THETACURVE CORRECTION= ......}
 The Nthetavalue values of the correction factor profile.









\index{wILSON PLOT}
\index{WILSON}
\section{WILSON PLOT  -  $\backslash$WILSON}


\small\begin{verbatim}
 \WILSON
 OUTPUT PLOT= NZ= STATS= EVALS=
 FILTER LIST28=
 WEIGHT UPDATE=
 END
\end{verbatim}\normalsize





PLOT, NZ and STATS take values of YES and NO, controlling whether output
is sent to the GUI.
EVALS takes the values NO/YES/PUNCH.



 NO/YES  controls whether E-values are computed.



 PUNCH enables E-values to be computed and output to a text file.












\index{THLIM}
\index{THLIM}
\section{THLIM  -  $\backslash$THLIM}


Hthetalim#

\small\begin{verbatim}
 \THLIM  INPUTLIST=
 OUTPUT PLOT= GLIST=
 END
\end{verbatim}\normalsize


 This command is only used in SCRIPTS to send information to the GUI



INPUTLIST IS EITHER 6 or 7



PLOT=YES sends plotting data to the GUI



GLIST=YES sends a list of missing reflections to the GUI














SIGMADIST#
\index{SIGMADIST}
\section{SIGMADIST  -  $\backslash$SIGMADIST}


Hsigmadist#

\small\begin{verbatim}
 \SIGMADIST  INPUTLIST=
 OUTPUT PLOT= RESOLUTIO= PHASE= FILTER=
 END
\end{verbatim}\normalsize


 This command is only used in SCRIPTS to send information to the GUI



PLOT=YES sends completeness data to GUI



RESOLUTIO=YES sends distribution as function of resolution to GUI



PHASE=YES   sends phase angle distribution to GUI



FILTER=YES enable LIST 28 to be used.









\chapter{Atomic And Structural Parameters}
\label{atomparams} \section{Scope of the atomic and structural parameters Section}






 The areas covered are:




\small\begin{verbatim}
 Specifications of atoms and other parameters
 Input of atoms and other parameters              - \LIST 5
 Re-order the atom list                           - \REGROUP
 Collect atoms together by symmetry               - \COLLECT
 Move the structure into the cell                 - \ORIGIN
 Modification of lists 5 and 10 on the disc       - \EDIT
 Applying permitted origin shifts                 - \ORIGIN
 Conversion of temperature factors                - \CONVERT
 Hydrogen placing                                 - \HYDROGENS
 Per-hydrogenation                                - \PERHYDRO
 Re-numbering hydrogen atoms                      - \HNAME
 Regularisation of groups in LIST 5               - \REGULARISE
\end{verbatim}\normalsize





\index{Atom name syntax}
\section{Specifications of atoms and other parameters}



There is a consistent syntax thoughout CRYSTALS for refering to atoms
 and atomic parameters. This was referred to briefly in Chapter 1, and
 will be defined more fully here.





\bigskip\Instruction{ATOM SPECIFICATION}




There are three different but related ways of specifying an atom
 or a group of atoms.


\bigskip\Directive{TYPE(SERIAL,S,L,TX,TY,TZ)}


This specification defines one atom.
 The various parts of the expression are :


\bigskip\Keyword{TYPE}
 The atom type, defined in Chapter 1 in the section on form-factors.


\bigskip\Keyword{SERIAL}
 The serial number, in the range 1-9999


\bigskip\Keyword{Checking of serial numbers}


Atoms of the same type are distinguished from one another by having
 different serial numbers. However, at no stage is a check made to ensure
 that there is not more than one atom in LIST 5 (atomic parameters) with 
 the same type and
 serial number. If a routine is searching for an atom with a given type
 and serial number, the first atom found will always be taken, and any
 subsequent atoms with the same type and serial number will be ignored.


Serial numbers are considered to be different if they differ from
 each other by more than 0.0005.



\bigskip\Keyword{S}
 'S' specifies a symmetry operator provided
 in the unit cell symmetry LIST (LIST 2 - see section \ref{LIST02}).
 'S' may take any value between '-NSYM' and
 '+NSYM', except zero, where 'NSYM' is the
 number of symmetry equivalent positions
 provided in LIST 2.
 if 'S' is less than zero, the coordinates
 of the atom stored in LIST 5 are negated (i.e.
 inverted through a centre of symmetry at
 the origin) and then multiplied by the
 operator specified by the absolute value of
 'S' to generate the new atomic coordinates.
 'S' may be less than zero even if the space
 group is non-centrosymmetric ( \emph{i.e.} introduce a false centre),
 but must not be greater than 'NSYM'.
 The default value for 'S' is '1', specifying the
 first matrix in LIST 2, usually the unit matrix.



\bigskip\Keyword{L}
 'L' specifies the non-primitive lattice translation
 that is to be added after the coordinates have been
 modified by the operations given by 'S'.
 'L' must not be greater than the number of allowed
 non-primitive translations in the space group.
 The translations provided by 'L' depend on
 the lattice type and are given by :

\small\begin{verbatim}
          L=    1             2                3                4

      P       0,0,0
      I       0,0,0      1/2,1/2,1/2
      R       0,0,0      1/3,2/3,2/3      2/3,1/3,1/3
      F       0,0,0        0,1/2,1/2      1/2, 0 ,1/2      1/2,1/2,0
      A       0,0,0        0,1/2,1/2
      B       0,0,0      1/2, 0 ,1/2
      C       0,0,0      1/2,1/2, 0
\end{verbatim}\normalsize


        the default value of 'L' is '1', specifying no
        non-primitve lattice translation.



\bigskip\Keyword{TX,TY,TZ}
 Unit cell translation along the x,y and z directions.



 The unit cell translations are added to the
 coordinates after the 'S' and 'L' operations
 have been performed.
 The translations may be positive or
 negative, but must refer to complete
 unit cell shifts.
 The default values for 'TX', 'TY' and 'TZ'
 are all zero, giving no unit cell translations.



 The symmetry operations are applied in the order :

\small\begin{verbatim}
      1.  Centre of symmetry if 'S' negative
      2.  Symmetry operator 'S'
      3.  Non-primitve lattice translation
      4.  Whole unit cell translations 'T(X)', 'T(Y)', 'T(Z)'.

    i.e.
           X'=  [R(s)](+X) + t(s) + L + T(X) + T(Y) + T(Z)
    or
           X'=  [R(s)](-X) + t(s) + L + T(X) + T(Y) + T(Z)
\end{verbatim}\normalsize




The format given above is a complete atom definition.
 For convenience the definition may sometimes be shortened.
 The obligatory parts are the  TYPE  and  SERIAL.
 The remaining parameters, S,  L,  TX,  TY, TZ, are optional.


An optional parameter taking its default value
 may be omitted, though its place must be marked by its associated
 comma. A series of trailing commas may be omitted.


The following are all equivalent :

\small\begin{verbatim}
      TYPE(SERIAL,1,1,0,0,0)
      TYPE(SERIAL,,,0,0,0)
      TYPE(SERIAL,1,,,,0)
      TYPE(SERIAL,,,,,)
\end{verbatim}\normalsize




The values of  S ,  L ,  TX ,  TY  and  TZ  are exactly those
 output and used by the distance angles routines under the headings
  S(I) ,  L ,  T(X) ,  T(Y)  and  T(Z)  respectively.
 (See the section of the user guide on 'results of refinement').


When the symmetry operators are applied, the actual values
 of  S  and  L  are checked to see that they are reasonable.
 If the values found are not reasonable, an error message will
 be output and the job terminated.


In some cases, the symmetry operators are accepted on input,
 but not used by the routine. The description of the routine will state this.





\bigskip\Directive{UNTIL sequences}
 When a group of atoms lie sequentially in the atom parameter list,
 there is an abbreviated way to refer to the group.
\small\begin{verbatim}
      TYPE1(SERIAL1,S,L,TX,TY,TZ)  UNTIL  TYPE2(SERIAL2)
\end{verbatim}\normalsize




This definition specifies all the
 atoms in the current list starting with the atom TYPE1(SERIAL1)
 The first atom in the specification must  occur before the second atom
 in the current parameter list,
 otherwise an error message will be output and the task aborted.
 If symmetry operators are used, they must be given for the first atom
 of the sequence, and will be appied to all  the atoms in the sequence.



\small\begin{verbatim}
      Examples

                  C(1) until C(6)

      Six atoms lying around a centre of symmetry:

                  C(1) until C(3) C(1,-1) UNTIL C(3)
\end{verbatim}\normalsize




\bigskip\Directive{FIRST   AND   LAST}


These specifications each define one atom.   FIRST  Refers to the
 first atom stored in LIST 5 (the model parameters) or LIST 10
 (Fourier peaks), and  LAST  refers to the last atom in the
 list. If these are used as atom designators, no serial number may be given,
 but symmetry operators may be. They may be used in until sequences.

\small\begin{verbatim}
      examples
                  LAST
                  FIRST(x)
                  FIRST(-1) UNTIL C(16)  C(23) UNTIL LAST
\end{verbatim}\normalsize




\bigskip\Directive{ALL}


This specifies all atoms in the list, can take symmetry
 operators or parameter names, but cannot be accompanied on the same line
 by any other atom specifiers.

\small\begin{verbatim}
      examples
                  ALL
                  ALL(x)
                  ALL(-1)
\end{verbatim}\normalsize




\bigskip\Directive{RESIDUE}


This specifies all atoms or parameters with the given residue number.

\small\begin{verbatim}
      examples
                  RESIDUE(3)
                  RESIDUE(3,X's)
\end{verbatim}\normalsize




\bigskip\Directive{PART}


This specifies all atoms or parameters with the given part number.

\small\begin{verbatim}
      examples
                  PART(3001)
                  PART(3001,X's)
\end{verbatim}\normalsize


The part number is constructed from two values, the assembly number and
the group number.
\small\begin{verbatim}
    PART NO. = 1000 * ASSEMBLY NO. + GROUP NO.
\end{verbatim}\normalsize





The assembly number is normally zero, but a value
can be given to all atoms that are involved in a particular disordered
area of the structure. E.g. on a disordered methyl all the H atoms could
be placed in assembly number 1.



The group number within an assembly groups together those atoms which
are simultaneously occupied. E.g. on a disordered methyl, all the H atoms
approximately 109 degrees apart would be given the same group number.



The part and group numbers affect the default bonds that are determined
by CRYSTALS, and subsequently output in the CIF or summary file. Some
bonding rules are applied in the following order of priority:
\small\begin{verbatim}
1. An atom in assembly 0, group 0, will bond to any other nearby atom.
2. Atoms in the same assembly, but with different, non-zero group numbers
   will not bond to each other.
3. Atoms in different assemblies with one zero group number
   will not bond to each other.
4. Atoms in the same assembly and group, but with a negative group number
   will not bond to symmetry related atoms in the same assembly and group.
5. All remaining close contacts will be bonded together.
\end{verbatim}\normalsize


Rule 3 may be ignored unless you're trying to set up something very
special. Rule 4 is useful if you have a group disordered across a
symmetry element.



Finally, you can select all atoms in a given part of group by using the number 999 as a wild-card.

\small\begin{verbatim}
      examples
                  PART(3999,X's)
      selects all atoms in assembly 3.
\end{verbatim}\normalsize







\bigskip\Directive{HPART}


This specifies all hydrogen (or deuterium) atoms or parameters with the given part number.

\small\begin{verbatim}
      examples
                  HPART(3001)
                  HPART(3001,X's)
\end{verbatim}\normalsize




\bigskip\Directive{NHPART}


This specifies all non-hydrogen (and deuterium) atoms or parameters with the given part number.

\small\begin{verbatim}
      examples
                  NHPART(3001)
                  NHPART(3001,X's)
\end{verbatim}\normalsize




\bigskip\Directive{TYPE}


This enable all atoms or parameters on atoms of the given type to be 
processed together.
\small\begin{verbatim}
      example
      /EDIT
      RESET 9900 PART TYPE(H)
      END
\end{verbatim}\normalsize




This puts all the hydrogen atoms into PART(9900). Since all other 
atoms are in PART(0) or a specified PART, the hydrogen will now be 
treated independently.









\bigskip\Instruction{ATOMIC PARAMETER SPECIFICATION}




Atomic parameters have a NAME. Some directives permit the use of the
 parameter name by itself, which implies that parameter for all atoms.
 The parameter name may be combined with an atom specifier, in which case
 only the parameter for that atom (or group in an UNTIL sequence) is
 referenced. Symmetry operators may be used. The normal drop-out rules
 apply.


\bigskip\Directive{Parameter NAMES}


The following NAMES are recognised.

\small\begin{verbatim}
      X      Y      Z      OCC      U[ISO]    SPARE
      U[11]  U[22]  U[33]  U[23]    U[13]     U[12]
      X'S    U'S    UIJ'S  UII'S
\end{verbatim}\normalsize



\small\begin{verbatim}
      Examples
        X            The 'x' coordinate for all atoms
        C(9,X,Y)     The 'x' and 'y' coordinates for atom C(9)
        FIRST(X'S)   The 'x','y' and 'z' coordinates for the first atom
        FIRST(U'S) UNTIL C(23)
                     The anisotropic temperature factors for all atoms
                     up to C(23).
\end{verbatim}\normalsize




\bigskip\Directive{Temperature factor definitions}


\bigskip\Keyword{Isotropic temperature factor}
\small\begin{verbatim}
 The isotropic temperature factor is defined by:

       T = exp(-8*pi*pi*U[iso]*s**2)
                                       where s = sin(theta)/lambda
\end{verbatim}\normalsize




\bigskip\Keyword{Anisotropic Temperature Factor}
\small\begin{verbatim}
      The anisotropic temperature factor (adp) is defined by:

       T = exp(-2*pi*pi*(h*h*a'*a*u[11]
                        +k*k*b'*b'*u[22]
                        +l*l*c'*c'*u[33]
                    +2.0*k*l*b'*c'*u[23]
                    +2.0*h*l*a'*c'*u[13]
                    +2.0*h*k*a'*b'*u[12])).
                                       where x' are the reciprocal
                                       cell parameters and h, k and
                                       l are the Miller indices
\end{verbatim}\normalsize

\bigskip\Keyword{Uequiv}


CRYSTALS contains two definitions od Uequiv. Both definitions are acceptable
 to Acta. The arithmetic mean of the principle axes is often similar
 to the refined value of Uiso. The geometric mean is more sensitive to
 long or short axes, and so is more useful in publications. Ugeom is the
 sphere with the same volume as the ellipsoid.
\small\begin{verbatim}
      U(arith) = (U1+U2+U3)/3
      U(geom)  = (U1*U2*U3)**1/3
                                    Where Ui are the principal axes of
                                    the orthogonalised tensor.
\end{verbatim}\normalsize





 {\bf CAUTION}


It should be noted that if a set of anisotropic atoms are input with
 the
  FLAG  key set to anything but 0, then the parameters will be
interpreted as Isotropic atoms, or special shapes.





\bigskip\Directive{Uequiv}
      Two expressions are available for the equivalent temperature factor
 (the geometric or arithmetric mean of the principal components).
 The Immediate Command 'SET UEQUIV' sets which definition will be used.


\small\begin{verbatim}
      Ugeom  = (Ui * Uj * Uk)**1/3

      Uarith = (Ui + Uj + Uk)/3
                                       Where Ui, Uj & Uk are the
                                       principal components of U

      Ugeom is the radius of the sphere with the same volume as the adp
      ellipsoid, and thus gives a good indication of the quality of the
      ellipsoid. Uarith is often closer to the value of Uiso, and so is
      useful for returning to an isotropic refinement.
\end{verbatim}\normalsize




\bigskip\Directive{The Special Shapes}



 The SPecial Shape keys are

\small\begin{verbatim}

      type serial occ FLAG x y z u[11]  u[22] u[33] u[23] u[13] u[12] spare
                                 U[ISO]                               spare
                                 U[ISO] SIZE                          spare
                                 U[ISO] SIZE  DECLINAT AZIMUTH        spare

\end{verbatim}\normalsize



The value of 'FLAG' is used on input of atoms to indicate what kind of
patameters will follow, and is used during calculations for the
interpretation of the parameters.



\bigskip\Keyword{FLAG interpretation}
The following table shows the interpretation of the FLAG parameter.
\small\begin{verbatim}

FLAG        meaning          parameters

'normal' types of atoms:

 0          Aniso ADP        u[11]  u[22] u[33] u[23] u[13] u[12]
 1          Iso ADP          U[ISO]

New 'special' shapes:

 2          Sphere           U[ISO] SIZE
 3          Line             U[ISO] SIZE  DECLINAT AZIMUTH
 4          Ring             U[ISO] SIZE  DECLINAT AZIMUTH

\end{verbatim}\normalsize





The parameters have the following meaning for the new special shapes:


\bigskip\Keyword{Special U[iso]}
U[iso] is related to the 'thickness' of the line, annulus or shell.


\bigskip\Directive{Special SIZE}
SIZE is the length of the line, or the radius of the annulus or shell.


\bigskip\Keyword{Special DECLINAT}
DECLINAT is the declination angle between the line axis or annulus normal and the
  \emph{z}
axis of the usual CRYSTALS orthogonal coordinate system, in
degrees/100.


\bigskip\Keyword{Special AZIMUTH}
AZIMUTH is the azimuthal angle between the projection of the
line axis or annulus normal onto the \emph{x} - \emph{y} plane and the \emph{x}
axis of the usual CRYSTALS orthogonal coordinate system, in
degrees/100.



If either of these angles is input with a value greater than 5.0, it
is assumed that the user has forgotten to divide by 100, which is thus
done automatically.








\bigskip\Instruction{OVERALL PARAMETER SPECIFICATION}




Overall parameters are specified simply by their keys.
 The following overall parameter keys may be given :

\small\begin{verbatim}
      SCALE      OU[ISO]      DU[ISO]      POLARITY
      ENANTIO    EXTPARAM
\end{verbatim}\normalsize




\bigskip\Keyword{SCALE}
 This parameter defines the overall scale factor and has a default value
 of unity.
 It is the number by which /FC/ must be multiplied to put
 it onto the scale of /FO/, i.e. /Fo/ = scale*/FC/.


\bigskip\Keyword{DU[ISO]}
 This parameter is the dummy overall isotropic temperature factor and has
 a default value of 0.05.


The dummy overall temperature factor is in no way related to the overall
 temperature factor, and its use is explained in the input of LIST 12,
 which comes in the section of the user guide on 'structure factors'.


\bigskip\Keyword{OU[ISO]}
 This parameter is the overall isotropic temperature factor and has a default
 value of 0.05.


\bigskip\Keyword{POLARITY}
 This is the Rogers \emph{eta} parameter, and  is a  multiplier for the
 imaginary part of the anomalous
 scattering factor.
 Setting the value to 1.0 (its default) has the effect of using the
 imaginary part
 of the anomalous scattering factor as given.
 Changing the value to
 -1.0 has the effect of changing the hand of the model. Setting the value
 at zero has the effect of removing the contribution of f". However, if
 contributions from f" are not required, IT IS MORE EFFICIENT to set ANOMALOUS
 = NO in LIST 23 (structure factor control, see section \ref{LIST23}). If you 
 need to use f", remember not to
 apply Friedel's law (LIST 13, section \ref{LIST13}) during data reduction
 (section \ref{DATAREDUC}),
 and to include anomalous scattering (LIST 3, section \ref{LIST03} and
 LIST 23, section \ref{LIST23}). See D. Rogers, Acta Cryst (1981), 
 A37,734-741. \emph{POLARITY}
 \emph{and} \emph{ENANTIO} \emph{should} \emph{not} \emph{be} \emph{used} \emph{simultaniously.}


\bigskip\Keyword{ENANTIO}
 This overall parameter is the fractional contribution of F(-h) to the observed
 structure amplitude, and like the POLARITY parameter is sensitive to the
 polarity of the structure. It is defined by
\small\begin{verbatim}
       Fo**2 =(1-x)* F(h)**2 + x*F(-h)**2
\end{verbatim}\normalsize


 where x is the ENANTIOpole parameter. A value of 0.0 means the structure
 stored in LIST 5 is of the correct hand. A value of 1.0 inverts the structure.
 Its effect on the structure factor is switched on or off by the parameter
 ENANTIO in LIST 23 (see section \ref{LIST23}). Computations are more efficient 
 when it is turned off.
 If the enantiopole is used (or refined) then Friedel's law must not be applied
 (LIST 13, section \ref{LIST13}) and anomaloue scattering must 
 be included (LIST 13 and LIST 23).
 See Howard Flack, Acta Cryst, 1983, A39, 876-881. This parameter
 is more robust than the POLARITY parameter.
 See also section in Results.


\bigskip\Keyword{EXTPARAM}
 This parameter is Larson's extinction parameter , r*, (equation 22 in
   A.C. Larson, Crystallographic Computing, 1970, 291-294, ed F.R.
 Ahmed, Munksgaard, Copenhagen , but with V
 replaced by the cell volume)
 and has a default value of zero.


Note that many other programs use expression (4),
  which cannot cope with Neutron data, and  gives a value for 'g'
  which is about 1,000,000 times smaller than 'r*'.

\small\begin{verbatim}
       g ~= [(e**2/mc**2)**2 . lambda**3/V**2 . Tbar ] . r*
\end{verbatim}\normalsize


 Tbar is the absorption weighted mean path length, and is assumed to be
 stored in
 LIST 6 (section \ref{LIST06}) with a key of  TBAR . If this key is 
 absent, a default value of 1.0
 is used. If extinction is to be included in the model, the mosaic spread
 should have been set in LIST 13 (section \ref{LIST13}).



\index{LIST 5}
\index{Input of atoms and other parameters}
\section{Input of atoms and other parameters  -  LIST 5}
\label{LIST05} 


\small\begin{verbatim}
 \LIST 5
 OVERALL SCALE= DU[ISO]= OU[ISO]= POLARITY= ENANTIO= EXTPARAM=
 READ NATOM= NLAYER= NELEMENT= NBATCH=
 either ATOM TYPE= SERIAL= OCC= FLAG= X= Y= Z= U[11]= ....U[12]=
 or     ATOM TYPE= SERIAL= OCC= FLAG= X= Y= Z= U[ISO]
 INDEX P= Q= R= S= ABSOLUTE=
 LAYERS SCALE=
 ELEMENTS SCALE=
 BATCH SCALE=

\end{verbatim}\normalsize



\small\begin{verbatim}
 \LIST 5
 OVERALL SCALE=0.123
 READ NATOM=2 NELEMENT=2
 ATOM PB 1 FLAG=0 .25 .25 .25 .03 .03 .03 .0 .0 .0
 ATOM C 2  X= .23 .13 .67
 ELEMENTS 0.8 0.2
 END

\end{verbatim}\normalsize




\bigskip\Instruction{$\backslash$LIST 5}




\bigskip\Directive{OVERALL SCALE= DU[ISO]= OU[ISO]= POLARITY= ENANTIO= EXTPARAM=}


This directive specifies various parameters that refer to the
 structure as a whole.


\bigskip\Keyword{SCALE=}
 The overall scale factor, default = 1.0


\bigskip\Keyword{DU[ISO]=}
 The dummy overall isotropic temperature factor, default = 0.05.


\bigskip\Keyword{OU[ISO]=}
 The overall isotropic temperature factor, default = 0.05.


\bigskip\Keyword{POLARITY=}
 Rogers \emph{eta} parameter (see above), default = 1.0.


\bigskip\Keyword{ENANTIO=}
 Flack enantiopole parameter (see above), default = 0.0.


\bigskip\Keyword{EXTPARAM=}
 Larson r* secondary extincion parameter, default = 0.0.


\bigskip\Directive{READ NATOM= NLAYER= NELEMENT= NBATCH=}


This directive specifies the number of atoms, layer scale factors,
 element scale factors, and batch scale factors that are to follow.


\bigskip\Keyword{NATOM=}
 The number of atom directives to follow, default = 0.


\bigskip\Keyword{NLAYER=}
 The number of layer scale factors to follow, default = 0.


\bigskip\Keyword{NELEMENT=}
 The number of element scale factors to follow, default = 0.


\bigskip\Keyword{NBATCH=}
 The number of batch scale factors to follow, default = 0.


\bigskip\Directive{ATOM TYPE= SERIAL= OCC= FLAG= X= Y= Z= U[11]= ..}


The parameters for an atom, repeated NATOM times.


\bigskip\Keyword{TYPE=}
 The atomic species, an entry for which should exist
 in LIST 3 (see section \ref{LIST03}). There is no default value.


\bigskip\Keyword{SERIAL=}
 The atoms serial number. There is no default value.


\bigskip\Keyword{OCC=}
 This parameter defines the site occupancy {\bf EXCLUDING} special position
 effects (i.e. is the 'chemical occupancy). The default is 1.0.
 Special position  effects are computed by CRYSTALS and multiplied onto
 this parameter.


\bigskip\Keyword{FLAG=}
 This parameter specifies the type of temperature factor for the atom,
 and if it is omitted a default value of 1 is assumed. {\bf NOTE} that it
 {\bf must} be set to 0 for anisotropic atoms.


\bigskip\Keyword{X= Y= Z=}
 These parameters specify the atomic coordinates for the atom, for which
 there are no default values.


\bigskip\Keyword{U[11]= U[22]= U[33]= U[23]= U[13]= U[12]=}
 These parameters have different interpretations depending upon the
value of FLAG



 If FLAG=0


 These parameters specify the anisotropic temperature factors for the atom
 and if they are omitted default values of zero are assumed.
 The order of the
 cross terms is obtained by dropping 1,2,3 sequentially from [123].



 If FLAG=1


 The first parameter specifies the isotropic temperature factor, which
 defaults to 0.05.



  If FLAG=2,3 or 4, the six parameters represented by u[ij] have the
following imterpretation:



\small\begin{verbatim}
KEY   shape      parameters

 2    Sphere     U[ISO] SIZE
 3    Line       U[ISO] SIZE  DECLINAT AZIMUTH
 4    Ring       U[ISO] SIZE  DECLINAT AZIMUTH

\end{verbatim}\normalsize





The parameters have the following meaning for the new special shapes:


\bigskip\Keyword{Special U[iso]}
U[iso] is related to the 'thickness' of the line, annulus or shell.


\bigskip\Directive{Special SIZE}
SIZE is the length of the line, or the radius of the annulus or shell.


\bigskip\Keyword{Special DECLINAT}
DECLINAT is the declination angle between the line axis or annulus normal and the
  \emph{z}
axis of the usual CRYSTALS orthogonal coordinate system, in
degrees/100.


\bigskip\Keyword{Special AZIMUTH}
AZIMUTH is the azimuthal angle between the projection of the
line axis or annulus normal onto the \emph{x} - \emph{y} plane and the \emph{x}
axis of the usual CRYSTALS orthogonal coordinate system, in
degrees/100.



If either of these angles is input with a value greater than 5.0, it
is assumed that the user has forgotten to divide by 100, which is thus
done automatically.





\bigskip\Directive{INDEX P= Q= R= S= ABSOLUTE=}


This directive is used to input the constants that define an index
 for layer scaling. The layer scale index for the reflection with indices
  HKL  is computed from

\small\begin{verbatim}
      index = (h*p + k*q + l*r + s)
\end{verbatim}\normalsize


 and the absolute value is taken if the parameter  ABSOLUTE  = yes.


\bigskip\Keyword{P= Q= R=}
 These parameters have default values of zero.


\bigskip\Keyword{S=}
 This parameter has a default value of unity. The zeroth layer must have
 an index of 1.


\bigskip\Keyword{ABSOLUTE=}

\small\begin{verbatim}
      NO
      YES  -  Default value
\end{verbatim}\normalsize




\bigskip\Directive{LAYERS SCALE=}


This directive defines the layer scale factors, starting with the scale
 for an index of 1.


\bigskip\Keyword{SCALE=}
 This parameter gives the layer scale, and has a default value of 1.
 It is repeated  NLAYER  times.


\bigskip\Directive{ELEMENTS SCALE=}


This directive defines the scale factors for the elements of a twinned
 structure. See the chapter on twinned structures.


\bigskip\Keyword{SCALE=}
 This parameter gives the element scale factor, and has a default value
 of 1. It is repeated  NELEMENT  times - the number of components
 in the twin.


\bigskip\Directive{BATCH SCALE=}


This directive defines the batch scale factors.


\bigskip\Keyword{SCALE=}
 This parameter gives the batch scale factor, and has a default value of 1.
 It is repeated  NBATCH  times. Remember to set appropriate keys in LIST
 6





\bigskip\Instruction{Further examples of parameter input}



\small\begin{verbatim}
 ATOM TYPE=C,SERIAL=4,OCC=1,U[ISO]=0,X=0.027,Y=0.384,Z=0.725,
 CONT U[11]=0.075,U[22]=0.048,U[33]=.069
 CONT U[23]=-.007,U[13]=.043,U[12]=-.001
 ATOM C 5 U[ISO]=0.0 .108,.365,.815,.074
 CONT .051 .065 -.015 .048 -.014
 ATOM C 2 1 0.05 0.149 0.411 0.651 0 0 0 0 0 0
 ATOM C 1 X=0.094,Y=0.343,Z=0.890
 ATOM C 3 X=0.050 0.406 0.648
\end{verbatim}\normalsize


\section{Printing and punching list 5}



\bigskip\Instruction{$\backslash$PRINT 5}


 Lists the current LIST 5 to the printer file.


\bigskip\Instruction{$\backslash$PUNCH 5 mode}


 Mode controls the format of the file.
\small\begin{verbatim}
        -  Punches the model parameters in CRYSTALS format.
      A -  Punches the model parameters in CRYSTALS format.
      B -  Punches the atomic parameters in XRAY format.
      C -  Punches the atomic parameters in SHELX format.
      E -  Punches atomic parameters and esds in a plain format
\end{verbatim}\normalsize




\bigskip\Instruction{Summary display of LIST 5 - $\backslash$DISPLAY}



\small\begin{verbatim}
 \DISPLAY LEVEL=
 END

 \DISP HIGH
 END
\end{verbatim}\normalsize




This allows the user to display a summary of the contents
 of list 5.  The  output is sent to both monitor and listing channels, so
 the contents of list 5 can be examined on-line during interactive work.
 The output produced is more compact than that from PRINT 5, and various
 levels of detail can be selected.
 The command required is :-


\bigskip\Instruction{$\backslash$DISPLAY    LEVEL=}




DISPLAY has one  optional  parameter.



\bigskip\Keyword{LEVEL}
\small\begin{verbatim}
      LOW
      MEDIUM
      HIGH
\end{verbatim}\normalsize




The effects of this parameter are :-


LOW            The names of the atoms, overall parameters,
 and any layer, batch, and element scales in list 5 are displayed.


MEDIUM         Each atom in list 5 is displayed with its type, serial,
 occupancy, isotropic temperature factor ( if any ), and positional parameters.
 The values of the overall parameters and of any layer, batch, and element
 scales are displayed.


HIGH           All of the parameters of each atom in list 5 are
 displayed. The values of the overall parameters, and of any layer, batch,
 and element scale factors are displayed.






\index{EDIT}
\index{Editing structural parameters}
\section{Editing structural parameters -  $\backslash$EDIT}

\small\begin{verbatim}
 \EDIT INPUTLIST= OUTPUTLIST=
 
 EXECUTE
 SAVE
 QUIT
 LIST LEVEL
 MONITOR LEVEL
 END

 ADD  VALUE PARAMETERS  ...
 AFTER  ATOM-SPECIFICATION
 ANISO  ATOM-SPECIFICATIONS  .  .
 ATOM TYPE= SERIAL= OCC= FLAG= X= Y= Z= U11= ..
 CENTROID Z ATOM-SPECIFICATION ...
 CHANGE  PARAMETER-SPECIFICATION VALUE ...
 COPYXYZ ATOM-PAIRS
 COPYUIJ ATOM-PAIRS
 CREATE Z ATOM-SPECIFICATION  ...
 DELETE  ATOM SPECIFICATIONS  .  .
 DEORTHOGINAL  ATOM-SPECIFICATION . .
 DIVIDE  VALUE  PARAMETERS  ...
 DSORT TYPE1 TYPE2 ...
 INSERT IDENTIFIER
 KEEP  Z ATOM-SPECIFICATIONS ...
 MOVE Z ATOM-SPECIFICATION  ...
 MULTIPLY  VALUE  PARAMETERS  ...
 PERTURB VALUE PARAMETERS ...
 REFORMAT
 RENAME ATOM1  ATOM2  (, ATOM1  ATOM2) ...
 RESET PARAMETER-NAME VALUE ATOM-NAMES
 ROTATE ANGLE ATOM VECTOR ATOM-SPECIFICATION
 ROTATE ANGLE ATOM1 ATOM2 ATOM-SPECIFICATION
 ROTATE ANGLE POINT VECTOR ATOM-SPECIFICATION
 SELECT ATOM-PARAMETER  OPERATOR  VALUE, . .
 SHIFT  V1, V2, V3   ATOM-SPECIFICATION . .
 SORT KEYWORD
 SORT TYPE1 TYPE2 ...
  SPLIT Z ATOM-SPECIFICATION ...
 SUBTRACT  VALUE  PARAMETERS  ...
 SWAP ATOM-PAIRS
 TRANSFORM  R11, R21, R31, . . . R33  ATOM-SPECIFICATION . .
 TYPECHANGE KEYWORD OPERATOR VALUE NEW-ATOM-TYPE
 UEQUIV  ATOM-SPECIFICATIONS  .  .

 LINE NEWSERIAL ATOMLIST
 RING NEWSERIAL ATOMLIST
SPHERE NEWSERIAL ATOMLIST
\end{verbatim}\normalsize



\small\begin{verbatim}
 LIST LOW
 TYPECHANGE TYPE EQ Q C
 SELECT U[ISO] LT 0.1
 ADD  0.25 X
 RENAME C(1) S(1)
 CHANGE  S(1,OCC) UNTIL O(1) .5
 KEEP  1 FIRST UNTIL LAST
 L L
 SPLIT 100 C(45)
 DELETE  C(46) UNTIL LAST
 RESET OCC 1.0 ALL
\end{verbatim}\normalsize








This is a powerful crystallographic editor for
 modifying a LIST 5 (the model parameters) or LIST 10 (Fourier peaks).
 It offers the editing facilities frequently needed
 for the management of atom parameters, including conditional operations
 and arithmetic.





EDIT is a semi-interactive command, in that each directive is
 computed as soon as its input is complete. Since CONTINUE can be used
 to extend a directive over several lines, completion is indicated be
 the start of a new directive, or the special directive EXECUTE.


After the terminating END, the resulting list is
 output to the disc.
 However if the list has not been changed, a new list will be created only
 if the list type is being changed ( e.g. 10 to 5 ).
 The current edited version of the list can be saved at
 any time to protect against future editing mistakes ( the SAVE directive ).
 It is also possible to abandon editing without creating a new list ( the QUIT
 directive ).


When used in interactive mode, a new list is created even though errors may
 have occured during command input unless the QUIT directive is used. In
 online and batch modes no new list will be created if errors occured
 during the edit. In this case an error message in generated.


Take care to note that some directives refer to atom or group of atoms,
 others refer to one or more parameters, and two
 (CHANGE  and  SELECT)will refer to either an atom specification
 or a parameter specification.
 Although atom definitions can include a series of symmetry operators,
 the only directives that will use them are those
 for which the subsequent description explicitly states that the
 symmetry operators are used.
 In all other cases, the symmetry information will be read in without any
 error messages and ignored.
 Those operations which require a single parameter type as argument
 (ADD,  MULTIPLY  etc ) will fail if composite parameters ( "U'S", etc )
 are given.





\bigskip\Instruction{$\backslash$EDIT INPUTLIST OUTPUTLIST}





\bigskip\Keyword{INPUTLIST}
\small\begin{verbatim}
      5   -  Default value, the atomic coordinates
      10                    the Fourier peaks search
\end{verbatim}\normalsize







\bigskip\Keyword{OUTPUTLIST}
\small\begin{verbatim}
      5   -  Default value, the atomic coordinates
      10                    the Fourier peaks search
\end{verbatim}\normalsize







\bigskip\Directive{EXECUTE}
 This directive which has no parameters does nothing to the edited list. It
 is provided to allow the user to see the results of one operation (
 initiated by the directive whose input is terminated by EXECUTE ) before
 attempting the next.


\bigskip\Directive{SAVE}
 Forces the current atom list to be writen to disk.


\bigskip\Directive{QUIT}
 This directive will cause the edit to be abandoned without the creation
 of a new list if it is followed by
 END . If it is followed by any other directive it is ignored.





\bigskip\Directive{LIST LEVEL}
 This directive produces a list of the current edited list in the monitor
 output stream and in the listing file. If KEEP has been used, the atoms
 which will be kept are indicated.
 The possible values for 'level' are :-

\small\begin{verbatim}
      OFF               No listing produced
      LOW               Type and serial listed
      MEDIUM            Type , serial , occ , u[iso] ,
                        x , y , z listed
      HIGH              All atomic parameters listed
\end{verbatim}\normalsize




\bigskip\Directive{MONITOR LEVEL}
 This directive controls the level of monitoring of editing operations. When
 each operation is performed, the results can be monitored in the monitor
 channel and in the listing file. Four levels of monitoring are provided. The
 inital level and the default level used when no value is specified is
 'MEDIUM'. The possible values of the parameter 'level' are :-

\small\begin{verbatim}
      OFF          No monitoring occurs
      LOW          Type and serial only are displayed
      MEDIUM       Program selects level of display   (default)
      HIGH         At least the level represented by
                   'MEDIUM' listing is displayed
\end{verbatim}\normalsize




When the program selects a monitor level account is taken of the
 amount of relevant information for the particular directive. Thus for
 DELETE only 'type' and 'serial' need be displayed whereas for CHANGE
 all parameter values are displayed.


\bigskip\Directive{END}
 This must be the last directive in the set of modification directives, 
 and writed the list to disk.

















\bigskip\Directive{ADD  VALUE  PARAMETERS  ...}





\bigskip\Directive{AFTER  ATOM-SPECIFICATION  ...}


This defines the atom in the list after which atoms
 that are  MOVEd  should be placed.
 (See  MOVE  below).
 If this directive is omitted, the default option places the first  MOVED
 atom at the head of the list, and successive atoms after it.
 Once one  AFTER  directive has been given, atoms are placed behind the
 given atom in the order in which they are presented on  MOVE  directives.
 If no atom specification is given on this directive, subsequent MOVEs will move
 the atoms  to the  head of the list.





\bigskip\Directive{ANISO  ATOM SPECIFICATIONS  .  .}
 This directive causes all the specified atoms to be converted so that they
 have anisotropic temperature factors.
 If an atom is already anisotropic, no action is taken, and any symmetry
 operators given are ignored.
 If this directive is given with no arguments, all the atoms in the current
 atomic parameter list are converted to anisotropic temperature factors.


Note that   the anisotropic temperature factor produced by this operation
 is in fact still spherically symmetrical, and
 that the s.f.l.s.
 routines automatically ensure that when the temperature factor of an atom
 is to be  refined, it is in the correct form.





\bigskip\Directive{ATOM TYPE SERIAL OCC FLAG X Y Z U11 ..}
 This directive causes the system to add an atom to the
 end of the edited list. The format is the same as that used in $\backslash$LIST 5 (see
section \ref{LIST05}).
 Values must be provided for 'type' , 'serial' , 'x' , 'y' , and 'z' .
 Default values are provided for the other parameters as in $\backslash$LIST 5.
 Example :
\small\begin{verbatim}
 ATOM O 1 X = 0.3427 .89004 .09181
\end{verbatim}\normalsize




\bigskip\Directive{CENTROID Z ATOM-SPECIFICATION ...}
 A new atom is created at the centroid of the specified atoms, and with
 a pseudo adp representing the inertial tensor (ie the 'shape' of the
 group). The atom TYPE is QC, and its serial Z.
 The sequence Z ATOM-SPECIFICATIONS can be repeated.





\bigskip\Directive{CHANGE  ARG(1)  ARG(2)  ARG(3)  .}
 There are two possible formats for each 'ARG(i)' on this directive.
 the first is :
\small\begin{verbatim}
 PARAMETER(i)  VALUE(i)
\end{verbatim}\normalsize


 If  ARG(i)  is of this form, the specified parameter or parameters
 are changed to the value  VALUE(i) .
 If  PARAMETER(i)  defines one or more atomic parameters, then the
 symmetry operators found or inserted by default
 are applied to the resulting set of atomic parameters.
 For overall parameters,
 no symmetry information can be provided.
 The  VALUE  associated with this argument must always be present.


The second form of  ARG(i)  on this directive is :
\small\begin{verbatim}
 ATOM-SPECIFICATION
\end{verbatim}\normalsize


 For this form of  ARG(i) , the symmetry operators given in the
 atom specification or assumed by default are applied, but no
 other atomic parameter is explicitly altered.
 There is no  VALUE  associated with  ARG(i)  in this format.


The two different types of argument on this directive may be used
 interchangeably :

\small\begin{verbatim}
      CHANGE  S(1,OCC) UNTIL O(1) .5
      CONT    C(1,-2,1) UNTIL C(12)
      CONT    C(13,X) .0179
\end{verbatim}\normalsize







\bigskip\Directive{COPYXYZ  ATOM-PAIRS  ...}
 This directive COPIES the XYZ coordinates of the first atom into
 the second atom.  The other parameters are left  unchanged.  








\bigskip\Directive{COPYUIJ  ATOM-PAIRS  ...}
 This directive COPIES the Uij coordinates of the first atom into
 the second atom.    The other parameters are left  unchanged.








\bigskip\Directive{CREATE Z ATOM-SPECIFICATION  ...}
 This directive applies the symmetry operators given or assumed by default
 in the atom specification, and creates a set of new atoms from those given.
 The new atoms are added at the end of the current list. The serial numbers
 of the new atoms are given by:
\small\begin{verbatim}
      NEWSERIAL = Z + OLDSERIAL
\end{verbatim}\normalsize


 The sequence Z ATOM-SPECIFICATIONS can be repeated.
 When moving from a centrosymmetric to a non-centrosymmetric space group,
 for example, atoms formerly related by the centre of symmetry can be
 generated :
\small\begin{verbatim}
      CREATE 30 MO(1,-1) UNTIL C(15)

      Creates atoms MO(31) until C(45)
\end{verbatim}\normalsize




\bigskip\Directive{DELETE  ATOM SPECIFICATIONS  .  .}
 All the specified atoms are removed
 from the current atomic parameter list.
 Deleted atoms should not be referenced by subsequent directives.





\bigskip\Directive{DEORTHOGINAL  ATOM SPECIFICATION . .}
 This directive applies the matrix vector saved by a previous MOLAX SAVE
 directive to the atoms given  in the atom specification. THEIR ORIGINAL
 COORDIANATES x,y,z MUST be in the MOLAX coordinate (Angstrom) system
 This directive does not create new atoms, but simply modifies
 those already present. Symmetry operators are not permitted.





\bigskip\Directive{DIVIDE  VALUE  PARAMETERS  ...}
 These directives causes the 'value' to be applied to the parameter.
 'PARAMETER(I)' may be an overall parameter, or a single atomic
 parameter of one  or more atoms, as defined above.
 Any symmetry operators given with this directive will be ignored.
 Note that the parameter SERIAL is numeric, and so can be arithmetically
 modified.





\bigskip\Directive{DRENAME ATOM1 ATOM2 }
 Used by SCRIPTS to avoid name clashes.





\bigskip\Directive{DSORT TYPE1 TYPE2 ...}


\bigskip\Directive{DSORT KEYWORD}
 This directive is exactly analagous to SORT, below, except that it sorts
 into descending order.





\bigskip\Directive{INSERT IDENTIFIER=NAME}
 This directive inserts the value of the named identifier into the
 parameter 'SPARE' in the atom list, replacing any previous value
 (except 'RESIDUE' which uses the 'RESIDUE' paramter in the atom list).
 SPARE is normally used to hold rho after Fourier maps.


 Currently available values for NAME are






 ELECTRON - This inserts the atomic electron count calculated
from the form factor


 WEIGHT - This inserts the atomic weight from LIST 29 (see section \ref{LIST29}).


 RESIDUE - This inserts a residue number into the 'RESIDUE' slot
 of list 5 replacing any previous value, such that all connected atoms
 have the same residue number and each molecule has a different residue number.


 NCONN - This inserts the number of atoms connected to an atom, using
 the list of bonds.


 RELAX - This inserts an ID, based upon the bonding topology and atomic
 types of the atoms. Atoms at topologically identical positions will be given
 the same ID. (e.g. the terminal F's in a CF3 group).





\bigskip\Directive{KEEP  Z ATOM-SPECIFICATIONS ...}


Only the atoms referenced in this directive will be kept in the list,
 all the others will be lost, even though they can be referenced
 right up until the final END.
 The sequence Z ATOM-SPECIFICATIONS can be repeated.



 Atoms that are  KEPT  are moved to the top of the list, and stored in the
 order in which they are specified on  the KEEP  directive. Only one KEEP
 directive may be given. Use CONTINUE if one line isn't long enough for the
 atom sequence.


The atom specifications may contain symmetry operators, which are used
 to generate the coordinates of the atoms that are to be retained.
  'Z' Is an optional parameter which defines the serial number
 of the first atom in the specification immediately following it.
 For each atom thereafter in the current atom specification, the serial
 number is incremented by one to generate the output
 serial number.
 Atoms whose serial numbers are changed in this way must be referred to
 in subsequent directives by their new serial numbers.
 If 'Z' is not given, the atoms retain
 their old serial numbers.


If an  UNTIL  sequence is used after a  KEEP  directive has
 been given, it should be used with care, since the order of the new
 parameter list is different from the input list.





\bigskip\Directive{MOVE Z ATOM-SPECIFICATION  ...}


This directive moves atoms about in the list and places them
 in the position defined by the latest  AFTER  directive.
 (See the previous directive).
 This directive does not remove atoms from the list, but simply
 reorders the list.
 The sequence Z ATOM-SPECIFICATIONS can be repeated.


The atom specifications may contain symmetry operators, which are used
 to generate the coordinates of the atoms that are to be moved.
  'Z'  is an optional parameter which defines the serial number
 of the first atom in the specification immediately following it.
 For each atom thereafter in the current atom specification, the serial
 number is incremented by one to generate the output
 serial number.
 Atoms whose serial numbers are changed in this way must be referred to
 in subsequent directives by their new serial numbers.
 If no  'Z' is given, the atoms retain
 their old serial numbers.


If an  UNTIL  sequence is used after one or more  MOVE  directives have
 been given, it should be used with care, since the order of the new
 parameter list is different from the input list.





\bigskip\Directive{MULTIPLY  VALUE  PARAMETERS  ...}





\bigskip\Directive{PERTURB VALUE PARAMETERS ...}
  This directive perturbs the specified parameters using a rnadom
 number generator. The VALUE is the requested rms perturbation, in the
 natural units of the parameters. The mean deviation applied should be
 approximately zero, and the rms deviation applied should be
 approximately that requested.





\bigskip\Directive{REFORMAT}
This directive converts an old (non-FLAG) version of LIST 5 (see
section \ref{LIST05}) to the new format (extra parameters, old U[iso] 
slot now used as a flag and u[11] used for u[iso]).





\bigskip\Directive{RENAME ATOM1  ATOM2  (, ATOM1  ATOM2) ...}
 This directive requires pairs of atom specifications (optionally separated
 by a comma). The TYPE and SERIAL of 'atom1' are changed to those of 'atom2'.
 Atom1 must exist in LIST 5, atom2 must NOT exist in LIST 5. An atom can be
 renamed repeatedly. If atom1 contains symmetry operators, these are applied
 to the coordinates of the renamed atom. An atom cannot be renamed to itself
 in a single step.





\bigskip\Directive{RESET PARAMETER-NAME  VALUE ATOM LIST}


This directive assigns the given value to the named parameter
 for all the atoms in the atom list

\small\begin{verbatim}
      RESET OCC 1.0 ALL
      RESET OCC .5 O(1) O(2) O(3)
      RESET U[11] .05 C(27) UNTIL C(50)
\end{verbatim}\normalsize




\bigskip\Directive{ROTATE}
This directive rotates a group of atoms a certain number of degrees
around a specified vector. The rotation is carried out in orthogonal
space so preserves the geometry of the group.



There are three options available:
ROTATE D X Y Z VX VY VZ atom-specification  


ROTATE D ATOM1 VX VY VZ atom-specification  


ROTATE D ATOM1 ATOMS2   atom-specification  





The first rotates the specified atoms, D degrees around the vector
VX,VY,VZ keeping point X,Y,Z fixed. (X,Y,Z and VX,VY,VZ are given in
crystal fractions).



The second notation uses ATOM1 instead of X,Y,Z to specify the fixed point.



The third notation uses ATOM1 to specify the fixed point and the vector
from ATOM1 to ATOM2 to rotate around.



The rotation is D degrees anti-clockwise, when the specified vector
is pointing towards you.



1) Rotate the hydrogens of a methyl group by sixty degrees.
\small\begin{verbatim}
 \EDIT
 ROTATE 60 C(1) C(2) H(20) H(21) H(22)
 END
\end{verbatim}\normalsize


2) Turn a phenyl ring through 30 degrees around its external connecting
bond, c(1) to c(20).
\small\begin{verbatim}
 \EDIT
 ROTATE 30 C(1) C(20) C(21) C(22) C(23) C(24) C(25)
 END
\end{verbatim}\normalsize


3) Rotate a residue 90 degrees about the a-direction from its centroid, QC(1)
 (see also CENTROID and INSERT RESIDUE directives)
\small\begin{verbatim}
 \EDIT
 INSERT RESIDUE
 CENTROID 1 RESIDUE(1)
 ROTATE 90 QC(1) 1 0 0 RESIDUE(1)
 END
\end{verbatim}\normalsize




\bigskip\Directive{SELECT ATOM-PARAMETER  OPERATOR  VALUE, . .}


This directive selects and retains atoms with parameters satisfying
 the specified conditions.
 Only atoms that satisfy ALL the selection criteria, whether these
 are in the same or different directives, will be kept.
 All other atoms will be deleted from the list.


The operators allowed are :

\small\begin{verbatim}
            EQ            equal
            NE            not equal
            GT            greater than
            GE            greater than or equal to
            LT            less than
            LE            less than or equal to
\end{verbatim}\normalsize


 Examples of the  SELECT  directive are :

\small\begin{verbatim}
      SELECT SERIAL LT 50
      SELECT OCC GT 0.5, OCC LT 1.5
      SELECT C(1,X) LT 1., C(1,X) GT 0.
      SELECT TYPE NE Q
\end{verbatim}\normalsize


 This example will only retain atoms with serial numbers less than 50
 and occupancies between 0.5 and 1.5.
 The  'X'  parameter of atom c(1) must also lie between 0.0 and
 1.0 oterwise it will be rejected, and any atoms of type Q  will be deleted.





\bigskip\Directive{SHIFT  V1, V2, V3   ATOM-SPECIFICATION . .}
 This directive reads the three numbers of a shift vector, which must
 be in the same coodrinate system as the atomic parameters,
 and applies it to the parameters in the atom specification.
 This directive does not create new atoms, but simply modifies
 those already present.
 Any symmetry operators given are applied before the translation.





\bigskip\Directive{SORT TYPE1 TYPE2 ...}


\bigskip\Directive{SORT KEYWORD}
 This directive has two formats, and is used to sort the atoms stored in
 LIST 5 into a user-defined order.
 The default action sorts the atoms  on their types and serial numbers.
 The types are taken in the order found in LIST 5, and atoms of each
 type are grouped together. In each group the atoms are
 arranged by ascending serial number.
 The order of the types of atoms may also be determined
 by specifying them explicitly on the SORT directive,  or by a mixture of
 these methods.





In the second format, a keyword corresponding to an atom parameter
 name (as defined in LIST 5, see
 section \ref{LIST05}) is given, and the whole list sorted on
 increasing value of the specified parameter. Note that sorting on TYPE
 will give results depending on the 'collating sequence' of the computer.
 Fortunately, this generally leads to alphabetic sorting.


SORT sorts the whole list 5, and cancels any existing KEEP
 directives.





\bigskip\Directive{SPLIT Z ATOM-SPECIFICATION ...}
 Two new isotropic atoms are added to the end of the atom list for every
 atom referenced in the atom-specification. These atoms lie on
 of the principal axis of the original atoms anisotropic adp ellipsoid 
 and U[iso] set to U[meadian] of the original adp.


The original atoms are not deleted.
 The sequence Z ATOM-SPECIFICATIONS can be repeated.
 The new serial numbers are given by
\small\begin{verbatim}
      NEWSERIAL(1) = Z* OLDSERIAL and
      NEWSERIAL(2) = Z* OLDSERIAL +1
\end{verbatim}\normalsize




\bigskip\Directive{SUBTRACT  VALUE  PARAMETERS  ...}





\bigskip\Directive{SWAP  ATOM-PAIRS  ...}
 This directive swaps the atomic coordinates (x's and U's) for the two 
 named atoms.  The occupation numbers, Residue and Parts are left 
 unchanged.  





\bigskip\Directive{TRANSFORM  R11, R21, R31, . . . R33  ATOM SPECIFICATION . .}
 This directive reads the nine numbers of a transformation matrix, which must
 be separated by commas or spaces, and applies the matrix to the atoms given
 in the atom specification.
 This directive does not create new atoms, but simply modifies
 those already present.
 Any symmetry operators given are applied before the rotation.





\bigskip\Directive{TYPECHANGE KEYWORD OPERATOR VALUE NEW-ATOM-TYPE}


This directive conditionally changes the TYPES of atoms. If an atomic
 parameter selected by the keyword (see sort above) satisfies the
 conditions defined by the 'operator' and 'value' (see SELECT above),
 then the TYPE of the atom is changed to 'new-atom-type'.

\small\begin{verbatim}
      TYPECHANGE OCC GT 1.2 O
                              If Occ large, convert to oxygen
      TYPECHANGE U[ISO] LE 0.03 N
                              If Uiso small, convert to nitrogen
      TYPECHANGE TYPE EQ Q C                              Convert peaks (type Q) to carbon
\end{verbatim}\normalsize




\bigskip\Directive{UEQUIV  ATOM SPECIFICATIONS  .  .}
 The specified atoms to be converted so that they
 have isotropic temperature factors,
 U(equiv), defined by the SET UEQUIV command.
 IT IS NOT simply related to the
 diagonal elements of U(aniso).
 If an atom is already isotropic, no action is taken.
 If this directive is given with no arguments, all the atoms in the current
 atomic parameter list are converted to isotropic temperature factors.
 Physically impossible values are not rejected.
 Symmetry operators are ignored.












The following directives are used to convert atomic parameters into 
 the special shape parameters









\bigskip\Directive{SPHERE NEWSERIAL ATOMLIST}
This creates a 'shell' shape from the specified atom list. The centre of
the shell is at the centre of gravity, the size is the mean distance of
the given atoms from the centre, and the occupancy is equal to  the sum of
the occupancies
of the atoms listed. U[iso] is the mean of the U[iso] or Ueqiv of the
listed atoms.
 The atom TYPE is QS, with the given serial number. The
original atoms are not deleted, though they should be or their occupancy
set to zero. The atom type, QS, should be changed to something
appropriate.






\bigskip\Directive{RING NEWSERIAL ATOMLIST}
This creates an 'annulus' shape from the specified atom list. The centre of
the ring is at the centre of gravity, the size is the mean distance of
the given atoms from the centre, and the occupancy is equal to the sum of
the occupancies
of the atoms listed. U[iso] is the mean of the U[iso] or Ueqiv of the
listed atoms.
The atom TYPE is QR, with the given serial number. The
original atoms are not deleted, though they should be or their occupancy
set to zero. The atom type, QS, should be changed to something
appropriate. The DECLINATION and AZIMUTH are computed from the
constituent atoms.





\bigskip\Directive{LINE NEWSERIAL ATOMLIST}
This creates an 'line' shape from the specified atom list. The centre of
the line is at the centre of gravity, the size is twice the mean distance of
the given atoms from the centre, and the occupancy is equal to the sum of
the occupancies
of the atoms listed. U[iso] is the mean of the U[iso] or Ueqiv of the
listed atoms.
The atom TYPE is QL, with the given serial number. The
original atoms are not deleted, though they should be or their occupancy
set to zero. The atom type, QS, should be changed to something
appropriate. The DECLINATION and AZIMUTH are computed from the
constituent atoms.







\index{REGROUP}
\index{Reorganising atoms and peaks}
\section{Reorganisation of lists 5 and 10  -  $\backslash$REGROUP}


\small\begin{verbatim}
 \REGROUP INPUTLIST= OUTPUTLIST=
 SELECT MOVE= KEEP= MONITOR= SEQUENCE= SYMMETRY= TRANSLATION= GROUP=
 END
\end{verbatim}\normalsize



\small\begin{verbatim}
 \REGROUP
 SELECT MOVE=1.6,MONITOR=HIGH
 END
\end{verbatim}\normalsize




This routine offers a
 way of re-ordering the atoms in LIST 5 (atomic parameters) or LIST 10
 (Fourier peaks), so that
 related atoms or peaks
 form a sequential group in the list, and the coordinates put the atoms
 as close together as possible.


THIS ROUTINE DOES NOT USE LIST 29 (atomic properties) to get bonding 
 distances, but uses a single overall distance.


In this routine, a set of distances is calculated about
 each atom or peak in the list in turn.
 For each atom or peak in the list below the current pivot, the
 minimum contact distance is chosen, and if this is less than a user
 specified maximum, the atom or peak is moved up the list to
 a position directly below the pivot. ( The  MOVE  parameter).
 When more than one atom or peak is moved, their relative order is
 preserved as they are inserted behind the current pivot atom.
 As well as reordering the list, the necessary symmetry operators are
 applied to the positional and thermal parameters to bring the atom
 or peak into the same part of the unit cell as the current pivot
 atom. The result of this process is to bring related atoms together
 in the list, and to place all the atoms in the same part of the unit cell. 
 Setting the GROUP parameter to YES causes the PART to be incremented 
 between isolated parts of the structure.


\bigskip\Instruction{$\backslash$REGROUP INPUTLIST= OUTPUTLIST=}




\bigskip\Keyword{INPUTLIST=}

\small\begin{verbatim}
      5   -  Default value, the atomic coordinates
      10                    the Fourier peaks search
\end{verbatim}\normalsize




\bigskip\Keyword{OUTPUTLIST=}

\small\begin{verbatim}
      5   -  Default value, the atomic coordinates
      10                    the Fourier peaks search
\end{verbatim}\normalsize




\bigskip\Directive{SELECT MOVE= KEEP= MONITOR= SEQUENCE= SYMMETRY= TRANSLATION= GROUP=}


\bigskip\Keyword{MOVE=}
 This parameter has a default value of 2.0, and is the distance
 below which atoms or peaks are considered to be bonded, and are thus
 moved about the cell and relocated in LIST 5 (atomic parameters).



 If the MOVE parameter is -ve, then a covalent radius used, and the
 absolute value of MOVE is used as a TOLERANCE, such that bonds are formed
 if D $<$ COV1+COV2+TOLERANCE.








\bigskip\Keyword{KEEP=}
 This is the maximum number of atoms that the final output list can
 contain. If this parameter is omitted, all the atoms are output.
 If MOVE is used to move the atoms around, it is unwise to use the  KEEP
 parameter,since some of the original input atoms may find their way
 to the bottom of the list and be eliminated. (The default value is
 1000000).


\bigskip\Keyword{MONITOR=}

\small\begin{verbatim}
      LOW   -  Default value
      HIGH
\end{verbatim}\normalsize


 If  MONITOR  is  HIGH, then each
 pivot atom and its associated moved atoms are listed, as well as any
 deleted atoms. If MONITOR is LOW,
 the moved atoms are not listed.


\bigskip\Keyword{SEQUENCE}

\small\begin{verbatim}
      NO   -  Default value
      YES
      EXHYD
\end{verbatim}\normalsize


 If  SEQUENCE  is  YES, the outputlist is resquenced as described above.



 If  SEQUENCE  is  NO, the serial numbers of the atoms are not
 changed from the  original list.



 If SEQUECE is EXHYD the hydrogen atoms are excluded from the 
renumbering.


\bigskip\Keyword{SYMMETRY=}
 This parameter controls the use of symmetry information in the calculation of
 contacts, and can take three values.

\small\begin{verbatim}
      SPACEGROUP  -  Default value. The full spacegroup symmetry is used in
                                    all computations
      PATTERSON.     A centre of symmetry in introduced, and the translational
                     parts of the symmetry operators are dropped.
      NONE.          Only the identity operator is used.
\end{verbatim}\normalsize




\bigskip\Keyword{TRANSLATION=}
 This parameter controls the application of cell translations in the
 calculation of contacts, and can take the values YES or NO





\bigskip\Keyword{GROUP}

\small\begin{verbatim}
      NO   -  Default value
      YES
\end{verbatim}\normalsize


 If  GROUP  is  YES, the PART parameter for each atom is set.



\index{COLLECT}
\index{Repositioning atoms and peaks}
\section{Repositioning of atoms  -  $\backslash$COLLECT}



This routine changes the atom
 coordinates so as to form a 'molecule' using the covalent radii given in
 LIST 29 (atomic properties - see section \ref{LIST29}). The atom TYPE, SERIAL 
 and order in LIST 5 (atomic parameters - see section \ref{LIST05}) is not changed.





\bigskip\Instruction{$\backslash$COLLECT INPUTLIST= OUTPUTLIST=}




\bigskip\Keyword{INPUTLIST=}

\small\begin{verbatim}
      5   -  Default value, the atomic coordinates
      10                    the Fourier peaks search
\end{verbatim}\normalsize




\bigskip\Keyword{OUTPUTLIST=}

\small\begin{verbatim}
      5   -  Default value, the atomic coordinates
      10                    the Fourier peaks search
\end{verbatim}\normalsize




\bigskip\Directive{SELECT MONITOR= TOLERANCE= TYPE= SYMMETRY= TRANSLATION=}


\bigskip\Keyword{MONITOR=}

\small\begin{verbatim}
      LOW   -  Default value
      HIGH
\end{verbatim}\normalsize


 If  MONITOR  is  HIGH, then each
 pivot atom and its associated moved atoms are listed, as well as any
 deleted atoms. If MONITOR is LOW,
 only deleted atoms are listed.





\bigskip\Keyword{TOLERANCE=}
 The tolerance is added to the sum of the co-valent radii
 taken from LIST 29 (atomic properties - see section \ref{LIST29}) to give a 
 value used for determining inter-atomic bonds.
 The default is 0.2 A.





\bigskip\Keyword{TYPE=}

\small\begin{verbatim}
      ALL
      PEAKS
      ATOMS
\end{verbatim}\normalsize


 If TYPE equals ALL, then the coodinates of all atoms and Q-peaks 
 are liable to be
 modified by the symmetry operators in order to assemble a single fragment.



 If TYPE equals PEAKS, then only the peaks are moved to bring them as close
 as possible to existing atoms.



 If TYPE equald ATOMS, only non-Q atoms are modified





\bigskip\Keyword{SYMMETRY=}
 This parameter controls the use of symmetry information in the calculation of
 contacts, and can take three values.

\small\begin{verbatim}
      SPACEGROUP  -  Default value. The full spacegroup symmetry is used in
                                    all computations
      PATTERSON.     A centre of symmetry in introduced, and the translational
                     parts of the symmetry operators are dropped.
      NONE.          Only the identity operator is used.
\end{verbatim}\normalsize




\bigskip\Keyword{TRANSLATION=}
 This parameter controls the application of cell translations in the
 calculation of contacts, and can take the values YES or NO






\index{ORIGIN}
\index{Shifting the molecule to a permitted alternative origin}
\section{Shifting the molecule to a permitted alternative origin - $\backslash$ORIGIN}


\small\begin{verbatim}
 \ORIGIN INPUTLIST= OUTPUTLIST= MODE=
 END

\end{verbatim}\normalsize




Attempt to move the structure to the centre of the unit cell
 using the permitted origin shifts.


\bigskip\Instruction{$\backslash$ORIGIN INPUTLIST= OUTPUTLIST= MODE=}










\bigskip\Keyword{INPUTLIST=}

\small\begin{verbatim}
      5   -  Default value, the atomic coordinates
      10                    the Fourier peaks search
\end{verbatim}\normalsize




\bigskip\Keyword{OUTPUTLIST=}
\small\begin{verbatim}
      5   -  Default value, the atomic coordinates
      10                    the Fourier peaks search
\end{verbatim}\normalsize




\bigskip\Keyword{MODE=}
\small\begin{verbatim}
      CENTROID   -  Default value.
      FIRST
\end{verbatim}\normalsize




CENTROID tries to move the centroid of LIST 5 as close to .5 .5 .5
 as is permitted by the permitted origin shifts.  Other connected atoms follow 
 the centroid.


FIRST  As above excpet that the first atom in LIST 5 is the 
 target atom.  This may be a user-computed partial centroid.
\small\begin{verbatim}
      \edit
      cent 100 residue 3
      move qc(100)
      end
      \origin mode=first
\end{verbatim}\normalsize




 Currently (April 2011) the code only processes primitive triclinic, 
monoclinic and orthorhombic cells, using the tables in Direct Methods in 
Crystallography, Giacovazzo, Academic press, 1980, pp 74 and 76.




\index{CONVERT}
\index{Converting B[ij's] to U[ij's]}
\section{Conversion of temperature factors - $\backslash$CONVERT}


\small\begin{verbatim}
 \CONVERT INPUTLIST= OUTPUTLIST= CROSSTERMS=
 END

 \CONVERT
 END
\end{verbatim}\normalsize




This routine will convert the temperature factors of a set of atoms
 into the correct form when their temperature factor, t, is given by :

\small\begin{verbatim}
       T = exp(-B[iso]*S**2)     where s = sin(theta)/lambda.

 or for an anisotropic atom :

       T = exp(-(h*h*b[11] + k*k*b[22] + l*l*b[33]
           + k*l*2*b[23] + h*l*2*b[13] + h*k*2*b[12]))
\end{verbatim}\normalsize


 The cross terms stored in the original LIST 5 (the model parameters) 
 may either be  B[IJ]  or 2*B[IJ] .
 (The correct form of the temperature factor, in terms of u[ii]'s
 and u[ij]'s, is given in the section on the input of LIST 5).
 After conversion, the atoms are output to the disc as a new LIST 5.
 Remember that if U[ISO] is non-zero, (its default at atom input is 0.05)
 the U[IJ] are ignored and so will not be converted.


\bigskip\Instruction{$\backslash$CONVERT INPUTLIST= OUTPUTLIST= CROSSTERMS=}




This is the command which initiates the routine
 to convert the temperature factors.


\bigskip\Keyword{INPUTLIST=}

\small\begin{verbatim}
      5   -  Default value, the atomic coordinates
      10                    the Fourier peaks search
\end{verbatim}\normalsize




\bigskip\Keyword{OUTPUTLIST=}
\small\begin{verbatim}
      5   -  Default value, the atomic coordinates
      10                    the Fourier peaks search
\end{verbatim}\normalsize




\bigskip\Keyword{CROSSTERMS=}
\small\begin{verbatim}
      B[IJ]   -  Default value.
      2B[IJ]
\end{verbatim}\normalsize


\index{Hydrogen placement}
\index{HYDROGENS}
\section{Hydrogen placing - $\backslash$HYDROGENS}


\small\begin{verbatim}
 \HYDROGENS INPUTLIST= OUTPUTLIST=
 DISTANCE  D
 SERIAL    N
 U[ISO]    U
 U[ISO]    NEXT   MULT
 AFTER     TYPE(SERIAL)
 PHENYL    X R(1) R(2) R(3) R(4) R(5)
 H33       X R(1) R(2)
 H23       X R(1) R(2)
 H13       X R(1) R(2) R(3)
 H22       X R(1) R(2)
 H12       X R(1) R(2)
 H11       X R(1)
 HBOND     DONOR ACCEPTOR
 END
\end{verbatim}\normalsize



\small\begin{verbatim}
 \HYDROGENS
 DISTANCE  1.09
 U[ISO]    NEXT   1.2
 H33     C(7) C(6) R(5)
 H22     C(14) C(15) C(13)
 END
\end{verbatim}\normalsize


 This routine computes the coordinates of hydrogen atoms bonded to a
 target atom. The hybridisation of the target atom and the identifiers of
 atoms bonded to it must be given.


\bigskip\Instruction{$\backslash$HYDROGENS INPUTLIST= OUTPUTLIST=}




\bigskip\Keyword{INPUTLIST=}
\small\begin{verbatim}
      5   -  Default value, the atomic coordinates
      10                    the Fourier peaks search
\end{verbatim}\normalsize




\bigskip\Keyword{OUTPUTLIST=}
\small\begin{verbatim}
      5   -  Default value, the atomic coordinates
      10                    the Fourier peaks search
\end{verbatim}\normalsize




\bigskip\Directive{DISTANCE  D}
 This sets the central atom-hydrogen atom distance to
 'D' angstroms. The default value is 1.0.
 The current value of 'D' remains in force until another 'DISTANCE'
 directive is given.


\bigskip\Directive{SERIAL  N}
 This sets the serial number of the next hydrogen atom
 to be added to LIST 5 (atomic parameters) to 'N'.
 The default value is 1.
 Subsequent hydrogen atoms will have the serial numbers 'N+1', 'N+2', etc.,
 until the next 'SERIAL' directive is input.


\bigskip\Directive{U[ISO]    U}
 This directive sets the isotropic temperature factor
 of each hydrogen atom
 to 'U' angstroms squared,
 and remains in force until another 'U[ISO]'
 directive is given.
 If no values is given for U, the next definition is used.


\bigskip\Directive{U[ISO]    NEXT   MULT}
 This is an alternatine form of the preceding directive. It sets the
 isotropic temperature factor  of each hydrogen atom
 to 'MULT' times the equivalent temperature factor of the atom it is
 bonded to.  The default value is 1.2.
 The directive remains in force until another 'U[ISO]'
 directive is given.


\bigskip\Directive{AFTER   TYPE(SERIAL)}
 The hydrogen atoms generated by the placing routines
 are inserted in the new LIST 5 (atomic parameters) after the atom
 'TYPE(SERIAL)'.
 This directive must appear immediately after the directive that generated
 the hydrogen atom coordinates, and applies only to that group of
 hydrogen atoms.
 If no 'AFTER' directive is
 given, the new hydrogen atoms are added at the end of the
 current LIST 5 (atomic parameters).


\bigskip\Directive{PHENYL  X R(1) R(2) R(3) R(4) R(5)}
 This generates the coordinates of  the five hydrogen atoms
 of a phenyl group. The first atom specified must be
 the atom that bonds the phenyl group
 to the rest of the structure, and the other atoms must be in the order
 of connectivity.


\bigskip\Directive{H33  X R(1) R(2)}
 This geneates the hydrogen atoms of a methyl
 group.
 The methyl carbon is the first atom specified, and
 the hydrogen atoms are generated so that one of them is trans
 with respect to the third atom specified, R(2).

\small\begin{verbatim}
      H
       \
      H-X-R(1)-R(2)
       /
      H
\end{verbatim}\normalsize




\bigskip\Directive{H23  X R(1) R(2)}
 This generates the coordinates of two hydrogen atoms on an sp3 atom X.

\small\begin{verbatim}
      H   R(1)
       \ /
        X
       / \
      H   R(2)
\end{verbatim}\normalsize





\bigskip\Directive{H13  X R(1) R(2) R(3)}
 This generates the coordinates of one hydrogen atom on an sp3 atom X.

\small\begin{verbatim}
          R(1)
         /
     H- X-R(2)
         \
          R(3)
\end{verbatim}\normalsize





\bigskip\Directive{H22  X R(1) R(2)}
 This generates the coordinates of two hydrogen atoms on an sp2  atom X

\small\begin{verbatim}
      H        R(2)
       \      /
        X=R(1)
       /
      H
\end{verbatim}\normalsize





\bigskip\Directive{H12  X R(1) R(2)}
 This generates the coordinates of one hydrogen atom on an sp2 atom X.

\small\begin{verbatim}
        H
         \
          X=R(1)
         /
      R(2)
\end{verbatim}\normalsize




\bigskip\Directive{H11  X  R}
 This generates the coordinates of the single
 hydrogen atom bonded to an SP hybridised atom.


\bigskip\Directive{HBOND X  R}
 This generates a single H atom 'DISTANCE' angstroms from the donor
 in the direction of the acceptor. X is the donor, R the acceptor.

\small\begin{verbatim}

        X-H....R
\end{verbatim}\normalsize



\small\begin{verbatim}
 Place Hydrogen atoms on the following fragment:

      C(1)          C(5)
          \        /
           C(2)=C(3)
                   \
                    C(4)-Br(1)

     \HYDROGENS
      DISTANCE 0.99
      U[ISO]   0.06
      H33 C(1) C(2) C(3)
      AFTER C(1)
      H12 C(2) C(1) C(3)
      AFTER C(2)
      H23 C(4) Br(1) C(3)
      AFTER C(4)
      H33 C(5) C(3) C(4)
      END
\end{verbatim}\normalsize


\index{Hydrogen placement - automatic}
\index{PERHYDRO}
\section{Perhydrogenation - $\backslash$PERHYDRO}


\small\begin{verbatim}
 \PERHYDRO INPUTLIST= OUTPUTLIST=
 DISTANCE  D
 SERIAL    N
 U[ISO]    U
 U[ISO]    NEXT   MULT
 ACTION    MODE
 TYPE      C or N
 END
\end{verbatim}\normalsize



\small\begin{verbatim}
 \PERHYDRO
 U[ISO] NEXT 1.0
 END
\end{verbatim}\normalsize


 This command scans the atomic coordinates for carbon atoms, attempts
 to assign their hybridisation state (on the basis of bond lengths) and
 then generates $\backslash$HYDROGEN commands to create any necessary hydrogen atoms.
 Existing Hydrogen atoms are not replaced by this routine.


The generated commands may be processed internally by CRYSTALS without
 the user needing to see them, or they may be sent to the external files for
 later use. This is the default mode. If no new hydrogen atoms are 
 generated, no new external files are created.


The external files are called DELH.DAT and PERH.DAT, with DELH 
 containing an entry for every atom created by PERH.  Executing DELH and 
 PERH will delete existing named atoms, and recreate them geometrically.


\bigskip\Instruction{$\backslash$PERHYDRO INPUTLIST= OUTPUTLIST=}




\bigskip\Keyword{INPUTLIST=}
\small\begin{verbatim}
      5   -  Default value, the atomic coordinates
      10                    the Fourier peaks search
\end{verbatim}\normalsize




\bigskip\Keyword{OUTPUTLIST}
\small\begin{verbatim}
      5   -  Default value, the atomic coordinates
      10                    the Fourier peaks search
\end{verbatim}\normalsize




\bigskip\Directive{DISTANCE  D}
 This sets the central atom-hydrogen atom distance to
 'D' angstroms. The default value is 1.0.
 The current value of 'D' remains in force until another 'DISTANCE'
 directive is given.


\bigskip\Directive{SERIAL  N}
 This sets the serial number of the next hydrogen atom
 to be added to LIST 5 to 'N'.
 The default value is 1.
 Subsequent hydrogen atoms will have the serial numbers 'N+1', 'N+2', etc.,
 until the next 'SERIAL' directive is input.


\bigskip\Directive{U[ISO]    U}
 This directive sets the isotropic temperature factor
 associated with each hydrogen atom
 to 'U' angstroms squared.
 The default value is 0.05.
 The directive remains in force until another 'U[ISO]'
 directive is given.


\bigskip\Directive{U[ISO]    NEXT   MULT}
 This is an alternatine form of the preceding directive. It sets the
 isotropic temperature factor  associated with each hydrogen atom
 to 'MULT' times the equivalent temperature factor of the atom it is
 bonded to.  The default value is 1.2.
 The directive remains in force until another 'U[ISO]'
 directive is given.


\bigskip\Directive{ACTION    MODE}





\bigskip\Keyword{MODE}
\small\begin{verbatim}
      NORMAL
      PUNCH
      BOTH     -  Default value.
\end{verbatim}\normalsize


 NORMAL causes internal commands to be generated and executed. PUNCH causes
 output to the PUNCH file only. BOTH forces both actions.








\bigskip\Directive{TYPE    MODE}





\bigskip\Keyword{MODE}
\small\begin{verbatim}
      C   -  Default value.
      N
\end{verbatim}\normalsize


 C enables the program to place hydrogen atoms on carbon atoms.



 N enables the program to place hydrogen atoms on nitrogen.



It is advisable to perform placement on C before N, since the
hybridisation states of C are more clearly defined.









\index{HNAME}
\index{Hydrogen - automatic renumbering  }
\section{Hydrogen re-numbering - $\backslash$HNAME}


\small\begin{verbatim}
 \HNAME INPUTLIST= OUTPUTLIST=
 END
\end{verbatim}\normalsize



\small\begin{verbatim}
 \HNAME
 END
\end{verbatim}\normalsize


 This command automatically renumbers hydrogen atoms so that their
 serial numbers are related to the bonded non-hydrogen atom.



\index{REGULARISE}
\index{Regularisation of atomic groups}
\section{Regularisation of atomic groups - $\backslash$REGULARISE}


\small\begin{verbatim}
 \REGULARISE    MODE
 COMPARE
 KEEP
 REPLACE
 AUGMENT
 METHOD NUMBER
 GROUP NUMBER
 TARGET Atom Specifications
 IDEAL  Atom Specifications
 RENAME offset number
 CAMERON
 MAP Atom Specifications
 ONTO Atom Specifications
 SYSTEM a b c alpha beta gamma
 ATOM    x    y    z
 CP-RING x
 HEXAGON x
 OCTAHEDRON x y z
 PHENYL
 SQP x y z
 SQUARE x y
 TBP x z
 TETRAHEDRON x
 END
\end{verbatim}\normalsize






\small\begin{verbatim}
 \REGULARISE REPLACE
 GROUP 6
 TARGET C(1) UNTIL C(6)
 PHENYL
 END
\end{verbatim}\normalsize




This routine calculates a fit between the
 coordinates of a group of atoms in LIST 5 (atomic parameters) and another 
 group.
 The calculated fitting matrix may be used to compare the geometry of two
 groups, or it may be applied to transform the new coordinates which will
 then replace the existing group in LIST 5 (D. J. Watkin, Act Cryst
 (1980). A36,975).


In this section, the group of atoms in LIST 5 to whose coordinates the
 fit is made is referred to as the 'TARGET atoms', and the group to be fitted
 onto that group is referred to as the 'IDEAL atoms'.


The source of the 'IDEAL atoms' can be the LIST 5,
 a pre-stored idealised geometry,  or values read in from the directives.
 Those directives that refer to LIST 5 use the usual CRYSTALS formats for
 atom specifications. Once a transformation has been found, this can be 
 used as the basis for naming one fragment based on the names of 
 another.



\bigskip\Instruction{Input for REGULARISE}




The input to REGULARISE must define the groups to be fitted together,
 the method used for fitting , and the use to be made of the results.
 The user must ensure that corresponding atoms
 are specified in the same positions of the 'TARGET' and 'IDEAL' group
 definitions, so the program knows which pairs of atoms are to be
 matched.
   It is not necessary to have co-ordinates of every atom
 in the TARGET fragment. The inclusion of atom specifications
 for which coordinates do not exist in the parameter list indicates
 that the procedure must generate coordinates for these atoms.
  This allows the user to give a type and serial to new atoms
 created by the procedure.
 Any 'atoms' without coordinates are not included in the fitting
 process.


The maximum number of atom  IDENTIFIERS permitted on an TARGET or
 IDEAL directive is about 250. Note that an UNTIL sequence only 
 counts as two identifiers.
 The number of implied atoms permitted is very large.


The 'IDEAL' group may be given in various ways.
 For calculations
 on a single structure,  it may be extracted from the stored data
 in the same way as the 'TARGET' group.  In this case however, all the atoms
 must previously exist.
 Alternatively, explicit co-ordinates may be  given in
 a system defined  by the user,   or a predefined group may be used.
 In any case all the positional parameters of
 the atoms in the  'IDEAL' group will be known before the
 calculation begins. Finally, various pre-defined geometrical groups are
 available.


\bigskip\Instruction{Output from REGULARISE}




The output from REGULARISE  includes the fragment centroids, their
 sums and differences and the
 transformation fitting the IDEAL onto the TARGET.



\bigskip\Instruction{Method of calculation}





The centroid of each fragment is moved to the origin.
  The atomic
 coordinates are converted to an orthogonal system and rotated  to an
 'inertial tensor' system (to help condition the L.S. matrix).


The fitting calculation is either constrained to be a pure rotation-
 inversion, or is a free linear transformation (rotation-diltion).
 If requested, the pure rotation component of the calculated
 rotation-dilation matrix is extracted.

 The calculated  matrix is applied to the co-ordinated of the
 'IDEAL' group, which is then converted back to crystal fractions,
 for comparison with the TARGET.


\bigskip\Directive{WARNING}


The  3 by 3  transformation  matrices  generated  at  various
 stages  may  well  be  singular,  especially if  no rotation  is
 defined about one of the axes.  To combat possible problems with
 matrix inversion, a Moore-Penrose type matrix inverter is used.
 Even so, the user should be aware that there may be no unique solution
 to his problem. For example, when a planar fragment is fitted to an
 almost planar fragment one fit may involve inversion of the non-planar
 fragment. Inversion can be prevented by using Method 3.
 Note also that if almost planar groups are being fitted, the dilation
 factor perpendicular to the plane may be very large, and thus have an
 undesirable effect if applied to atoms far from the plane.





\bigskip\Instruction{$\backslash$REGULARISE    MODE}


 MODE is an  optional  parameter.


\bigskip\Keyword{MODE}
\small\begin{verbatim}
      COMPARE      -      Default value
      KEEP
      REPLACE
      AUGMENT
\end{verbatim}\normalsize




The effects are :-


COMPARE        The specified groups are only compared. The
 translations and rotations necessary to match the
 groups will be calculated but not applied.


KEEP           The specified groups will be compared and the
 calculated transformations applied. The TARGET atoms are
 kept, and atoms  whose parameters have been
 calculated will be stored at the end of the new LIST 5.
 NOTE. If KEEP is given as a keyword, it can be followed by an offset
 to be used for the new serial numbers


REPLACE        The specified groups will be compared and the
 calculated transformations applied. The new atoms whose parameters
 have been calculated will
 be placed at the end of LIST 5 and the old atoms deleted form the list.


AUGMENT        The specified groups will be compared and the
 calculated transformations applied. The TARGET atoms which actually
 exist in LIST 5 are retained unaltered.
 Parameters that have been  calculated for dummy atoms (represented by a
 name only in the TARGET list) will be placed at the end of the new LIST 5.


For REPLACE and KEEP the 'IDEAL' coordinates define the geometry to
 be preserved, i.e. the model, and the 'TARGET' coordinates specify where,
 in what orientation and with what atom identifiers the model is to be placed.
 That is, the TARGET structure is replaced by the IDEAL.





\bigskip\Directive{KEEP  Z}


\bigskip\Directive{COMPARE}


\bigskip\Directive{REPLACE}


\bigskip\Directive{AUGMENT}


These 4 directives override the option specified by the MODE parameter of
 the REGULARISE command. The next group calculated will be treated
 in the specified mode. See the description of MODE for
 details. If the mode is KEEP, an offset Z can be given to be added to 
ther SERIAL of kept atoms (default 0) otherwise there are no parameters.


\bigskip\Directive{METHOD NUMBER}


This directive selects the method for matching the
 groups by giving its number from the
 following list:-

\small\begin{verbatim}
      Number     Method
      ------     ------
       1        Rotation  component of  rotation-dilation
                matrix applied. ( default )
       2        Rotation-dilation  matrix  calculated and
                applied.
       3        Pure  rotation matrix  calculated  by the
                Kabsch method and applied. This algorithm
                preserves chirality.
       4        Enable improper rotation in Kabsch method
       5        Use identity matrix as rotation component
\end{verbatim}\normalsize




\bigskip\Directive{GROUP NUMBER}


This directive specifies the number of atoms in the groups to be matched.
 It should be the first directive for each group of atoms. The appearance
 of a second or subsequent GROUP directive in the input
 initiates the calculation for the previous group.


\bigskip\Directive{TARGET Atom Specifications}


This directive is used to specify the 'TARGET' group of atoms. The directive
 will carry a series of atom specifications which will define the positions
 of the 'TARGET' atoms and the names of any atoms to be created by the
 routine. Atoms which exist in LIST 5 and atoms to be created can appear in
 any order in the TARGET group , although the order should be such that
 corresponding pairs of atoms appear at the same relative positions in the
 'TARGET' and 'IDEAL' groups.


\bigskip\Directive{IDEAL Atom Specifications}


This directive is used to specify a group of 'IDEAL' atoms to be taken
 from the stored LIST 5. Every atom on this directive must exist.


\bigskip\Directive{SYSTEM a b c alpha beta gamma}


This directive is will change the co-ordinate system used to interpret any
 subsequent ATOM directives.


The initial co-ordinate system has orthogonal axes of unit length and is
 equivalent to :-
\small\begin{verbatim}
 SYSTEM  1.0  1.0  1.0  90.0  90.0  90.0
\end{verbatim}\normalsize




Values must be given for a', b', and c', the angles default to 90.0.


\bigskip\Directive{ATOM    x    y    z}
 This directive allows the cordinates of a single atom to be specified,
 in fractional co-ordinates in the current co-ordintate system. It must be
 followed by three decimal numbers which will be the X, Y, and Z coordinates
 of the atom.


\bigskip\Directive{RENAME offset number}
 This directive can only be used after previous directives have been 
 used to match one group onto another (REGULARISE COMPARE),
 and enables the use of the MAP 
 and ONTO directives. The MAP list of atoms is transformed by the 
 existing transformation matrix (which may have been computed from only a 
 few specified atoms). Each atom is then compared with the ONTO list, 
 and the TYPE and SERIAL of the MAP atom used to generate a TYPE and 
 SERIAL for the closest ONTO atom. 


\bigskip\Keyword{OFFSET}
      The serial numbers of the atoms in the group being re-named are 
  related to those of the master group by an increment of 'OFFSET'. The 
 default value is 100


\bigskip\Keyword{NUMBER}
 If the number of atoms supplied on the following MAP and ONTO 
 directives does not match NUMBER, a warning is printed.


\bigskip\Directive{CAMERON}
 This matches atoms as in RENUMBER, but only creates CAMERON files
 with atoms transformed into the common coordinate system.


\bigskip\Directive{MAP  Atom Specifications}
 This specifies the atoms whose TYPE and SERIAL are to be propogated 
 into the ONTO atoms. The atoms can be in any order.


\bigskip\Directive{ONTO  Atom Specifications}
 This specifies the atoms to be renamed. The atoms may be in any order 
 and have any TYPE, but there must be EXACTLY as many as on the MAP 
 directive. The atoms can have any TYPE, but must have unique SERIAL 
 numbers.



\bigskip\Directive{HEXAGON X}
 The 'IDEAL' group is a regular hexagon with
 a side of length 'X'. The default for x is 1.0.


\bigskip\Directive{PHENYL}
 The same as HEXAGON with a fixed side of 1.39.


\bigskip\Directive{CP-RING X}
 The 'IDEAL' group is a regular pentagon with
 a side of length 'X'. The default for x is 1.4.


\bigskip\Directive{SQUARE X Y}
 The 'IDEAL' group is a rectangle with atoms
 at (x,0,0) , (0,y,0) , (-x,0,0) , (0,-y,0) . The parameters
 X and Y specify the size of the group to be used.


\bigskip\Directive{OCTAHEDRON X Y Z}
 The 'IDEAL' group is an octahedron with
 atoms at (0,0,0) , (-x,0,0) , (0,y,0) , (x,0,0) , (0,-y,0) ,
 (0,0,z) , (0,0,-z).
 The parameters X, Y and Z specify the size of the octahedron.
 'z' defaults to 'y' defaults to 'x'
 defaults to '1.0'


\bigskip\Directive{SQP X Y Z}
 The 'IDEAL' group is a square pyramid with
 atoms at (0,0,0) , (x,0,0) , (0,y,0) , (-x,0,0) , (0,-y,0) ,
 (0,0,z).
 The parameters X, Y and Z specify the size of the octahedron.
 'z' defaults to 'y' defaults to 'x'
 defaults to '1.0'


\bigskip\Directive{TBP X Z}
 The 'IDEAL' group is a trigonal bipyramid with
 atoms at (0,0,0) , (x,0,0) , (-x/2,0.86603x,0), (-x/2,-0.86603x,0) ,
 (0,0,z) , (0,0,-z) . The parameters X and Z specify the scale in the
 xy plane  and z directions.


\bigskip\Directive{TETRAHEDRON X}
 The 'IDEAL' group is a regular tetrahedron with
 an atom at the centre.
 'x' is the distance in Angstrom from the
 centre to an apex and defaults to '1.0'


\bigskip\Directive{ORIGIN}


This directive is not yet implemented.





\bigskip\Instruction{Uses of $\backslash$REGULARISE}







\bigskip\Directive{1 - Extending a fragment to a complete molecule}





Three atoms of a phenyl group ( C(1), C(2) C((6)) have been located.
 Fill in the missing  atoms from a non-dilated idealised phenyl group.

\small\begin{verbatim}
      \REGULARISE AUGMENT
      GROUP 6
      METHOD 1
      \ C(3), C(4), and C(5) do not yet exist.
      TARGET C(1) C(2) C(3) C(4) C(5) C(6)
      PHENYL
      END
\end{verbatim}\normalsize







\bigskip\Directive{2 - Forcing a regular shape on a group of atoms}


A group of atoms is approximately  octahedral.
 Replace them by a (posibly dilated) regular octahedron.

\small\begin{verbatim}
      \REGULARISE REPLACE
      GROUP 7
      METHOD 2
      TARGET CO(1) N(1) N(2) N(3) N(4) N(5) N(6)
      OCTAHEDRON
      END
\end{verbatim}\normalsize




\bigskip\Directive{3 - Checking for an additional symmetry element}


Determine whether the two molecules in an asymmetric unit are
 related  by a  symmetry operation not  expected for the space  group.
 The matrix relating the molecules and the translation required  to
 make their centroids  coincide should display any additional
 (approximate)  symmetry  present. Remember that if one molecule is
 the enantiomer of the other, Method 3 will lead to an unsatisfactory
 fitting unless  one molecule is inverted, (by using the operator -1 in
 the atom specifications e.g. FIRST(-1) UNTIL C(23). This can be done
 even if the space group is non-centrosymmetric ).




\small\begin{verbatim}
      \REGULARISE COMPARE
      GROUP 16
      TARGET C(101) UNTIL N(102)
      IDEAL  C(201) UNTIL N(202)
      END
\end{verbatim}\normalsize




\bigskip\Directive{4 - Renaming a group of atoms}
 A second group of atoms is given new TYPES and SERIAL numbers 
 so that the atom names are related to  a previously named group.



 In the example, the user has identified two sets of four non-coplanar 
 atoms in each 
 group e.g. C(1) with Q(103), C(3) with  Q(99) etc. The transformation 
 is then used to map the whole of the first group (C(1) until O(25)) onto 
 the second group (Q(96) until Q(120)). Both of these groups must contain 
 the same number of atoms, but they may be in any order. Atom Q(103) will 
 be renamed to C(101), atom Q(100) to C(107) etc. Once all the atoms 
 have been renamed, the list could be sorted based on the serial numbers.



  The MAP group of atoms must correspond to (though not necessarily include)
  the IDEAL, and similarly the ONTO and TARGET groups.

\small\begin{verbatim}
      \REGULARISE
      GROUP 4
      IDEAL C(1) C(3) C(5) C(7)
      TARGET Q(103) Q(99) C(116) Q(100)
      RENAME 100
      MAP C(1) UNTIL O(25)
      ONTO Q(96) UNTIL Q(120)
      END
      \EDIT
      SORT SERIAL
      END
\end{verbatim}\normalsize




\bigskip\Directive{5 - Viewing matched molecules in CAMERON}
 This does the mapping as RENAME, but doesn't
 rename the atoms, just outputs CAMERON input files showing the two molecules
 superimposed.



  The MAP group of atoms must correspond to (though not necessarily include)
  the IDEAL, and similarly the ONTO and TARGET groups.  Use as follows:

\small\begin{verbatim}
   \REGULARISE
   group 16
   target C(10) until C(26)
   ideal  C(60) until C(76)
   cameron
   onto   C(1) until C(49)
   map    C(51) until C(99)
   end
\end{verbatim}\normalsize

 This produces a cameron.ini, regular.l5i and regular.oby which may be viewed
 by choosing Graphics-$>$Special-$>$Cameron (use existing...) from the menu.
 Then type "obey regular.oby" in Cameron to colour the molecules nicely.
 The TARGET and IDEAL are used to obtain the mapping. The atoms in MAP and ONTO
 are just the ones you want to be included.
 Don't read the atoms back into CRYSTALS when closing CAMERON - they're in
 orthogonal coordinates.





\bigskip\Directive{ - Comparing two structures}
 The SYSTEM and ATOM directives enable one to compare one structure with
atoms from a second structure. However, since the second structure is 
not part of the main model, CRYSTALS knows nothing about the 
connectivity.  Using the KEEP z directive, the second strcuture can be 
added to the DSC file, enabling a complete calculation to be performed.


 In the following example, O(16) is in a quite distinctly different 
position in the two structures, so place holder Q(16) is used during the 
first mapping. The input coordinates are added to the DSC file with 
SERIAL numbers off-set by 400.



In the second calculation O(16) of the original structure is compared 
with Q(416) of the input structure. 
\small\begin{verbatim}
\regular keep
keep 400
group 7
old mo(1) o(11) o(12) o(13) o(14) o(15) q(16)
system 8.4830 10.1870 11.0340 105.260 95.290 95.100 909.60
atom 0.1570 0.5269 0.2514
atom 0.1356 0.5975 0.1278  
atom -0.0296 0.4567 0.2632  
atom 0.2258 0.3448 0.1750  
atom 0.1693 0.6928 0.3850  
atom 0.4211 0.5669 0.2567  
atom 0.7960 0.1904 0.1727  


\regular compare
group 7
old mo(1) o(11) until o(16)
new mo(401) until q(416)
end
\end{verbatim}\normalsize






\index{MATCH}
\index{Map two atomic groups together}
\section{Map two atomic groups together - $\backslash$MATCH}


\small\begin{verbatim}
 \MATCH
 MAP Atom Specifications
 ONTO Atom Specifications
 RENAME n
 EQUALATOM
 METHOD
 END
\end{verbatim}\normalsize






\small\begin{verbatim}
 \MATCH
 METHOD 3
 MAP RESIDUE(1)
 ONTO RESIDUE(2)
 RENAME 100
 END
\end{verbatim}\normalsize




This routine  uses the mapping routines in REGULARISE to compare
two residues.  Unlike REGULARISE itself, the user does not have
to list the atoms in any special order - the routine attempts to make
pairwise assigments. Initially this is done via a topographical search,
and refined by minimising cartesian residuals whenever degeneracy is 
found.



\bigskip\Directive{MAP}


The following list of atoms is the ideal fragment with ideal atom 
types and numbers





\bigskip\Directive{ONTO}


The following list of atoms is the fragment ot be compared with the 
ideal.  There must be the same number of atoms in both fragments, but 
not necessarily in the same order. Inclusion of any Q atoms sets the 
EQUALATOM flag, ie the atom types are ignored.





\bigskip\Directive{EQUALATOM}


If this parameter is included, the atom types in the "ONTO" list are 
ignored - they can even be Q atoms





\bigskip\Directive{METHOD}


One of the METHODS available in REGULARISE


The default is method 4.





\bigskip\Directive{RENAME  n}


If the mapping is succerssful, atoms in the ONTO list are given the 
same type as the corresponding atom in the MAP list, and the same SERIAL 
plus "n".





\bigskip\Directive{OUTPUT LIST= PUNCH=}


lIST takes the values OFF, LOW, MEDIUM, HIGH.


unch takes the values OFF, RESULTS




Results creates an ASCII file that could be processed by EXCEL or 
other spreadsheet.
The fields are tab-deliminated, and have the following attributes.
\small\begin{verbatim}
CSD_CIF_BAZHAM01  (title)
Asymmetric  (potential internal symmetry)
:Centroids   (x,y,z)A and (x,y,z)B  
:Axes of Inertia   (long, medium, short)A and (long, medium, short)B 
:Sum dev sq  After matching, Sum(delta_x^2), y and z.  Sum Separation^2 
:RMS dev  As above, but root mean square deviation rather than sums  
:RMS bond and tors dev  RMS differences of equivalent bond lengths and torsion angles  
:Min and Max bond dev  Minimum and maximum bond length differences.  
:Min and Max tors dev   Minimum and maximum torsion angle differences. 
:Mean & delta Centroids  Mean(xA,xB), (yA,yB), (zA,zB) and (xA-xB), (yA-yB), (zA-zB)
:Transformation   (3x3) transformation matrix
:Det and trace   Determinant and trace of matrix
:Closeness to ideal rotation  RMS of the deviations of matrix elements from integers  
:Closeness to group operator  RMS of the deviations of translations from integers  
:Combined measure of closeness    
:Rworst & Raverage  Worst and Average for to an operator, operator identifier  
:Symmetry  Local point group relating molecules.
:Pseudo  Pseudo point group  relating molecules.
:Operator  Pseudo symmetry operator.
:No_Atoms  Number of atoms in each molecule.
:S.G.  Original space group.
:Cell  Original unit cell parameters.
\end{verbatim}\normalsize




\bigskip\Directive{Comparing lots of Z'=2 structures}


If one has a single cif file containing many Z'=2 structures, the whole 
 file can be processed structure-by-structure automatically, with 
 automatic matching of the structures. See
 Structure matching: measures of similarity and pseudosymmetry. 
 A.  Collins, R. I.  Cooper and D. J.  Watkin. 
 Journal of Applied Crystallography 2006;39(6):842-849.


To do this, extract the structures from the CSD saving the results to 
 a single file using the operation 'Import CIF file' from the drop-down 
 menu 'X-ray Data'.  Before starting, 
 ensure that a file titled "cifproc.dat" is in the same folder as the 
 composite cif file and that it says:

\small\begin{verbatim}
/edit
mon off
list off
insert resi
sel type ne h
end
/match
output punch=results
map resi(1)
onto resi(2)
end
/purge
end
\end{verbatim}\normalsize







\bigskip\Directive{Output from MATCH}


If the output is set to PUNCH=RESULTS, a summary of the result of 
each matching operation is copied to the Punch file (Usually BFILE.PCH)
All the information for each match is appended to a single line. The 
items are "tab deliminated" to facilitate reading them into 
spreadsheets. The keywords are preceeded by a ":" so that an editor can 
be used to break the line as necessary.


Description of output:


\small\begin{verbatim}
CSD_CIF_AANHOX01  
:Centroids    (the x,y and z coordinates of both centroids)
:Axes of Inertia    (the three axes of inertia of both molecules)
:Sum dev sq   (sum of the squares of the deviations in x,y,z and delta-d)
:RMS dev   (rms deviations in x,y,z and delta-d)
:RMS bond and tors dev     
:Min and Max bond dev     
:Min and Max tors dev   
:Sum & delta Centroids   
:Transformation    (matrix transforming one molecule to the other)
:Det and trace     (of the matrix - -ve indicates inversion)
:Closeness to ideal rotation   
:Closeness to group operator   
:Combined measure of closeness   
:Rworst & Raverage   
:Symmetry   
:Pseudo   
:Operator   
:No_Atoms   
:S.G.   
:Cell  
\end{verbatim}\normalsize







Example of output


\small\begin{verbatim}
CSD_CIF_AANHOX01 
:Centroids       0.9169 0.6281 0.3346 1.0255 0.8741 0.8117 
:Axes of Inertia 59.3690 10.0284 0.1392 59.1659 10.1380 0.0762 
:Sum dev sq      0.0014 0.0014 0.0726 0.0754 
:RMS dev         0.0115 0.0114 0.0812 0.0828 
:RMS bond and tors dev 0.0040 2.3978 
:Min and Max bond dev  0.0003 0.0094 
:Min and Max tors dev  0.2952 6.1112 
:Sum & delta Centroids 0.9712 0.7511 0.5731 0.1086 0.2461 0.4771 
:Transformation 0.9662 0.1296 -0.071 -0.237 0.961 0.146 -0.476 0.292 -0.960
:Det and trace -1.0000 0.9679 
:Closeness to ideal rotation 0.21507 
:Closeness to group operator 0.21567 
:Combined measure of closeness 0.21522 
:Rworst & Raverage 0.203432098 0.186725736 10 
:Symmetry m 
:Pseudo m 
:Operator 0.11+X 0.25+Y 1.15-Z 
:No_Atoms 11 
:S.G. P N A 21 
:Cell 7.5570 11.4580 17.6020 90.0000 90.0000 90.0000
\end{verbatim}\normalsize






\index{BONDCALC}
\index{Calculation of interatomic bonds}
\section{Calculation of interatomic bonds - $\backslash$BONDCALC}
\label{LIST41} 


\small\begin{verbatim}
 \BONDCALC
  END
\end{verbatim}\normalsize






\small\begin{verbatim}
 \BONDCALC FORCE
 END
\end{verbatim}\normalsize




This routine calculates a list of unique bonds between
atoms in LIST 5 including bonds to symmetry related atoms.
The bonds are stored in LIST 41.



\bigskip\Instruction{ Method of calculation}





The BONDCALC routine uses the atomic positions from LIST 5 (the
model parameters, see \ref{LIST05}) (together
with cell (LIST 1, see \ref{LIST01}) and spacegroup information 
(LIST 2, see \ref{LIST02}),
the covalent radii from LIST 29 (atomic properties, see \ref{LIST29}), and any 
additional bonding information in LIST 40 to calculate a list of bonds. 
The algorithm and tolerances used depend upon settings in LIST 40.



LIST 41 is only updated by $\backslash$BONDCALC if there has been a change
to LISTS 5 or 40 OR if $\backslash$BONDCALC FORCE is issued.





\index{$\backslash$LIST 40}
\section{Bonding information - $\backslash$LIST 40}
\label{LIST40} 


\small\begin{verbatim}
 \LIST 40
 DEFAULTS TOLTYPE= TOLERANCE= MAXBONDS= NOSYMMETRY= SIGCHANGE=
 READ NELEMENTS= NPAIRS= NMAKE= NBREAK=
 ELEMENT TYPE= RADIUS= MAXBONDS=
 PAIR TYPE1= TYPE2= MIN= MAX= BONDTYPE=
 MAKE TYPE= SERIAL= S= L= TX= TY= TZ=
      TYPE2= SERIAL2= S2= L2= TX2= TY2= TZ2= BONDTYPE=
 BREAK TYPE= SERIAL= S= L= TX= TY= TZ=
       TYPE2= SERIAL2= S2= L2= TX2= TY2= TZ2=
 END
\end{verbatim}\normalsize




\bigskip\Directive{DEFAULTS TOLTYPE= TOLERANCE= MAXBONDS= NOSYMMETRY= SIGCHANGE=}


 This directive may only appear once. It affects the algorithm used
to update LIST 41.


\bigskip\Keyword{TOLTYPE=}
 A value of 1 (default) causes $\backslash$BONDCALC to use as a threshold for
 bonding, the sum of the covalent radii * the tolerance given.
 A value of 0 causes $\backslash$BONDCALC to use the sum of the covalent radii + the
 tolerance given (in Angstroms), as a threshold.


\bigskip\Keyword{TOLERANCE=}
 The tolerance to be used in the $\backslash$BONDCALC calculation as a threshold
 for bonds. Exact use depends on the value of the TOLTYPE keyword above.


\bigskip\Keyword{MAXBONDS=}
 Specifies the maximum number of bonds that may be formed to an atom. The
 BONDCALC calculation proceeds through the list of atoms searching for bonds,
 according to the TOLERANCE criteria. If more than MAXBONDS bonds are found,
 the best MAXBONDS will be kept. Best bonds are those where the sum of the
 covalent radii is closest to the actual bond length. (Where a PAIR 
 directive has been used, the best are the closest to the mean of the min and 
 max values  on the PAIR directive.)
 Note well: The calculation proceeds through the list of atoms, so bonds
 are formed from atoms near the top of the list to those lower down. While
 atoms lower down will still only form at most MAXBONDS bonds, they are 
 less likely to be the 'best' bonds since they are formed from atoms higher
 up the list. E.g. You have an H right at the end of the list, and you
 set MAXBONDS=1 for H (see ELEMENT). If the first atom forms a bond to that
 H, then no more bonds can be formed to that H even if they are better.
 If the H were at the top of the list it would get the choice of which bonds
 to pick. This is fairly unimportant stuff, it is rare that there will
 be ambiguities over whether something is bonded or not. The default value
 of MAXBONDS is therefore 15.


\bigskip\Keyword{NOSYMMETRY=}
 0 (default) searches for all symmetry related bonds.
 1 ignores symmetry, will not find bonds across operators, may speed
   up bond bond calculation slightly.


\bigskip\Keyword{SIGCHANGE=}
 Number of angstroms that any atom in LIST 5 must move during refinement
 for it to be considered a significant change resulting in a recalculation
 of bonding.


\bigskip\Directive{READ NELEMENTS= NPAIRS= NMAKE= NBREAK=}
 Specify the number of ELEMENT, PAIR, MAKE and BREAK directives that are
 to follow.


\bigskip\Directive{ELEMENT TYPE= RADIUS= MAXBONDS=}
 Override the covalent radius in L29 and the MAXBONDS value on the DEFAULTS
 directive for a specific element.


\bigskip\Keyword{TYPE=}
 The element type. E.g. C


\bigskip\Keyword{RADIUS=}
 The covalent radius to use for this element.


\bigskip\Keyword{MAXBONDS=}
 The maximum number of bonds to this element.


DPAIR TYPE1= TYPE2= MIN= MAX= BONDTYPE=
 Override the covalent based calculation altogether.


\bigskip\Keyword{TYPE1=}
 An element type, e.g. C


\bigskip\Keyword{TYPE2=}
 An element type, e.g. O


\bigskip\Keyword{MIN=}
 The minimum length of a bond.


\bigskip\Keyword{MAX=}
 The maximum length of a bond.


\bigskip\Keyword{BONDTYPE=}
 The bondtype to be assigned to this bond. BONDCALC will eventually have a
 go at bond type assignment, if you are forced to add in extra PAIR
 commands then there is not much chance that the assignment will be correct
 so it can be specified here. Use 0 for unknown.


More than one pair of the same elements can be used at once:

\small\begin{verbatim}
e.g.
  PAIR C O 1.0 1.2 BONDTYPE=2
  PAIR C O 1.2 1.4 BONDTYPE=1
\end{verbatim}\normalsize




\bigskip\Directive{MAKE TYPE= SERIAL= S= L= TX= TY= TZ= TYPE2= SERIAL2= S2= L2= TX2= TY2= TZ2= BONDTYPE=}
Makes a bond between two atoms (possibly symmetry related).


\bigskip\Keyword{TYPE= TYPE2=}
An element type, e.g. C.


\bigskip\Keyword{SERIAL= SERIAL2=}
The serial number of the atom. (From List 5, atomic parameters).


\bigskip\Keyword{S= S2=}
The number of the symmetry matrix used from List 2 (list of space group 
symmetry operators, see section \ref{LIST02}) (default, unity = 1).
Negative indicates centre of symmetry applied aswell.


\bigskip\Keyword{L= L2=}
The number of the non-primitive lattice translation from List 2. (default =1, 
see section \ref{LIST02})


\bigskip\Keyword{TX= TY= TZ= TX2= TY2= TZ2=}
Translations from asymmetric unit co-ordinates.


\bigskip\Keyword{BONDTYPE=}
 The bondtype to be assigned to this bond. BONDCALC will eventually have a
 go at bond type assignment, if you are forced to add in extra MAKE
 commands then there is not much chance that the assignment will be correct
 so it can be specified here. Use 0 for unknown.


\bigskip\Directive{BREAK TYPE= SERIAL= S= L= TX= TY= TZ= TYPE2= SERIAL2= S2= L2= TX2= TY2= TZ2=}
As for MAKE, but without the BONDTYPE keyword.



\index{$\backslash$BONDING}
\section{Bonding information - $\backslash$BONDING}


\small\begin{verbatim}
 \BONDING ACTION
 DEFAULTS TOLTYPE= TOLERANCE= MAXBONDS= NOSYMMETRY= SIGCHANGE=
 ELEMENT TYPE= RADIUS= MAXBONDS=
 PAIR TYPE1= TYPE2= MIN= MAX= BONDTYPE=
 MAKE  atom-specification TO atom-specification bondtype
 BREAK atom-specification TO atom-specification
 END
\end{verbatim}\normalsize




THis is a more user-friendly alternative to inputting a LIST 40.
Directive syntax is like $\backslash$LIST 40 with the following exceptions:


1) ACTION. This can take two values:

\small\begin{verbatim}
      REPLACE (Default, and replace previous LIST 40 with a new on)
      EXTEND  (adds new commands to end of existing LIST 40)
\end{verbatim}\normalsize




2) The MAKE and BREAK directives look like this:

\small\begin{verbatim}
 MAKE C(1) TO C(4) 8
 BREAK N(1) TO H(14)
\end{verbatim}\normalsize




Symmetry may be specified in the standard CRYSTALS way, the numbers
in parenthesis are serial,S,L,Tx,Ty,Tz (see above) the list may be
truncated when the rest are default values: (serial,1,1,0,0,0):

\small\begin{verbatim}
 MAKE C(1,2,1,0,1,1) TO C(8) 4
\end{verbatim}\normalsize




3) The READ directive need not be given. This makes it easier to edit
text files containing the command as you don't have to remember to alter
the values on the READ directive.


3) The command may be given as $\backslash$BONDING EXTEND, in which case it takes
any directives given and adds them to the existing LIST 40.



\section{Printing of LIST 40}



LIST 40 may be listed with either


\bigskip\Instruction{$\backslash$PRINT 40}


 or


\bigskip\Instruction{$\backslash$SUMMARY 40}




LIST 40 may be punched with


\bigskip\Instruction{$\backslash$PUNCH 40}


which will produce a standard List 40 in CRYSTALS format, or


\bigskip\Instruction{$\backslash$PUNCH 40 B}

 
which will produce a $\backslash$BONDING command which is easier to edit.



\section{Creating a null LIST 40}



A null LIST 40, containing no extra information, may be created with

\small\begin{verbatim}
 \LIST 40
 END
\end{verbatim}\normalsize


or

\small\begin{verbatim} 
 \BONDING
 END
\end{verbatim}\normalsize


\section{Printing of LIST 41}



LIST 41 may be listed with either


\bigskip\Instruction{$\backslash$PRINT 41}


 or


\bigskip\Instruction{$\backslash$SUMMARY 41}




Issuing $\backslash$BONDCALC when there is no LIST 40 will cause a null list 40
to be created.






\chapter{Structure Factors And Least Squares}
\label{sfandls} \section{Scope of the structure factors and least squares section}






This section describes the necessary LISTS and explains how structure factor
 calculation and
 least squares refinement can be carried out.






 Refinement


 LIST 23 - Structure factor calculation control list (\ref{LIST23})


 SPECIAL - Special positions constraints (\ref{SPECIAL})


 LIST 12 - Input of refinement directives (\ref{LIST12})


 LIST 16 - Input of the restraints (\ref{LIST16})


 LIST 4 - Weighting the reflections (\ref{LIST04})


 LIST 28 - Reflection restriction list (\ref{LIST28})


 CHECK - Checking the refinement and restraint directives


 SFLS - Structure factor least squares calculations (\ref{SFLS})


 ESD - Creates an esd list, LIST 9 (\ref{LIST9})


 ANALYSE - Systematic comparisons of Fo and Fc


 DIFABS - Least squares absorption correction





 Internal workings





 LIST  9 - Parameter esds (\ref{LIST9})


 LIST 22 - Refinement parameter map (\ref{LIST22})


 LIST 17 - Input of the special restraints (\ref{LIST17})


 LIST 11 - The least squares matrix (\ref{LIST11})


 LIST 24 - The least squares shift list (\ref{LIST24})


 LIST 26 - Restraints in internal format (\ref{LIST26})



\index{Refinement}
\section{Refinement}






Before a stucture factor-least squares calculation is performed, the
 following lists must exist in the .DSC file





 LIST  1  Cell parameters (section \ref{LIST01})


 LIST  2  Symmetry inforamation (section \ref{LIST02})


 LIST  3  Scattering factors (section \ref{LIST03})


 LIST  4  Weighting scheme


 LIST  5  Atomic and other model parameters (section \ref{LIST05})


 LIST  6  Reflection data (section \ref{LIST05})


 LIST 12  Refinement definitions (section \ref{LIST12})


 LIST 16  Restraints (section \ref{LIST16})


 LIST 17  Special position restraints (section \ref{LIST17})


 LIST 18  SMILES string (section \ref{LIST18})


 LIST 23  Structure factor control list (section \ref{LIST23})


 LIST 25  Twin laws, only for twinned refinements (section \ref{LIST25})


 LIST 28  Reflection control list (section \ref{LIST28})





 LISTS 12,16 and 17 (constraints \ref{LIST12}, restraints 
  and special restraints  \ref{LIST16}RLIST17

) 
are not required if structure factors 
are only going to be calculated.



 The refinement directives specify which model
 parameters are to be refined, and the control directives control the
 terms in the minimisation function.


During structure factor least squares calculations,
 the partial derivatives with respect to each of the parameters is
 calculated for each structure factor and added into the 'normal
 equations'.
 This system of equations may be represented in matrix notation as :

\small\begin{verbatim}
       A.x = b
       WHERE :
       A      'A' is a symmetric n*n matrix. an element
              'A(i,j)' of 'A'is given by :
               A(i,j) = Sum [ w(k)*Q(i,k)*Q(j,k) ] over k.
       n       number of parameters being refined.
       k       indicates reflection number 'k'.
       w(k)    weight of reflection k.
       Q(i,k)  the partial differential of Fc(k) with
               respect to parameter i.
       x       a column vector of order n, containing
               the shifts in the parameters.
       b       also a column vector, an element
               of which is given by :
               b(i) = Sum [ w(k)*DF(k)*Q(i,k) ] over k.
       DF(k)   delta for reflection k, given by :
               DF(k) = [Fo(k) - Fc(k)]

\end{verbatim}\normalsize




As the matrix  A  is symmetric, only (n(n+1))/2 of its elements
 need to be calculated and stored, together with a few house keeping items.


In some cases, because of either storage or time considerations,
 it is impractical to use the full normal matrix  A .
 In this situation, it is necessary to use a 'block diagonal approximation'
 to the full matrix, in which interactions between parameters which are known
 not to be highly correlated are ignored.
 The effect of ignoring such interactions is to leave blank areas of
 the full matrix, related symmetrically across the diagonal, which
 do not need to be stored or accumulated.
 A common (but not very efficient or stable)
 example of this approach is to place one atom in each of the blocks
 used to approximate the normal matrix, so that each block is of order
 either 4 (x, y, z and u[iso]) or 9 (x, y, z and the anisotropic
 thermal parameters).


One of the main purposes of the refinement directives is to
 describe the areas of the matrix  A  that are to be calculated.
 If the matrix  A  is approximated by  m  blocks of order
 n(1), n(2),.....n(m),
 The total amount of memory needed to hold the matrix
 and vector is:

\small\begin{verbatim}
      Elements = 12 + 4*m + Sum n(i)*(5 +n(i))/2,

                                                 i = 1 to m

      Currently (June 2003) elements=8,388,608, giving over 4000
 parameters in a single block.
\end{verbatim}\normalsize




The formation of the blocks that are to be used to
 approximate the normal matrix  A  is controlled in the refinement
 directive list by a series of  BLOCK  directives, each of which
 contains the coordinates that are to be included in the newly
 specified block.
 Further control instructions for the current block may appear on
 subsequent directives until a new  BLOCK  directive is found, when the formation
 of another block with its associated parameters is started.


Two special directives are provided to allow for the most common
 cases required, full matrix refinement (a  FULL  directive)
 and one atom per block (a  DIAGONAL  directive).
 For all these cases only the parameters specified on the control directives
 and the following directives are refined.


\bigskip\Directive{Correlations in Refinement}


Highly correleated parameters {\bf MUST} be refined together. Refining
 them in different cycles or different blocks will lead to an incorrect
 structure.


As a rough guide, the following groups of parameters are in general
 highly correlated and should be refined in the same block if possible :

\small\begin{verbatim}
 1.  Temperature factors, scale factors, the extinction
     parameter, the polarity parameter and the
     enantiopole parameter.
 2.  Coordinates of bonded atoms.
 3.  Non-orthogonal coordinates of the same atom.
 4.  U(11), U(22) and U(33) of the same atom.
\end{verbatim}\normalsize




If it is necessary to split the temperature factors and scale
 factor into different blocks, their interactions must not be neglected
 but must be allowed for by using a 'dummy overall isotropic temperature
 factor'.
 In this case, the scale factor and the dummy temperature factor
 must be put into a block of order 2 by themselves, and the program
 will make the appropriate corrections to all the temperature factors.
 This dummy temperature factor should not be confused with the
 'overall temperature factor' which is a temperature factor that
 applies to all the atoms and is therefore just a convenience and requires
 no special treatment.


For further details,  Computing Methods in
 Crystallography, edited by J. S. Rollett, page 50, and Crystalographic
 Computing, ed Ahmed, 1970,  page 174.



Although it is possible to input an overall temperature factor as one
 of the overall parameters, it is not possible to use it under all
 circumstances. The structure factor routines always take the
 temperature factor of an individual atom as the value or values stored
 for that atom. If the overall temperature factor is to be refined, the
 system will ensure that the current value of the overall temperature
 factor is inserted for the temperature factor of all the atoms. When the
 new parameters are computed after the solution of the normal equations,
 this substitution is again made, so that all the atoms have the same
 overall isotropic temperature factor. However, if the overall temperature
 factor is not refined, or no refinement is done, the individual temperature
 factor for each atom will be used, and the overall temperature factor ignored.


\bigskip\Directive{CAUTION}


It should be noted that if a set of anisotropic atoms are input with no
  U[ISO]  key and  U[ISO]  data, then the default value of 0.05 will be
 inserted by the sfls routines. This implies that all such atoms are
 isotropic, so that the anisotropic temperature factors will be set to
 zero, and the calculation will proceed for isotropic atoms.





\bigskip\Directive{F or Fsq refinement?}
 Both type of refinement have been available in CRYSTALS since the early
 70's. For most data sets, there is little difference between the two
 correclty weighted refinements. One of the current reasons for choosing
 Fsq refinement is 'so that -ve observations may be used'. Such a choice
 is based on the misapprehension that the moduli in /Fo/ are the result
 of taking the square root of Fsq. In fact, it indicates that the phase
 cannot be observed experimentally. The experimental value of Fo takes
 the sign of Fsq and the positive square root. With proper weighting,
 both refinemets converge to the same minima
 (Rollett, J.S., McKinlay,T.G. and Haigh, N.P.,  1976, Crystallographic
 Computing Techniques, pp 413-415,  ed F.R.
 Ahmed,Munksgaard; and Prince,E. 1994, Mathematical Techniques in
 Crystalography and Materials Science, pp 124-125.Springer-Verlag).
 However, the path to the
 minima will be different, and there is some evidence that Fsq refinement
 has less false minima. Using all data, including -ve observations,
 increases the observation:variable ratio, but it is not evident that a
 large number of essentially unobserved data will improve the refinement.
 If the difference between F and Fsq refinement is significant, then the
 analysis requires care and attention.





\bigskip\Directive{Hydrogen Atom Refinement}


Several strategies are available for refining hydrogen atoms.
 Which you use is probably a matter of taste.


\bigskip\Keyword{Geometric re-placement}
 The command $\backslash$HYDROGEN or $\backslash$PERHYDRO is used to compute geometrically
 suitable positions for the H atoms. These are {\bf not} refined (either
 they are left out of LIST 12, or a fixed with the FIX directive). After
 a few cycles of refinement of the remaining parameters, they are deleted
 ($\backslash$EDIT $<$cr$>$ SELECT TYPE NE H) and new positions computed. This ensures
 optimal geometry, ensures that Fcalc is optimal, but avoids the cost of
 including the deviatives in the normal matrix.


\bigskip\Keyword{Riding hydrogens}
 As above, the hydrogens are placed geomtrically, but they are included
 in the formation of the least squares matrix. Their derivatives are
 added to those of the corresponding carbon, and a composite shift
 computed for each carbon and its riding hydrogens. This preserves the
 C-H vector, but can distort C-C-H angles. A cycle of refinement takes
 almost twice as long as the re-placement method.


\bigskip\Keyword{Restrained hydrogens}
 In this method, starting positions are hound for the hydrogen atoms
 (either from Fourier maps of geometrically), and the hydrogen positions
 are refined along with other atoms. The C-H distances and C-C-H angles
 are restrained to acceptable values in LIST 16. This calculation is even
 slower than the riding model, and would normally only be applied to an
 atom of special significance ( \emph{e.g.} a hydrogen bond H atom).


\bigskip\Keyword{Free refinement}
 The hydrogen atom is treated like any other atom. Requires good data,
 and may be applied to atoms of special interest.





Note that the different methods can be mixed in any way, with some
 hydrogens placed geometrically, and others refined.





\bigskip\Directive{R-Factor and minimisation function definitions}


\bigskip\Directive{Conventional R-value}
 This is defined as:

\small\begin{verbatim}
      R = 100*Sum[//Fo/-/Fc//]/Sum[/Fo/]
\end{verbatim}\normalsize


 The summation is over all the reflections accepted by LIST 28. This
 definition is used for both conventional and F-squared refinement.


\bigskip\Directive{Weighted R-value}
 The Hamilton weighted  R-value  is defined as :

\small\begin{verbatim}
      100*Sqrt(Sum[ w(i)*D'(i)*D'(i) ]/SUM[ w(i)*Fo'(i)*Fo'(i) ])

      D'  = Fo'-Fc'
      Fo' = Fo for normal refinement, Fsq for F-squared refinement.
      Fc' = Fc for normal refinement, Fc*Fc for F-squared refinement.
\end{verbatim}\normalsize




\bigskip\Keyword{Minimisation function}
 This is defined by :

\small\begin{verbatim}
      MINFUNC = Sum[ w(i)*D(i)*D(i) ]

      D', Fo', Fc' defined above.
\end{verbatim}\normalsize




\bigskip\Keyword{Residual}
 The residual and weighted residual are defined by:

\small\begin{verbatim}
      residual = Sum D'(i)**2
      weighted residual = Sum w(i)*D'(i)**2
\end{verbatim}\normalsize






\index{Special positions}
\section{Special positions}



The second major purpose of the refinement directives is to
 allow for atoms on special positions.
 For example, the atom at the Wyckoff site  H  in the space group
 P6(3)/mmc (no. 194) has coordinates  X,2X,Z .
 In a least squares refinement, the  X  and  Y  coordinates of this atom
 must be set to the same variable, i.e. they become equivalent.

 The command
 $\backslash$SPECIAL (section \ref{SPECIAL}) can be used to generate the necessary 
 constraints or restraints,
 and may be invoked automatically before structure factor calculations
 by setting the appropriate parameters in  LIST 23 (structure factor
 control settings, see section \ref{LIST23})

 The user can do this manually via  the refinement directives, LIST 12.
 The relationship is set up
 by an  EQUIVALENCE  directive, which sets all the parameters
 on the directive to the same least squares parameter.
 In this example, it is also necessary to alter the contribution
 of the  Y  coordinates to the normal matrix by multiplying the derivatives
 by 2.
 This facility is provided by the  WEIGHT  directive, which should not be confused
 with the weight ascribed to each reflection in
 the refinement.
 For a full treatment of atoms on special positions, see
 Crystallographic Computing, edited by F. R. Ahmed, page 187,
 or Computing Methods in Crystallography, page 51.


Similar relationships also hold for the anisotropic temperature factors.


The relationships between the variable  parameters in a refinement
 may also be defined by RESTRAINTS. These are held in LIST 17 (see \ref{LIST17}),
 and are particularly usefull if a complex matrix has been defined (e.g.
 using RIDE, LINK, EQUIVALENCE, WEIGHT, BLOCK, GROUP or COMBINE).

\index{Refinement - atomic parameters}
\section{Atomic parameter refinement}



Atomic parameters may be specified in three different ways.
 Firstly, there is an  {\bf IMPLICIT}  definition, in which parameters for all the
 atoms are specified simply by giving the appropriate key or keys.


 Hydrogen atoms are automatically excluded from implicit definitions.


Secondly, there is an  {\bf EXPLICIT}  definition, in which the parameters of one
 atom are specified by giving the atom name followed by the appropriate
 keys.


 Lastly, the parameters for a continuous group of atoms in LIST 5 may be
 specified by an UNTIL sequence.
 This type of parameter definition is taken to be implicit.


\bigskip\Directive{KEY[1] . . . KEY[K]}
 parameters defined by the keys KEY[1] . . KEY[K]
 are included (or excluded) for all the atoms in LIST 5,
 e.g. X U[ISO]  implies that the 'X' and 'U[ISO]'
 parameters of all the atoms in the current LIST 5
 will be used. This is an {\bf implicit} definition,
 since parameters for
 all the atoms in LIST 5 are specified simply by
 giving the appropiate key.


\bigskip\Directive{TYPE(SERIAL,KEY[1], . . ,KEY[K])}
 parameters defined by the keys KEY[1] . . . KEY[K]
 are included (or excluded) for the atom of type 'TYPE' with
 the serial number 'SERIAL', e.g. C(21,X,U[ISO])
 implies that the 'X' and 'U[ISO]' parameters of
 atom C(21) will be used.
 This is an  {\bf explicit} definition.


\bigskip\Directive{TYPE1(SERIAL1,KEY[1], . ,KEY[K])  UNTIL  TYPE2(SERIAL2)}
 the parameters defined by the keys KEY[1] . . KEY[K]
 are included (or excluded) for atoms in LIST 5 starting at the
 atom with type 'TYPE1' and serial 'SERIAL1', and
 finishing with the atom of type 'TYPE2' and
 serial 'SERIAL2'.
 This definition is {\bf implicit,}
 since the number of atoms
 included by this definition depends on the number
 and order of the atoms in LIST 5.


Parameter definitions of all three types may appear on any directive
 in any desired combination.

\small\begin{verbatim}
 EXAMPLE
       LIST 5 contains  FE(1) C(1) C(2) C(3) C(4) C(5) C(6) N(1)

       \LIST 12
       BLOCK X'S C(1,U[ISO]) UNTIL C(6) FE(1,U'S)
       END

      This refines x,y,z of all atoms, u[11]...u[12] of iron, and
      u[iso] of the other atoms.
\end{verbatim}\normalsize




The following parameter keys may be given in an atom
 definition :

\small\begin{verbatim}
 OCC     X       Y         Z
 U[ISO]  SIZE    DECLINAT  AZIMUTH
 U[11]   U[22]   U[33]     U[23]    U[13]    U[12]

 X'S      Indicating  X,Y,Z
 U'S      Indicating  U[11],U[22],U[33],U[23],U[13],U[12]
 UII'S    Indicating  U[11],U[22],U[33]
 UIJ'S    Indicating  U[23],U[13],U[12]
\end{verbatim}\normalsize


\index{Refinement - overall parameters}
\section{Overall parameter refinement}



Overall parameters, apart from the layer scale factors and the
 element scale factors, are specified  simply by their keys.
 Such a specification is considered to be an explicit definition.
 The following overall parameter keys may be given :

\small\begin{verbatim}
 SCALE       OU[ISO]       DU[ISO]
 POLARITY    ENANTIO       EXTPARAM
\end{verbatim}\normalsize


\section{Scale factor definitions}



The OVERALL scale factor is always applied to the structure factor
 calculation, though it need not necessarily be refined.
 LAYER and BATCH  scale factors are applied only if indicated in LIST 23
 (structure factor control settings, see section \ref{LIST23}),
 and ELEMENT scales only if the crystal is marked as being twinned in LIST
 13. Note that all of these scale factors can be expected to be correlated
 with each other, and the overall parameters.


The layer scale factors, batch scale factors
  and the element scale factors may be
 given in three different ways, all of which are
 considered to be explicit :


\bigskip\Directive{LAYER(M), BATCH(M)   OR   ELEMENT(M)}
 this indicates only scale factor 'M' of the specified type.
 'M' must be in the correct range, which for 'N' layer scale factors
 is 0 to 'N-1', and for 'N' element scale factors is
 1 to N.


\bigskip\Directive{LAYER(P) UNTIL LAYER(Q)   OR   BATCH(P) .....}
 this indicates all the scale factors of the specified type from
 'P' to 'Q'.
 'P' and 'Q' must be in the correct range, as defined for 'M'
 in the previous section.


\bigskip\Directive{LAYER SCALES, BATCH SCALES   OR   ELEMENT SCALES}
 this indicates all the scale factors of the given type.



\index{Structure factor calculation control}
\index{LIST 23}
\section{Structure factor calculation control list  -  LIST 23}
\label{LIST23} 


\small\begin{verbatim}
 \LIST 23
 MODIFY ANOM= EXTINCT= LAYERSCALE= BATCHSCALE= PARTIAL= UPDATE= ENANTIO=
 MINIMISE NSINGULARITY= F-SQUARED= REFLECTIONS= RESTRAIN=
 REFINE  SPECIAL= UPDATE= TOLERANCE=
 ALLCYCLES MIN-R= MAX-R= *-WR= *-SUMSQ= *-MINFUNC= U[MIN]=
 INTERCYCLE MIN-DR= MAX-DR= *-DWR= *-DSUMSQ= *-DMINFUNC=
 END
\end{verbatim}\normalsize



\small\begin{verbatim}
 \LIST 23
 MODIFY EXTINCTION=YES, ANOMALOUS=YES
 END
\end{verbatim}\normalsize







This LIST controls the structure factor calculation. The
 default calculation involves the minimum of computation (atomic
 parameters and overall sale factor).  More extensive calculations have
 to be indicated by entries in this list. The presence of a parameter in
 the parameter list (LIST 5) does not automatically mean that it will be
 included in the structure factor calculation.


This list also controls the treatment of atoms on special positions,
 the use of F or Fsq, and the use of restraints.


The presence of information in the DSC file does
 not ensure that it will be used by the structure factor
 routines. Thus, the operations corresponding to
  RESTRAIN ,  ANOMALOUS ,  EXTINCTION ,  PARTIAL ,  BATCHSCALES,
 LAYERSCALES and ENANTIO are not performed unless they are explicitly
 asked for in a LIST 23.





\bigskip\Instruction{$\backslash$LIST 23}




\bigskip\Directive{MODIFY ANOM= EXTINCT= LAYERSCALE= BATCHSCALE= PARTIAL= UPDATE= ENANTIO=}


This directive controls modifications that can be applied to Fo
 and Fc.


\bigskip\Keyword{ANOMALOUS=}
\small\begin{verbatim}
      NO   -  Default value
      YES
\end{verbatim}\normalsize


 If  ANOMALOUS  is  YES , the imaginary part of the anomalous dispersion
 correction, input in LIST 3 (see section \ref{LIST03}, will be included in 
 the s.f.l.s. calculations.
 For computational efficiency, it is recommended only to use the value YES
 towards the end of a refinement. Note that the value YES should be used
 even for centro-symmetric crystals if they contain a heavy atom.


\bigskip\Keyword{EXTINCTION=}

\small\begin{verbatim}
      NO   -  Default value
      YES
\end{verbatim}\normalsize


 If  EXTINCTION  is  YES , the calculated structure factors are modified
 to allow for the effects of extinction by the method of A. C. Larson.
 See Atomic and Structural Parameters for the definition.





\bigskip\Keyword{LAYERSCALES=}


\bigskip\Keyword{BATCHSCALES=}
 SCALE keys have two alternatives:

\small\begin{verbatim}
      NO   -  Default value
      YES
\end{verbatim}\normalsize


 If either SCALE key is  YES , the corresponding scale factors stored in
 LIST 5 (the model parameters) are
 applied to the reflection data. If this parameter is omitted, the
 scale factors are not applied, even if they exist in LIST 5


\bigskip\Keyword{PARTIAL=}

\small\begin{verbatim}
      NO   -  Default value
      YES
\end{verbatim}\normalsize


 If  PARTIAL  is  YES , the fixed partial contributions stored in LIST 6 
 (section \ref{LIST06}) are
 added in during the calculation of Fc and the phase.
 The partial contributions must already be present in LIST 6, and should
 have the keys  A-PART  and  B-PART . The atoms which have contributed to
 the partial terms should be omitted from LIST 5 whenever  PARTIAL  is
  YES .


\bigskip\Keyword{UPDATE=}

\small\begin{verbatim}
      NO   -  Default value
      YES
\end{verbatim}\normalsize


 If  UPDATE  is  YES , the contributions of the atoms to  A  and  B  are
 output to LIST 6 with the keys  A-PART  and  B-PART . If  UPDATE  is
  NO , its default value, the partial contributions are not changed
 during structure factor calculations.
 This requires that LIST 6 contain the keys A-PART and B-PART.


\bigskip\Keyword{ENANTIO=}
\small\begin{verbatim}
      NO   -  Default value
      YES
\end{verbatim}\normalsize


 If ENANTIO is YES, then Fc is computed with
\small\begin{verbatim}
      Fc = SQRT( (1-x)*F(h)**2 + x*F(-h)**2 )
\end{verbatim}\normalsize


 Where x is the enantiopole parameter from LIST 5. Once the correct
 enantiomer has been established, set this parameter back to NO.


\bigskip\Directive{MINIMISE= NSINGULARITY= F-SQUARED= REFLECTIONS= RESTRAIN= }


This directive controls modifications made to the minimisation
 function during s.f.l.s.


\bigskip\Keyword{NSINGULARITY=}
 The default value is zero.
 If this parameter is omitted, any singularities  discovered during the
 inversion of the normal matrix will cause the program to
 terminate after the current cycle of refinement.
 If  NSINGULARITY  is greater than zero, it represents the number of
 singularities allowed before the program will terminate.


\bigskip\Keyword{F-SQUARED=}
\small\begin{verbatim}
      NO   -  Default value
      YES
\end{verbatim}\normalsize


 If F-SQUARED is NO, the traditional minimisation function is:
\small\begin{verbatim}
      Minimisation function = Sum[ w*(Fo - Fc)**2 ]
\end{verbatim}\normalsize


 If  F-SQUARED  is  YES , the minimisation function is:
\small\begin{verbatim}
      Minimisation function = Sum[ w*(Fo**2 - Fc**2)**2 ]
\end{verbatim}\normalsize


 If  F-SQUARED  is  YES , the weights given by  w  in the above expression
 are assumed to be on the correct scale and to refer to Fsq
 rather than Fo's. Note that refinement can be against Fo or Fsq independent of
 whether the input was Fo or Fsq.


\bigskip\Keyword{REFLECTIONS=}
 REFLECTIONS has two alternatives:
\small\begin{verbatim}
      NO
      YES  -  Default value
\end{verbatim}\normalsize


 If REFLECTIONS is YES, the reflections stored in LIST 6 (and subject to the
 checks in LIST 28) are used for computing structure factors and the
 derivatives added into the matrix if required.


If REFLECTIONS is NO,  LIST 6 is not used, whether it is
 present or not. This setting could be used for refinement against restraints
 only. See the section DLS, 'Distance Least Squares'.


\bigskip\Keyword{RESTRAIN=}
 RESTRAIN has two alternatives:
\small\begin{verbatim}
      NO
      YES  -  Default value
\end{verbatim}\normalsize


 If  RESTRAIN  is  YES, the restraints in LIST 16 (section \ref{LIST16}) 
 and LIST 17 (section \ref{LIST17})
 are added into the normal equations.





\bigskip\Directive{REFINE  SPECIAL= UPDATE= TOLERANCE=}
 This directive controls the  refinement of atoms on special positions
 and the control of floating origins.
 The default action for atoms is to try to constrain them.
 However, if an atom is already the subject of a user defined constraint,
 the symmetry requirements are imposed by restraints. The site occupancy,
 positional and thermal parameters can be set to satisfy the
 site symmetry. The site occupancy is indepentant of any chemical or
 physical partial occupancy by an atom.


Floating origins are controlled by
 restraining the center of gravity of the structure along the axis to remain
 fixed.


\bigskip\Keyword{SPECIAL=}

\small\begin{verbatim}X
     SPECIAL = NO                  No action
             = TEST                Displays but does not store any restrictions
             = ORIGIN              Tests for and restrains floating origins
             = RESTRAIN            Creates and stores restraints
             = CONSTRAIN (Default) Attempt to create constraints
\end{verbatim}\normalsize




\bigskip\Keyword{UPDATE=}

\small\begin{verbatim}
      UPDATE = NO                    Nothing updated
             = OCCUPATION            Site occupancies modified
             = PARAMETERS  (Default) All adjustable parameters modified
\end{verbatim}\normalsize




\bigskip\Keyword{REWEIGHT=}
 Not currently used


\bigskip\Keyword{GROUPS=}
\small\begin{verbatim}
      NO  -  Default value
      YES
\end{verbatim}\normalsize


 GROUPS is automatically set to YES if LIST 12
 contains any GROUP directives. It forces the group derivatives
 to be recalculated between each cycle or refinement.
 Not currently used





\bigskip\Keyword{COMPOSITE=}
 Not currently used



\bigskip\Keyword{TOLERANCE=}
 Atoms within 'TOLERANCE' Angstrom of a symmetry equivalent atom are
 regarded as being on a special position. The default is 0.6A. For high
 symmetry spacegroups with disorder, the value might need reducing if
 multiplicities are incorrectly calculated.





\bigskip\Directive{ALLCYCLES MIN-R= MAX-R= *-WR= *-SUMSQ= *-MINFUNC= U[MIN]=}
 This directive controls conditions that must be satisfied after
 each cycle if refinement is to continue. It can be used to detect
 converged or 'blown-up' refinements.
 The heading has been abbreviated, the  *  representing  MIN  and  MAX .


\bigskip\Keyword{MIN-R=, MAX-R=}
 The normal  R-value  must lie between  MIN-R  and  MAX-R, otherwise
 refinement is terminated after
 the current cycle.
 The default values for
  MIN-R  and  MAX-R  are 0.0 and 100.0 percent.


\bigskip\Keyword{MIN-WR MAX-WR}
 The Hamilton weighted  R-VALUE  must lie between  MIN-WR  and
  MAX-WR, otherwise the refinement is
 terminated after the current cycle.
 The default values for  MIN-WR  and  MAX-WR  are 0.0 and 100.0
 percent respectively.


\bigskip\Keyword{MIN-SUMSQ=, MAX-SHUMSQ=}
 The rms (shift/e.s.d.) fo all parameters in the refinement must
 lie between  MIN-SUMSQ  and  MAX-SUMSQ, otherwise
 the refinement is terminated after the current cycle.
 The sum of the squares of the ratios is defined as :
\small\begin{verbatim}
      SUMSQ = SQRT(SIGMA(SHIFT/ESD))/N)
  The default values
 of  MIN-SUMSQ  and  MAX-SUMSQ  are 0.03 and 10000.0, .
\end{verbatim}\normalsize

\bigskip\Keyword{MIN-MINFUNC= MAX-MINFUNC=}
 The minimisation function, on the scale of Fo, must lie between
  MIN-MINFUNC  and  MAX-MINFUNC, otherwise
 the refinement is terminated after the current cycle.
 The default values of  MIN-MINFUNC  and  MAX-MINFUNC  are 0.0 and
 1000000000000000.0.


\bigskip\Keyword{U[MIN]=}
 If Uiso or a principal component of the adp of any atom is
 less than   U[MIN] , then a warning is issued and the idp
 reset to u[min], or the components of the adp reset to MAX(Uii,U[MIN])
 or MAX(Uij,0.01U[min]).
 If this parameter
 is omitted, a default value of 0.0 is assumed.


\bigskip\Directive{INTERCYCLE MIN-DR= MAX-DR= *-DWR= *-DSHIFT/ESD= *-DMINFUNC=}


This directive refers to conditions that must be obeyed before the
 next cycle of least squares refinement can proceed. (A quantity
 undergoes a positive change if  OLD - NEW  is positive, not
  NEW - OLD ). The definitions are similar to ALLCYCLES.
 The abbreviation  '*'  represents  MIN  and  MAX .


\bigskip\Keyword{MIN-DR= MAX-DR=}
 Between two cycles of least squares, the change in
  R-VALUE  must lie between  MIN-DR  and  MAX-DR, otherwise
 the refinement is terminated.
 The default values are -5.0 and 100.0.


\bigskip\Keyword{MIN-DWR MAX-DWR}
 The default values are -5.0 and 100.0.


\bigskip\Keyword{MIN-DSUMSQ MAX-DSUMSQ}
 The default values are -10. and 10000.0.


\bigskip\Keyword{MIN-DMINFUNC MAX-DMINFUNC}
 The default values are 0.0 and 1000000000000000.0.



\section{Printing the SLFS control list}






\bigskip\Instruction{$\backslash$PRINT 23}




This prints LIST 23. There is no command for punching LIST 23.



\index{Special position constraints}
\section{Special position constraints - $\backslash$SPECIAL}
\label{SPECIAL} 


\small\begin{verbatim}
 \SPECIAL  ACTION= UPDATE= TOLERANCE=
 END

 \SPECIAL
 END
\end{verbatim}\normalsize




$\backslash$SPECIAL can be issued at any time to get information about atoms on
 special positions. However, normally it is called automatically by
 setting the SPECIAL keyword in LIST 23 (section \ref{LIST23}).


Atoms on special positions may be constrained through LIST 12
(section ), or restrained through LIST 17 (section  \ref{LIST12}RLIST17

).
 CRYSTALS  will attempt to generate the special position conditions when
 requested via the
 SPECIAL command, and also  update coordinates of atoms on
 special positions.


If the RESTRAIN option is chosen, then the special conditions are imposed
 on the refinement by restraints, which are generated without reference to
 what is being specified in LIST 12, the refined parameter definition list.


If the CONSTRAIN option is chosen, then CRYSTALS examines the site
 restrictions as it processes LIST 12. If an atom on a special position is
 being refined without any user defined conditions (EQUIVALENCE, RIDE, LINK,
 COMBINE, GROUP, WEIGHT), and the related coordinates are in the same matrix
 block, then the internal representation of LIST 12 (LIST 22)
 is dynamically modified to include the necessary
 constraints. If the atom is already the
 object of a constraint,
 then LIST 12 cannot safely be modified, and the special condition
 is applied as a restraint. In either case, CRYSTALS warns the user about
 what is being done.


The origins of polar space groups are always fixed by restraints, since this
 produces a better conditioned matrix than one from just fixing atomic
 coordinates.


The UPDATE directive controls whether parameters of atoms near special
 positions will be modified to make them exact. The routine will update
 just the site occupancies, or the occupancies and the other variable
 parameters. The crystallographic site occupancy is held temporarily in the
 key SPARE, leaving the key OCC available for a refinable chemical
 occupancy. Take care if an atom refines onto (or off) a special position.


The function SPECIAL is actioned automatically for every round of least
 squares refinement. Its action is then determined by values held in LIST 23 
(structure factor control, see section \ref{LIST23})


\bigskip\Instruction{$\backslash$SPECIAL  ACTION= UPDATE= TOLERANCE=}




\bigskip\Keyword{ACTION}

\small\begin{verbatim}
      ACTION = NONE      No action
             = TEST      Displays but does not store any restrictions
             = ORIGIN    Tests for and restrains floating origins
             = RESTRAIN  Creates and store a LIST 17
             = CONSTRAIN Attempt to create constraints.
             = LIST23    (Default) Takes the action defined in LIST 23
\end{verbatim}\normalsize




\bigskip\Keyword{UPDATE}

\small\begin{verbatim}
      UPDATE = NONE        Nothing updated
             = OCCUPATION  Site occupancies modified
             = PARAMETERS  All adjustable parameters modified
             = LIST23      (Default) Takes action defined in LIST 23
\end{verbatim}\normalsize




\bigskip\Keyword{TOLERANCE}


TOLERANCE is the maximum separation, in Angstrom, between nominally
 equivalent sites. The default is 0.6A.
\section{Printing the special position information}



Force the atom parameter list (LIST 5) to be updated and send it to
 the PCH file.
\small\begin{verbatim}
      \SPECIAL TEST PARAMETER
      END
      \PUNCH 5  (to get a listing with 5 decimal places)
      END
\end{verbatim}\normalsize


\index{LIST 12}
\index{Refinement directives}
\section{Refinement directives  -  LIST 12}

\label{LIST12} 


This list defines the parameters to be refined in the
 least squares calculation, and
 specifies relationships between those parameters.

\small\begin{verbatim}
 \LIST 12
 BLOCK  PARAMETERS ...
 FIX  PARAMETERS ...
 EQUIVALENCE  PARAMETERS ...
 RIDE  ATOM_PARAMETER SPECIFICATIONS ...
 LINK PARAMETER_LIST AND PARAMETER_LIST AND PARAMETER_LIST.
 COMBINE PARAMETERS_LIST AND PARAMETERS_LIST
 GROUP  ATOM SPECIFICATIONS
 WEIGHT F1 PARAMETERS F2 PARAMETERS ...
 FULL  PARAMETERS
 DIAGONAL  PARAMETERS
 PLUS  PARAMETERS
 SUMFIX PARAMETERS
 REM text
 END
\end{verbatim}\normalsize



\small\begin{verbatim}
 \LIST 12
 BLOCK SCALE X'S U'S
 END
\end{verbatim}\normalsize




\bigskip\Instruction{$\backslash$LIST 12}




\bigskip\Directive{BLOCK  PARAMETERS}
 This directive defines the start of a new matrix
 block.
 Any parameters that come on this directive and any directives until another BLOCK
 directive are put into the same matrix block. If only one BLOCK directive is
 given, then the refinement is 'full matrix'.


\bigskip\Directive{FIX  PARAMETERS}
 The specified parameters are not to be refined.


\bigskip\Directive{EQUIVALENCE  PARAMETERS}
 Sets the given parameters to a single least squares parameter (see the
 examples).


\bigskip\Directive{RIDE  ATOM\_PARAMETER SPECIFICATIONS}
 This directive links corresponding parameters for all the atoms
 specified on the directive.
 The parameters specified for the first atom given on this directive
 are each assigned to individual least squares parameters, and
 parameters for subsequent atoms are EQUIVALENCED,
 in the order given, to the corresponding least squares parameter.
  Only explicit atom parameters can be used on this directive. Usually, the same
  parameter keys will be given in the same order for all atoms referenced,
 though this may not be true for high symmetry space groups.


\bigskip\Directive{LINK PARAMETER\_LIST AND PARAMETER\_LIST ( AND PARAMETER\_LIST.)}
  Links the parameters defined after the AND with those specified in the
 first parameter list. A least squares parameter is assigned to each
 physical parameter in the first list. Physical parameters specified in the
 second (and subsequent if present) lists are then assigned IN THE ORDER
 GIVEN to these least squares parameters. There must be the same number
 of parameters in each parameter list. The parameter list may contain
 more than one atom, and is terminated by the 'AND' or the end of the
 directive.  Overall and implicit parameters may  be given.


\bigskip\Directive{COMBINE PARAMETERS\_1 AND PARAMETERS\_2}
  Combines the parameters defined before the AND with those defined after.
 Physical parameters are taken pairwise in the order given
 from parameter list 1 and 2 and two least-squares parameters defined such
 that one is the sum and the other is the difference of the physical
 parameters.

\small\begin{verbatim}
      x' = x1 + x2
      x" = x1 - x2
                        where x1 and x2 are physical parameters,
                          and x' and x" are least squares parameters.
\end{verbatim}\normalsize


 Such a re-parameterisation is useful for dealing with certain sorts of ill-
 conditioning, such as that due to pseudo-symmetry,
 of the normal matrix (see Edward Prince, Mathematical Techniques in
 Crystallography and Material Science, 1982, Springer-Verlag, page 113).
 NOTE that only one AND can be given.


\bigskip\Directive{GROUP  ATOM SPECIFICATIONS}
 The positional coordinates of the
 atoms given in the ATOM SPECIFICATIONS are refined as a rigid group.
 Parameter specifications MUST NOT be included. The first atom specified
 is taken as the pivot atom of the group. All atoms in the group may be
 the subject of restraints to atoms in other parts of the structure, or in
 other groups. Use LINK, RIDE or EQUIVALENCE to build a suitable model
 for the temperature factors.


Because of the linearisation algorithm used, some distortion of the group
 will occur if there are large parameter shifts. Use REGULARISE to re-form it.


\bigskip\Directive{WEIGHT w1 PARAMETERS w2 PARAMETERS . .}
 Before the contributions of the specified parameters
 are included in the normal equations, they are
 multiplied by the number wI .  Similarly ,
 when the normal equations are solved, the shifts
 and e.s.d.'s are multiplied by the same wI.
 The default  value of wI is 1.0.
 The parameters are multiplied by the value of
 wI that precedes them (see the examples).
\section{Obsolete Refinement directives}




 The following directives may be removed in some future release.


\bigskip\Directive{FULL  PARAMETERS}
 The parameters on the directive directive plus
 any other parameters defined on subsequent directives
 are to be included in a full matrix refinement.
 The scale factor is automatically included, while
 the dummy overall isotropic temperature factor
 is fixed. This is equivalent to:


BLOCK SCALE PARAMETERS





\bigskip\Directive{PLUS  PARAMETERS}
 The specified parameters are to be refined, and
 they will be placed in the current block of
 the normal matrix. This is equivalent to:


CONTINUE PARAMETERS after the BLOCK directive.





\bigskip\Directive{DIAGONAL  PARAMETERS}
 All the specified parameters in the LIST 12 are
 included in a block diagonal approximation to
 the full matrix, based on one block for each atom.
 Both the SCALE FACTOR and the DUMMY OVERALL ISOTROPIC
 TEMPERATURE FACTOR are automatically included.





\bigskip\Directive{SUMFIX PARAMETERS [ AND PARAMETERS ] [ AND PARAMETERS ]...}
 Constrains the sum of the specified parameters (or groups of parameters) to
 be constant.
 If no AND is given, then sum of the parameters specified is constrained, by
 making the total of the shifts add to zero. If AND is given then the total of
 the shifts for each group add to zero, and each group is equivalenced to a
 single least squares parameter.
 E.g. constrain sum of three occupancies to be constant.

\small\begin{verbatim}
      \LIST 12
      BLOCK SCALE X'S
      SUMFIX Fe(1,OCC) Al(1,OCC) Si(1,OCC)
      END
\end{verbatim}\normalsize


 E.g. constrain sum of three disordered parts to be constant.

\small\begin{verbatim}
      \LIST 12
      BLOCK SCALE X'S
      SUMFIX PART(1001,OCC) AND PART(1002,OCC) AND PART(1003,OCC)
      END
\end{verbatim}\normalsize


\section{Defining the least squares matrix}



Parameters may be referred to either \emph{implicitly,} by just giving the
 parameter name (in which case that parameter is referenced for all
 atoms), or \emph{explicitly} by specifying the parameter for an atom
 or group of atoms. All implicit specifications ignore H atoms.

\small\begin{verbatim}
 e.g.
      IMPLICIT: x, u's
      EXPLICIT C(1,X), O(1,U'S) UNTIL O(14)
\end{verbatim}\normalsize




A parameter may not be referenced more than once either explicitly or
 implicitly. A parameter \emph{may} be referenced both implicitly and
 explicitly, in which case the explicit reference takes precedence.

\small\begin{verbatim}
 e.g.
      BLOCK x's            (implicit reference)
      FIX Pb(1,y)          (explicit reference)
                  This establishes the refinement of z,y,z for all atoms
                  except Pb(1), for which only x and z are refined.
\end{verbatim}\normalsize



\small\begin{verbatim}
 EXAMPLES :
      1. BLOCK SCALE X
         FIX C(1,X)     ALLOWED
      2. BLOCK SCALE  X
         FIX X          NOT ALLOWED
\end{verbatim}\normalsize




The refinement directives are read and
 stored on the disc.
 Before the structure factor least squares routines
 can use the information in LIST 12 (constraint directives), it is validated against 
 LIST 5 (the model parameters) and stored
 symbolically as a LIST 22.
 This is done automatically by the SFLS routines (section \ref{SFLS}), but
 the user can force the verification of LIST 12 by issuing the command
 $\backslash$LIST 22.
\section{Printing of LIST 12}



LIST 12 may be listed with either


\bigskip\Instruction{$\backslash$PRINT 12}


 or


\bigskip\Instruction{$\backslash$SUMMARY 12}




LIST 12 may be punched with


\bigskip\Instruction{$\backslash$PUNCH 12}





\section{Creating a null LIST 12 }



A null LIST 12, containing no refinement directives, may be created with


\bigskip\Instruction{$\backslash$CLEAR 12}





\section{Processing of LIST 12}



LIST 12 is processed to greate a LIST 22 with


\bigskip\Instruction{$\backslash$LIST 22}



\small\begin{verbatim}
 Examples.
 1. Full matrix isotropic refinement of a structure without H atoms

      \LIST 12
      BLOCK SCALE X'S U[ISO]
      END

 2. Full matrix anisotropic of a structure with C(25) as the last
 non-hydrogen, not refining the H atoms.

      \LIST 12
      BLOCK SCALE FIRST(X'S,U'S) UNTIL C(25)
      END

 3. Refine all positions, aniso non-H, iso H atoms

      \LIST 12
      BLOCK SCALE X'S
      CONTINUE FIRST(U'S) UNTIL C(25)
      CONTINUE H(1,U[ISO]) UNTIL LAST
      END

 4. Ride H(1) positions on C(21) positions, etc. There are 2 H on C(25)

      \LIST 12
      BLOCK SCALE X'S
      CONTINUE FIRST(U'S) UNTIL C(25)
      CONTINUE H(1,U[ISO]) UNTIL LAST
      RIDE C(21,X'S) H(1,X'S)
      RIDE C(22,X'S) H(2,X'S)
      RIDE C(23,X'S) H(3,X'S)
      RIDE C(24,X'S) H(4,X'S)
      RIDE C(25,X'S) H(51,X'S) H(52,X'S)
      END

 5. A fragment is distributed over 2 sites. The fragments are
 C(100) C(101) O(102) C(103) and C(200) C(201) O(202) C(203)

      \LIST 12
      BLOCK SCALE X'S
      ... ...
      EQUIVALENCE C(100,OCC) UNTIL C(103) C(200,OCC) UNTIL C(203)
      WEIGHT -1 C(200,OCC) UNTIL C(203)
      END

\end{verbatim}\normalsize


\index{LIST 16}
\index{Restraint directives}
\section{Restraints  -  LIST 16}
\label{LIST16} 



This list defines the restraint to be used as supplemental
 observations.

\small\begin{verbatim}
 \LIST 16
 DISTANCES  VALUE, E.S.D= BOND1, BOND2
 A-DISTANCES  VALUE, E.S.D= BOND1, BOND2
 DISTANCES  VALUE, E.S.D= MEAN BOND1, BOND2
 DISTANCES  VALUE, E.S.D= DIFFERENCE BOND1, BOND2
 NONBONDED  VALUE, POWERFACTOR=  BOND1, BOND2
 ANGLES     VALUE, E.S.D= ANGLE1, ANGLE2
 ANGLES     VALUE, E.S.D= MEAN ANGLE1, ANGLE2
 ANGLES     VALUE, E.S.D= DIFFERENCE ANGLE1, ANGLE2
 VIBRATIONS VALUE, E.S.D= BOND1, BOND2
 U(IJ)'S    VALUE, E.S.D= BOND1, BOND2
 A-VIBRATIONS VALUE, E.S.D= BOND1, BOND2
 A-U(IJ)'S    VALUE, E.S.D= BOND1, BOND2
 UTLS              E.S.D. [ m ] ATOM1 [ ATOM2 ] ...
 UEQIV             E.S.D. ATOM1 [ ATOM2 ] ...
 UVOL              E.S.D. ATOM1 [ ATOM2 ] ...
 UQISO             E.S.D. ATOM1 [ ATOM2 ] ...
 UEIG              E.S.D. ATOM1 [ ATOM2 ] ...
 UALIGN            E.S.D. ATOM1 [ ATOM2 ] ...
 URIGU             E.S.D. ATOM1 TO ATOM2 [, ATOM3 TO ATOM4 ] ...
 UPERP             E.S.D. ATOM1 TO ATOM2 [ TO ATOM3 ] [, ATOM4 TO ATOM5 [ TO ATOM6 ] ] ...
 UPLANE            E.S.D. ATOM1 TO ATOM2 TO ATOM3 [, ATOM4 TO ATOM5 TO ATOM6 ] ...
 ULIJ              E.S.D. ATOM1 TO ATOM2 TO ATOM3 [, ATOM4 TO ATOM5 TO ATOM6 ] ...
 PLANAR            E.S.D  FOR 'ATOM SPECIFICATIONS'
 LIMIT             E.S.D  FOR 'PARAMETER SPECIFICATIONS'
 ORIGIN            E.S.D  FOR 'PARAMETER SPECIFICATIONS'
 SUM               E.S.D  FOR 'PARAMETER SPECIFICATIONS'
 AVERAGE           E.S.D  FOR 'PARAMETER SPECIFICATIONS'
 SAME       BOND-ESD ANGLE-ESD FOR GROUP-1 AND GROUP-2 AND ...       
 DELU       ADP-ESD  GROUP-1 (AND GROUP-2 AND ...)       
 SIMU       ADP-ESD  GROUP-1 (AND GROUP-2 AND ...)       
 RESTRAIN   VALUE, E.S.D= TEXT
 DEFINE NAME = TEXT
 COMPILER
 EXECUTION
 END
\end{verbatim}\normalsize



\small\begin{verbatim}
 \LIST 16
 DIST  1.39 , .01 =      C(1) to C(2), C(2) to C(3), C(3) to C(4)
 DIST  0.0  , .01 = MEAN C(1) to C(2), C(2) to C(3), C(3) to C(4)
 VIBR  0.0  , .01 =      C(1) to C(2), C(2) to C(3), C(3) to C(4)
 U(IJ) 0.0  , .02 =      C(1) to C(2), C(2) to C(3), C(3) to C(4)
 PLANAR                  C(1) until C(6)
 SUM                     K(1,OCC), K(2,OCC) K(3,OCC)
 SUM                     ELEMENT SCALES  (twin element scale factors)
 LIMIT                   U[11] U[22] U[33]
 END
\end{verbatim}\normalsize




The restraints that can be applied under this system are of a type
 originally described by J. Waser, Acta Cryst. 1963, 16, 1091.
 A good summary of the present facilities and aims is provided
 by J.S. Rollett in Crystallographic Computing, p170.


In this method of restraints, the user provides  a set of
 physical or chemical restraints that are to be applied to the
 proposed model.
 These restraints are usually based upon observations of similar
 compounds (for example, bond lengths or bond angles) or upon
 known physical laws (for example, the difference in mean square displacement
 of two atoms along the bond that joins them).
 These restraints are not
 rigidly applied to the model, but each restraint has
 associated with it an e.s.d., which is used to calculate a weight
 so that the restraint can then be added into the normal
 equations.
 (The e.s.d.'s are provided on an absolute scale, and rescaled by the
 program onto the same scale as the xray data).
 In this way, the importance of the restraints, which are
 treated as extra observations, can be varied with respect to the
 importance of the X-ray data.
 If the structure is required to adhere closely to the proposed
 model, the restraints are given high weights (i.e. small e.s.d.'s)
 otherwise they can be given smaller weights.


If, at the end of a refinement, the restraints are not compatible with
 the Xray data, this is shown by a discrepancy
 between the requested value for the restraint, and that computed from
 the refine parameters.
  If this is found, the validity of the restraints that have been imposed
 should be carefully checked.


In order that the restraint routines should be completely general,
 each atom that is part of a restraint can be modified by a set of
 symmetry operators before the restraint is applied.
 (This is vital for molecules that lie across a symmetry element,
 as all the atoms that constitute the molecule are not present
 in LIST 5).


If a structure uses symmetry related atoms to form bonds, the
 command $\backslash$DISTANCE with OUTPUT PUNCH=RESTRAIN can be used to set up a
 proforma restraints list, including symmetry codes.
 The distances and e.s.ds
 will have to be edited to the correct target values. Use appropriate values
 on the SELECT, INCLUDE and EXCLUDE directives for DISTANCE to
 tailor the generated list.


Note that restraints may be used without
 diffraction data, see the chapter 'Distance Least Squares' for examples.


\bigskip\Directive{NOTE}
 The restraint directives are read and
 stored on the disc.
 Before the structure factor least squares routines
 can use the information in LIST 16 (restraints), it is validated against LIST 5
 (the model parameters) and stored symbolically as a LIST 26 (see \ref{LIST26}).
 This is done automatically by the SFLS routines (section \ref{SFLS}), but
 the user can force the verification of LIST 16 by issuing the command
  $\backslash$CHECK (see later).
\section{Parameter, atom, bond and angle specifications}



Composite parameter specifications are not permitted ( \emph{e.g.} U's), atom
 specifications are as in Chapter 4.


Two atoms that are bonded together are defined in the following
 way :
\small\begin{verbatim}
      atom1 to atom2,
\end{verbatim}\normalsize


  'atom1'  and  'atom2'  are standard atom specifications as described
 in chapter 4, and are separated from any other text on the line by
 at least one space. If there is more than one bond specification on  a
 line, it may be separated from another by either a space or a comma.
 The 'TO'  is mandatory, and is terminated by
 one or more spaces.


The definition of an angle is an extension of the definition
 of a bond:
\small\begin{verbatim}
      atom1 to atom2 to atom3,
\end{verbatim}\normalsize


 The angle is defined as the angle subtended at  atom2  by  atom1
 and  atom3.
 The restraints routines apply all the required symmetry if specified in
 an atom definition, while still
 conserving the partial derivatives in their correct form.


\bigskip\Instruction{$\backslash$LIST 16}


 The restraints routines regard all
 continuation directives as part of the original
 directive, so that the column of a
 character on a continuation directive will
 have had '80*n' added to it, where 'n' is
 the number of directives between the current
 continuation directive and the start of
 the directive. The ',', '=' signs and separator 'MEAN' are mandatory
 if shown in  the definition.


\bigskip\Directive{DISTANCES  VALUE, E.S.D. = BOND1, BOND2, . . . . .}
 The bonds specified after the '=' sign are
 restrained to have a length of 'VALUE', with
 an e.s.d. of 'E.S.D.'.


\bigskip\Directive{A-DISTANCES  VALUE, E.S.D. = BOND1, BOND2, . . . . .}
 Asymmetric distance restraint. The second atom in the bond is 
 restrained to be the given distance from the first, which is unrestrained.


 Note that the qualifiers MEAN and DIFFERENCE cannot be used in this 
 context.


\bigskip\Directive{DISTANCES VALUE, E.S.D. = MEAN BOND1, BOND2, . . . .}
 Initially the restraints routines calculate the
 'MEAN' value of all the bonds specified by
 the directive. Each of the bonds
 specified is then restrained to be
 equal to 'MEAN' + 'VALUE', with an e.s.d. of
 'E.S.D.'. The 'DELTA' used in the
 right hand sides of the normal equations is
 defined by :
\small\begin{verbatim}
      DELTA = MEAN + VALUE - BOND CALCULATED.
\end{verbatim}\normalsize




\bigskip\Directive{DISTANCES VALUE, E.S.D. = DIFFERENCE BOND1, BOND2, . .}
 Each of the bonds in this directive is restrained 
 to be equal to 'VALUE'
 plus the length of each of the bonds that
 follow it.
 The computed value of 'DELTA' used in
 the right hand sides of the normal
 equations is thus given by :
\small\begin{verbatim}
      DELTA = VALUE + BOND(N) - BOND(M)
\end{verbatim}\normalsize


 Where BOND(N) occurs after BOND(M) in
 the directive.


Each such restraint is added into the
 normal equations with an e.s.d. Of  E.S.D. .
 However, as each bond is restrained to each
 of the bonds that follow it,  (N*(N-1))/2
 separate restraints are generated.
 Many of these restraints involve the same
 bond lengths and are thus not independent.
 To be strictly accurate, a non-diagonal
 weight matrix should be used with this
 restraint but such a facility
 is not available.


The letters  DIFFERENCE  are terminated
 by one or more spaces and may be abbreviated to
 DIFF.


\bigskip\Directive{NONBONDED  VALUE, POWERFACTOR =, BOND1, BOND2, . . . . .}
 This restraint is similar to the 'DISTANCE' restraint in that the pairs of
 atoms defining the bond are restrained to be at the 'VALUE' distance appart.
 However, the weight to be given to the restraint is computed from the
 difference between the observed and the requested contact distance using the
 expression:

\small\begin{verbatim}
       weight = 10000*(requested/observed)**(powerfactor*12)
\end{verbatim}\normalsize




When the observed equals the requested distance,
 the weight corresponds to an e.s.d.
 of .01. If the requested is less than the observed, the weight is reduced
 slowly as a function of the discrepancy. If the requested is greater than the
 observed, the weight rises rapidly with discrepancy. The function is like the
 repulsive part of a 6-12 energy expression, having greatest effect on
 anomalously short contacts. Powerfactors of between 1 and 4 seem to be
 suitable.





\bigskip\Directive{ANGLES VALUE, E.S.D. = ANGLE1, ANGLE2, . . . .}
 Each of the angles given in the directive 
 is restrained to a value of 'VALUE',
 with an e.s.d. of 'E.S.D.'. The angles must
 be in degrees.


\bigskip\Directive{ANGLES VALUE, E.S.D. = MEAN ANGLE1, ANGLE2, . . . . .}
 This is the analagous to the MEAN distance restaint,
 except that the mean value is computed for
 the specified angles and each of the angles
 is then restrained to 'MEAN' + 'VALUE',
 with an e.s.d. of 'E.S.D.'.
 The 'DELTA' values and the syntax rules are all the
 same as for the equivalent distance restraint.


\bigskip\Directive{ANGLES VALUE, E.S.D. = DIFFERENCE ANGLE1, ANGLE2, . .}
 This restraint is analogous to the
 DIFFERENCE restraint  for bond lengths.
 Each of the angles in the directive is
 restrained to be equal to 'VALUE' plus each of
 the angles after it in the input.
 Although each such restraint is applied with
 an e.s.d. of 'E.S.D.', the same reservations about
 the validity of the weighting scheme exist
 here as for the equivalent distance restraint.


\bigskip\Directive{VIBRATIONS VALUE, E.S.D. = BOND1, BOND2, . . . .}
 The difference in mean square displacement
 along the bond direction
 of the two atoms that form the bond is
 restrained to be 'VALUE', with an e.s.d. of
 'E.S.D.'. In general, 'VALUE' is assumed to be
 zero, while the e.s.d. reflects the maximum
 discrepancy  in m.s.d. that would be expected for
 the type of bond being considered.
 If either or both of the given atoms is
 isotropic, the program will convert the m.s.d.
 into the appropriate form and calculate the
 derivatives for the isotropic atom correctly.


  Note that the atoms defining a 'bond' need not actually be bonded, but
 merely serve to define a direction. For really bonded atoms, try an esd
 of  .002; for 1-3 atoms or diagonals of phenyl groups, try .005.


\bigskip\Directive{A-VIBRATIONS VALUE, E.S.D. = BOND1, BOND2, . . . .}
 Asymmetric vibration restraint. The second atom in the bond is 
 restrained to be similar to the first, which is unrestrained.


\bigskip\Directive{U(IJ)'S VALUE, E.S.D. =, BOND1, BOND2, . . . .}
 This is a similarity restraint, and may be used to ensure that the
  vibration parameters of adjacent atoms are similar, as must be the case
 even for flexible systems. The esd used must be softer than for a
 VIBRATION restraint, typically 0.01.
 In this restraint, the difference
 between corresponding u(ii) and u(ij) terms
 is restrained to be 'VALUE', with an
 e.s.d. of 'E.S.D.'.
 Each bond that is specified generates therefore
 six separate restraints, one for each of
 the anisotropic temperature parameters. If
 an atom with an isotropic temperature factor
 is included in this restraint, the specified bond
 and all six restraints are ignored.


\bigskip\Directive{A-U(IJ)'S VALUE, E.S.D. =, BOND1, BOND2, . . . .}

 Asymmetric Uij restraint. The second atom in the bond is 
 restrained to be similar to the first, which is unrestrained.


\bigskip\Directive{PLANAR E.S.D. FOR 'ATOM SPECIFICATIONS'}
 This directive instructs the system to compute the mean plane
 through the atoms given in the atom specifications,
 and then to restrain each of the atoms to lie in the plane.
 The 'E.S.D.' with which each atom is restrained to be on
 the plane is given in angstrom.
 This parameter is optional and has a default value of 0.01.
 'FOR' is optional.
 'ATOM SPECIFICATIONS' define the atoms that are on the plane.
 Each 'ATOM SPECIFICATIONS' may consist of one atom, together
 with symmetry data, or two atoms separated by 'UNTIL'.
 One or more specifications must be given.

\small\begin{verbatim}
      Examples :
       PLANAR C(1,2) UNTIL C(6) C(9) C(10,2,2)
       PLANAR 0.05 C(1) C(2) UNTIL C(6)
       PLANAR 0.05 FOR C(1) C(2) UNTIL C(6)
       PLANAR FOR C(1,2) UNTIL C(6)
\end{verbatim}\normalsize




\bigskip\Directive{LIMIT E.S.D. FOR 'PARAMETER SPECIFICATIONS'}
 This restraint seta a target shift of zero for the specified
 parameters, with the specified esd, and thus tries to limit the shift
 in the parameters.
 Since it modifies the normal matrix, it does not have the same effect as
 partial shifts (SHIFT,MAXIMUM,and FORCE in SFLS [section \ref{SFLS}]). 
 In particular, the
 e.s.d. on the parameter will depend upon the E.S.D. given to this restraint.
 The default for E.S.D. is .001. Reducing this to about .00001 will have almost
 the same effect as FIX in LIST 12. Increasing it to 10.0  will cause the
 restraint to have almost no effect unless the parameter involved is almost
 singular with respect to some other parameter. Note that this is only a
 restraint, and if the medel and X-ray data are good, the specified parameters
 will still shift. This restraint is valuable during the development of
 a poor starting model.





\bigskip\Directive{ORIGIN E.S.D. FOR 'PARAMETER SPECIFICATIONS'}
 This is used  for polar space groups,
 where the singularity up the polar axis may be removed by  holding
 the electron weighted sum of all the coordinates up that axis constant.

\small\begin{verbatim}
      Example
       ORIGIN Y
\end{verbatim}\normalsize




\bigskip\Directive{SUM E.S.D. FOR 'PARAMETER SPECIFICATIONS'}
 This restraint holds the sum of the parameters on the directive constant
 during the refinement.
 A typical case is where several (more than 2, which are better
 treated with EQUIVALENCE, in LIST 12) atoms share a site.
 'E.S.D.' is the e.s.d. with which the sum of the parameters is
 held constant.
 This is an optional parameter and has a default value of 0.0001.
 'FOR' is optional.
 'PARAMETER SPECIFICATIONS' define the parameters that are to be
 summed. They may be given as :

\small\begin{verbatim}
       overall parameters e.g. SCALE,
       all atomic parameters of one type e.g. X, Y, U[11],
       atomic parameters of one type for a group of atoms
             e.g. NA(1,OCC) UNTIL RB(6),
\end{verbatim}\normalsize



\small\begin{verbatim}
      Examples :
       SUM 0.0001 NA(1,OCC) UNTIL RB(6)
       SUM LAYER SCALES
\end{verbatim}\normalsize




\bigskip\Directive{AVERAGE E.S.D. FOR 'PARAMETER SPECIFICATIONS'}
 For this directive, the system computes the mean of the given
 parameters,
 and then restrains each to have the mean value
 with an e.s.d. of 'E.S.D.'.
 The parameters are as for the 'SUM' directive above.








\bigskip\Directive{SAME BOND-ESD ANGLE-ESD FOR GROUP-1 AND GROUP-2 AND ...}
 The first group on the card is the 'target' - 
 all following groups are mapped onto it (in order specified) and the distances and
 angles restrained - using the connectivity of the first group.



 The first two arguments are the e.s.d for bond length restraints and the e.s.d
 for angle restraints. Groups are seperated by the word 'AND'.
 NOTE the absence of the usual '=' sign.
 I.E:
\small\begin{verbatim}
 SAME 0.01, 0.1 FOR RESI(1) AND RESI(2)
 SAME  PART(1001) PART(1002)
\end{verbatim}\normalsize


   maps all atoms in the first argument onto all the atoms in 
   the second argument.


TAKE CARE
   Although this shorthand is appealing, the order of the atoms in LIST 
   5 must be identical in both arguments, although the atoms do not have 
   to be adjacent.
\small\begin{verbatim}
 SAME 0.01 , 0.1 
 CONT C(17)  C(18)  H(183) H(182) H(181) AND
 CONT C(17)  C(18)  H(182) H(181) H(183)
   imposes 3-fold symmetry on a single methyl group.

 SAME 0.01 , 0.1 
 CONT C(17)  C(18)  H(183) H(182) H(181) AND
 CONT C(17)  C(19)  H(193) H(192) H(191) AND
 CONT C(8)   C(9)   H(93)  H(92)  H(91) AND
 CONT C(8)   C(10)  H(103) H(102) H(101) AND
 CONT C(14)  C(15)  H(153) H(152) H(151) AND
 CONT C(14)  C(16)  H(163) H(162) H(161)
   restrains six methyl groups to have the same geometry as each
   other. Combining the last two restraints would make all the
   methyls have 3 fold symmetry, and all be the same.
\end{verbatim}\normalsize




 Errors are generated if
   1) the size of any of the groups on the SAME card is not the
      same as the first group.
   2) the element type in a group does not match the corresponding
      element type in the first group.



 Warnings are printed if there are zero bonds to any of the atoms
 in the first group.



 The comma separating the e.s.d arguments, and the 'FOR' separating the
 e.s.d.s from the atom specifications are optional.
 The second e.s.d is optional, the default is 0.1 degrees.
 The first e.s.d is optional unless you wish to specify the second, the
 default is 0.01 Angstroms.



 List 41 (bonds) is loaded by the restraint generating routine, if it
 does not exist an error will occur. (By default L41 is kept up to date
 with the current model.)







\bigskip\Directive{DELU ADP-ESD  FOR GROUP-1 (AND GROUP-2 AND ...)}
 The adps of all pairs of bonded atoms in each group are restrained
 to be equal in the direction of the bond.  Unlike SAME, a single
 group can be given.  The RESIDUES are NOT restrained to be similar.



 The first argument is the e.s.d for adp-restraint,  
 I.E:
\small\begin{verbatim}
 DELU 0.01  FOR RESI(1) AND RESI(2)
\end{verbatim}\normalsize




 Errors are generated if
   1) the size of any of the groups on the DELU card is not the
      same as the first group.
   2) the element type in a group does not match the corresponding
      element type in the first group.



 Warnings are printed if there are zero bonds to any of the atoms
 in the first group.



 The  'FOR' separating the
 e.s.d.s from the atom specifications is optional.
 The e.s.d is optional, the  default is 0.01 Angstroms.



 List 41 (bonds) is loaded by the restraint generating routine, if it
 does not exist an error will occur. (By default L41 is kept up to date
 with the current model.)







\bigskip\Directive{SIMU ADP-ESD  FOR GROUP-1 (AND GROUP-2 AND ...)}
   Restrains equivalent elements of the  adps of all pairs on bonded atoms 
   in each residue. The RESIDUES are NOT restrained to be similar. Unlike
   SAME, a single group can be given



 The first argument is the e.s.d for adp-restraint.
 I.E:
\small\begin{verbatim}
 SIMU 0.04  FOR RESI(1) AND RESI(2)
\end{verbatim}\normalsize




 Errors are generated if
   1) the size of any of the groups on the DELU card is not the
      same as the first group.
   2) the element type in a group does not match the corresponding
      element type in the first group.



 Warnings are printed if there are zero bonds to any of the atoms
 in the first group.



 The  'FOR' separating the
 e.s.d.s from the atom specifications is optional.
 The e.s.d is optional, the  default is 0.04 Angstroms.



 List 41 (bonds) is loaded by the restraint generating routine, if it
 does not exist an error will occur. (By default L41 is kept up to date
 with the current model.)







\bigskip\Directive{UTLS              E.S.D. [ m ] ATOM1 [ ATOM2 ] ...}
   Restrain ADPs of all or the m first atoms listed to calculated ADPs derived 
   from a TLS model from all the atom list.
\small\begin{verbatim}
 UTLS 0.01 3 F(1) F(2) F(3) C(1) C(2)
\end{verbatim}\normalsize




 A TLS model is generated using F(1), F(2), F(3), C(1) and C(2). This model 
 is then used to calculate theoritical ADPs. These ADPs are then used as targets
 for the restraints on atoms F(1), F(2) and F(3) with an e.s.d. of 0.01.




   


\bigskip\Directive{UEQIV             E.S.D. ATOM1 [ ATOM2 ] ...}
   Restrain Ueqiv of the ADPs of all the atom listed to be equal
\small\begin{verbatim}
 UEQIV 0.01 F(1) F(2) F(3)
\end{verbatim}\normalsize




 Ueq is the arthmetic mean of the principal axes of an ADP. 
 the current Ueq is calculated for each atom and an average calculated.
 If an atom is non positive definite, its Ueq is not used for the calculation of the mean
 but the atom is included in the restraint. Ueq of each atom is then restrained to
 the average.



   
   


\bigskip\Directive{UVOL              E.S.D. ATOM1 [ ATOM2 ] ...}
   Restrain Ugeom of the ADPs of all the atom listed to be equal
\small\begin{verbatim}
 UVOL 0.01 F(1) F(2) F(3)
\end{verbatim}\normalsize




 Ugeom is the geometric mean of the principal axes of an ADP. 
 the current Ugeom is calculated for each atom and an average calculated.
 If an atom is non positive definite, the atom is excluded from the restraint. 
 Ugeom of each atom is then restrained to the average.







\bigskip\Directive{UQISO             E.S.D. ATOM1 [ ATOM2 ] ...}
   Restrain each atom listed to be isotropic
\small\begin{verbatim}
 UQISO 0.01 F(1) F(2) F(3)
\end{verbatim}\normalsize




 The length of the principal component of each atom are restrained to its average.
 Each atom is treated independantly.



   
   


\bigskip\Directive{UEIG              E.S.D. ATOM1 [ ATOM2 ] ...} 
   Restrain each atom listed to be spheroid (2 smallest principle axes to be equal)
\small\begin{verbatim}
 UEIG 0.01 F(1) F(2) F(3)
\end{verbatim}\normalsize




 The length of the two smallest principal component of each atom are restrained to its average.
 Each atom is treated independantly and the third principal component is free.



  
   


\bigskip\Directive{UALIGN            E.S.D. ATOM1 [ ATOM2 ] ...}
   Restrain the ADPs of all the atom listed to be aligned to their average direction
\small\begin{verbatim}
 UALIGN 0.01 F(1) F(2) F(3)
\end{verbatim}\normalsize




 The matrix of eigenvectors is calculated for each ADP and an average calculated.
 If an atom is non positive definite, the atom is excluded from average but keep for the restraint. 
 Each matrix of eigenvectors is then restrained to the average. The eigenvalues remain free.




  


\bigskip\Directive{URIGU             E.S.D. ATOM1 TO ATOM2 [, ATOM3 TO ATOM4 ] ...}
   Restrain the ADPs listed using enhanced rigibody restraints
\small\begin{verbatim}
 URIGU 0.01 F(1) TO F(2), F(2) TO F(3)
\end{verbatim}\normalsize




 Implementation of the RIGU restraint from SHELXL. See 
 \url{https://dx.doi.org/10.1107\%2FS0108767312014535}{Thorn, A., et. al., Acta Cryst. Sect A, 2012, Vol. 68, pp. 448-451}


   




\bigskip\Directive{UPERP             E.S.D. ATOM1 TO ATOM2 [ TO ATOM3 ] [, ATOM4 TO ATOM5 [ TO ATOM6 ] ] ...}
   Restrain the ADP of the first atom of the group to be perpendicular to the direction 
   of the 2 atoms of the group or the bissector define by ATOM1 TO ATOM2 and ATOM1 TO ATOM3
\small\begin{verbatim}
 UPERP 0.01 F(1) TO C(1), F(2) TO C(1)
\end{verbatim}\normalsize




 The ADP of the atom F(1) and F(2) are restrained to be perpendicular to the direction
 F(1)-C(1) and F(2)-C(2) respectively.




\small\begin{verbatim}
 UPERP 0.01 C(2) TO C(1) TO C(3)
\end{verbatim}\normalsize




 The ADP of the atom C(2) is restrained to be perpendicular to the direction
 define by the bissector of C(1)-C(2)-C(3).







\bigskip\Directive{UPLANE            E.S.D. ATOM1 TO ATOM2 TO ATOM3 [, ATOM4 TO ATOM5 TO ATOM6 ] ...}   
   Restrain the ADP of the first atom of the group to be perpendicular to the plane 
   formed by the 3 atoms of the group
\small\begin{verbatim}
 UPLANE 0.01 F(1) TO C(1) TO C(2)
\end{verbatim}\normalsize




 The ADP of the atom F(1) is restrained to be colinear to the direction
 define by the normal of the plane defined by F(1), C(1) and C(2).




 


\bigskip\Directive{ULIJ              E.S.D. ATOM1 TO ATOM2 TO ATOM3 [, ATOM4 TO ATOM5 TO ATOM6 ] ...}
   Restrain the ADP of all the first atom (ATOM1, ATOM4...) of the groups to be equal in the local coordinate 
   basis defined by all the atom of each group (ATOM1, ATOM2, ATOM3; ATOM4, ATOM5, ATOM6...).
\small\begin{verbatim}
 ULIJ 0.01 F(1) TO C(1) TO C(2), F(2) TO C(1) TO C(2)
\end{verbatim}\normalsize




 The Uij terms of the ADP of the atom F(1) and F(2) are restrained to be equal
 in their repesctive local coordinate system defined by F(1), C(1), C(2) and
 F(2), C(1), C(2).




\small\begin{verbatim}
 Info coppied from Github october, 2019:

 It's in the checkin comments for the work on lexical_preprocessor and
 associated file from October '18. I like the look of the first too,
 but wonder if we can make them more specific, e.g. restraining all
 C-H bonds to be equivalent is quite unusual, and similarly restraining
 all element types to have same Ueq would be unusual.

 The DEFINE option does work, and it means you can adjust the e.s.d's
 on a load of restraints with one edit. However, I haven't checked if
 it screws up the inbuilt DEFINE directive in L16 - might need to
 change the name before publicising this one.

 Rich

 + replace bonds (C--H, C==C, ...) with C() TO H()... 
   you can now write DIST 0.0, 0.01 = MEAN C--H to restrain
   all C--H distances to their mean

 + replace generic type C(*) with C(), C()... 
   UEQIV 0.01 C(*) restrain all C atoms to the same Ueq

 + replace variables definitions (define a = 0.1), then $a is
   used instead of 0.1
   DEFINE a = 0.01
   DIST 0.0, $a = MEAN C--H

 + implement the RIGU restraint as in shelx (using connectivity)

 + implement chiv as in shelx (not working)

 - update urigu restraint to accept new syntax URIGU esd12 esd13
   num_12 bond_list_12 bond_list_13
   (num_12 is the number of 1,2 bond definition)


 On Mon, Oct 28, 2019 at 10:32 AM David Watkin
 <david.watkin@chem.ox.ac.uk> wrote:

    Hi Richard

    Last week you discovered a new bond restraint syntax.  Does it work?
    It cant find anything about it in CRYSTALS.MAN and I could not spot
    anything in the code.

    Cheers
    D
\end{verbatim}\normalsize



\section{General restraints}



The 'general restraint' enables
 the user to write out a restraining equation explicitly.
 The system
 automatically calculates the value of the restraint and then
 evaluates the partial derivatives for each of the refinable  parameters


Thes restraints look like  simple
 fortran statements involving operators and operands.


\bigskip\Directive{OPERATORS}
 The available operators are :

\small\begin{verbatim}
      (
      )
      **            must be followed by an operand.
      *             must join two operands.
      /             must join two operands.
      +             must precede an operand.
      -             must precede an operand.
\end{verbatim}\normalsize


 An operand may be a simple variable or
 an expression enclosed in parentheses.


The operators above assume their normal FORTRAN meanings,
 and the combination of operands and operators is the same as in
 standard FORTRAN, except that all calculations are done
 in floating point.


\bigskip\Directive{ATOMIC COORDINATES}
 These are specified by a modified form of the
 atom definition given above. This is :
\small\begin{verbatim}
       TYPE(SERIAL,S,L,TX,TY,TZ,KEY)
\end{verbatim}\normalsize


  KEY  Specifies the relevant coordinate of the
 atom. The  KEY  is regarded as an
 obligatory parameter, but for the remaining
 symmetry parameters, the drop
 out rules and default
 settings described under the atom definition may be
 applied, so that the simplest form of
 coordinate definition is  TYPE(SERIAL,KEY), similar to a LIST 12
 definition .
 The usual parameter keys are recognized.





\bigskip\Directive{OVERALL PARAMETERS}
 The usual overall paameter keys are recognized.


\bigskip\Directive{VARIABLES}
 These are unsubscripted variables specified by up to
 8 characters, of which the first must be a letter.
 Many commonly occurring crystallographic
 quantities are already prestored by the system,
 and the user has the ability to declare new
 constants with a 'DEFINE' directive, which
 is described below.
 When a user defines a new variable, he must not use a name that
 has already been declared by the system.
 The system variables are:


\bigskip\Directive{ARRAY VARIABLES}


The system has pre-stored various arrays and variables holding
 useful crystallographic information, and
 users may not define or declare  new arrays.
 The addressing is done in
 the normal Fortran manner, except that
 the element required must be specified by numeric
 arguments, and not variables.
 Thus A(3,1) is allowed, but A(I,J) is illegal.
\small\begin{verbatim}
      A(6)       the cell parameters (angles in radians)
      CV         real cell volume
      AR(6)      reciprocal cell parameters (angles in radians)
      RCV        reciprocal cell volume
      G(3,3)     real metric tensor
      GR(3,3)    reciprocal metric tensor
      L(3,3)     real orthogonalization matrix
      LR(3,3)    reciprocal orthogonalization matrix
      CONV(3)    conversion factor for the 'U(ij)'s' from 'U[iso]'
      RIJ(6)     coefficients needed to calculate [sin(theta)/l]**2
      ANIS(6)    coefficients needed to calculate the temperature
                 factor from the anisotropic temperature factors
      SM(3,4,p)  symmetry matrix 'p', where the translational
                 part is stored in sm(i,4,p)
      SMI(3,4,p) inverse symmetry operators
      NPLT(3,n)  non-primitive lattice translations
      PI         3.141......... etc.
      TPI        2*Pi
      TPIS       2*pi*pi
      DTR        conversion of degrees to radians
      RTD        conversion of radians to degrees
      ZERO       0.0
\end{verbatim}\normalsize


 The following functions
 are also recognized :

\small\begin{verbatim}
      SIN(ARG)      COS(ARG)      TAN(ARG)      ACOS(ARG)
      ASIN(ARG)     ATAN(ARG)     EXP(ARG)      SQRT(ARG)
\end{verbatim}\normalsize


\section{General restraints}



There are two directives.


\bigskip\Directive{DEFINE NAME = TEXT}
 This may be used to set up a user-defined constant which is evaluated
 at run time, and  which may be referred to later on by 'NAME'.
 The text comprises  a series of variables and
 numeric constants interspersed with operators.
 The 'NAME' must not be one of the standard functions or
 variables, and may be overwritten several times  -  i.e.
 its value may be redefined. Derivatives of a DEFINEd constant are not
 evaluated as part of the restraint, though its numerical value will
 change in successive cycles of refinement.


\bigskip\Directive{RESTRAIN VALUE, E.S.D. = TEXT}
 The physical or chemical quantity defined by the
 'TEXT' is restrained to be 'VALUE', with
 an e.s.d. of 'E.S.D.'.
 The text is comprised of operands separated by
 operators.
 The system will differentiate the 'TEXT' with
 respect to each of the refinable coordinates
 that it contains and add the derivatives to
 the normal matrix in the usual way.
\section{Debugging restraints}



Debugging commands are available to help with the creation of general
 restraints


\bigskip\Directive{COMPILER }
 During the formation of LIST 26 (see \ref{LIST26}),
 the input directives are listed, together
 with various internal stacks.


\bigskip\Directive{EXECUTION }
 During the application of the restraints to
 the normal equations, various stacks are
 printed and all the calculated derivatives
 are printed (use with care).
\section{Printing the contents of LIST 16}



The contents of LIST 16 may be listed with:


\bigskip\Instruction{$\backslash$PRINT 16}


 or


\bigskip\Instruction{$\backslash$SUMMARY LIST 16}




LIST 16 may be punched with:


\bigskip\Instruction{$\backslash$PUNCH 16}





\section{Creating a null LIST 16 }



A null LIST 16, containing no restraints, may be created with


\bigskip\Instruction{$\backslash$CLEAR 16}





\small\begin{verbatim}
 restrain a set of distances to 1.5 angstrom with an
 e.s.d. of 0.03, note the use of symmetry indicators.

      DISTANCE 1.5 , 0.03 = C(1) TO S(1) , C(1,5) TO S(1,5)
      CONT                  S(1,7,1,-1) TO C(1,7,1,-1)

 restrain the first distance above explicitly, by a user defined
 restraint

      RESTRAIN 1.5 , 0.03 = SQRT
      CONT ((C(1,5,X)-S(1,5,X))*(C(1,5,X)-S(1,5,X))*G(1,1)
      CONT +(C(1,5,X)-S(1,5,X))*(C(1,5,Y)-S(1,5,Y))*G(1,2)
      CONT +(C(1,5,X)-S(1,5,X))*(C(1,5,Z)-S(1,5,Z))*G(1,3)
      CONT +(C(1,5,Y)-S(1,5,Y))*(C(1,5,X)-S(1,5,X))*G(2,1)
      CONT +(C(1,5,Y)-S(1,5,Y))*(C(1,5,Y)-S(1,5,Y))*G(2,2)
      CONT +(C(1,5,Y)-S(1,5,Y))*(C(1,5,Z)-S(1,5,Z))*G(2,3)
      CONT +(C(1,5,Z)-S(1,5,Z))*(C(1,5,X)-S(1,5,X))*G(3,1)
      CONT +(C(1,5,Z)-S(1,5,Z))*(C(1,5,Y)-S(1,5,Y))*G(3,2)
      CONT +(C(1,5,Z)-S(1,5,Z))*(C(1,5,Z)-S(1,5,Z))*G(3,3))

 restrain some distances to their mean

      DISTANCE 0.0 , 0.03 = MEAN O(1) TO S(1) O(2) TO S(1)
      CONT                       O(1,2) TO S(1) O(1,7) TO S(1)

 vibration restraints along a bond

      VIBRATION 0.0 , 0.01 = S(1,5) TO O(1,5) S(1,7) TO C(1,7)
      CONT                  S(1) TO O(1) S(1) TO C(1)

 thermal similarity restraints

      U(IJ)  0.0 , 0.01 = S(1,5) TO O(1,5) S(1,7) TO C(1,7)
      CONT                S(1) TO O(1) S(1) TO C(1)

 user defined restraints to some of the U(IJ)'S. This might cure a npd
 atom

      RESTRAIN 0.0,0.01=S(1,U[11])-S(1,U[33])
      RESTRAIN 0.0,0.01=S(1,U[12])
      RESTRAIN 0.0,0.01=S(1,U[13])
      RESTRAIN 0.0,0.01=S(1,U[23])

\end{verbatim}\normalsize


\section{The special restraints  -  LIST 17}
\label{LIST17} 



\bigskip\Instruction{$\backslash$LIST 17}




LIST 17 is  generated automatically
 by the command $\backslash$SPECIAL (section \ref{SPECIAL}), 
 and is intended to take care of floating
 origins and atoms on special positions.
 The user may create their own LIST 17, but this will be over written by
 SPECIAL unless it this is deactivated.



\section{Printing and punching LIST 17}

 LIST 17 may be printed with:


\bigskip\Instruction{$\backslash$PRINT 17}


 or


\bigskip\Instruction{$\backslash$SUMMARY LIST 17}





 It is punched with:


\bigskip\Instruction{$\backslash$PUNCH 17}





\section{Creating a null LIST 17 }



A null LIST 17, containing no restraints, may be created with


\bigskip\Instruction{$\backslash$CLEAR 17}





\section{Checking restraints - CHECK}


 $\backslash$CHECK LEVEL=
 END

 $\backslash$CHECK HI
 END





The target values for the restraints can be checked against
 the calculated values by issuing the following command :


\bigskip\Instruction{$\backslash$CHECK LEVEL=}




\bigskip\Keyword{LEVEL=}

\small\begin{verbatim}
      LOW      Default value
      HIGH
\end{verbatim}\normalsize




This command causes the restraints to be calculated,
 but not added into the normal equations.
 The observed and calculated values are output to the listing file,
 with a summary on the terminal. If the LEVEL is LOW, only restraints
 where the calculated value differs significantly from the target are
 printed, otherwise all restraints are printed.


If a cycle of refinement is done before issuing the command, the leverages 
 of all the restraints are calculated and printed on the screen and the lis file.
 A leverage of 0 means that the restraint has no influence on the parameter, 
 a leverage of 1 means that the restraint completely determine the value of the parameter,
 a value between 0 and 1 indicate that bot the X-ray data and the restraint contribute to the parameter.



\index{LIST 4}
\index{Weighting schemes}
\section{Weighting schemes for refinement-  LIST 4}
\label{LIST04} 


\small\begin{verbatim}
 \LIST 4
 SCHEME NUMBER= NPARAMETERS= TYPE= WEIGHT= MAXIMUM=
 PARAMETERS P=
 END

 \LIST 4
 SCHEME 14 3
 END

\end{verbatim}\normalsize




The weighting of least squares refinement is still very controversial.
 The matter is discussed at some length by Schwartzenbach \emph{et} \emph{al} in
 Statistical Descriptors, and further insights may be gleaned from
 Numerical Recipies. Weighting the refinement can serve several purposes,
 and the weighting
 may need to be changed as the refinement proceeds. The weighting of Fo
 and Fsq refinements will be different. To a first approximation,
\small\begin{verbatim}
      w(Fsq) = w(Fo)/2Fo
                              note the problem as Fo approaches 0.0
\end{verbatim}\normalsize




Initially the analyst
 must choose a scheme which will hasten the rate of convergence, and
 reduce the risk of the refinement falling
 into a false minimum.
 Towards the end of the refinement, once all the parameters have been
 approximately refined, a different scheme will be necessary to generate
 reliable parameter s.u.s (e.s.d.s)



My advice (DJW,1996) is
 to use unit weights for Fo refinement (1./4Fsq for Fsq refinement) until
 the structure is fully parameterised, and then an empirical scheme for
 the final refinement. It seems that pure 'statistical' weights are
 rarely satisfactory. The crucial thing is to look at the analysis of
 variance (/ANALYSE). The weighted residual (see definition of Fo' etc above)
 w(Fo'-Fc')**2 should be invariant for any rational
 ranking of the data. If there are any trends, then either the model is
 wrong or the estimates of w are wrong. If the model is believed to be
 full parameterised and substantially correct, the trend in residual can
 be used to estimate the weights.
\section{Weighting for refinement against Fo}



This set of weighting schemes should be selected when
 the minimisation function that is to be used during the least squares
 process is given by :
\small\begin{verbatim}
       SUM( w*(Fo - Fc)**2 )
\end{verbatim}\normalsize


 Where the summation is over all the reflections.
\section{Weighting for refinement against Fsq}



Refinement against Fo or Fsq is also controversial. The controversy is
 not really concerned with negative observations, since Fo can be given
 the sign of Io. The real problem is that the error distribution for Fo
 is not the same as that of Fsq, and is not simply related to it for very
 weak reflections. However, the argument is academic, since the error
 estimates for Fsq are not really known.


CRYSTALS provides two different alternatives for the case in which the
 minimisation function is given by :
\small\begin{verbatim}
          SUM( w*(Fo**2 - Fc**2)**2 )
\end{verbatim}\normalsize


 In the first of these options, the weights are calculated normally for Fo,
 and then converted so that they apply to Fsq
 by the operation :
\small\begin{verbatim}
          w' = w/(4*Fo*Fo)
\end{verbatim}\normalsize


 Where w' is the weight for Fsq and  w  is the weight for Fo.
 This option is selected by the parameter  TYPE  =  1/2Fo
 in the  SCHEME  directive above.
 For example, pseudo unit weights are selected by the input :

\small\begin{verbatim}
      \LIST 4
      SCHEME 9 0 1/2Fo
      END
\end{verbatim}\normalsize


 This option may be used with any of the weighting schemes above.


The second option also uses the weighting scheme types for Fo, except
 that Fsq is substituted for Fo in the equations.
 This option is selected by the parameter  TYPE  =  Fo**2
 in the  SCHEME  directive above. This choice would be suitable for the
 Chebychev weighting schemes.


\bigskip\Instruction{$\backslash$LIST 4}




\bigskip\Directive{SCHEME NUMBER= NPARAMETERS= TYPE= WEIGHT= MAXIMUM=}


\bigskip\Keyword{NUMBER}
 The number of the weighting scheme to be used
 (see below). The default value is  9 (unit weights).


\bigskip\Keyword{NPARAMETERS=}
 The number of parameters to be provided for the weighting
 scheme. The default value is zero,


\bigskip\Keyword{TYPE}



\small\begin{verbatim}
      NORMAL
      1/2Fo
      Fo**2
      CHOOSE    -  Default value
\end{verbatim}\normalsize


 The value of  NORMAL  indicates that the weighting scheme type
 is for refinement against Fo.
 If  TYPE  is  1/2Fo  or  Fo**2  the weighting scheme type is for
 refinement against Fsq (see above).


If TYPE equals CHOOSE, one of the three previous type is chosen depending
 on the scheme number and the refinement type set in LIST 23 (structure
factor control, see section \ref{LIST23}).


\bigskip\Keyword{WEIGHT=}
 This parameter determines the weight assigned to reflections during the
 determination of Chebychev coefficients.
 For each reflection the weight with which it is added into the
 Chebychev normal equations is given by :
\small\begin{verbatim}
      W = 1/[1+Fo**WEIGHT].
\end{verbatim}\normalsize


 Thus if WEIGHT is equal to zero, all the weights will be the same and equal
 to  0.5.
 The default value is 2.0


\bigskip\Keyword{MAXIMUM=}
 This parameter is used to set the maximum weight that can be applied, and
 is usefull for the Dunitz-Seiler scheme (13), and the Chebyshev
 schemes (10 and 14).








\bigskip\Directive{PARAMETERS P=}


The parameters that are to be used to compute the weight for a given
 reflection are specified with this directive.


\bigskip\Keyword{P=}
 This directive contains  NPARAMETERS  values. If this parameter is omitted,
 default values of zero are assumed for  P.


The parameters must always be provided on the scale of Fo, not
 on the scale of Fc.
 For example, the agreement analysis programs can work on the scale
 of Fc, so that constants derived from such output must be put
 on the scale of Fo by multiplying them by the scale factor in LIST 5 
 (the model parameters).
\section{Weights stored in LIST 6}



If w is the weight to be applied to a reflection in the least squares
 refinement, the value to be stored in LIST 6  (section \ref{LIST06}) 
 is sqrt(w), given the key
 SQRTW. If weights are computed by some external utiity, then either is
 should generate sqrt(w), or the values be converted after input to
 CRYSTALS - see scheme 5 below.
\section{Weighting schemes}



In the equations and explanations below,  NP  is an
 abbreviation of  NPARAMETERS , the number of parameters required
 to define the weighting scheme, P(1) is the first such
 parameter and P(NP) the last parameter.


The available weighting schemes are :

\small\begin{verbatim}
 1.     sqrt(W) = Fo/P(1), Fo < P(1) OR Fo = P(1)
        sqrt(W) = P(1)/Fo, Fo > P(1)

 2.     sqrt(W) = 1      , Fo < P(1) OR Fo = P(1)
        sqrt(W) = P(1)/Fo, Fo > P(1)

 3.     sqrt(W) = sqrt(1/(1 + [(Fo - P(2))/P(1)]**2))

 4.     sqrt(W) = sqrt(1/[P(1) + Fo + P(2)*Fo**2 + . . + P(NP)*Fo**NP])

        try P(1) = 2*FMIN and P(2) = 2/FMAX,
        Cruickshank, Computing Methods and the Phase
        Problem, Pepinsky et al, 1961, page 45

 5.     sqrt(W) = SQRT(data with the key 'SQRTW' in list 6)

 6.     sqrt(W) = (data with the key 'SQRTW' in list 6)

 7.     sqrt(W) = SQRT(1/(data with the key 'sigma(Fo)' in LIST 6))

 8.     sqrt(W) = 1/(DATA WITH THE KEY 'SIGMA(Fo)' IN LIST 6)
                  ** remember that for schemes 7 & 8, LIST 6  **
                  ** must store both weight and sigma.        **

 9.     sqrt(W) = 1.0 (Unit weights, default)

 10.    sqrt(W) = sqrt(1.0/[A[0]*T[0]'(X) + A[1]*T[1]'(X) . .
                  +A[NP-1]*T[NP-1]'(X)])
                  Chebychev weighting - see below for details

 11.     As for 10, but only applying previously determined parameters.

 12.    sqrt(W) = sqrt([SIN(THETA)/LAMBDA]**P(1))
                  If  NP  is zero, a value of -1 is assumed for P(1) .

 13.    sqrt(W) = sqrt([weight] * exp[8*(p(1)/p(2))*(pi*s)**2])
                  Dunitz Seiler weighting - see below for details

 14.    sqrt(W) = sqrt(W' * (1. - (delta(F)/ 6* del(F)est)**2)**2)
             W' = 1.0/[A[0]*T[0]'(X) + A[1]*T[1]'(X) . .
                  +A[NP-1]*T[NP-1]'(X)]
                   Robust-resistant refinement - see below

 15.     As for 14, but only applying previously determined parameters.

 16.     sqrt(W) = Sheldrick SHELX-97 weights (page 7-31). The P1-P6
                   correspond to Sheldricks parameters a-f, but are not
                   refined automatically. Fo and Fc replace Fosq and
                   Fcsq for Fo refinement.
                   Use 0.1 0 0 0 0 .333 to get Sheldrick defaults.

 17.    Automatically determine the parameters for 16

\end{verbatim}\normalsize


\section{Dunitz Seiler weighting - scheme 13}

 S is sin(theta)/lambda, pi is 3.141...  Use p(1) = 1 and
 p(2) = 4 to simulate p=3, q=9 in Dunitz and Seiler
 Acta(1973),B29,589. Set MAXIMUM to 100. This scheme may be used for
 refinement before looking for bonding electrons.
\section{Chebychev weighting schemes 10, 11}




 A[i] are the coefficients of a Chebyshev series in t[i]'(x),
 where x = Fo/Fo(max).
 (There is an account of CHEBYSHEV series in Computing Methods
 in Crystallography, edited by J.S. Rollett, p40).
 For this weighting scheme, the coefficients a[i] are calculated
 by the program using a least squares procedure which
 minimizes sum[(Fo - Fc)**4] over all the reflections.
 The resulting coefficients are stored in a new LIST 4
 as weighting scheme type 11 (see below),
 and then used to calculate the weights
 for each of the reflections.
 It is recommended that several different values of
  NP  are used (e.g 3 to 5), so that series of various orders are tested
 to see which gives the best fit. If negative or very small reciprocal
 weights are computed (i.e. the computed curve fall close to or crosses
 the ordinate axis), the parameter MAXIMUM can be used to restrict the
 maximum weight. For data on 'ordinary' scales, this will require a value
 of about 100.
 (This is best seen by computing an agreement analysis
 once the new weights have been calculated).
 The parameters P(i) need not be given,  because they are to be computed.
 When the Chebyshev coefficients have been determined, p(1) is
 overwritten by the value determined for a[1]. (Carruthers and Watkin,
 Acta Cryst (1979) A35, 698).
 Scheme 10 generates the parameters needed for a scheme 11.
\section{Robust-resistant weighting schemes 14, 15}

 This scheme should only be used towards the end of a refinement, when
 all of the expected variables have been introduced.
 It is usefull when there is suspicion of uncorrelated but significant
 errors in the data, and its effect is similar to scheme 10 in the absence
 of such errors.
 The expression for W' is as in Scheme 10 above, except that X is Fc/Fc(max).
 This weight is then modified by a function expressing confidence in
 Fo-Fc. If the observed delta is large compared with the delta estimated
 from the Chebychev fitting, the reflection is down weighted. It it is more
 than 6 times the estimate, the weight is set to zero and the reflection
 flagged as an 'OUTLIER'. This scheme is recommended in preference to Scheme
 10, which is kept for old times sake. See E. Prince, Mathematical Techniques
 in Crystallography, Springer-Verlag, for the background.



\section{Statistical Weights, 16}

 This scheme can in principle be introduced at any time, but the
 parameters P(i) are best optimised near the end of a refinement.
 Typical values for CAD4 data are: p(1)=.001, p(2)=3.0 and p(4)=1.



\section{Auto-Statistical Weights, 17}

 This scheme can in principle be introduced at any time, but the
 parameters P(i) are best optimised near the end of a refinement.
 The parameters are optimised automatically by a grid-search of 
 the residual as a function of intensity and resolution.



\section{Printing LIST 4}



of LIST 4 may be printed with:


\bigskip\Instruction{$\backslash$PRINT 4}




There is no command for punching LIST 4.

\small\begin{verbatim}
 Example
      \ Weighting scheme type 10 (Chebyshev) with 3 parameters
      \LIST 4
      SCHEME NUMBER = 10,NPARAM = 3
      END
\end{verbatim}\normalsize


\section{Weighting the reflections - WEIGHT}



If the weighting scheme is changed, new weights are automatically
 computed for the next structure factor calculation. The computation of
 weights can be forced at any time with $\backslash$WEIGHT.


\bigskip\Instruction{$\backslash$WEIGHT INPUT= FACTOR=}




\bigskip\Keyword{INPUT}
Indicates which reflection list to use.
\small\begin{verbatim}
      6      Default
      7      Alternative reflection list
\end{verbatim}\normalsize





\bigskip\Keyword{FACTOR}


 If FACTOR is less or equal zero, the code tries to compute a value
 which will make S (GoF) about unity.  It is important that the GoF 
 stored in LIST 30 is up to date. The SCRIPT REWEIGHT will try to do this 
 for you.


 If FACTOR is unity, the re-weighting factor is taken from LIST 4.
 This is the default action



If FACTOR is greater than zero and not exactly unity,
 all weights will be multiplied by this amount.  









\bigskip\Directive{LIST LEVEL=}


\bigskip\Keyword{LEVEL}
LEVEL is OFF, LOW, MEDIUM or HIGH 








\index{LIST 28}
\index{Reflection restriction}
\index{Omitting reflections}
\index{Cutting reflection data}
\section{Reflection restriction list  -  LIST 28}
\label{LIST28} 


\small\begin{verbatim}
 \LIST 28
 MINIMA COEFFICIENT(1)= COEFFICIENT(2)= ...
 MAXIMA COEFFICIENT(1)= COEFFICIENT(2)= ...
 READ NSLICES= NOMISSION= NCONDITION=
 SLICE P= Q= R= S= T= TYPE=
 OMIT H= K= L=
 CONDITION P= Q= R= S= T= TYPE
 SKIP STEP=
 END
\end{verbatim}\normalsize



\small\begin{verbatim}
 \LIST 28
 MINIMA RATIO=3
 READ NOMIS=1
 OMIT 2 0 0
 END
\end{verbatim}\normalsize




LIST 6 (section \ref{LIST06}) should contain all the reflections, 
including negative ones.
 LIST 28 can then be used to dynamically select which ones are to be
 omitted from a calculation. Several conditions may be specified, and
 ALL the conditions must be satisfied for a reflection to be used,
 i.e. the conditions are ANDed  together.


It is also possible to specify individual reflections which
 are to be omitted.


TAKE CARE WHEN CHANGING LIST 28. If the conditions are relaxed, reflections
 may become acceptable for which Fc and phase have not been recomputed
 because they were rejected at an earlier stage. Recompute them all.





\bigskip\Instruction{$\backslash$LIST 28}




\bigskip\Directive{MINIMA COEFFICIENT(1)= COEFFICIENT(2)= .  .}


This defines the coefficients whose minimum values are
 to be restricted.


\bigskip\Keyword{COEFFICIENT=  VALUE}
 Each such parameter defines one coefficient and its minimum value.
 The following are known to the system, BUT REMEMBER, with the exception of
 (sintheta/lambda)**2, which is computed for each reflection from the cell
 parameters, only those coefficients specifically stored in the LIST 6
 (see section \ref{LIST06}) will have values.
\small\begin{verbatim}
      H             K             L             /FO/
      SQRTW         FCALC         PHASE         A-PART
      B-PART        TBAR          FOT           ELEMENTS
      SIGMA(F)      BATCH         INDICES       BATCH/PHASE
      SINTH/L**2    FO/FC         JCODE         SERIAL
      RATIO         THETA         OMEGA         CHI
      PHI           KAPPA         PSI           CORRECTIONS
      FACTOR1       FACTOR2       FACTOR3       RATIO/JCODE
\end{verbatim}\normalsize




\bigskip\Directive{MAXIMA COEFFICIENT(1) COEFFICIENT(2) .  .}


This defines the coefficients whose maximum values are
 to be restricted. See MINIMA above.





\bigskip\Directive{READ NSLICES= NOMISSION= NCONDITION=}


This gives the number of conditional directives to follow.


\bigskip\Keyword{NSLICES}
 This specifies the number of  SLICE directives, default is zero.


\bigskip\Keyword{NOMISSIONS}
 This specifies the number of OMIT directives, default is zero.


\bigskip\Keyword{NCONDITION}
 This specifies the number of  CONDITION directives, default is zero.


\bigskip\Directive{SLICE P= Q= R= S= T= TYPE=}


This directive selects reflections to those giving
 values of (h*p + k*q + l*r) in the range s to t.
 The number
 of such directives is specified on the  READ  directive above. TYPE
 indicates whether the selected reflections are accepted or rejected.

 The records are processed in the order given. If a reflection matches 
 the conditions, the specified action is taken and no further slice 
 directives are considered. This enables quite fancy intersections to be 
 specified.



 For example, a single layer of
 reciprocal points, or a set of adjacent layers, oriented in any
 desired crystallographic direction, can be selected.



\bigskip\Keyword{P= Q= R= S= T=}
 These parameters, whose default values are zero, specify selected
 slices of reciprocal space.


\bigskip\Keyword{TYPE=}
\small\begin{verbatim}
      REJECT (default) causes rejection of selected reflections.
      ACCEPT           accepts reflections
\end{verbatim}\normalsize







\bigskip\Directive{OMIT H= K= L=}


This directive causes the reflection with the indices H, K, and L
 to be omitted.


\bigskip\Keyword{H= K= L=}
 These parameters specify the indices of the reflection to be omitted.


\bigskip\Directive{CONDITION P= Q= R= S= T= TYPE=}


This directive causes selection of reflections giving
 values of (h*p + k*q + l*r + s) exact multiples of 't'.  TYPE
 indicates whether the selected reflections are accepted or rejected.
 The number
 of such directives is specified on the  READ  directive above.
 The records are processed in the order given. If a reflection matches 
 the conditions, the specified action is taken and no further slice 
 directives are considered. This enables quite fancy intersections to be 
 specified.



 For example, l odd layers can be rejected by setting
 'r' and 's' to 1, 't' to 2.



\bigskip\Keyword{P= Q= R= S= T=}
 These parameters, whose default values are zero, specify selected
 slices of reciprocal space.


\bigskip\Keyword{TYPE=}
\small\begin{verbatim}
      REJECT (default) causes rejection of selected reflections.
      ACCEPT           accepts reflections
\end{verbatim}\normalsize




\bigskip\Directive{SKIP STEP=}


This directive can be used sample the data by skipping through LIST 6,
 (reflections, section \ref{LIST06}) and may be usefull to speed up 
 initial refinement.


\bigskip\Keyword{STEP=}
 This is the skip step length, and has a default of 1, i.e. all reflections
 are accepted. A value of 3 selects every third reflection for use in 
 calculations (i.e. 2 out of 3 are skipped).



\section{Creating a null LIST 28}

\small\begin{verbatim}
 \LIST 28
 END
\end{verbatim}\normalsize


 Allows all the reflections in LIST 28 to be used.
\section{Printing the contents of LIST 28}



LIST 28 may be listed by the command :


\bigskip\Instruction{$\backslash$PRINT 28}




There is no command for punching LIST 28.

\small\begin{verbatim}
      Example 1
      \LIST 28
      \ Set the minimum ratio I/sigma(i) to 3.0,
      \ a maximum Fo to 1000
      \ and omit the 0 2 0 reflection
      \
      MINIMA Ratio=3
      MAXIMA Fo=1000
      READ NOMIS = 1
      OMIT 0 2 0
      END

      Example 2. To reject h and k simultaneously even:

      condit p=1 s=1 t=1 type=accept    \lets ALL with h odd through
      condit q=1 s=1 t=1 type=accept    \lets ALL with k odd through
      condit s=1 t=1 type=reject        \rejects remaining.

      Example 3. To reject all k=0, k=2:

      slice q=1 s=0 t=0 type=reject
      slice q=1 s=2 t=2 type=reject

      Example 4. To reject all k=0, k=2 but keep the l=0 row:

      slice r=1 s=0 t=0 type=accept
      slice q=1 s=0 t=0 type=reject
      slice q=1 s=2 t=2 type=reject

      Example 5. To only allow specific zones, the ones wanted are 
      selected, and then the rest rejected, eg for h=0:

      slice p=1 s=0 t=0 type=accept              \ accept the h00 zone
      slice p=1 q=1 r=1 s=-500 t=500 type=reject \ reject everything else

\end{verbatim}\normalsize


\index{SFLS}
\index{Structure factor least squares calculation}
\section{Structure Factor Least Squares Calculations  -  $\backslash$SFLS}
\label{SFLS} 


\small\begin{verbatim}
 \SFLS  INPUT=
 CALCULATE LIST= MAP= /Fo/= THRESHHOLD=
 SCALE LIST= MAP= /Fo/=
 REFINE LIST= MAP= /Fo/= PUNCH= MATRIX= MONITOR= INVERTOR= CALCULATE=
 SHIFT  KEY= KEY=
 MAXIMUM  KEY= KEY=
 FORCE  KEY= KEY=
 SOLVE MONITOR= MA=P /Fo/= PUNCH= MATRIX=
 VECTOR MONITOR= MAP= /Fo/= PUNCH= MATRIX=
 HUGE=
 END
\end{verbatim}\normalsize


\small\begin{verbatim}
 \SFLS
 REFINE
 REFINE
 END
\end{verbatim}\normalsize


\section{Definitions}



\bigskip\Directive{Minimisation funtion for Fsq}
\small\begin{verbatim}
      Minimisation function = Sum[ w*(Fo**2 - Fc**2)**2 ]
\end{verbatim}\normalsize




\bigskip\Directive{Minmisation function for Fo}
\small\begin{verbatim}
      Minimisation function = Sum[ w*(Fo - Fc)**2 ]
\end{verbatim}\normalsize




\bigskip\Directive{R-factor for Fo}
\small\begin{verbatim}
      R = 100*Sum[//Fo/-/Fc//]/Sum[/Fo/]
\end{verbatim}\normalsize


 The summation is over all the reflections accepted by LIST 28. This
 definition is used for both conventional and F-squared refinement.


\bigskip\Directive{R-Factor, Hamilton weighted}
\small\begin{verbatim}
      100*Sqrt(Sum[ w(i)*D'(i)*D'(i) ]/SUM[ w(i)*Fo'(i)*Fo'(i) ])

      Fo' = Fo for normal refinement, Fsq for F-squared refinement.
      Fc' = Fc for normal refinement, Fc*Fc for F-squared refinement.
      D'  = Fo'-Fc'
\end{verbatim}\normalsize




The weighted R-factor stored in  LIST 6 (section \ref{LIST06}) and LIST 30 
 (section \ref{LIST30}) is that computed
during a structure factor calculation. The conventional R-factor is
updated by either an SFLS calculation or a SUMMARY of LIST 6.





\bigskip\Directive{Minimisation function}

\small\begin{verbatim}
      Fo' = Fo for normal refinement, Fo*Fo for F-squared refinement.
      Fc' = Fc for normal refinement, Fc*Fc for F-squared refinement.
      D'  = Fo'-Fc'

      MINFUNC = Sum[ w(i)*D(i)*D(i) ]
\end{verbatim}\normalsize




Good references to the theory and practice of structure factor least squares
 are in the chapters by J. S. Rollett and D. W. J.
 Cruickshank in Crystallographic Computing, edited by F. R. Ahmed, and
 chapters 4, 5 and 6 in Computing Methods in Crystallography, edited by
 J. S. Rollett.

\section{Unstable refinements}



If a refinement 'blows up', i.e. diverges rapidly, the user should
 seek out the physical cause (wrong space group, pseudo symmetry,
 incorrect data processing, disorder, twinning etc). If the cause of the
 divergence is simply that the model is too inaccurate, the divergence
 can by controlled by limiting the shifts applied in the first few
 cycles. The modern way to do this is via 'shift limiting restraints'
 (Marquardt modifier) in LIST 16. An older method was to use partial
 shift factors. These are set up by directives to the $\backslash$SFLS command 
 (section \ref{SFLS}).

 During the solution of the normal equations, the user may specify
 that more or less than the whole calculated shift should be
 applied.
 Alternatively, the program can be instructed to scale the shifts so that
 the maximum shift for any parameter group is limited to a given value.
 (The  SHIFT ,  MAXIMUM  and  FORCE  directives).



\section{Sorting of LIST 6 for structure factor calculations}



During a structure factor least squares calculation, the values
 for the real and imaginary parts of  A  and  B  and their derivatives
 are computed and stored.
 These values are then taken and formed into Fc and its derivatives,
 which are added into the normal matrix.
 Between reflections, the values for A and B and their derivatives are
 retained.
 If the next reflection in LIST 6 (section \ref{LIST06}) has a set of indices 
 which are equivalent
 to the
 last reflection, the same values for the real and imaginary parts of A and B
 and their derivatives can be used.


This type of situation can arise either when anomalous scatterers
 are present, implying that F(h,k,l) is not equal to F(-h,-k,-l), or
 when an extinction parameter is being refined and formally equivalent
 reflections have different Fo values and mean path lengths.
 In this sort of case, the time for a structure factor calculation
 can be significantly reduced if reflections with symmetry related sets
 of indices are adjacent in LIST 6, when the conserved values
 of A and B can be used repeatedly.


In a similar way,
 during a structure factor calculation for a twinned crystal, the
 contribution and derivatives for each element are stored as they are
 calculated and then combined to produce /FCT/ when all the contributions
 have been accumulated.
 Between reflections this stored information is retained, so that if the
 next reflection contains contributions from elements with the same
 indices as the previous reflection,
 it is unnecessary to re-compute  the A and B parts.
 Obviously,
 reflections with common contributors must again be adjacent in LIST 6,
 in which case a structure factor calculation, with or without least
 squares, takes only slightly longer than the corresponding normal calculation
 with the same number of observations.


\bigskip\Instruction{$\backslash$SFLS}




The directives are carried out in the order in
 which they appear.


The directives  REFINE,  SCALE,  and CALCULATE
 initiate cycles of S.F.L.S. calculations. If one of the
 directives  SHIFT ,  MAXIMUM  or  FORCE  is given following  REFINE,
 a scaled shift will be applied to that cycle of refinement.


\bigskip\Directive{SFLS INPUT=}


\bigskip\Keyword{INPUT}
Indicates which reflection list to use.
\small\begin{verbatim}
      6      Default
      7      Alternative reflection list
\end{verbatim}\normalsize







\bigskip\Directive{CALCULATE LIST= MAP= /Fo/= THRESHOLD=}


If CALCULATE is included with other commands within a single $\backslash$SFLS 
block, it {\bf MUST} {\bf BE} the last command.


This directive indicates that structure factors should be
 calculated, but that no refinement of any type should be done. 
Structure factors are computed for every reflection and used to compute 
R and Rw for all data. R and Rw are also computed for reflections with 
I$>$threshold Sigma(I).  The default value for the threshold is 4.



 The directives
  SHIFT ,  MAXIMUM  and  FORCE  may not be given before the next
 REFINE directive.



\bigskip\Keyword{LIST}
 Controls the listing of reflection information.

\small\begin{verbatim}
      OFF   -  Default
      MEDIUM
      HIGH
\end{verbatim}\normalsize


 The value OFF  indicates that the
 discrepancy for each reflection (/Fo-Fc/)/Fo is computed and if greater than
 3*(overall R factor) from the previous cycle of structure factors, a warning
 is printed. Only the first 25 such reflections are listed.


If the ENANTIOPOLE parameter is activated in LIST 23 (structure factor
  control, see section \ref{LIST23}), sensitive reflections,
 for which /(F+)-(F-)/ .$>$ .05 *((F+)+(F-)/2), are also listed.


If  LIST  is  MEDIUM , the structure factors are listed as they
 are computed. The output contains h, k, l, Fo
 (on the scale of Fc), Fc, the phase and sin(theta)/lambda,
 the unweighted and weighted delta's.
 (Fo - Fc or Fo**2 - Fc**2, depending upon the type of refinement
 being done), and information which is useful when anomalous
 dispersion effects are present, and contains the real part of Fc (/Fc'/),
 the imaginary part of Fc (/Fc"/), the computed difference between
 /F(h,k,l)/**2 and /F(-h,-k,-l)/**2, and the calculated or theoretical
 Bijvoet ratio (t.b.r.).
\index{Twinning}


When a twinned crystal structure is being refined,  LIST  =  HIGH
 gives FoT and /FcT/ in place of Fo and Fc, respectively.
 Also, the contributions of each element to each reflection of a twinned
 crystal are listed. As well as /FcT/ and the indices,
 Fc, multiplied by the square root of the element scale factor, and the
 element number are also printed for each component under the column headed
 by /FC'/. This option is only obeyed if LIST 13 (section \ref{LIST13})
 indicates that a twinned crystal structure is being refined.


List MEDIUM and HIGH produces one line of output for each reflection.


\bigskip\Keyword{MAP}
 Controls printing of the memory map - mainly used by programmers

\small\begin{verbatim}
      NONE  -  Default value
      PART
      FULL
\end{verbatim}\normalsize


 If  MAP  is  PART , a list of core addresses is printed,
 together with any unused locations. If  MAP  is  FULL , the addresses and
 contents of the areas of code claimed by each list as it is brought down
 are printed on the line printer as the list is loaded.
 This option produces reams of output and should never be used
 except by programmers.
 If  MAP  is  NONE , its default value, a core map is
 not printed.


\bigskip\Keyword{/Fo/}
 Controls the treatment of twinned data.

\small\begin{verbatim}
      FoT        -  Default value
      Scaled-FoT
\end{verbatim}\normalsize


 In the refinement of a twinned crystal, if  /Fo/  =  FoT , its
 default value, the FoT is output as the data for the key  /Fo/ , the
 corresponding /FcT/ is output for  Fc , and the phase is arbitrarily
 set to zero. If  /Fo/  is  SCALED-FoT , the data for  Fo ,  Fc
 and  PHASE  contain an estimate of the required quantities for the
 element in whose reference system the nominal indices are given, i.e.
 estimates of the resolved data are produced.


\bigskip\Keyword{THRESHOLD}
 Sets a sigma(I) threshold for computing the restricted Rfactor. The 
default value is 4.0


\bigskip\Directive{SCALE LIST= MAP= /Fo/=}


This directive indicates that structure factors should be calculated
 and the overall scale factor should be refined.
 The directives
  SHIFT ,  MAXIMUM  and  FORCE  may not be given before the next
 REFINE directive.


\bigskip\Keyword{LIST}
 This parameter has the same options as for
 the  CALCULATE  directive above.


\bigskip\Keyword{MAP}
 This parameter has the same options as for
 the  CALCULATE  directive above.


\bigskip\Keyword{Fo}
 This parameter has the same options as for
 the  CALCULATE  directive above.


\bigskip\Directive{REFINE LIST= MAP= /Fo/= PUNCH= MATRIX= MONITOR= INVERTOR= }


This directive indicates a complete structure factor least squares
 calculation.


\bigskip\Keyword{LIST}
 This parameter has the same options as for
 the  CALCULATE  directive above.


\bigskip\Keyword{MAP}
 This parameter has the same options as for
 the  CALCULATE  directive above.


\bigskip\Keyword{Fo}
 This parameter has the same options as for
 the  CALCULATE  directive above.


\bigskip\Keyword{PUNCH}
 Controls punching results to files

\small\begin{verbatim}
      NO  -  Default value
      YES  -  write the model parameters to a *.pch file
      MATLAB  - write the design matrix, normal matrix, inversion and weights to matlabs files *.m
      TEXT  - write the design matrix, normal matrix, inversion and weights to ASCII files *.dat
      NUMPY  - write numpy files for python plus a python script for the user to use
\end{verbatim}\normalsize

\bigskip\Keyword{MATRIX}
 Controls re-use of the normal matrix

\small\begin{verbatim}
      NEW  -  Default value
      OLD
\end{verbatim}\normalsize


 If MATRIX is NEW, a new matrix is computed for the current cycle.





 If  MATRIX  is  OLD , the left hand side of the normal matrix is not
 accumulated during the cycle of refinement. Instead, the version that
 already exists is used with the new right hand sides. This option
 is particularly useful at the end of a refinement of a large structure
 when the left hand side does not change appreciably from cycle to cycle.
 It greatly reduces the time for a cycle.


\bigskip\Keyword{MONITOR}
 Controls the shift information printed out.

\small\begin{verbatim}
      LOW  -  Default value
      MEDIUM
\end{verbatim}\normalsize


 The MEDIUM listing outputs details about all parameters refined, and lists
 the values, shifts and e.s.d.s of all parameters in LIST 5. The LOW
 listing outputs information only for those l.s. parameters for which the
 SHIFT RATIO exceeds 3.0, and/or the SHIFT/ESD  exceeds 1.0 . Only those
 atoms in LIST 5 containing one or more refinable parameters are listed.


\bigskip\Keyword{INVERTOR}
 Six matrix inversion methods are provided.

\small\begin{verbatim}
      AUTO - Default value (Auto-adaptive single precision/double precision LDLT inversion)
      LDLT - single precision Robust Cholesky decomposition of a matrix with pivoting
      EIGENVALUE - Eigenvalue decomposition + filtering on small eigenvalues then inversion
      DP_LDLT - double precision Robust Cholesky decomposition of a matrix with pivoting
      XCHOLESKY - original crystals Cholesky decomposition
      CHOLESKY - Standard Cholesky inversion
\end{verbatim}\normalsize


 The AUTO method is suitable for most problem including moderatly ill condition problems.
 Crystals will switch automatically from single precision to double precision for the inversion when needed.
 The eigenvalue decomposition should be used with care, the XCHOLESKY method is kept for historical purpose,
 The LDLT and DP\_LDLT can be use to forced crystals to use single or double precision. Beware of precision loss
 in single precision. The CHOLESKY method is deprecated.


\bigskip\Directive{SHIFT  KEY = VALUE  KEY = VALUE .  .  .  .}


This directive sets the shift factor for the specified cycle of
 refinement. (The shift factor is the amount by which the calculated
 shift is multiplied before it is applied to generate the new parameters).
 For each of the parameters given by the  KEYS  on the directive,
 the shift factor is changed to the value given by  VALUE.
 The  =  sign is not optional. 



 If more
 than one shift directive ( SHIFT ,  MAXIMUM  or  FORCE ) is given for
 the same parameter, only the last is obeyed.



 The following  KEYS  are recognized :

\small\begin{verbatim}
      GENERAL   This refers to all the atomic, batch and element parameters
      OVERALL   This refers to the overall parameters

      OCC     U[ISO]      X       Y       Z
      U[11]   U[22]   U[33]   U[23]   U[13]   U[12]
      SPISO   SPSIZE  LINISO  LINSIZE LINDEC  LINAZI
      RINGISO RINGSIZ RINGDEC RINGAZI 
\end{verbatim}\normalsize


 This is the default for GENERAL, OVERALL and 
 SPECIAL parameters, with default shift factors 1.0.


\bigskip\Directive{MAXIMUM  KEY = VALUE  KEY = VALUE .  .  .  .}


This directive is similar to the directive  SHIFT  above, except that the
 maximum shift that is applied for the given parameters cannot be
 greater than  VALUE. The units of  VALUE  are conventional,
 WITH x, y, z measured in angstrom, and ADPs in Angstrom sq. If none of 
 the shifts exceend VALUE, then they are applied unmodified.
 This provides a
 method of automatically scaling down the applied shifts if the matrix 
 inversion has become unstable. Shift limiting restraints (LIST 16) are a 
 more controlled alternative



 The KEYS are the same as in SHIFT above, and this is the default 
 action for Occ, adps and positions, with maximal values:
\small\begin{verbatim}
      OCC 1.0
      U*  0.05 (Angstrom sq)
      X's 1.0  (Angstom)

\end{verbatim}\normalsize




\bigskip\Directive{FORCE  KEY = VALUE  KEY = VALUE .  .  .  .}


This is similar to MAXIMUM  above, except that the
 maximum shift is scaled to VALUE even if it is less than VALUE. 






 The KEYS are the same as in SHIFT above. There is no default
 shift.


\bigskip\Directive{VECTOR MONITOR MAP Fo PUNCH MATRIX}


This is an obsolete feature, and will be removed at a later date.


This directive indicates that structure factors are to be calculated
 and then the shift vector stored in LIST 24 (see \ref{LIST24}) is to be applied.
 This is used to apply a shift vector calculated from one of
 the eigenvalues of the normal matrix. Although no new matrix is produced
 by this directive, sufficient space must be allocated for the normal
 matrix, since it is loaded when the new coordinates are calculated.


\bigskip\Keyword{HUGE}


This directive tuned the refinement for big structures. It performs the
 the normal matrix accumulation in single precision, 
 the accumulation of the right and side is also done in single precision.
 The inversion is still using double precision but with a 
 Cholesky decomposition faster than the default auto-adaptive LDL^t decomposition. 
 Bigger structures tend to be more ill-conditioned so inversion needs to be done in double precision.
 These changes could potentially lead to a less stable refinement but it is unlikely, 
 the default refinement is just very conservative.


KCALCULATE
 LEVERAGES  - CALCULATE the leverages and t-values. Warning: shifts are not applied in this case; 
 default output is numpy but can be changed to TEXT using PUNCH. MATLAB and pch is not supported at the moment.



 This directive CALCULATE is differente from the previous directive and is applied under REFINE.
 


\bigskip\Keyword{MONITOR}
 This parameter has the same options as for
 the  CALCULATE  directive above.


\bigskip\Keyword{MAP}
 This parameter has the same options as for
 the  CALCULATE  directive above.


\bigskip\Keyword{Fo}
 This parameter has the same options as for
 the  CALCULATE  directive above.


\bigskip\Keyword{PUNCH}
 This parameter has the same options as for
 the  REFINE directive above.


\bigskip\Keyword{MATRIX}
 This parameter has the same options as for
 the  REFINE  directive above.
\section{Processing of the refinement directives}



The program expands the CALCULATE, SCALE or REFINE directives into
 sub-directives. These sub-directives {\bf MUST} {\bf NOT} be
  given by a user:

\small\begin{verbatim}
 1.   \REFINE
                  Compute structure factors and
                  derivatives. No refinement is
                  actually done.
 2.   \SCALE
                  Calculate structure factors and
                  refine the overall scale factor.
 3.   \CALCULATE
                  Calculate structure factors.
 4.   \RESTRAIN
                  Apply the restraints stored in
                  the current lists 16 and 17.
 5.   \INVERT
                  Invert the current normal matrix
                  and store a shift list as list 24.
 6.   \SOLVE
                  Take the current list 5 (the model parameters)
                  and apply the shifts given in the current
                  list 24.
 7.   \NEWSHIFTS
                  Allocate space for list 24.
 8.   \CYCLENDS
\end{verbatim}\normalsize





\index{ANALYSE}
\index{Residual analysis}
\section{Analysis of residuals -  $\backslash$ANALYSE}


\small\begin{verbatim}
 \ANALYSE INPUT=
 FO INTERVAL= TYPE= SCALE=
 THETA INTERVAL=
 LIST LEVEL=
 LAYERSCALE AXIS= APPLY= ANALYSE=
 END
\end{verbatim}\normalsize



\small\begin{verbatim}
 \ANALYSE
 LIST HIGH
 END
\end{verbatim}\normalsize




ANALYSE provides a comparison between Fo and Fc as a
 function of the indices, various parity groups, ranges of F and
 ranges of sin(theta)/lambda. For a well refined structure with suitable
 weights, $<$Fo$>$/$<$Fc$>$ should be about unity for all ranges, and $<$wdeltasq$>$
 should also be about unity for all ranges. A serious imbalance in Fo/Fc
 may mean the structure is incomplete, or unsuitable data reduction 
 (section \ref{DATAREDUC}). A
 systemstic trend in $<$wdeltasw$>$ may mean unsuitable weights are being
 used.


The monitor listing is always just as a funtion of F. The output to
 the listing file is user controlled.


This routine will also compute approximate layer scale factors for
 data which has been collected by layers. These can be refined in the
 least squares to complete a refinement.


\bigskip\Instruction{$\backslash$ANALYSE INPUT=}




\bigskip\Keyword{INPUT}
Indicates which reflection list to use.
\small\begin{verbatim}
      6      Default
      7      Alternative reflection list
\end{verbatim}\normalsize





\bigskip\Directive{FO INTERVAL= TYPE= SCALE=}
 Controls the analysis as a function of F.


\bigskip\Keyword{INTERVAL=}
 The interval between successive ranges of F.
 Its value should be determined in combination with the
 parameter  TYPE.


\bigskip\Keyword{TYPE}
 Controls how F is sampled.

\small\begin{verbatim}
      sqrt(Fc)  -  Default value
      Fc
      sqrt(Fo)
      Fo
\end{verbatim}\normalsize


 If  TYPE  is  sqrt(Fc),
 (its default value), the interval between successive F ranges is based
 on 'the square root of Fc' Values of Fc in the interval
 '0 to INTERVAL**2' will be analysed in the first range, Fc values that lie
 in the range INTERVAL**2 to 4*INTERVAL**2 in the second and so on).
  INTERVAL  is thus the increment in the square root of Fc between
 successive ranges and, in this case, has a default value of 1.
 If  TYPE  is  Fc , the interval between successive Fc ranges is
 based on the value of Fc. In this case  INTERVAL  is the increment
 in Fc and has a default value of 2.5.


\bigskip\Keyword{SCALE}
 Controls the scale of the listing

\small\begin{verbatim}
      Fo  -  Default value
      Fc
\end{verbatim}\normalsize


 If  SCALE  is  Fo, the reflection information
 is printed on the scale of Fo.
 (This is useful as the weighting parameters in LIST 4 (section \ref{LIST04})
 must be provided on the scale of Fo).
 If  SCALE  is  Fc, the reflection information is printed on the scale
 of Fc.


\bigskip\Directive{THETA INTERVAL=}


This directive determines the interval between successive
 sin(theta)/lambda squared ranges.


\bigskip\Keyword{INTERVAL=}
 The default is 0.04.


\bigskip\Directive{LAYERSCALE AXIS= APPLY= ANALYSE=}


This directive allows the results of layer scaling to be
 investigated.


\bigskip\Keyword{AXIS=}
 Selects the axis for layer scaling

\small\begin{verbatim}
      NONE  -  Default value
      H
      K
      L
\end{verbatim}\normalsize


 The default value of  NONE  indicates that no layer scaling is to be
 done.  H,  K  and  L  indicate the axes up which layer scaling
 is to be done.


\bigskip\Keyword{APPLY=}

\small\begin{verbatim}
      NO   -  Default value
      YES
\end{verbatim}\normalsize


 When layer scaling has been completed and the results printed,
 the calculated scale factors will be applied to the stored Fo data
 if  APPLY  is  YES . If  APPLY  is  NO , its default setting,
 then the new scales will not be applied to the data.
 If  AXIS  is  NONE , then  APPLY  is ignored.


\bigskip\Keyword{ANALYSE}

\small\begin{verbatim}
      NO
      YES  -  Default value
\end{verbatim}\normalsize


 If  ANALYSE  is  YES, a second agreement analysis
 will be performed after the layer scaling so that the results of the
 new scales can be seen. (This is true whether the new scales are
 applied or not, i.e. independent of the value of  APPLY ).
 In this way the effects of layer scaling can be seen without damaging
 the data. If  ANALYSE  is  NO , the second agreement analysis is
 suppressed.


\bigskip\Directive{LIST LEVEL=}


This directive determines the amount of output produced.


\bigskip\Keyword{LEVEL=}

\small\begin{verbatim}
      HIGH
      LOW   -  Default value
\end{verbatim}\normalsize


 If LEVEL is LOW the  analysis is against Fo and sin theta only. 
 


 The high level listing  sends 2 dimensional (Fc vs resolution)
 analyses ($<$Fo$>$/$<$Fc$>$ and $<$Fo-Fc$>$) 
 to the listing file and as a punch file, which could 
 be imported into EXCEL and used to create a 'heat map'.
 (In EXCEL see Conditional Formatting).



 The averages are over unweighed values of Fo and Fc because of issues
 with converting the weights if refinement was against F^2.



\index{DIFABS}
\index{Least squares absorption corrrection}
\section{Least squares absorption correction  - $\backslash$DIFABS}
\label{DIFABS} 


\small\begin{verbatim}
 \DIFABS  ACTION= MODE= INPUT=
 CORRECTION THETA=
 DIFFRACTION GEOMETRY= MODE=
 END

 \DIFABS ACTION=NEW MODE=FC
 END

\end{verbatim}\normalsize




Although this is a least squares fitting technique for an arbitary
 model, it does not form part of the main refinement module. The DIFABS
 parameters cannot be refined simultaneously with the atomic parameters.


A low order term Fourier series is used to model an absorption surface for
 differences between the observed structure factors and those obtained from a
 structure factor calculation after isotropic least squares refinement.

 For SERIAL diffractometrs (e.g. CAD4)
 Spherical polar angles are used to define the incident and diffracted
 beam path
 directions so that each reflection is characterised by four angles - viz.
 PHI(p),
 MU(p), PHI(s), and MU(s).  A theta-dependent correction is evaluated to allow
 for diffracted beams with different path lengths occurring at the same polar
 angles.  A low order term Fourier series is used in Bragg angle THETA,
 but is highly correlated with the temperature factors, and not normally
 recommended.
  This version is general for any SERIAL diffractometer data collection
 geometry.


For AREA detector diffractometers equivalent reflections are usually 
 measured at different setting angles and then averaged.   
 The concept of incident and emergent beams has no meaning for merged 
 data, so instead DIFABS computes the azimuth and declination of the
 scattering vector.  This cannot be reliably used as the basis for a 
 correction, but the "absorption" surface is a measure of the mis-match
 between Fo and Fc.


The quantity minimised is the sum of the squares of the residuals, Rj, where



\small\begin{verbatim}
               Rj = ( //Fc/-/Fo//)wj
\end{verbatim}\normalsize




The weighting function, wj, used is derived from the overall scale factor,
 the counting statistics standard deviation, and the Lorentz-polarisation
 factor.


In the original implementation, the correction factor was applied to
 Fo. This lead to criticism in the literature that the observations were
 being tampered with. In the current implementaion in CRYSTALS, the
 correction can be applied to Fo or Fc.








References:
 N. Walker and D. Stuart, 1983, Acta Cryst., A39, 158 - 166.
 The code is incorporated with the permission of Dr N. Walker





Implementation





The correction is evaluated using observed structure factors, /Fo/, corrected
 for Lorentz-polarisation effects and any decay in intensity standards during
 data collection, with systematically absent reflections removed. Since
 equivalent reflections will be measured at different diffractometer
 settings, the  correction should be  calculated and  applied to the data set
 without any transformation of the reflection indices, and without
 symmetry- equivalent or Friedel-pair reflections being averaged.
 Calculated  amplitudes must be obtained from  the  isotropic
 refinement of an as-complete a model as practical from the unique (merged)
 data set. Such a LIST 6 (reflections, section \ref{LIST06}) will probably be 
 unsuitable for Fourier or
 difference maps (since these expect a unique segment of data only) unless
 you then remerge the data. The best maps must be computed
 with the correction applied to Fo before the data is merged.  In addition,
 the most reliable merging R factor (Rint) must be computed from
 corrected Fos.



\bigskip\Directive{WARNING }
 To use DIFABS most successfully, you should probably do data-reduction
 again from scratch, inhibiting the merging of all but exactly equivalent
 reflections.


In favourable cases, when the observed data is the unique segment plus a
 small redundant volume (e.g. often the -1 layers at Oxford), you may get away
 with applying the correction to normally (merged) processed data during
 structure development.


Once the structure is fully developed (ie all atoms found
 and partially refined with an extinction correction if necessary), data
 reduction should be repeated inhibiting all index transformations.
 New values of Fc must be computed from
 isotropic atoms (Use UEQUIV in $\backslash$EDIT to recover equivalent isotropic
 temperature factors, and then do a few cylces of isotropic refinement)
 and the DIFABS correction applied to Fc. Anisotropic refinement can be
 computed to completion (including optimisation of weights) using
 unmerged data. If you wish to see an  absorption correctd Rint
 and compute a final difference map, the data must be re-merged. Use
 DIFABS with MODE = TRANSFER to move the correction onto Fo before
 transforming indices, sorting and merging the data.




\bigskip\Instruction{$\backslash$DIFABS  ACTION= MODE= INPUT=}




\bigskip\Keyword{INPUT}
Indicates which reflection list to use.
\small\begin{verbatim}
      6      Default
      7      Alternative reflection list
\end{verbatim}\normalsize




\bigskip\Keyword{ACTION=}
 Controls the action on LIST 6 (reflections, section \ref{LIST06}), and 
has three values:

\small\begin{verbatim}
      TEST   -   Computes the correction, but does not apply it
      UPDATE -   Tries to update LIST 6
      NEW    -   DEFAULT. Creates new LIST 6
\end{verbatim}\normalsize




If UPDATE is specified, the stored values of Fo are over written. If
 NEW is specified, a new LIST 6 is written to disc. The
 disc will be extended sufficiently to accommodate the new list.



\bigskip\Keyword{MODE=}
 Controls the mode of application of the correction, and
 has three values:

\small\begin{verbatim}
      FO        - Applies the correction to Fo
      FC        - Applies the correction to Fc
      TRANSFER  - Applies the inverse of the Fc correction to Fo
\end{verbatim}\normalsize




\bigskip\Directive{CORRECTION THETA=}


 Controls whether a theta-dependent correction is to be
 applied. - NOT RECOMMENDED.


\bigskip\Keyword{THETA=}

\small\begin{verbatim}
      NO - Default value
      YES
\end{verbatim}\normalsize




\bigskip\Directive{DIFFRACTION GEOMETRY= MODE=}


 Controls the geometry used for data collection to be input.


\bigskip\Keyword{GEOMETRY=}
 The type of diffractometer used is specified:

\small\begin{verbatim}
      CAD4          - Default value
      SYNTEX-P1
      SYNTEX-P21
      PICKER        - Picker FACS-I
      PW1100        - Philips PW1100
      AREA          - Any Area detector instrument
\end{verbatim}\normalsize




\bigskip\Keyword{MODE=}
 The mode of data collection is given:

\small\begin{verbatim}
      BISECTING      - Default value
      PARALLEL
      GENERAL
\end{verbatim}\normalsize



\small\begin{verbatim}
 This example assumes that there are no equivalent reflections.
      \DIFABS
      DIFFRACTION GEOMETRY=SYNTEX-P1
      END
\end{verbatim}\normalsize


\small\begin{verbatim}
 This example demonstrates a total re-processing of the data, including
 converting atoms to isotropic if they have previously been refined
 anisotropically. Note that a theta dependent correction from
 International Tables is applied during data reduction
 (see also section \end{verbatim}\normalsize\ref{DATAREDUC}). The theta
 dependant correction in DIFABS is ill-conditioned and unstable.

      $\backslash$ save the contents of the old dsc file
      $\backslash$PURGE NEW
      END
      $\backslash$ Connect the reflection file to HKLI
      $\backslash$OPEN HKLI ZNCPD.HKL
      $\backslash$ Use an $\backslash$HKLI command to apply the tabulated theta correction
      $\backslash$HKLI
      READ NCOEF=12 FORMAT=FIXED UNIT=HKLI F'S=FSQ CHECK=NO
      INPUT H K L /FO/ SIGMA(/FO/) JCODE SERIAL BATCH THETA PHI OMEGA KAPPA
      FORMAT (5X,3F4.0,F9.0,F7.0,F4.0,F9.0,F4.0,4F7.2)
      STORE NCOEF=6
      OUTPUT INDICES /FO/ BATCH RATIO/JCODE SIGMA(/FO/) CORRECTIONS SERIAL
      ABSORPTION PHI=NO  THETA=YES PRINT=NONE
      THETA 16
      THETAVALUES
      CONT 0  5 10 15 20 25 30 35 40 45 50 55 60 65 70 75
      THETACURVE
      CONT 3.61  3.60  3.58  3.54  3.50  3.44  3.37  3.30
      CONT 3.23  3.16  3.09  3.02  2.96  2.91  2.86  2.82
      END
      $\backslash$LP
      END
      $\backslash$SYSTEMATIC
      $\backslash$ preserve the original indices
      STORE NEWINDICES=NO
      END
      $\backslash$SORT
      END
      $\backslash$ copy from workfile to disk
      $\backslash$LIST 6
      READ TYPE=COPY
      END
      $\backslash$ unit weights
      $\backslash$LIST 4
      END
      $\backslash$WEIGHT
      END
      $\backslash$EDIT
      UEQUIV FIRST UNTIL LAST
      END
      $\backslash$LIST 28
      MINIMA RATIO =  3.0
      END
      $\backslash$SFLS
      SCALE
      END
      $\backslash$ assume there are no H atoms
      $\backslash$LIST 12
      FULL FIRST(U[ISO]) UNTIL LAST
      END
      $\backslash$SFLS
      REFINE
      REFINE
      CALC
      END
      $\backslash$LIST 28
      $\backslash$ remove all restrictions to get Fcs.
      END
      $\backslash$DIFABS UPDATE FC
      END
      $\backslash$ Complete anisotropic refinement, produce publication tables etc
      $\backslash$LIST 12
      FULL X'S U'S
      END
      $\backslash$LIST 28
      MINIMA RATIO=3
      END
      $\backslash$SFLS
      REFINE
      .....
      $\backslash$CIF
      END
      $\backslash$ reprocess data so that it can be merged for the final
      $\backslash$ difference map
      $\backslash$DIFABS UPDATE TRANSFER
      END
      $\backslash$SYST
      END
      $\backslash$MERGE
      END
      $\backslash$FOURIER
      $\backslash$ etc etc etc ---



\section{Internal workings}



SOme understanding of the internal data management in CRYSTALS may
 help the user to sort out unexplained failures.





\bigskip\Instruction{Parameter esds - LIST 9}


\label{LIST9} 


This list contains the refineable parameter esds.


It is created from LIST 5 and LIST 11 with the instruction
\small\begin{verbatim}

  \ESD
  END

\end{verbatim}\normalsize

The list can be printed with
\small\begin{verbatim}
  \PRINT 9
  END
\end{verbatim}\normalsize




The instruction $\backslash$PUNCH 5 E creates a plain format punch file with
the atomic parameters and esds.





\bigskip\Instruction{Refinement parameter map  -  LIST 22}


\label{LIST22} 


This list contains the refinement directives in internal format
 and it can only be generated by the computer.
 After the refinement directives have been read in, they are
 stored on the disc in binary format ready for processing.
 Before the structure factor least squares routines can
 use the information in LIST 12, it is necessary to convert them to
 a LIST 22.


If the conversion fails, or the input of LIST 5 or LIST 12
 is in error, LIST 22 will be marked as an error list, and any job
 that attempts to reference LIST 22 will terminate in error.


For complex LIST 12s, i.e. those
 containing EQUIVALENCE, LINK, RIDE, GROUP, WEIGHT or COMBINE, the user is
 strongly advised to issue $\backslash$LIST 22 and then $\backslash$PRINT 22, and look at the
 LIST 22
 generated. The ouput, which is set out like a LIST 5, shows the
 relationship between the physical and the least squares parameters.





\bigskip\Instruction{The least squares matrix  -  LIST 11}


\label{LIST11} 



The matrix that is produced by the structure factor least
 squares process is stored on the disc as a LIST 11. This list may be
 massive, so it is wise to purge the disk regularly with large
 structures. To recover the maximum space, delete the LIST 11 before
 purging.
\small\begin{verbatim}
      \DISK
      DELETE 11
      END
      \PURGE
      END
\end{verbatim}\normalsize




\bigskip\Instruction{Printing the contents of LIST 11}





LIST 11 is printed by :


\bigskip\Directive{$\backslash$PRINT 11}


\bigskip\Directive{$\backslash$PRINT 11 A}
 Prints the largest 10 correlation coefficients whose magnitude is
 greater than 0.25.


\bigskip\Directive{$\backslash$PRINT 11 B}
 Prints the correlation matrix.


\bigskip\Directive{$\backslash$PRINT 11 C}
 Prints the current matrix (usually the inverse matrix).


\bigskip\Instruction{Least squares shift list  -  LIST 24}


\label{LIST24} 



When the normal matrix produced by the least squares process
 has been inverted, a set of shifts is calculated,
 suitably scaled if necessary, to apply to the atomic parameters.
 These shifts are output to the disc as a LIST 24,
 and then applied by the routines that compute the new parameters.
 List 24 can only be generated in the machine.


\bigskip\Instruction{Restraints in internal format  -  LIST 26}



\label{LIST26} 


This list contains the restraints in internal format.
 Before the structure factor least squares routines can
 use the information in LIST 16 and 17, it is necessary to
 convert it to an internal format held in LIST 26.


If this operation fails, or the input of LIST 12 or LIST 16
 goes wrong, LIST 26 will be marked as an error list, and any job
 that attempts to reference LIST 26 will terminate in error.



\chapter{Fourier Routines}
\label{fouandpat} 

\index{Fourier routines}\section{Scope of the Fourier section of the user guide}



In this section of the user guide, the lists and commands
 relating to the Fourier routines are described.

\small\begin{verbatim}
      Input of the Fourier section limits                  -  \LIST 14
      Compute Fourier linits from the symmetry operators   -  \FLIMIT
      Fourier calculations                                 -  \FOURIER
      Processing of the peaks list                         -   LIST 10
      Elimination of duplicated entries in LISTS 5 and 10  -  \PEAKS
      Slant fourier calculations                           -  \SLANT
\end{verbatim}\normalsize





\index{LIST 14}
\index{Fourier limits}
\section{Input of the Fourier section limits - LIST 14}
\label{LIST14} 


\small\begin{verbatim}
 \LIST 14
 X-AXIS MINIMUM= STEP= MAXIMUM= DIVISION=
 Y-AXIS MINIMUM= STEP= MAXIMUM= DIVISION=
 Z-AXIS MINIMUM= STEP= MAXIMUM= DIVISION=
 X-PAT MINIMUM= STEP= MAXIMUM= DIVISION=
 Y-PAT MINIMUM= STEP= MAXIMUM= DIVISION=
 Z-PAT MINIMUM= STEP= MAXIMUM= DIVISION=
 ORIENTATION DOWN= ACROSS= THROUGH=
 SCALEFACTOR VALUE=
\end{verbatim}\normalsize



\small\begin{verbatim}
 \LIST 14
 X-AXIS 0.0 0.0 0.5 0.0
 Y-AXIS 0.0 0.0 0.9 0.0
 Z-AXIS -2 2 32 60
 ORIENTATION Z X Y
 SCALE VALUE = 10
 END
\end{verbatim}\normalsize




The Fourier routines will calculate a map with section edges
 parallel to any two of the cell axes (a, b or c). The starting and
 stopping points must be given for each direction (in crystal fractions).
 The user should choose the asymmetric unit to have one
 range as small as possible, and the other two approximately equal.
 Orientate the computation so that the sections are perpendicular to the
 short range direction.
 If the command $\backslash$SPACEGROUP has been used to input the symmetry
 information, a LIST 14 will  have been generated. This will be a valid
 choice, but may not be optimal.





\bigskip\Instruction{$\backslash$LIST 14}







\bigskip\Directive{X-AXIS MINIMUM= STEP= MAXIMUM= DIVISION=}


This directive specifies how the x-axis is to be divided.


\bigskip\Keyword{MINIMUM=}
 This parameter gives the initial value along the x-direction.
 If it is omitted, a default value of 0.0 is assumed for  MINIMUM.


\bigskip\Keyword{STEP=}
 This parameter, which has a default value of 0.3, gives the
 step along the x-direction.


\bigskip\Keyword{MAXIMUM=}
 This parameter, which has a default value of 1.0, gives the
 final value along the x-direction.


\bigskip\Keyword{DIVISION=}
 If  DIVISION  is greater than zero, it defines the number of
 divisions into which the x-axis is to be divided.
 In this case, the three remaining parameters are expressed in
 terms of  DIVISION  and give the first point ( MIN ), the
 increment between successive points ( STEP ) and the final
 point to be calculated ( MAX ). If the divisions of the unit
 cell along the x-axis are given in this way, the user must
 ensure that sufficient map is calculated for the map scan, by
 adding one extra point beyond the asymmetric unit at both ends
 along the x-axis. If this is not done, peaks at the edge of the
 asymmetric unit may be missed by the peak search.


If  DIVISION  is equal to zero, which is its default value,
 the Fourier  routines will calculate the number of divisions
 required along the x-axis. In this case,  STEP  is the interval
 between successive points along the axis in angstrom.
 If this parameter is less than 0.05, a default value of 0.3 angstrom
 is used.  MINIMUM  And  MAXIMUM  define the first and last points to be
 calculated and are given in fractional coordinates.
 When the values of  MIN  and  MAX  are converted into unit cell
 divisions, an extra point is added at each end to ensure that the
 peak search functions correctly.





\bigskip\Directive{Y-AXIS MINIMUM= STEP= MAXIMUM= DIVISION=}
 Similar to X-AXIS above.





\bigskip\Directive{Z-AXIS MINIMUM= STEP= MAXIMUM= DIVISION=}
 Similar to X-AXIS above.








\bigskip\Directive{X-PAT MINIMUM= STEP= MAXIMUM= DIVISION=}
 This directive is similar to the X-AXIS directive, but refers to the
 Patterson asymmetric unit.





\bigskip\Directive{Y-PAT MINIMUM= STEP MAXIMUM= DIVISION=}
 Similar to X-PAT above.





\bigskip\Directive{Z-PAT MINIMUM= STEP= MAXIMUM= DIVISION=}
 Similar to X-PAT above.





\bigskip\Directive{ORIENTATION DOWN= ACROSS= THROUGH=}


Controls the orientation parameters for the map
 calculation and printing.


\bigskip\Keyword{DOWN=}

\small\begin{verbatim}
      X  -  Default value
      Y
      Z
\end{verbatim}\normalsize


 The default value X
 indicates that the x coordinate goes down the printed page.


\bigskip\Keyword{ACROSS=}
 As DOWN above, but with the default value Y
 indicating that the y coordinate goes across the page.


\bigskip\Keyword{THROUGH=}
 As DOWN above, but with the default value Z
 indicating that the z coordinate changes from section to section.





\bigskip\Directive{SCALEFACTOR VALUE=}


\bigskip\Keyword{VALUE=}
 This parameter specifies the value by which the electron density,
 on the scale of /Fc/, is multiplied before it is printed.
 If this parameter is omitted, a default value of 10 is assumed.



\section{Printing the contents of LIST 14}



The contents of LIST 14 can be listed to the line printer
 by issuing the command :





\bigskip\Instruction{$\backslash$PRINT 14}




There is no command available for punching LIST 14.






\index{FLIMIT}
\index{Conpute Fourier limits}
\section{Compute Fourier limits from the symmetry}
\label{FLIMIT} 


\small\begin{verbatim}
 \FLIMIT LAUE=
\end{verbatim}\normalsize



\small\begin{verbatim}
 \FLIMIT
 END
\end{verbatim}\normalsize




This command uses the same algorithms as $\backslash$SPACEGROUP 
 to create a LIST 14. This will be a valid
 choice, but may not be optimal.  The parameter LAUE takes a
 value from this table:

\small\begin{verbatim}

  Laue Group     Number      Nx  Ny  Nz         Comment
     -1                   Default value     Compute from operators
      1             1         4   4   4     Triclinic
     2/m            2         8   8   8     Monoclinic
     mmm            3         8   8   8     Orthorhombic  (Fddd 16)
     4/m            4         8   8  16     Tetragonal
     4/mmm          5         8   8  16     Tetragonal
     -3R            6         8   8   8     Rhombohedral
     -3mR           7         8   8   8     Rhombohedral
     -3             8        12  12  24     Hexagonal
     -3m1           9        12  12  24     Hexagonal
     -31m          10        12  12  24     Hexagonal
     6/m           11        12  12  24     Hexagonal
     6/mmm         12        12  12  24     Hexagonal
     m3            13        16  16  16     Cubic
     m3m           14        16  16  16     Cubic
  The values for groups 8 and 9 are OK for the order X,Y,Z, if the 2
  other orders are searched NX and NY should be 24
\end{verbatim}\normalsize





\index{FOURIER}
\section{Fourier calculations  -  $\backslash$FOURIER}
\label{FOURIER} 


\small\begin{verbatim}
 \FOURIER INPUT=
 MAP TYPE= NE= PRINT= SCAN= SCALE= ORIGIN= NMAP= MONITOR=
 REFLECTIONS WEIGHT= REJECT= F000= CALC=
 LAYOUT NLINE= NCHARACTER= MARGIN= NSPACE= MIN-RHO= MAX-RHO=
 PEAKS HEIGHT= NPEAK= REJECT=
 TAPES INPUT= OUTPUT=
 END


 \FOURIER
 MAP TYPE=DIFF
 PEAK HEIGHT = 3
 END
\end{verbatim}\normalsize




Before a Fourier is computed, a LIST 14 must have been created or
 input. The routine will compute a map in any space group,
 the relevant symmetry being found in LIST 2 (space group information,
see section \ref{LIST02}).


In the ouput listing, new peaks are labelled, with the following
 meanings

\small\begin{verbatim}
      GOOD PEAK - The peak centre was determined by Least-Squares.
      POOR PEAK - The peak centre was determined by interpolation.
      DUBIUOS PEAK - The peak centre is only a local maximum.
      MALFORMED PEAK - The peak centre is extrapolated to be out side
                 of the asymmetric unit - usually due to very poor phasing.
\end{verbatim}\normalsize







\bigskip\Instruction{$\backslash$FOURIER INPUT=}




\bigskip\Keyword{INPUT}
Indicates which reflection list to use.
\small\begin{verbatim}
      6      Default
      7      Alternative reflection list
\end{verbatim}\normalsize







\bigskip\Directive{MAP TYPE= NE= PRINT= SCAN= SCALE= ORIGIN= NMAP= MONITOR=}





\bigskip\Keyword{TYPE=}

\small\begin{verbatim}
      F-OBS     -  Default value
      F-CALC
      DIFFERENCE
      2F0-FC
      OPTIMAL
      FO-PATTERSON
      FC-PATTERSON
      EXTERNAL
\end{verbatim}\normalsize


  The map type 'OPTIMAL' implements a suggestion of Peter Main. It is
 a form of weighted Fo map, with coefficients w*Fo if the reflection is
 in a centro-symmtric class, otherwise (2*w*Fo)-Fc, where w is the Simm weight.
 NOTE this is not the same as w(2*Fo-Fc), a Sim weighted 2Fo-Fc map. It has
 the property that known and unknown atom peak heights are approximately the
 same, and should be usefull for Fourier refinement.


\bigskip\Keyword{NE=}
 This parameter indicates which solution should be used to compute the
 externally phased map, and has a default value of 1.
  NE  is only used in conjunction with TYPE = EXTERNAL.


\bigskip\Keyword{PRINT=}
 Controls the printing pf the map.

\small\begin{verbatim}
      NO   -  Default value
      YES
\end{verbatim}\normalsize




\bigskip\Keyword{SCAN=}
 Controls automatic scanningof the map for peaks.

\small\begin{verbatim}
      NO
      YES  -  Default value
\end{verbatim}\normalsize




\bigskip\Keyword{SCALE=}
 Controls the scaling of the electron density in the map.

\small\begin{verbatim}
      NO
      AUTOMATIC  -  Default value
      YES
\end{verbatim}\normalsize


 If SCALE is YES,
 the program computes a scale factor rather than take one
 from LIST 14 (Fourier control - section \ref{LIST14}). The 
 scale factor is computed by summing the modulus of
 all the contributors to the map, and dividing this total into
 ORIGIN  (see the next parameter).
 For a Patterson, therefore, the origin is scaled to be  ORIGIN,
 while for other maps a scale factor is computed which guarantees
 that every number is less than  ORIGIN.


If  SCALE  is  NO, the scale factor is taken from LIST 14
 for all types of Fourier maps.
 If SCALE is AUTOMATIC,
 there is automatic scaling for an external or Patterson map,
 while other maps take their scale factors from LIST 14.


\bigskip\Keyword{ORIGIN=}
 The default value for this parameter is 999, and is used when
 the program calculates a scale factor (see  SCALE  above).


\bigskip\Keyword{NMAP}
 Controls negation of the density values, with default NO.
 Use YES, in which case the density values are negated,
 when looking for minima. This
 feature permits location of hydrogen in Neutron maps, and the location of
 minima (which become maxima) generally. Set the Peak Height positive
 even when searching for minina, since at the time of the search the
 minima are inverted. The out put density values have the correct sign.
 Use $\backslash$COLLECT 10 5 rather than $\backslash$PEAKS on negated maps, since PEAKS
 cannot handle minima.


\bigskip\Keyword{MONITOR=}

\small\begin{verbatim}
      LOW
      MEDIUM  -  Default value
      HIGH
\end{verbatim}\normalsize


 If MONITOR is MEDIUM the, the peak coordinates are printed as they
 are found. If HIGH, density at known sites is also printed.








\bigskip\Directive{REFLECTIONS WEIGHT= REJECT= F000= CALC=}





\bigskip\Keyword{WEIGHT=}

\small\begin{verbatim}
      SIM
      NO     -  Default value
      LIST-6
\end{verbatim}\normalsize




If  WEIGHT  is  NO , its default value, then the map is not weighted.


If  WEIGHT  is set equal to  SIM , then SIM weights are computed.
 This option requires both LIST 29 (atomic properties, section \ref{LIST29}
 and LIST 5. The occupation factors in LIST 5 are used to determine how many
 atoms of each type are present, and LIST 29 indicates how many should be
 present. See the notes under 'TYPE', above.


If  WEIGHT  is  LIST-6 , then the map is weighted with the weight
 stored in LIST 6 (section \ref{LIST06}).


\bigskip\Keyword{REJECT=}

\small\begin{verbatim}
      NONE
      SMALL  -  Default value
      QUARTER
      HALF
\end{verbatim}\normalsize


 If REJECT  is NONE, all the reflections in LIST 6 which are
 allowed by LIST 28 are included.
 In this case, no check is made on the /Fc/ value.
 For an /Fo/, /Fc/ and difference Fourier, the program expects that there
 should be an /Fc/ value if the phase is to be defined.
 Accordingly, reflections where /Fc/ $<$ 0.001 are normally rejected
 for such Fouriers, and this is the default option of  SMALL.


Some users like to omit reflections if Fc is smaller then a fraction of
 Fo. The options QUARTER and HALF are available.


\bigskip\Keyword{F000=}
 The default value for this parameter is zero, and specifies the value
 of F(000) to be used.


\bigskip\Keyword{CALC}

\small\begin{verbatim}
      NO   -  Default value
      YES
\end{verbatim}\normalsize


 Value YES causes structure factors (i.e. Fc and phase) to be calculated
 immediately before the map is computed. This option can only be activated
 if some previous task with the current DSC file
 has computed phases via a $\backslash$SFLS command (section \ref{SFLS}) and left a 
 LIST 33 on the disk (List 33 is the stored representation of the SFLS
 command, so that the program can rememeber how the last refinement was
 carried out, see section \ref{SFLS})





\bigskip\Directive{LAYOUT NLINE= NCHARACTER= MARGIN= NSPACE= MIN-RHO= MAX-RHO=}


This directive specifies how the map should be printed, if the value of the
 PRINT parameter on the MAP directive is YES.


\bigskip\Keyword{NLINE=}
 This parameter sets the number of lines per row of map, and has a default
 value of 2.


\bigskip\Keyword{NCHARACTER=}
 This parameter controls the number of characters for each grid
 point, and has a default value of 4.


\bigskip\Keyword{MARGIN=}
 This parameter, whose default value is 4, defines the number of
 characters per division number down each side of the map.


\bigskip\Keyword{NSPACE=}
 This parameter has a default value of 2, and defines the number of
 spaces between the division number and the grid number down each
 side of the map. The minimum value for  NSPACE  is 2.


\bigskip\Keyword{MIN-RHO=}
 This parameter has a default value of -1000000, and points less than
  MIN-RHO  are left blank when the map is printed.


\bigskip\Keyword{MAX-RHO=}
 This parameter has a default value of 1000000, and points greater than
  MAX-RHO  are left blank when the map is printed.





\bigskip\Directive{PEAKS HEIGHT= NPEAK= REJECT=}


Controls the search for peaks when the map is
 searched, i.e. if the value of the SCAN  parameter on the MAP directive
 is YES.


\bigskip\Keyword{HEIGHT=}
 This parameter sets the search of the map for all
 peaks with an electron density greater than  HEIGHT. If this
 parameter is omitted, a default value of 50 is assumed
 for an external or Patterson map. For all other maps, the map is scanned
 for peaks greater than 1.5*SCALE, where  SCALE  is the map scale factor,
 either taken from LIST 14 (Fourier control - section \ref{LIST14}) 
or computed using  SCALE  =  YES  above.


\bigskip\Keyword{NPEAK=}
 This parameter, whose default value is 0, determines the number
 of peaks to be retained after they have been ranked by peak height.
 If NPEAK is zero or negative, the number of peaks saved is computed from

\small\begin{verbatim}
      NPEAK = (Cell volume) / (18 * Space Group multiplicity)

                               18 is an average atomic volume.
\end{verbatim}\normalsize




\bigskip\Keyword{REJECT=}
 This parameter, with a default value of 0.01, specifies that peaks
 within a distance of  REJECT  angstrom of a peak already ranked on
 peak height, will be rejected from the list.





\bigskip\Directive{TAPES INPUT= OUTPUT=}
 This directive is used if a map is to be read off magnetic tape,
 or a computed map is to be written to a
 magnetic tape. Remember that CRYSTALS will use scratch files unless given
 named files. To assign a named  output file, issue
\small\begin{verbatim}
      \OPEN MT1 filename
\end{verbatim}\normalsize




The tape is unformatted.

\small\begin{verbatim}
 Record 1: 'INFO  DOWN ACROSS SECTION'
 Record 2: 'TRAN'       9 elements of a transformation matrix
 Record 3: 'CELL'       Cell parameters, angles in radians
 Record 4: 'L14 '       List 14 information
 Record 5: 'SIZE'       number of points down, across, and number of sections
 Record 6:  number of values,  values for a section
           Record 6 is repeated for every section.
 Record n:  number of atoms, number of items per atom
 Record n+1: Items for an atom, repeated for all atoms
\end{verbatim}\normalsize




Record 4 contains 6 integers, (No of points down
 and across the page, number of sections, and the index of these
 directions, 1 = x). Subsequent records contain a whole section line by line,
 prefixed by the total number of points in the section.


\bigskip\Keyword{INPUT=}

\small\begin{verbatim}
      NO   -  Default value
      YES
\end{verbatim}\normalsize


 If INPUT is YES, a map will be
 read in from the 'input magnetic tape', and the resulting map
 will be the minimum of each point of the calculated and input maps.
 The input map sections must be on device 'MT2'


*** THIS FACILITY IS NOT CURRENTLY IMPLEMENTED ***





\bigskip\Keyword{OUTPUT=}

\small\begin{verbatim}
      NO   -  Default value
      YES
\end{verbatim}\normalsize


 If  OUTPUT  is YES, the map produced is written to the
 'output magnetic tape'. You may need to OPEN a permanent file on
 device 'MT1'.



\section{Calculation of superposition minimum functions}



(Issue 7 - implementation incomplete, 1984)


(Issue 9 - implementation still incomplete, 1993 - no one seems to want it
 anyway!- use SHELXS if you need to).


(Issue 10 - still no change, 1996)


The Fourier routine provides a way of
 calculating superposition minimum functions.
 For each map that is produced, it is possible to specify that another
 map should be read in from magnetic tape at the same time (the  TAPES
 directive). Each point of the resulting map is taken as the minimum of
 the newly computed map and that read off the magnetic tape. This output
 map may be written to a second magnetic tape, also by use of the  TAPES
 directive.


When the input map and the calculated map are superposed, the first
 point calculated and the first point read off the tape are compared,
 the second point calculated and the second point input are compared,
 and so on. This implies that the first point on each map must represent
 the same point in real space for the output map, and that each map must
 contain the same number of points.
 The origin of each map that is to be calculated is altered by
 changing LIST 14 (Fourier limits - section \ref{LIST14}). For example, 
 if a 2x, 2y, 2z vector has been identified
 at 0.36, 0.14 and 0.28, and the 2x, 1/2-2y, 0 vector resulting from a
 two-fold axis has been found at 0.36, 0.36, 0, then the two LIST 14's
 for the superposition function might appear as :

\small\begin{verbatim}
 \LIST 14
 X-AXIS 14 4 122 400
 Y-AXIS 5 2 59 100
 Z-AXIS 12 2 66 100
 ORIENT X Y Z
 SCALE 10
 END

      and

 \LIST 14
 X-AXIS 14 4 122 400
 Y-AXIS 16 2 70 100
 Z-AXIS -2 2 52 100
 ORIENT X Y Z
 SCALE 10
 END
\end{verbatim}\normalsize




For the first map, the origin of real space is at 0.18, 0.07 and 0.14
 in vector space. This point is moved so that it is one grid point
 in along each axial direction, to allow for the map scan.
 For the second peak, the origin in real space is at 0.18, 0.18 and 0.0.
 The second LIST 14 places this point one grid point in along each of the
 axial directions so that the real space origin of the two maps
 coincides. To convert the coordinates that result from the second map
 scan to real space coordinates, it is necessary to subtract 0.18
 from x and 0.18 from y, since the coordinates are printed in
 Patterson space for all the maps calculated.



\index{LIST 10}
\index{Peaks list}
\section{Processing of the peaks list  -  LIST 10}
\label{LIST10} 






\bigskip\Directive{$\backslash$LIST 10}


LIST 10 cannot be input by the user. When the map scan has been completed,
 the resulting peaks are output to the disc as a LIST 10.
 Except for an external or  Patterson map, the atoms already
 in LIST 5 are placed at the beginning of the LIST 10.


A LIST 10 is usually converted to a LIST 5
 by one of the following commands :

\small\begin{verbatim}
 \EDIT 10 5                \PEAKS 10 5
 \COLLECT 10 5             \REGROUP  10 5
\end{verbatim}\normalsize


 $\backslash$PEAKS  is the normal choice,
 since duplicate peaks related by symmetry, or peaks corresponding to
 known atoms  can be eliminated. It is described below; EDIT, COLLECT
 and REGROUP are in the section on Atomic and Structural Parameters.



\section{Printing the contents of LIST 10}



The contents of LIST 10 can be listed with:





\bigskip\Instruction{$\backslash$PRINT 10}




There is no command available for punching LIST 10 out to a file.



\index{PEAKS}
\section{Elimination of duplicated entries in LISTS 5 and 10  -  $\backslash$PEAKS}


\small\begin{verbatim}
 \PEAKS INPUTLIST= OUTPUTLIST=
 SELECT REJECT= KEEP= MONI= SEQ= TYPE= REGROUP= MOVE= SYMM= TRANS=
 REFINE DISTANCE= MULTIPLIER=
 END

 \PEAKS
 SELECT REJECT=0.0001
 REFINE DISTANCE=.5
 END
\end{verbatim}\normalsize




This routine eliminates
 atoms or peaks which duplicate other entries in an atomic
 parameter list.
 When using this routine, a set of distances is calculated about each
 atom or peak in turn. Atoms or peaks further down the list than the
 current pivot are then eliminated if they have a contact distance less
 than a user specified maximum (the  REJECT  parameter).
 Thus, when peaks have been added to a
 LIST 5, the peaks corresponding to the atoms can be eliminated.





\bigskip\Instruction{$\backslash$PEAKS INPUTLIST= OUTPUTLIST=}


 INPUTLIST and OUTPUTLIST specify where the atoms are to be taken from,
 and where they will be put.


\bigskip\Keyword{INPUTLIST=}

\small\begin{verbatim}
      5
      10  -  Default value
\end{verbatim}\normalsize




\bigskip\Keyword{OUTPUTLIST=}

\small\begin{verbatim}
      5   -  Default value
      10
\end{verbatim}\normalsize







\bigskip\Directive{SELECT REJECT= KEEP= MONI= SEQ= TYPE= REGROUP= MOVE= SYMM= TRANS=}


\bigskip\Keyword{REJECT=}
 REJECT is the distance above which connected atoms or peaks are assumed to
 be distinct. If a contact is found which is less than  REJECT
 the second atom or peak of the pair in the list is eliminated, and
 defaults to 0.5.


\bigskip\Keyword{KEEP=}
 This parameter indicates how many entries are to be kept in the
 output list. The default value of 1000000 is the maximum possible.


\bigskip\Keyword{MONITOR=}

\small\begin{verbatim}
      LOW
      HIGH  -  Default value
\end{verbatim}\normalsize


 If  MONITOR  is given as  LOW only the atoms or peaks that are
 deleted because of the REJECT limit are listed.
 If  MONITOR  is  HIGH, all the atoms deleted because of both  KEEP
 and  REJECT  are listed.


\bigskip\Keyword{SEQUENCE=}

\small\begin{verbatim}
      NO  -  Default value
      YES
      EXHYD
\end{verbatim}\normalsize


 If  SEQUENCE  is  YES, then the program will
 give sequential serial numbers to the atoms and
 peaks in the final output  list .



 If SEQUECE is EXHYD the hydrogen atoms are excluded from the 
renumbering.


\bigskip\Keyword{TYPE=}

\small\begin{verbatim}
      PEAK  -  Default value
      ALL
      AVERAGE
\end{verbatim}\normalsize


 If  TYPE is  PEAK, then the program will only delete PEAKS which are
 within REJECT of an existing atom. It TYPE is ALL, atoms are also
 deleted.



 If TYPE is AVERAGE, coincident atoms or peaks are averaged. The radius
 for coincidence is taken from the DISTANCE keyword on the REFINE
 directive. The default radius is .5 Angstrom.


\bigskip\Keyword{REGROUP=}
 This parameter has two allowed values :

\small\begin{verbatim}
      NO  -  Default value
      YES
\end{verbatim}\normalsize


 If  REGROUP  is  YES, then the program will reorganise LIST 5 so that
 bonded atoms and peaks are adjacent.


\bigskip\Keyword{MOVE=}
 The value of this parameter is the maximum separation for 'bonded' atoms.
 The default is 2.0 A.


\bigskip\Keyword{SYMMETRY=}
 This parameter controls the use of symmetry information in the calculation of
 contacts, and can take three values.

\small\begin{verbatim}
      SPACEGROUP  -  Default value. The full spacegroup symmetry is used in
                                    all computations
      PATTERSON.     A centre of symmetry in introduced, and the translational
                     parts of the symmetry operators are dropped.
      NONE.          Only the identity operator is used.
\end{verbatim}\normalsize




\bigskip\Keyword{TRANSLATION=}
 This parameter controls the application of cell translations in the
 calculation of contacts, and can take the values YES or NO








\bigskip\Directive{REFINE DISTANCE= MULTIPLIER=}


Controls action of Fourier refinement.


\bigskip\Keyword{DISTANCE=}
 This parameter has a default value of zero, and is
 the distance below which atoms and peaks are considered
 to be coincident.  The coordinates of an existing atom are replaced
 by those of a coincident peak. Refinement takes precedence
 over deletion of peaks.


\bigskip\Keyword{MULTIPLIER=}
 This parameter has a default value to give automatic refinement.
 It is set to 1 for a centric space group and is set to
 2 for a non-centric space group. It can be set to 0.0 to preserve original
 coordinates but be given new peak heights.

\small\begin{verbatim}
      X(new) = x(atom) + mult(x(peak) - x(atom)).
\end{verbatim}\normalsize



\small\begin{verbatim}
 \ reject atoms or peaks with contact distances less than 0.7
 \ keep 30 entries in the output list
 \ list the atoms and peaks rejected because of both 'KEEP'
 \ and 'REJECT'
 \
 \PEAKS 10 5
 SELECT REJECT=0.7,KEEP=30,MONITOR=HIGH
 END
\end{verbatim}\normalsize





\index{SLANT}
\index{Slant Fourier calculation}
\section{Slant fourier calculations  -  $\backslash$SLANT}


\small\begin{verbatim}
 \SLANT INPUT=
 MAP TYPE= MIN-RHO= SCALE= WEIGHT=
 SAVED MATRIX=
 CENTROID XO= YO= ZO=
 MATRIX R(11)= R(12)= R(13)= R(21)=  .  .  . R(33)=
 DOWN MINIMUM= NUMBER= STEP=
 ACROSS MINIMUM= NUMBER= STEP=
 SECTION MINIMUM= NUMBER= STEP=
 END

\end{verbatim}\normalsize




A Slant Fourier is one that is calculated through any general plane
 of the unit cell. For such a Fourier, the normal Beevers-Lipson
 expansion of the summation cannot be used, so that it
 will take many orders of magnitude longer than a
 conventional one.
 The algorithm adopted here is as follows :

\small\begin{verbatim}
 X    A general vector expressed in fractions of the
     unit cell edges (i.e. x/a, y/b and z/c)
 XO   The centroid of the required general fourier section,
     also expressed in crystal fractions.
 XP   The coordinates of the point 'X' when expressed
     in the coordinate system used to define the
     plane of the general section.
 'X' and 'XP' are related by the expression :  XP = R.(X-XO)
 R    'R' is the matrix that describes the transformation
     of a set of coordinates in the crystal system to
     a set of coordinates in the required plane.
 therefore :  X = S.XP + XO

     'S' is the inverse matrix of 'R'.

 The required expression in the fourier is :

     H'.X = H'.S.XP + H'XO

 H    H is a vector containing the Miller indices of
     a reflection and H' is the transpose of H.
 This may be re-expressed as :

     H'.X = H'.S.DXP + H'.(S.XPS + XO)

 DXP  'DXP' represents the increment in going from the
     first point on the section to be calculated.
 XPS  'XPS' is the coordinate of the first point on the
     section to be calculated.
     obviously :  XP = XPS + DXP.
\end{verbatim}\normalsize




When the Fourier is calculated, the term \emph{H'.(S.XPS} \emph{+} \emph{XO)}
 is constant for each section to be calculated. The term \emph{H'.S} ,
 which may be regarded as the transformed indices, is also constant
 for each reflection, so that a two dimensional recurrence relation
 may be used to change  \emph{DXP}  and thus  \emph{Cos(2*PI*H.X} \emph{-} \emph{ALPHA)'} over
 the required section for each reflection. ( \emph{ALPHA}  is the phase
 angle for the current reflection).


The input for the slant Fourier thus must include the rotation
 matrix  \emph{R,} the centroid  \emph{XO,} and the steps and divisions in the
 required plane.





\bigskip\Instruction{$\backslash$SLANT INPUT=}




This is the command which initiates the slant fourier routines.


\bigskip\Keyword{INPUT}
Indicates which reflection list to use.
\small\begin{verbatim}
      6      Default
      7      Alternative reflection list
\end{verbatim}\normalsize







\bigskip\Directive{MAP TYPE= MIN-RHO= SCALE= WEIGHT=}


\bigskip\Keyword{TYPE=}

\small\begin{verbatim}
      F-OBS
      F-CALC
      DIFFERENCE
      FO-PATTERSON
      FC-PATTERSON
\end{verbatim}\normalsize


 There is no default value for this parameter


\bigskip\Keyword{MIN-RHO=}
 This parameter has a default value of zero, and
 is the value below which all numbers on the map are replaced by
  MIN-RHO.


\bigskip\Keyword{SCALE=}
 The terms used in the Fourier are put on the same scale as  Fc,
 and then before the map is printed the numbers are multiplied
 by  SCALE . (i.e.  SCALE  is the map scale factor).
 The default is 10.


\bigskip\Keyword{WEIGHT=}

\small\begin{verbatim}
      NO   -  Default value
      YES
\end{verbatim}\normalsize




If WEIGHT = YES, the observed and calculated structure factors are
 multiplied by  the weights in LIST 6 (usually SQRT(w)). The user should
 be aware that this might have a major effect on the scale if the map
 density, and that SCALE may need adjusting.





\bigskip\Directive{SAVED MATRIX=}


This directive, which excludes CENTRIOD and MATRIX, uses the matrix and
 centroid stored in LIST 20 by a previous GEOMETRY, MOLAX or ANISO command (see
section \ref{LIST20}).


\bigskip\Keyword{MATRIX=}

\small\begin{verbatim}
      MOLAX
      TLS
      AXES
\end{verbatim}\normalsize







\bigskip\Directive{CENTROID XO= YO= ZO=}


This specifies the slant Fourier map centroid, in crystal fractions,
 and excludes SAVED.


\bigskip\Keyword{XO=}


\bigskip\Keyword{YO=}


\bigskip\Keyword{ZO=}
 The defaults value for  XO,YO,ZO, the coordinates  of the centroid,
 are 0.0.





\bigskip\Directive{MATRIX R(11)= R(12)= R(13)= R(21)=  .  .  . R(33)=}


This gives the elements of the rotation matrix  R, and
 excludes SAVED. The trnsformation generally used is from crystal
 fractions to orthogonal Angstroms.


\bigskip\Keyword{R(11)= R(12)= R(13=) R(21)=  .  .  . R(33)=}
 There are no default values for any of these parameters.





\bigskip\Directive{DOWN MINIMUM= NUMBER= STEP=}


This directive defines the printing of the map down the page.


\bigskip\Keyword{MINIMUM=}
 There is no default value for this parameter, the first point,
 in Angstrom,
 down the page of the plane to be calculated.


\bigskip\Keyword{NUMBER=}
 There is no default value for this parameter, the number of points
 of the plane to be printed down the page


\bigskip\Keyword{STEP=}
 There is no default value for this parameter, the interval
 in Angstrom between successive points down the page.





\bigskip\Directive{ACROSS MINIMUM= NUMBER= STEP=}


This directive defines the printing of the map across the page. The
 parameters have similar meanings to those for 'DOWN'.






\bigskip\Directive{SECTION MINIMUM= NUMBER= STEP=}


This directive defines the printing of the map sections. The
 parameters have similar meanings to those for 'DOWN'.



The units of  MINIMUM  and  STEP  are based on the coordinate system
 used to describe the plane, with the new 'x' axis going down the page and
 'y' across. In general the most convenient axial
 system for the plane is one expressed in Angstrom, so that the initial
 points and the steps are all expressed in Angstrom. (The
 least squares best plane program
 prints out the centroid in crystal fractions
 and the rotation matrix from crystal fractions to best plane coordinates
 in Angstrom, which are the numbers required, and may be  saved for use in
 SLANT by the directive 'SAVE').

\small\begin{verbatim}
 \ the map will be a difference map
 \ we wish to compute the section 0.3 anstrom above the plane
 \ numbers less than zero will be printed as zero
 \ the molecule lies at a centre of symmetry
 \ so that the centroid in crystal fractions is 0, 0, 0
 \ the plane coordinates are in angstrom
 \ for printing the plane both across and down the page,
 \ we will start 4 angstrom from the centroid,
 \ and go 4 angstrom the other side of the centroid,
 \ making a grid 8 angstrom by 8 angstrom
 \
 MAP DIFFERENCE 0.3 0
 CENTROID 0 0 0
 MATRIX 3.4076 10.0498 6.1794
 CONT   5.0606  8.287 -9.5483
 CONT  -6.9181 11.0121 1.546
 DOWN -4 33 0.25
 ACROSS -4 33 0.25
 END
\end{verbatim}\normalsize


\chapter{Analysis Of Results}


\label{geomandres} \section{Scope of this section of the user guide}


\small\begin{verbatim}
 Analysis of residuals                          ANALYSE
 Output of atom esds                            ESD
 Distance and angles calculations               DISTANCES
 Void search                                    VOIDS
 Global Geometry (planes,lines & libration)     GEOMETRY
 Torsion angles                                 TORSION
 Absolute Configuration                         TON
 Publication listing of the atomic parameters   PARAMETERS
 Publication listing of the reflections         REFLECTIONS
 Summary of data lists                          SUMMARY
 CIF files                                      CIF
 Graphics                                       CAMERON
\end{verbatim}\normalsize





\index{ANALYSE}
\index{Residual analysis}
\section{Analysis of residuals - $\backslash$ANALYSE}



This analyses the residual, Fo-Fc, for systematic trends, which might
 either indiacate an incomplete model, or an unsatisfactory weighting
 scheme. It is described in the chapter Structure Factors and Least
 Squares.








\bigskip\Keyword{e.s.d.s}
 Most publication listings require e.s.ds. These are computed
 from the normal matrix. If LIST 5 (the model parameters) has been 
 modified in {\bf ANY} {\bf WAY}
 (including simply renaming or ordering atoms) since the last
 refinement cycle, the matrix will be invalid.


CRYSTALS will warn you that LIST 11, the normal matrix, cannot be
 loaded.  To create a valid matrix
 without changing the parameter values, compute a refinement cycle but set
 all the shifts to zero.

\small\begin{verbatim}
      \SFLS
      REFINE
      SHIFT GENERAL = 0.0
      END
\end{verbatim}\normalsize


\index{ESD}
\index{Create esd list, LIST 9}
\section{Create esd list - $\backslash$ESD}

\small\begin{verbatim}

 \ESD
 END
 
\end{verbatim}\normalsize




The current LIST 5 must belong to the current VcV matrix (See
 the warning above).



\index{VCV}
\index{Output VcV matrix of selected atoms}
\section{Output VcV matrix of selected atoms}


HVcV#
\small\begin{verbatim}
 \VCV
 ATOM C(1) C(2) C(3)
 ACTION
 END

\end{verbatim}\normalsize




q




\index{DISTANCES}
\index{Distance calculations}
\index{Angle calculations}
\index{Geometry calculations}
\section{Distance angles calculations  -  $\backslash$DISTANCES}


\small\begin{verbatim}
 \DISTANCES INPUTLIST=
 OUTPUT MONITOR=  LIST= PUNCH= HESD=
 SELECT ALLDIST= COORD= SORTED= TYPE= RANGE= SYMM= TRANS=
 LIMITS DMINIMUM= DMAXIMUM= AMINIMUM= AMAXIMUM=
 E.S.D.S COMPUTE= CELL=
 INCLUDE atoms
 EXCLUDE atoms
 ONLY atoms
 PIVOT atoms
 BONDED atoms
 END

 \DIST
 E.S.D YES
 END
\end{verbatim}\normalsize




The distance angles routine is completely
 general with respect to crystal and lattice symmetry.
 For distances, the user may either use elemental radii specified in
 LIST 29 (see section \ref{LIST29} for input details), or specify  minimum and 
 maximum limits,
 and the program then calculates all possible contacts within these
 limits.
 All symmetry operations and unit cell translations are automatically
 generated.
 For the angles, LIST 29 or a separate set of distance limits may be used.
 At a given atom, angles are then calculated between all the atoms
 which bond to the central atom within the given limits.


The distance-angles routines can calculate the estimated
 standard deviations of the distances and angles that they produce.
 These e.s.d.'s are based upon the matrix stored in LIST 11 (see section
 \ref{LIST11}), and
 as many variance and covariance terms as are present are used.
 (For a full matrix, therefore, the full variance-covariance matrix
 is used).
 For this reason, the calculation of e.s.d.'s
 takes at least ten times as long
 as a simple distance angles calculation.


When  a set of e.s.d.'s are calculated, the variance-covariance matrix
 for the cell parameters (LIST 31, section \ref{LIST31}) may also be used.





\bigskip\Instruction{$\backslash$DISTANCES INPUTLIST=}




\bigskip\Keyword{INPUTLIST=}

\small\begin{verbatim}
      5   -  Default value
      10
\end{verbatim}\normalsize


 The default is to use the normal atom coordinate list.





\bigskip\Directive{OUTPUT MONITOR= LIST= PUNCH= HESD=}


\bigskip\Keyword{MONITOR=}
  This controls the monitoring information.

\small\begin{verbatim}
      OFF         -  no output
      DISTANCES   -  only monitors distances. (Default)
      ANGLES      -  only monitors angles.
      ALL         -  monitors distances and angles.
\end{verbatim}\normalsize




\bigskip\Keyword{LIST=}
 This controls the format of the listing.

\small\begin{verbatim}
      OFF
      LOW  
      MEDIUM
      HIGH   -  Default
      DISTANCE
      (DEBUG)
\end{verbatim}\normalsize


 If  LIST  is  LOW the the listing is in a compressed
 format, without symmetry information. If LIST is OFF, no output is sent
 to the listing file unless PUNCH is PUBLISH, when a copy of the publication
 listing appears in the listing file. If LIST is DISTANCE, a simplified table of
 distances is output.





\bigskip\Keyword{PUNCH=}
 This controls the output sent to the 'punch' file.



 PUBLISH     - Produce a listing suitable for publication.


 HTML        - Produce an HTML format listing


 CIF         - Produce a listing in CIF format.


 H-CIF       - Produce a listing of the H-bonds in CIF format.


 SCRIPT      - Lists bonds in a easily machine readable format.


 RESTRAIN    - Produce a proforma LIST 16 (restraints - \ref{LIST16}).  Use the RANGE, 
  LIMIT, TYPE INCLUDE and EXCLUDE parameters to restrict the restraints produced.


 DELU        - Proforma LIST 16 for delta U restraints


 SIMU        - Proforma LIST 16 for U-similarity restraints


 NONBONDED   - Proforma LIST 16 with anti-bumping restraints


 H-RESTRAIN  - Produces a list of H-C,N and O distance and angle 
 restraints in the PUNCH file, and a list of the referenced H atoms in 
 the SCRIPTQUEUE file.


 H-RIDE      - Produces a list of H-C,N and O RIDE instructions.


 H-CIF       - Puts hydrogen bond donor and acceptors into the cif 
file.






 If hydrogen atom restraints are being generated, the following target 
values are used:
\small\begin{verbatim}
No.H  No.H   U mult     dist
C-H
>4            1.5      .96 disorder
1      1      1.2      .93 C C-H (acetylene)
1      2      1.2      .93 C-C(H)-C
1      3      1.2      .98 (C)3-C-H
2      1      1.2      .93 C=C-H(2)
2      2      1.2      .97 (C)2-C-(H)2
3      1      1.5      .96 C-C-(H)3
N-H
>4            1.5      .89 NH4 or disorder
1      1      1.2      .86 N-N/H
1      2      1.2      .86 (C)2-N-H
1      3      1.2      .89 (C)3-N-H
2      1      1.2      .86 C-N-(H)2
2      2      1.2      .89 (C)2-N-(H)2
3      1      1.2      .89 C-H-(H)3
O-H
1      1      1.5      .82 O-H

Dist      esd = 0.02
Vib       esd = 0.002
Angle     esd = 2.0

\end{verbatim}\normalsize








\bigskip\Keyword{HESD=}
 This controls the output of ESDs to the CIF file.

\small\begin{verbatim}
      ALL      - (Default) Output all bond length and angle standard
                 uncertainties (if requested) to the CIF (if requested),
                 including those of bonds to fixed atoms (i.e. to atoms on
                 special positions, or to atoms that are not refined).
      NONFIXED - Exclude standard uncertainties of bond distances and angles
                 to Hydrogen atoms that have not been refined. (as
                 required by Acta's notes for authors).
\end{verbatim}\normalsize




\bigskip\Directive{SELECT ALLDIST= COORD= SORTED= TYPE= RANGE= SYMMETRY= TRANS=}



\bigskip\Keyword{ALLDISTANCES=}

\small\begin{verbatim}
      NO   -  Default value
      YES
\end{verbatim}\normalsize





 If  ALLDISTANCES  is  NO,
 the distances calculated about each atom will only
 be those to atoms that occur after the central
 atom in LIST 5. (i.e. each distance
 is only printed once).



 If  ALLDISTANCES  is
  YES , then the distances from each atom to all
 the other atoms are calculated for all the atoms.
 (In this case, each distance will appear twice in the list).


\bigskip\Keyword{COORDINATES=}

\small\begin{verbatim}
      NO   -  Default value
      YES
\end{verbatim}\normalsize


 If  COORDINATES  is  YES, the transformed
 coordinates of each atom in a distance calculation
 are printed.
 If  COORDINATES  is  NO,
 the transformed coordinates are not printed.


\bigskip\Keyword{SORTED=}

\small\begin{verbatim}
      NO   -  Default value
      YES
\end{verbatim}\normalsize


 If  SORTED  is  NO,
 the distances from the central atom are in the order
 in which the other atoms occur in LIST 5.
  If  SORTED  is  YES , the distances are printed in
 order of increasing magnitude.


\bigskip\Keyword{TYPE=}
 This parameter indicates the type of distances which will be
 calculated.

\small\begin{verbatim}
      ALL   -  Default value
      INTRA
      INTER
\end{verbatim}\normalsize


 If  TYPE  is  ALL,
 then all distances are printed; if  TYPE  is  INTRA  then only
 intramolecular distances are printed, and if  TYPE  is  INTER
 then the intermolecular distances are printed (Note that the whole
 asymmetric unit is regarded as a 'molecule'.


\bigskip\Keyword{RANGE=}
 This parameter defines how the range is to be selected.
 Except when RANGE = LIMITS (when the lowest acceptable distance is
 user-specified) contacts of zero angstrom are suppressed.

\small\begin{verbatim}
      COVALENT      Use 'covalent' radii from LIST 29.
      VANDERWAALS.  Use 'VanderWaals' radii from LIST 29, but angles are
                    suppressed.
      IONIC.        Use 'ionic' radii from LIST 29.
      LIMITS.       Use specified or default ranges set by the LIMIT directive.
\end{verbatim}\normalsize




\bigskip\Keyword{SYMMETRY=}
 This parameter controls the use of symmetry information in the calculation of
 contacts, and can take three values.

\small\begin{verbatim}
      SPACEGROUP  -  Default value. The full spacegroup symmetry is used in
                                    all computations
      PATTERSON.     A centre of symmetry in introduced, and the translational
                     parts of the symmetry operators are dropped.
      NONE.          Only the identity operator is used.
\end{verbatim}\normalsize




\bigskip\Keyword{TRANSLATION=}
 This parameter controls the application of cell translations in the
 calculation of contacts, and can take the values YES or NO








\bigskip\Directive{LIMITS DMINIMUM= DMAXIMUM= AMINIMUM= AMAXIMUM=}


This directive specifies the limits for the distance angles
 calculations, and may only be given if RANGE = LIMITS has been specified
 on a
 preceding SELECT directive.


\bigskip\Keyword{DMINIMUM}
 This defines the distance below which distances are not
 calculated or printed. The default is zero.


\bigskip\Keyword{DMAXIMUM}
 This parameter defines the maximum distance above which distances are not
 calculated or printed.
 Use $\backslash$COMMANDS DISTANCES to find the default value for  DMAXIMUM.
 All the distances that are to be calculated and printed
 must lie between  DMINIMUM  and  DMAXIMUM.


\bigskip\Keyword{AMINIMUM}
 For a given central atom, other atoms which
 make contacts that are less than  AMINIMUM  will
 not be considered when the angles at the
 central atom are computed.
 The default is zero.


\bigskip\Keyword{AMAXIMUM}
 For a given central atom, other atoms which
 make contacts that are greater than  AMAXIMUM
 will not be considered when angles at the
 central atom are computed.
 The default value for  AMAXIMUM  is set in the COMMAND file.
 AMAXIMUM  And  AMINIMUM  define a shell about each
 pivot atom outside of which angles are not computed.








\bigskip\Directive{E.S.D.S COMPUTE= CELL=}


This directive determines whether estimated standard deviations
 of the distances and angles are calculated.


\bigskip\Keyword{COMPUTE}

\small\begin{verbatim}
      NO   -  Default value
      YES
\end{verbatim}\normalsize


 If this parameter is NO,
 standard deviations are not
 computed.
 Note that if e.s.d.'s are to be calculated, i.e.  COMPUTE  is
 set equal to  YES , then a suitable least squares matrix (LIST 11, see
 section \ref{LIST11}) must be
 available.


\bigskip\Keyword{CELL=}

\small\begin{verbatim}
      NO   -  Default value
      YES
\end{verbatim}\normalsize


 If this parameter is NO,
 the variance-covariance matrix for the
 cell parameters is not included when the e.s.d.'s are calculated.





\bigskip\Directive{INCLUDE atoms}


This directive determines which atoms are included as pivot atoms
 in the calculation.
 The arguments may be either a type of atom , or an atom specification of
 the 'type(serial)' or 'type(serial) UNTIL type(serial)' kind described
 elsewhere in the manual. Only INCLUDEd atoms are used as pivots, but distances
 and angles are computed to all other atoms in the current LIST 5 within
 the ranges specified on the SELECT directive.





\bigskip\Directive{ONLY atoms}


Similar to INCLUDE, except that specified atoms may be pivot or
 bonded.
 The arguments may be either a type of atom , or an atom specification of
 the 'type(serial)' or 'type(serial) UNTIL type(serial)' kind described
 elsewhere in the manual. Distances
 and angles are computed only to specified atoms in the current LIST 5 within
 the ranges specified on the SELECT directive.





\bigskip\Directive{PIVOT atoms}


Similar to INCLUDE, except that atoms excluded with an EXCLUDE
directive can still be used to bond to.
 The arguments may be either a type of atom , or an atom specification of
 the 'type(serial)' or 'type(serial) UNTIL type(serial)' kind described
 elsewhere in the manual. Distances
 and angles are computed only to specified atoms in the current LIST 5 within
 the ranges specified on the SELECT directive.





\bigskip\Directive{BONDED atoms}


Similar to INCLUDE, except that non-included atoms can still be used as
pivots.
 The arguments may be either a type of atom , or an atom specification of
 the 'type(serial)' or 'type(serial) UNTIL type(serial)' kind described
 elsewhere in the manual. Distances
 and angles are computed only to specified atoms in the current LIST 5 within
 the ranges specified on the SELECT directive.





\bigskip\Directive{EXCLUDE atoms}


This directive determines which atoms are excluded as pivots in the
 calculation.
 The arguments may be either a type of atom , or an atom specification of
 the 'type(serial)' or 'type(serial) UNTIL type(serial)' kind described
 elsewhere in the manual. If EXCLUDE directives alone are used, all
 atoms except those EXCLUDEd either explicitly or by
 type, are used as pivot atoms in the calculation.
 However, if both INCLUDE and EXCLUDE are used, the only atoms used in
 the calculation will be those INCLUDEd and not EXCLUDEd.






\section{Distance-angles symmetry operations}



Accompanying each atom in a distance or angle calculation
 with  LIST  equal to  HIGH
 are the symmetry operators that are necessary to bring the atom
 into the correct position in the cell to make a contact with
 the central atom.
 These symmetry operations are divided into six parts, which are
 indicated by five flags. These are explained in the section on Atomic
 and Structural Parameters.

\small\begin{verbatim}
 \
 \ distances from 0 to 2.5
 \ angles from 0 to 2.0
 \ the e.s.d.'s of the distances and angles are calculated
 \ distances from each atom to all other atoms are printed
 \ transformed coordinates are printed
 \ the distances are sorted in order of increasing magnitude
 \
 \DISTANCES
 SELECT ALL=YES,COORD=YES,SORT=YES,RANGE=LIMITS
 LIMITS DMAX=2.5, AMAX=2.0
 E.S.D. YES
 END
\end{verbatim}\normalsize






\small\begin{verbatim}
 \DIST
 EXCLUDE ALL
 ONLY C(1) C(3) C(4)
 END
\end{verbatim}\normalsize








\index{VOIDS}
\index{Locating voids in the model}
\section{Void Location  -  $\backslash$VOIDS}


\small\begin{verbatim}
 \VOIDS INPUTLIST=
 DISTANCE
 TOLERANCE
 CONTACTS
 RESOLUTION
 END

 \VOIDS
 DISTANCE 2.2
 END
\end{verbatim}\normalsize




This utility searches for the asymmetric unit for points which lie
outside the known atoms. The 'radii' of the known atoms is independent
of type, and in an input value. A pseudo atom in inserted at every point
on a search grid outside the known atoms. The pseudo atoms are given a
'TYPE' dependant upon the number of neighbouring pseudo atoms. Atoms of
type R are at the core of large voids, type L are intermediate, and M at
the surface.





\bigskip\Instruction{$\backslash$VOIDS INPUTLIST=}




\bigskip\Keyword{INPUTLIST=}

\small\begin{verbatim}
      5   -  Default value
      10
\end{verbatim}\normalsize


 The default is to use the normal atom coordinate list.





\bigskip\Directive{DISTANCE value}


This sets the radii of the known atoms, default 2.5A.





\bigskip\Directive{RESOLUTION value}


This sets the sampling interval for the search grid, default 0.8 A.





\bigskip\Directive{CONTACT value1 value2}


This sets the number of pseudo-atom contacts required for the core
 and intermediate pseudo atoms. The defaults are 27 (R type atoms), 15
 (L type atoms). All other atoms are of type M.


$\backslash$COLLECT and  $\backslash$REGROUP can be used to re-group the pseudo-atoms, and
 the augmented structure can be viewed in CAMERON.







\index{GEOMETRY}
\index{Best lines and planes}
\index{Planes}
\index{Lines}
\index{Inertial Tensor}
\index{Centre of Gravity}
\index{Librational tensor}
\index{Analysis of thermal displacement parameters}
\index{Principal atomic displacement directions}
\section{TLS analysis, best planes and lines  -  $\backslash$GEOMETRY}


\small\begin{verbatim}
 \GEOMETRY INPUTLIST=
  ATOMS  W(1)  SPECIFICATION(1)  W(2)  SPECIFICATION(2) .
  PLANE
  LINE
  AXES
  TLS 
  EXECUTE
  EVALUATE  ATOM SPECIFICATIONS . . . .
  REPLACE ATOM SPECIFICATIONS . . .
  PUNCH
  SAVE
  DIHEDRAL  NP(1)  AND  NP(2)
  QUIT
  CENTRE   X=, Y=, Z=
  REJECT   NV=
  LIMITS   VALUE=   RATIO= 
  MODL L(11), L(22) L(33) L(23) L(13) L(12)
  MODT T(11), T(22) T(33) T(23) T(13) T(12)
  ZEROS
  DISTANCES  DL=   AL=
  ANGLES  AL=
  PLOT
  END
\end{verbatim}\normalsize



\small\begin{verbatim}
 \GEOMETRY
 ATOMS FIRST UNTIL LAST
 PLANE
 EXECUTE
 SAVE
 ATOMS FIRST UNTIL LAST
 TLS
 EXECUTE
 ANGLE 1 AND 2
 EXECUTE
 DISTANCES
 END
\end{verbatim}\normalsize







GEOMETRY is used for computing the following global derived 
 parameters:

\small\begin{verbatim}
      Centroid (centre of gravity)
      Inertial Tensor
      Best Plane
      Best Line
      Shape Indices
      Principal Axes of adps
      Librational and Translational Thermal Tensors
      Dihedral Angles
\end{verbatim}\normalsize




It replaces the old $\backslash$MOLAX, $\backslash$AXES and $\backslash$ANISO commands
 


PLANE \& LINE are used for computing the principal axes of inertia
 through groups of atoms using  the routines described in
 Computing Methods in Crystallography, edited by J. S. Rollett,
 Pergamon Press, 1965, p67-68.


The  best  plane for a series of  N
 atoms whose positions have varying reliability, such that they can
 be assigned weights, w(1), w(2), . . . w(n), is defined as
 that for which
 the sum of the squares of the distances (in angstroms) of the
 atoms from the plane, multiplied by the weights, w(i), of
 the atomic positions, is a minimum.
 Note that the normal to the 'worst plane' is the 'best line',
 and if masses are used for weights, then the calculation gives the
 principal inertial axes.


The atomic positions are taken from LIST 5, possibly modified by
 symmetry information, to compute
 inertial axes \& deviations of atoms from  the planes or lines.


Each time a line or plane
 is computed, the direction cosines of the relevent axis are stored as AXIS
 number 'n'. The dihedral angles between these axes can be computed.
 Three geometry indices are also
 computed. The geometry is best described by the index closest to unity.
 (Mingos,D.P.M \& Rohl,A.L., J.Chem.Soc. Dalton Trans (1991) pp 3419 - 3425)


If the three principal axes are "big, medium, small", then




      Spherical   = small/big
      Cylindrical = 1-((medium+small)/(2.big))
      discoidal   = 1-2.small/(big+medium)





TLS. This routine calculates the overall rigid-body
 motion tensors T, L, S (Shoemaker and Trueblood, Acta Cryst. B24,
 63, 1968) by a least-squares fit to the individual anisotropic
 temperature factor components, together with librational corrections
 to bond lengths and angles.


Shoemaker and Trueblood's conventions and reductions
 are followed throughout; in particular, the trace of S, which is
 indeterminant, is set to zero. The program therefore determines
 20 overall tensor components - the upper triangles of T and L
 together with the whole of S apart from S(33). (See also:
 Johnson in Crystallographic Computing, 
 ed R.Ahmed, Munksgaard, 1970, pp 207-219)


Even when the trace-of-S singularity has been removed, however,
 the nature of the rigid body problem is such that ill-conditioned
 and singular normal matrices are much more common than in
 structure refinement and the program therefore proceeds via
 the eigenvalues and eigenvectors of the normal matrix.  In most
 cases the largest and smallest eigenvalues are output for
 inspection, but if the ratio of these quantities is less than
 the LIMITing RATIO, a full eigenvalue/vector listing is produced. Further,
 if any eigenvalue is itself less than the LIMITing VALUE, the corresponding
 parameter combination is set to zero, thus removing the near-
 singularity. These actions can be modified by the use of the
 LIMIT  and  REJECT  directives described below. If the TLS calcuation 
 cannot be stabilised by means of these filters, the user can modify 
 either T, L or S directly before applying the REPLACE or PUNCH 
 commands. Though here is some danger in this, especially if the supposed rigid 
 group is infact flexible, it may be preferable to using a model 
 yielding negative vibrational or librational amplitudes.






The direction cosines of the 
 principal axis of L are stored for use in inter-axis angle comutations.



Immediate execution of a directive can be forced by issuing an EXECUTE
 directive.





\bigskip\Instruction{$\backslash$GEOMETRY INPUTLIST=}




\bigskip\Keyword{INPUTLIST=}

\small\begin{verbatim}
      5   -  Default value
      10
\end{verbatim}\normalsize







\bigskip\Directive{ATOMS  W(1)  SPECIFICATION(1)  W(2)  SPECIFICATION(2) .}


This specifies atoms to be used in the calculation of the
 best plane.
  W(1)  Is the weight assigned to the atoms
 contained in the first atom specification,  W(2)  is the weight
 assigned to the second group of atoms, and so on.
 If  W(1)  is omitted, a default value of 1 is used,
 but any other  W(I)  term applies to all the atoms following it,
 until another  W  is found or the end of the directive is
 encountered.
 At least one  ATOM  directive must precede each  PLANE, LINE, TLS, AXES or
 PLOT directive.
 An ATOM directive will over-rule an immediately preceding ATOM directive. If an
 input line is not long enough for the full atom list, use CONTINUE.





\bigskip\Directive{PLANE}


This directive, (or  LINE, TLS, AXES, PLOT)
 must follow immediately after an  ATOM  directive and
 causes the calculation of a least squares best plane.





\bigskip\Directive{LINE}


This directive, (or  PLANE, TLS, AXES, PLOT)
 must follow immediately after an  ATOM  directive and
 causes the calculation of a least squares best line.





\bigskip\Directive{AXES}


This directive (like $\backslash$AXES) computes the principal axis lengths
 and directions for the atoms specified on a preceding ATOM directive.





\bigskip\Directive{TLS }


This causes the TLS calculation to be initiated. It MUST have been preceded
 by an ATOM directive.








\bigskip\Directive{EXECUTE}


This forces the execution of preceding directives.





\bigskip\Directive{EVALUATE  ATOM SPECIFICATIONS . . . .}


If present, this directive
 must appear after a  PLANE, LINE, TLS or PLOT directive,
 and causes the co-ordinates or adps of the atoms specified
 to be calculated and printed with respect to the current axial system.





\bigskip\Directive{REPLACE ATOM SPECIFICATIONS . . .}


if present, this directive
 must appear after a  PLANE, LINE, TLS or PLOT directive,
 and causes the co-ordinates or adps of the atoms specified to be modified so
 that they conform to the most recent geometry calculation.
 The LIST 5 in core is immediately
 updated, so that the new coordinates will be used for any subsequent
 computation. A LIST 5 is only written to the disc on a satisfactory exit from
 GEOMETRY.





\bigskip\Directive{PUNCH}


This directive causes the orthogonal coordinates of the atoms of any plane or
 line computed or EVALUATED in subsequent tasks to be output to the 'punch'
 file. For a TLS calculation, it causes  a restraint list to be output 
 to TLSREST.DAT



\label{LIST20} 


\bigskip\Directive{SAVE}


This directive is optional. 


If it follows a PLANE, LINE or TLS directive, it
 causes the latest rotation matrix and CENTRE  to be stored in the 
 appropriate position in LIST 20. 


If it
 follows an AXES directive, the direction cosines and centre if the ellipse FOR
 THE LAST ATOM are stored in LIST 20.


A LIST 20 is only written to the disc on
 a satisfactory exit from ANISO.





\bigskip\Directive{DIHEDRAL  NP(1)  AND  NP(2)}


If present, thus directive must follow at least
 two  PLANE, LINE or TLS computations.
 It causes the program
 to calculate the angle between the axes with  serial numbers
   NP(1)  and  NP(2) .
 The   AND  must be present.








\bigskip\Directive{QUIT}


This directive abandons the calculation without modifying the disc LISTs.













\bigskip\Directive{CENTRE   X=, Y=, Z=}


This directive specifies the centre of libration,
 in crystal fractions, to be used in the original derivation of
 the overall motion tensors. The program derives and uses a unique
 origin at a later stage in the calculations. This directive
 is optional, the default centre being (0,0,0).
 If a centre of (0,0,0) is given or set by default, the program computes
 and uses the mean position of the given atoms, INCLUDING any which are
 isotropic, even though these are not used to compute TLS. The stored CENTRE
 is updated during TLS, and a second TLS computation may be performed using
 this new value as CENTRE. This may help stabilise certain forms of
 ill-conditioning.





\bigskip\Directive{REJECT   NV=}


Overrides normal action and sets the parameter combination
 corresponding to eigenvector number nv to zero.
 Eigenvectors are numbered in ascending order of their eigenvalues,
 so that nv
 is in the range 1 to 20 inclusive and will usually have been obtained
 from a full eigenvalue/vector listing produced in
 a previous run.





\bigskip\Directive{LIMITS   VALUE=   RATIO= }


If an eigenvalue is less than  VALUE  or its size is less than
  RATIO * (the next bigger), it is eliminated from the analysis.
 VALUE is currently .000001 and RATIO .01 .





\bigskip\Directive{MODL L(11), L(22) L(33) L(23) L(13) L(12)}


This directive enables the user to change the values of the L tensor before
 EVALUATING or REPLACING the Uij. The L tensor changed is that with 
 respect to the inertial axes and the input centre of libration. It does 
 not depend upon S. All six values must be given.





\bigskip\Directive{MODT T(11), T(22) T(33) T(23) T(13) T(12)}


This directive enables the user to change the values of the T tensor before
 EVALUATING or REPLACING the Uij. The T tensor changed is that with 
 respect to the inertial axes and the input centre of libration, NOT the 
 final tensor, since this involves an interaction with S and L.
  All six values must be given.








DZEROS





This directive enables the user to set the S tensor to zero before
 EVALUATING or REPLACING the Uij. It decouples T from L.


\bigskip\Directive{DISTANCES  DL=   AL=}


This directive calculates all interatomic distances less than
 DL angstroms with librational corrections. If this directive is omitted,
 no distances are calculated; if DL is absent, a default value of 1.8 is
 inserted. If AL is present, angles between atoms separated by less than AL
 angstroms are computed.





\bigskip\Directive{ANGLES  AL=}


This directive calculates angles between all bonds less than
 AL angstroms. If this directive is omitted, no angles are calculated;
 if AL is absent, a default value of 1.8 is inserted.






 \emph{***********************} WARNING *************************





The directive DISTANCE may only be
 followed by ATOM, EXECUTE, or END.





\bigskip\Directive{PLOT}


This obsolete directive produces a join-the-dots diagram
 on the monitor or printer. It (or  PLANE, LINE, TLS, AXES)
 must follow immediately after an  ATOM  directive and
 causes the calculation of inertial axes.
 Details of the computation are suppressed on the Monitor,
 but a line drawing projected onto the best plane is produced.
  MOLAX  Can thus be used as a means of displaying some or all
 of the atoms in a structure.




\small\begin{verbatim}
   Examples:
 
  These instructions compute a plane
  involving n(1),n(2),n(3) and c(1), and
  prints the co-ordinates of all the atoms with
  respect to this plane.  The positions of the
  nitrogen atoms have double weight
 
 \GEOMETRY
 ATOMS 2 N(1) UNTIL N(3)  1 C(1) C(2)
 PLANE
 EVALUATE ALL
 
  These instructions calculate another plane,
  printing only the co-ordinates of c(5) with respect to
  the second plane.  The angle between the two planes
  is then calculated
 
 ATOMS C(1) S(1) N(1)
 PLANE
 EVALUATE C(5)
 DIHEDRAL 1 AND 2
 END
 
 
  These instructions compute a TLS tensor for the specified atoms
  and then set up restraints to encourage the Uij to conform to a
  rigid body.  You might want to tighten the esd in TLSREST.dat to 0.005.
 
 \GEOM                                                                           
 ATOM ALL
 TLS
 PUNCH
 EVAL ALL
 EX
 END
 \LIST 16                                                                        
 \USE TLSREST.DAT                                                                
 END
\end{verbatim}\normalsize









\index{TORSION}
\index{Torsion angles}
\section{Torsion angles  -  $\backslash$TORSION}


\small\begin{verbatim}
 \TORSION INPUTLIST=
 ATOMS  SPECIFICATIONS
 PUBLICATION  PRINT= LEVEL= E.S.D=
 INLCUDE TYPE=
 END

 \TORSION
 ATOM C(1) C(2) C(3) C(4)
 END
\end{verbatim}\normalsize




Calculation of the torsion angles for atoms i,j,k,l. 
 The torsion angle about the bond j-k
 is the angle the bond i-j is rotated about j-k to bring it into 
 coincidence with k-l.
 It is positive when  the rotation is
 clockwise on looking from j to k.


The program uses atomic positions taken from
 LIST 5.  These can be
 modified by the space group symmetry operators stored in LIST 2 (space
 group information, see section \ref{LIST02})


 ESDs (SUs) are computed by the method of Nardelli (using code given 
 to us from his program PARST) if only average atomic esds are 
 avaialble.  If the full VcV matrix is avaiable, they are computed by 
the method of Schaik et al (R. van Schaik, H Berendsen, A Torda \& W 
Gunsteren, J. Mol. Biol. (1993) 234, 751-762. The paper by U. Shmueli, 
Acta Cryst (1974) A30, 848-849 explains the need for the full vcv matrix 
and the effect of symmetry. The singularity in his equations (6) when tau is 
0 or 180 is circumvented in Schaiks formulation (which is singular if 
atoms j and k are coincident).



If three (or more) atoms are approximately co-linear the torsion angle is ill
defined. The current threshold for co-linearity is 5 degrees.






\bigskip\Instruction{$\backslash$TORSION INPUTLIST=}




\bigskip\Keyword{INPUTLIST}

\small\begin{verbatim}
      5   -  Default value
      10
\end{verbatim}\normalsize







\bigskip\Directive{ATOMS  SPECIFICATIONS}


This directive specifies atoms that are to be used in the calculation
 of the torsion angle. More than one  ATOMS  directive can be given. Each
 directive must define at least four atoms, the torsion angle being
 computed with respect to the first three atoms and each of the
 subsequent ones.


Atoms cannot be give if INCLUDE is ALL or NO-HYDROGEN





\bigskip\Directive{PUBLICATION  PRINT=}


The parameter PRINT controls the publication listing, which is sent to 
the file open on the CRYSTALS PUNCH unit.

\small\begin{verbatim}
      NO   -  DEFAULT.  There is no publication listing
      YES    There is a publication listing sent to the PUNCH file
      CIF    The listing is in CIF format
      PUNCH  Creates a file containing all the quartets of atoms forming
             bonded torsion angles.
\end{verbatim}\normalsize




\bigskip\Directive{ LEVEL=}


If LEVEL is HIGH, the VcV marix of the atoms in each torsion angle
 is printed.





\bigskip\Directive{ E.S.D.=}


NO suppresses the computation of standard uncertainties, EXCLRH excludes the
 computation of the su of torsion angles involving a riding hydroen (to keep checkcif
 happy).
\small\begin{verbatim}
      NO
      YES The default value
      EXCLRH  Exclude ridning hydrogen atoms
\end{verbatim}\normalsize




\bigskip\Directive{INCLUDE TYPE=}


The default (NULL) causes this directive to be skipped so that ATOM directives
  can be given, otherwise quartets of atom names are sent to the PUNCH file



\small\begin{verbatim}
     NULL  (The default). This directive is ignored
     ALL   Torsion quartets for all atoms are sent to the punch file.
     NO-HYDROGEN  Only quartets not involving hydrogen are punch out.
\end{verbatim}\normalsize
\small\begin{verbatim}
 Example.
 \ the torsion angle about C(3)-C(4) is calculated
 \ two torsion angles about C(4)-C(5) are calculated
 \
 \TORSION
 ATOMS N(2) C(3) C(4) C(5)
 ATOMS C(3) C(4) C(5) C(6) O(1,2,2)
 END
\end{verbatim}\normalsize


\small\begin{verbatim}
 Example.
 PART 1 sends a list of all quartets to a file called TORS-LIST.DAT

      \ PART 1
      \RELE PUNCH TORS-LIST.DAT
      \tors
      PUB PRINT=PUNCH
      incl all
      end
      \RELE PUNCH logs/bfile/pch
      
 PART 2 uses this file to create a partial cif file containing the angles
 and esds.
      \ PART 2
      \tors
      pub print=cif e.s.d=yes
      \use TORS-LIST.DAT
      END
\end{verbatim}\normalsize





\index{TON}
\index{Hooft-Straver-Spek}
\section{Absolute Configuration  -  $\backslash$TON}


\small\begin{verbatim}
 \TON 
 END

\end{verbatim}\normalsize




Compute the Hooft-Straver-Spek parameter for comparisonm with the 
Flack parameter





\bigskip\Instruction{$\backslash$TON}







This instruction evaluates a function of (Fo+h - Fo-h) and  (Fo+h - Fo-h)
 for all Friedel pairs. It uses LIST 7, which must be sorted to that 
 Friedel pairs are adjacent and with the same index signs, and with a 
 flag to indicate if the sign has been changed in 'phase'.


 The data is most conveniently prepared useing the SCRIPT TON towards 
the end of a refinement.















\index{PARAMETERS}
\index{Publication tables of atomic parameters}
\section{Publication listing of the atomic parameters  -  $\backslash$PARAMETERS}


\small\begin{verbatim}
 \PARAMETERS
 LAYOUT INSET= ATOM= DOUBLE= CHOOSE= FLOAT= NCHAR= NLINE= LISTAXES= ESD=
 COORDINATES NCHAR= NDECIMAL= SELECT= TYPE= DISPLAY= PRINT= PUNCH=
 U'S NCHAR= NDEC= SELECT= TYPE= DISPLAY= PRINT= PUNCH=
 END
\end{verbatim}\normalsize
\small\begin{verbatim}
 \PARAMETERS
 LAYOUT ATOM-NAME=6,DOUBLE=YES
 END
\end{verbatim}\normalsize




This routine sends the atomic parameters to the PUNCH file  in a suitable
 format  for publication or binding into a thesis.  As well as the current
 atomic parameters in LIST 5, the estimated  standard deviations derived from
 the least squares normal matrix are also  printed.
 {\bf THIS} {\bf ROUTINE} {\bf WILL} {\bf NOT} {\bf WORK} if LIST
 5  is modified \emph{in} \emph{any} \emph{way} since the last round of
 refinement. If any changes, including  renaming, are made, a further round of
 refinement must be done. If you wish  to preserve parameter values,
 and create a valid matrix
 without changing the parameter values, compute a refinement cycle but set
 all the shifts to zero.

\small\begin{verbatim}
      \SFLS
      REFINE
      SHIFT GENERAL = 0.0
      END
\end{verbatim}\normalsize


 The output is in two halves, the first
 containing  the positional coordinates and any isotropic temperature factors,
 and the  second containing all the anisotropic temperature parameters.


For the first part, a page is split into 6 separate  fields.
  The first field
 is blank, and is an offset so that the information  is  centred on
 the page.  The remaining fields contain the atom type and serial
 number, the  three positional parameters, and a temperature factor.
 This will be the value of U(iso) with its e.s.d for isotropic atoms,
 otherwise U(equiv), without an e.s.d, for anisotropic atoms.
 U(equiv) {\bf is} {\bf not} simply related to the
 diagonal elements of U(aniso), and may be computed as either the
 arithmetic or geometric mean of the principal axes of the ellipsoid.
 See $\backslash$SET UEQUIV in the chapter on IMMEDIATE commands.
 The width of each type of field may be altered by the user, using
 respectively the  INSET ,  ATOM-NAME , and  NCHARACTER  parameters.  The
 default length
 of a page of this type of output is that required for A4 paper.


The second part contains the anisotropic temperature factors, and each page
 is split into eight fields.
 As for the atomic coordinates, the first field is blank and represents
 an offset.
 The second field contains the atom type and serial number, and
 the remaining six fields contain the components of the anisotropic
 temperature factors.
 The width of each type of field may be adjusted by the user, using
 respectively the  INSET ,  ATOM-NAME  and  NCHARACTER  parameters.
 If a different value for  INSET  or  ATOM-NAME  is required in the
 first and second parts of the output, the job must be run twice.
 Depending upon the width across the page, the second part of the output
 occupies one sheet of A4 paper either across the page or down the page.


For both types of output, the user can select double spacing
 down the page with the  DOUBLE  parameter.
 Similarly for each of the numeric fields, the user can choose the number
 of decimal places to be printed
 (the  NDECIMAL  parameter),
 and whether the numbers are printed as integers or in floating point
 with a decimal point.
 (The  FLOATING  parameter).
 The e.s.d.'s are printed to the same accuracy as the atomic parameters,
 so that if the chosen field is too small and an e.s.d. appears to be
 zero, it will be omitted in exactly the same way as for a parameter
 that has not been refined.
 A parameter printed with 4 decimal places might thus appear as :

\small\begin{verbatim}
 0.0123(4)
 OR
    123(4)
\end{verbatim}\normalsize


 Depending upon the format.
 In either case, the numbers are right justified in their field.


As an alternative to the user selecting the number of decimal
 places that should be printed, it is possible to get the program
 to choose the number of decimal places required for each parameter
 automatically.
 (The  CHOOSE  parameter).
 If the parameters are to be printed in floating point, the number
 of decimal places is chosen so that the e.s.d. Can be represented
 as a one digit number in the last decimal place.
 For numbers that are to be printed as integers, the field used
 is never less than that given by the  NDECIMAL  parameter.
 If the required field is larger than that defined by these 
s,
 a decimal point is inserted and the required number of extra digits
 is output.
 For example, if the number of decimal places required is four, but the e.s.d.
 is too small, it would appear as :

\small\begin{verbatim}
 0.12345(6)
 OR
  1234.5(6)
\end{verbatim}\normalsize


 Depending upon whether floating point or integer output was required.
 For either type, if the parameter has not been refined, the number of
 decimal places is that given by the  NDECIMAL  instruction.


Since this routine prints the e.s.d.'s, it is vital that the least squares
 matrix (LIST 11, see section \ref{LIST11}) belongs to the current 
 LIST 5 (the model parameters). If LIST 5 has been modified
 in any way since the last Least Squares, this routine will abort.


When anisotropic atoms are present in LIST 5, U[EQUIV] is calculated
according to the current setting of $\backslash$SET UEQUIV.








\bigskip\Instruction{$\backslash$PARAMETERS}




This command initiates the routines for printing of the atomic
 parameters in a suitable format for publication.





\bigskip\Directive{LAYOUT= INSET= ATOM= DOUBLE= CHOOSE= FLOAT= NCHAR= NLINE= LISTAXES= ESD=}


This directive defines how the atomic parameters, both positional
 and thermal, are to be laid out on the page.


\bigskip\Keyword{INSET}
 This parameter sets the number of blank spaces on each line before the
 atom type and serial number. If this parameter is omitted
 a default value of 1 is assumed.


\bigskip\Keyword{ATOM-NAME}
 This parameter sets the width of the field that contains the atom
 type and serial number.
 The characters are left justified in the field, and the format is
 as follows :





TYPE(SERIAL)



 The serial number is printed as an integer, and the unoccupied
 spaces are filled with blanks.
 If this parameter is omitted, a default value of 6 is assumed.


\bigskip\Keyword{DOUBLE}
 This parameter has two possible values :





NO   -  DEFAULT VALUE


YES



 If  DOUBLE  is  YES  each line of parameters is double spaced.
 The default option if this parameter is omitted is single
 spacing, with no interleaving blank lines.


\bigskip\Keyword{CHOOSE}
 This parameter has two possible values :





NO


YES  -  DEFAULT VALUE




 If  CHOOSE  is  YES  the program chooses the number of decimal places that
 need to be printed for each parameter, depending upon its e.s.d..
 The format of the output depends upon whether a decimal point
 is being used, as explained above.


\bigskip\Keyword{FLOATING}
 This parameter has two possible values :





YES  -  DEFAULT VALUE


NO



 If  FLOATING  is  NO , the parameters are printed as integers,
 with an accuracy given either by the  NDECIMAL  parameters
 to the directives  COORDINATES  and "U'S, or by the 'CHOOSE' parameter.
 parameter.


\bigskip\Keyword{NCHARACTER}
 This parameter indicates the total number of printing positions
 on the output device.
 If this parameter is omitted, a default value of 118 is assumed.



\bigskip\Keyword{NLINE}
 This parameter indicates the total number of lines on the
 on the output media. Set a very lartge value (1000) to get continuous
 output.



\bigskip\Keyword{LISTAXES}
 This parameter can have two values





YES


NO   -  DEFAULT VALUE



 If the value is YES the principal axes of the temperature factors
 are printed.





\bigskip\Keyword{ESDS}
 This parameter can take 3 values





NO


YES  -  DEFAULT VALUE


EXCLRH



 EXCLRH inhibits printing the e.s.ds for riding hydrogen atoms





\bigskip\Directive{COORDINATES NCHAR= NDECIMAL= SELECT= TYPE= DISPLAY= PRINT= PUNCH=}


This directive defines how the positional coordinates are to be
 set out on the page.


\bigskip\Keyword{NCHARACTER}
 This parameter sets the width of the field that contains the
 positional coordinates. The characters are right
 justified in the field, and if this parameter is omitted,
 a default value of 14 is assumed.


\bigskip\Keyword{NDECIMAL}
 This parameter sets the number of decimal places to be printed for
 the positional parameters.
 It may be partially or completely overriden by the  CHOOSE
 parameter, depending upon the format of the output.
 If this parameter is omitted, a default value of 4 is assumed.


\bigskip\Keyword{SELECT}
 This parameter selects the kinds of data to be printed, and
 can have five values.





ALL     - Default. All atoms are printed.


NONE    - No atoms are printed.


ONLY    - Only atoms with TYPEs given on a TYPE directive are printed.


EXCLUDE - Atoms with TYPEs given on a TYPE directive are not printed.


SEPARATE- Atoms with TYPEs given on a TYPE directive are printed separately





\bigskip\Keyword{TYPE}



 Used in conjunction with SELECT to determine which atom types to
 INCLUDE,EXCLUDE or SEPARATE. TYPE is ignored if SELECT is ALL or NONE.
 Its default value is 'H'.


\bigskip\Keyword{DISPLAY}
 This parameter has two possible values

\small\begin{verbatim}
 NO   No output is displayed on the terminal.
 YES  Output is displayed on the terminal.
\end{verbatim}\normalsize




\bigskip\Keyword{PRINT}
 This parameter has two possible values

\small\begin{verbatim}
 NO       No output is sent to the listing file
 YES      Output is sent to the listing file
\end{verbatim}\normalsize




\bigskip\Keyword{PUNCH}
 This parameter has three possible values

\small\begin{verbatim}
 NO       No output is sent to the punch file
 YES      Output is sent to the punch file
 CIF      Output is in CIF format
\end{verbatim}\normalsize







\bigskip\Directive{U'S NCHAR= NDEC= SELECT= TYPE= DISPLAY= PRINT= PUNCH=}


This directive defines how the thermal parameter are to be
 set out on the page.


\bigskip\Keyword{NCHARACTER}
 This parameter sets the width of the field that contains the thermal
 parameters. The characters are right justified in the field, and
 if this parameter is omitted, a default value of 11 is assumed.


\bigskip\Keyword{NDECIMAL}
 This parameter sets the number of decimal places to be printed for
 the thermal parameters.
 If this parameter is omitted, a default value of 4 is assumed.


\bigskip\Keyword{SELECT}
 This parameter selects the kinds of data to be printed, and
 can have five values.





ALL     - Default. All atoms are printed.


NONE    - No atoms are printed.


ONLY    - Only atoms with TYPEs given on a TYPE directive are printed.


EXCLUDE - Atoms with TYPEs given on a TYPE directive are not printed.


SEPARATE- Atoms with TYPEs given on a TYPE directive are printed separately





\bigskip\Keyword{TYPE}



 Used in conjunction with SELECT to determine which atom types to
 INCLUDE,EXCLUDE or SEPARATE. TYPE is ignored if SELECT is ALL or NONE.
 Its default value is 'H'.


\bigskip\Keyword{MONITOR}
 This parameter has two possible values

\small\begin{verbatim}
 OFF   No output is displayed on the terminal.
 HIGH  Output is displayed on the terminal.
\end{verbatim}\normalsize




\bigskip\Keyword{PRINT}
 This parameter has two possible values

\small\begin{verbatim}
 NO       No output is sent to the listing file
 YES      Output is sent to the listing file
\end{verbatim}\normalsize




\bigskip\Keyword{PUNCH}
 This parameter has three possible values

\small\begin{verbatim}
 NO       No output is sent to the punch file
 YES      Output is sent to the punch file
 CIF      Output is in CIF format
\end{verbatim}\normalsize


\index{REFLECTIONS}
\index{Publication tables of reflection data}
\section{Publication listing of reflection data  - $\backslash$REFLECTIONS}
\label{REFLECTIONS} 


\small\begin{verbatim}
\REFLECTIONS INPUT=
LAYOUT NCOLUMNS= NLINES= INSET= NSPACE= SCALE= NCHARACTER=
OUTPUT PRINT= PUNCH= LIST28=
END
\end{verbatim}\normalsize




This routine prints the reflection data in LIST 6 (section \ref{LIST06}) 
 in a format suitable 
 for publication or binding into a thesis. 



 (For an unformatted output you might prefer $\backslash$PUNCH 6 I or $\backslash$PUNCH 6 IF)


The information printed falls
 into one or more columns, each of which contains h, k, l, /Fo/, /Fc/, and the
 phase angle in degrees.
 Each column is 18 characters wide. Although the user has no control
 over the contents of each column, it is possible to vary the number of
 blank spaces at the start of each line, the number of columns across the
 page,  the number of spaces between successive columns, and the number of
 lines per page. (The  INSET ,  NCOLUMNS ,  NSPACE  and  NLINES  parameters,
 respectively).
 /Fo/ and /Fc/ are both put on the same scale of /Fc/, using the scale factor
 in LIST 5, and both these two numbers may be modified by a scaling constant
 before they are printed. (The  SCALE  parameter). However, all the values of
 both /Fo/ and /Fc/ must be less than 10000 when they are printed.


LIST 28 is used for checking whether or not to print a reflection. Remember
 that if LIST 28 was used to reject some reflections when structure factors
 were last calculated, removing these restrictions before printing LIST 6
 will mean that some reflections will have incorrect values of Fc and phase.


\bigskip\Instruction{$\backslash$REFLECTIONS INPUT=}




\bigskip\Keyword{INPUT}
Indicates which reflection list to use.
\small\begin{verbatim}
      6      Default
      7      Alternative reflection list
\end{verbatim}\normalsize







\bigskip\Directive{LAYOUT NCOLUMNS= NLINES= INSET= NSPACE= SCALE= NCHARACTER=}


This directive defines how the reflection data is to be printed.


\bigskip\Keyword{NCOLUMNS=}
 This parameter indicates the number of columns of reflection data to be
 printed across the page. If this parameter is omitted, a default value of
 3 is assumed.


\bigskip\Keyword{NLINES=}
 This parameter indicates how many lines should be on each page of output.
 If this parameter is omitted a default value of 52 is assumed.


\bigskip\Keyword{INSET=}
 This parameter indicates how many blank spaces should be inset at
 the beginning of each line. If this parameter is omitted a default value
 of 30 is assumed.


\bigskip\Keyword{NSPACE=}
 This parameter indicates the number of spaces separating successive
 columns across the page. If this parameter is omitted a default value
 of 3 is assumed, which means that each column occupies 21
 characters across the page.


\bigskip\Keyword{SCALE=}
 This parameter indicates the scaling constant by which /Fo/ and /Fc/
 should be multiplied before they are printed,
 after they have been put on the same scale (the scale of /Fc/).
 If this parameter is omitted, a default value of 10 is assumed.


\bigskip\Keyword{NCHARACTER=}
 This parameter indicates the total number of printing positions
 on the output device.
 If this parameter is omitted, a default value of 120 is assumed.


\bigskip\Directive{OUTPUT PRINT PUNCH LIST28}


This directive defines where the reflection data is to be printed.


\bigskip\Keyword{PRINT=}
 This has two allowed values :-

\small\begin{verbatim}
 NO       No output is sent to the listing file
 YES      Output is sent to the listing file
\end{verbatim}\normalsize




\bigskip\Keyword{PUNCH=}
 This has two allowed values :-

\small\begin{verbatim}
      NO       No output is sent to the punch file
      YES      Output is sent to the punch file
\end{verbatim}\normalsize





\index{SUMMARY}
\index{Summary of data}
\section{Summary of data lists - $\backslash$SUMMARY}


\small\begin{verbatim}
 \SUMMARY OF= TYPE= LEVEL=

 \SUMMARY LIST 5 HIGH
 END
 \SUMMARY EVERYTHING
 END
\end{verbatim}\normalsize




This command produces a summary on the terminal of the contents of a
 list. Use $\backslash$PRINT if you need full details.





\bigskip\Instruction{$\backslash$SUMMARY OF= TYPE= LEVEL=}




\bigskip\Keyword{OF}

 LIST         Default, also requires TYPE to be set
 EVERYTHING



 The value EVERYTHING generates a summary of all LISTS.


\bigskip\Keyword{TYPE}
 This parameter requires a list type number if the previous parameter
      was 'LIST"


\bigskip\Keyword{LEVEL}
 Some lists may be listed in more or less detail.

 OFF
 LOW
 MEDIUM    Default
 HIGH





\bigskip\Directive{$\backslash$SUMMARY 6}
Unlike all the other SUMMARIES, which only generate readable output,
SUMMARY 6 computes the conventional R on the basis of the current Fo, Fc
and LIST 28 (section \ref{}LIST28), and updates the value stored in 
LIST 6 (section \ref{LIST06}) 
and LIST30 (section \ref{LIST30}). The
weighted R is not affected.







\index{TONSPEK}
\index{Estimation of Absolute configuration} 
\section{Estimation of Absolute configuration - $\backslash$TONSPEK }
\label{TONSPEK} 



\small\begin{verbatim}
\TONSPEK INPUT= TYPE= PLOT= PUNCH= FILTER1= ... FILTER5= WEIGHT=
END
\end{verbatim}\normalsize




This routine contains code kindly donated by Ton Spek for the 
 computation of the Hooft 'y' parameter and probabiliies. 
 Rob W. W. Hooft, Leo H. Straver and Anthony L. Spek, 
 J. Appl. Cryst. (2008). 41, 96�103.  It requires the current LIST 6 to 
 be converted to a LIST 7 in which Friedel pairs are adjacent and in 
 which Fc has been computed with the Flack 'x' parameter set to zero.


The pre-preparation for using this routine has been packaged into a 
 script available from the Analyse menu.  The current value of the 
 Flack parameter should already be stored in LIST 30.





\bigskip\Instruction{$\backslash$TONSPEK INPUT= TYPE= PLOT= PUNCH= FILTER1= ... FILTER5= WEIGHT=}




\bigskip\Keyword{INPUT=}
Indicates which reflection list to use.
\small\begin{verbatim}
      6      Default
      7      Alternative reflection list
\end{verbatim}\normalsize







\bigskip\Keyword{TYPE=}
The computations can be carried out using the observed and computed
Bijvoet differences, or the observed and computed Parsons Quotients.



\small\begin{verbatim}
      DODC      Bijvoet Differences
      QOQC      Parsons Quotients
\end{verbatim}\normalsize

\bigskip\Keyword{PLOT=}
 When used from the GUI causes the various graphs to be displayed 
\small\begin{verbatim}
      YES     Default value
      NO      
\end{verbatim}\normalsize




\bigskip\Keyword{PUNCH=}
 Controls output to the punch file, BFILE.CPH



\small\begin{verbatim}
      NONE           Default
      TABLE          Table of all the Bijvoet pairs.
      RESTRAINT      List of restriants suitable for adding to a LIST 16.
      GRAPH          Values suitable for input to an external program
      SUMMARY        One-line summary of the various parameters and sus.
\end{verbatim}\normalsize




\bigskip\Keyword{FILTER1=}
 The minimum acceptable value of Dcalc/sigma(Dobs).  This is the minimum 
'ideal' theoretical signal:noise ratio.  The default value of 0.5 can be 
increased for materials with a large Friedif, or may need to be reduced 
for small Freidif.


\bigskip\Keyword{FILTER2=}
 The minimum acceptable value for the ration of the average of the observed Bijvoet 
pairs to the su of the average, default is 1.0


\bigskip\Keyword{FILTER3=}
 The observed Bijvoet differences muse bt less than FILTER3 times the 
maximum calculated difference, default value is 2.0


\bigskip\Keyword{FILTER4=}
 Minimum accaptable weight modifier from the Blessing algorithm.


\bigskip\Keyword{FILTER5=}
 Maximum values of Max(Ao,Ac) / Min (Ao,Ac).  Used to filter out 
reflections where are serious discrepancies between the observed and calculated 
structure factors. Default value is 2.5.










\index{CIF}
\section{CIF lists - $\backslash$CIF}

 Data can be produced in CIF format for direct deposition at CCDC or
 submission to journals. The required information is taken from several
 lists, including LIST 30 (see section \ref{LIST30}). F000, Mu etc are also 
 computed and inserted in LIST 30.





\bigskip\Instruction{$\backslash$CIF}




There are no qualifiers.


See $\backslash$PARAMETERS and $\backslash$REFLECTIONS for the CIF printing of parameters
 and reflections .





\bigskip\Directive{CheckCIF}
  CheckCIF and other validators are continuously updated to meet the 
 changing needs of the community. It is unlikely that a CRYSTALS .cif 
 will pass all checks first time, and edits may be necessary to 
 accommodate special situations. Some of these have been foreseen, and the 
 .cif contains possible alternative texts as 'comments'. These can be 
 found by searching for the text 'choose'.


\bigskip\Directive{References}
 The SCRIPT directory contains two text files that contain information 
 copied into the cif file. The user may edit them.


\bigskip\Keyword{Refcif.dat}
   This file is copied in its entirety to the head of the cif file. If 
 it is edited, care must be taken to follow the rules about text 
 delimiters.


\bigskip\Keyword{Reftab}
    This is a loosely formatted file containing the references to be 
transcribed into the cif. 


Every reference is composed of 2 parts - a short text used as a data 
 item in the cif, and the full reference. The two parts must be kept 
 together, be separated from each other by a blank line, and be separated 
 from any other item by a blank line. 
\small\begin{verbatim}
      n  a four-digit number giving the number of references to follow. 
         Other text on the line is ignored.
Next items repeated 'n' times:
      m  a three digit number preceded by a 'hash' symbol used as an 
         identifier for the reference. The numbers must be unique, not 
         necessarily in any order, with the largest one equal to 'n'
      
      The full reference. References are put in the file in alphabetic 
      order.
\end{verbatim}\normalsize


 Items 001 to 004 are associated with the keywords 'unknown' for the 
 corresponding items in LIST 30 (see section \ref{LIST30}), and thus 
 enable the user to insert 
 their own references. Don't forget to move them to their correct 
 alphabetical place.
      


\bigskip\Directive{e.s.d.s}
 The esds output in CIF files try to follow the 'Rule of 19', as
requested by Acta Cryst.
Syd Hall, former Editor for Acta C, summarised the rule as follows:



'This method of handling the su (esd) values has been in force
with Acta since about 1984 apparently. In my time it came up for
discussion about two years ago (1996) and after much to-ing and fro-ing
it was readopted as the preferred level of precision for su's.



What it means is as follows....
\small\begin{verbatim}
(1) if one adopts esd values to one digit precision (rule of 9) the values

      5.548(1)      1.453(2)  3.921(3)  1.2287(8)  are acceptable.

(2) if one permits two digits precision with a limit of 19 (rule of 19)...

      5.5483(9)     1.4532(16)  3.921(3)  1.2287(8)  are acceptable.

(3) if one permits two digits precision with a limit of 29 (rule of 29)...

      5.5483(9)     1.4532(16)  3.9214(28)  1.2287(8)  are acceptable.

\end{verbatim}\normalsize


The object of this approach is to provide a more consistent distribution
of precision across all values. These particular matters are not really
my responsibility but we try to conform to recommendation of the
nomenclature people. This is one such occasion.'



\index{ADDARC}
\section{Adding Archive files - ADDARC}
\label{addarc} 






\bigskip\Instruction{ADDARC filename}


 This command is usually called from SCRIPTS and inserts the contents
 of the file 'filename' into the current cif file, usually 'publish.cif'
















\index{CAMERON}
\section{Graphics - CAMERON}
\label{grcameron} 






\bigskip\Instruction{CAMERON}


 The graphics module CAMERON is part of the graphical user interface, and can
 only be started from the GUI. Like CRYSTALS, a sub-set of the possible
 commands are packaged up into menus, but the advanced full potential
 is still available from the command line. There is a separate guide for
 CAMERON



On exit from CAMERON the current image of the structure is padded back to
 CRYSTALS in the file  CAMERON.L5.
 This contains {\bf all} and {\bf only} the atoms last displayed by CAMERON.
 Be careful -
 it could be a packing diagram!









\chapter{Twinned Crystals}
\label{twinning} \label{LIST25} \index{Twinning}\section{Twinning - introduction}



The terminology in articles on twinning is complicated and sometimes
contradictory, with the same term being used in different contexts by
different authors. We shall use the following terms, based upon
observations made from the {\bf whole} reciprocal lattice.



It is assumed that sufficient reflections are measured to give a
complete coverage of the asymmetric part of the r.l. for at least one
(called the major) component of the twinned crystal.



\bigskip\Instruction{TLQS twins}




  Some, but possibly not all, of the reflections from the major
component contain contributions from other twin components.  Overlap is
controlled by accidental relationships between cell parameters. If the
relationship is very exact, so that all reflections are overlapped, the
sample is a {\bf pseudo} {\bf TLS} twin.


\bigskip\Instruction{TLS twins}




 Every reflection from the major component contains a
constant fractional contribution from other components. The overlap is
controlled by the crystal class rather than accidental relationships
between cell parameters.


\bigskip\Directive{TLS twins - Class I}
 Except for the effect of anomalous dispersion, the Laue
symmetry of the diffracion pattern is the same as that of an un-twinned
crystal.


\bigskip\Directive{TLS twins - Class II}
The Laue
symmetry of the diffracion pattern is {\bf not} the same as that of an un-twinned
crystal.



\section{Twinning Problems}

The analysis of twinned structures is complicated by several issues.



1. Identification that the crystal is indeed twinned.





\bigskip\Instruction{Twinning - Initial clues}


These may include may include:



      a. Evident interpenetrating reciprocal lattices.



      b. Split reflections, with a varying intensity ratio.



      c. Systematic absences not conforming to any space group.



      d. The ratio of intensties of equivalent reflections from
         different samples is not constant.



Other clues are:



      a. Failure to solve the structure from apparently good data.



      b. Irreducible R factor from seemingly good quality data.



      c. Inexplicable strong residual peaks in the difference density
         map.






\bigskip\Instruction{Twinning - Data collection and processing}








     a.



There is usually no difficulty in collecting data for TLS twins.
For TLQS twins, each observation needs to be tagged to indicate which
twin components (elements) contribute to the observation. This may be
simply computed from the indices if the different lattices have a
more-or-less exact relationship between them, of may need to be assigned
more carefully if the twin obliquity causes only partial overlapping of
some reflections. For doublet spots, it is important that either the
whole doublet is integrated (tag '12'), or the principal component is
separated out (tag '1').






      b.



There may be serious difficulties in determining the space
group. Trial and error may be the only procedure available.






      c.



 The space group used for data reduction  (section \ref{DATAREDUC})
 and merging may not be
that of the major component. A Space group showing the symmetry of the
twinned diffraction data should be used initially. The correct space group
should be used once data reduction is complete.





\bigskip\Instruction{Twinning - Structure solution}



      In general, structure solution is the major difficulty in working
with twinned crystals.



      a.



For TLQS structures, if a substantial number of reflections are
from the major component only, the structure may solve by traditional
methods.






      b.



For Class I TLS structures, structure solution is usually
straight forward, the components of the twin differing only by the
effects of anomalous scattering. Such twins (merohedral twins, or
twinning by inversion) can be processed without further reference to this
part of the manual. All that needs to be refined is the Flack
enantiopole parameter. See the main chapter on refinement.






      c.



For Class II TLS structures, if the twin ratio is far from 50:50, the
structure may solve by traditional methods.





\bigskip\Instruction{Twinning - Structure Refinement}



      If the space group, trial structure, twin law and reflection
components are known, this is straight forward. The sum of the twin
fractions must be 1.0






\bigskip\Instruction{Twin Data stored by CRYSTALS}





      For a twinned crystal the following equation holds.
\small\begin{verbatim}
      Fsq(obs) = v1.Fsq(1) + v2.Fsq(2) ....
\end{verbatim}\normalsize


and similarly for F(calc). The v(i) are the volume fractions of the
components contributing to the observation. A Fourier synthesis using /Fobs/ as
coefficient is meaningless, since the phase alpha(calc) will belong to
only one of the components. The terms needed for Fourier and other
calculations are Fcalc(1), alpha(1) Fobs.vol-fract(1), i.e. only that
 contribution to Fo due to the principal element.




For a twin with two components, each observation may contain a
contribution from each component, or from both. The reflections have to
be 'tagged' to indicate which components are contributing, the ELEMENT
coefficient in LIST 6  (section \ref{LIST06})




For a TLS twin, every observation contains a contribution from both
components (though if it is a systematic absence for one component, the
contribution will be zero). Since the tagging is the same for every
reflection, it can be inserted automatically by CRYSTALS




For a TLQS twin, some observations will contain a contribution from the
principal component, and some from both components, giving ELEMENT tags
of '1' and '12' respectively. If additional observations have been made
based on the reciprocal lattice of component 2, and are indexed with
respect to lattice 2, they are given the tag '2'. If any of these
also contain a contribution from component 1, the tag will be '21'.




Example 1. An orthorhombic space group with a~b, twinned by interchange
of 'a' and 'b'. If 'a' is very similar to 'b', every observation 'hkl' will
overlap with  twin component 'khl', and the ELEMENT tag will be '12',
the default. If a systematic absence from element 1 falls on element 2,
the reflection should not be eliminated during data reduction, and will
have the tag '12', even though the contrinbution from 1 is zero.




Example 2. A monoclinic crystal with 2cCos(beta)/a about 1/3. Twinning
by a 2 fold rotation about 'a' gives a twin law
\small\begin{verbatim}
        1   0   0
        0  -1   0
      -1/3  0  -1
\end{verbatim}\normalsize


Overlap of reflections from both components will only occur when 'h' =
3n, giving the ELEMENT tag '12'. If the lattice is only sampled at r.l.
points corresponding to the principal indexing, reflections with 'h' 


3n will have the tag '1'.





\bigskip\Instruction{Twinning - LISTS affected}


\small\begin{verbatim}
LIST 5  - Parameters: the number of twin elements and their values must be set.
LIST 6  - Reflections: the observed twinned data must be stored as /FOT/, and the
          twin element tags be set.
LIST 12 - Constraint matrix: the twin elements must be refined, and possibly constrained.
LIST 13 - Experimental info: the key CRYSTAL TWIN=YES must be set.
LIST 16 - Restraints: the twin elements may be restrained.
LIST 25 - This contains the twin laws themselves.
\end{verbatim}\normalsize





\bigskip\Instruction{Twin List 5}


The number of twin elements and their values must be given. Currently,
the number of elements and their starting values cannot be input in
$\backslash$EDIT (though values can be changed later). Punch LIST 5, edit it, and
re-input it, or use the SCRIPT EDLIST5.

\small\begin{verbatim}
\LIST 5
READ NATOM=  NELEMENT=
ELEMENT value(1) value(2) ...
ATOM ..........
......
END
\end{verbatim}\normalsize





\bigskip\Instruction{Twin List 6}





For TLQS twins, the element tags (section \ref{LIST06})
really depend upon exact experimental conditions, and should be computed
by the data collection software. If a reflection is entered without a
twin element tag (eg a SHELX HKL 4 file),
CRYSTALS tries to compute the tag from the twin laws
as follows:
\small\begin{verbatim}
      h      the index with respect to LIST 1 (cell) and LIST 2 (space group)
             (this is the index in LIST 6)
      T      The twin law matrix.
      n      the nominal index for the twinned reflection.
             n = T.h
      d      the difference between an exact lattice point and the
             generated point.
             n-nint(n)
      s      The length**2 of the difference vector, in A-2.

\end{verbatim}\normalsize




If 's' is less than the TWINTOLERANCE given on the LIST 6 MATRIX
directive, the twinned reflection is regarded as falling upon a primary
element reflection, and the element tag is updated to indicate this.
This method is only an approximation, but may help to make otherwise
useless data useable. LIST 13 (section \ref{LIST13}) will be automatically
 updated to indicate that twinned data are being refined.





a)Analysis was started as untwinned, and the user wishes to convert to
a twinned refinement


The twin laws must be entered and CRYSTALS instructed to convert the
reflection list to a twinned list.
\small\begin{verbatim}
      \LIST 25
      READ NELEMENT=2
      MATRIX 1 0 0  0 1 0  0 0 1
      MATRIX 0 1 0  1 0 0  0 0 1
      END
      \LIST 6
      READ TYPE=TWIN
      MATRIX TWINTOL=.001
      END
      
\end{verbatim}\normalsize





b)Crystal identified as twinned, and data reduction, sorting and 
merging done outside of CRYSTALS





If the reflection data has been preprocessed so that it is a full,
unique, set for the corret space group, then the correct space group
should be entered, and the reflections input as FOT directly. This tells
CRYSTALS that the data is twinned. 
\small\begin{verbatim}
      \LIST 25
      READ NELEMENT=2
      MATRIX 1 0 0  0 1 0  0 0 1
      MATRIX 0 1 0  1 0 0  0 0 1
      END
      \OPEN HKLI TWINREF.HKL
      \LIST 6
      READ   F'S=FSQ  NCOEF = 5  TYPE = FIXED CHECK = NO
      INPUT H K L /FOT/ SIGMA(/FO/)
      FORMAT (3F4.0,2F8.0)
      STORE NCOEF=9
      OUTPUT   INDICES /FO/ SIGMA(/FO/) /FOT/ /FC/ SQRTW ELEMENT
      CONTINUE RATIO/JCODE CORRECT
      MATRIX TWINTOL=.001
      END
\end{verbatim}\normalsize







c)Data reduction, sorting and merging to be done in CRYSTALS






During initial data reduction  (section \ref{DATAREDUC}) the crystal must 
be given as {\bf untwinned}
in LIST 13 (section \ref{LIST13}), and the 'space group' should be that of the
Laue Class of the intensity data, so that the symmetry of the data is
preserved. In  general, systematic absences should be preserved, unless
centring of the cell matches for all twin components. Twin elelemt tags
may be provided by an external program, or computed by CRYSTALS.



If there are special ELEMENT tags, use something like the following:

\small\begin{verbatim}
\OPEN HKLI twin.hkl
\LIST 6
READ   F'S=FSQ  NCOEF = 6  TYPE = FIXED CHECK = NO
INPUT H K L /FO/ SIGMA(/FO/) ELEMENTS
FORMAT (3F4.0, 2F8.0, F3.0)
STORE NCOEF=7
OUTPUT INDICES /FO/ SIGMA(/FO/) ELEMENTS RATIO/JCODE CORRECTIONS SERIAL
END
\end{verbatim}\normalsize


After initial processing, LIST 13 (section \ref{LIST13}) should be changed 
to twinned, the
correct space group entered, and the value of the observed structure
factor stored as FOT, the Total or Twinned structure factor. This is
done by a special call to the LIST 6 instruction (which also sets the
TWIN flag in LIST 13).
\small\begin{verbatim}
\LIST 6
READ TYPE=TWIN
MATRIX TWINTOL=.001
END
\end{verbatim}\normalsize









\bigskip\Instruction{TWIN LIST 13}



The keyword TWINNED must be set to YES for structure factor
calculations. Because different components of a twin will probably have
different extinction corrections, refinement of extinction is deprocated
for twins. CRYSTALS prints a warning, then lets you continue at your own 
risk.
The special use on the LIST 6 command (above) will update
LIST 13 automatically.

\small\begin{verbatim}
\LIST 13
....
CRYSTALS FRIEDEL=NO  TWIN=YES  EXTINCTION=NO
\end{verbatim}\normalsize





\bigskip\Instruction{Twin List 12}



If all the element scale factors are refined simultaneously with the
overall scale factor, the calculation will be singular. In general, the
sum of the element scale factors is held at unity. For only two twin
componenets, this can be done in LIST 12 as a constraint. For more, it
can be done in LIST 16 as a restraint. The sum of the elements in input
to LIST 5 should be unity.
\small\begin{verbatim}
\LIST 12
FULL ........
EQUIVALENCE ELEMENT(1) ELEMENT(2)
WEIGHT -1 ELEMENT(2)
END
\end{verbatim}\normalsize





For a twin with more that 2 components (for example, twinning by some rotation/reflection
plus enantiomeric (Flack) twinning), an alternative LIST 12 construct should be used.
\small\begin{verbatim}
\end{verbatim}\normalsize

LIST     12                                                                    
BLOCK SCALE X'S, U'S 
SUMFIX ELEMENT SCALES 
END





The Flack parameter is just a short-cut to twinning by inversion. If you have additional twinning, then you have to do it the hard way by including all the twin laws in LIST 25, setting the appropriate ELEMENT flags in LIST 6, and appropriate instructions in LIST 12



If for example the macroscopic twin law is a reflection in P41, eg (0 1 0,  1 0 0, 0 0 1), then LIST 25 would be
\small\begin{verbatim}
1 0 0  0 1 0  0 0 1  (the unit operator)
1 0 0  0 1 0  0 0 -1 (inversion)
0 1 0  1 0 0  0 0 1  (reflection)
0 1 0  1 0 0  0 0 -1 (inversion and reflection)
\end{verbatim}\normalsize





and every reflection would, have the ELEMENT key 1234.



\bigskip\Instruction{Twin List 16}



The sum of the element scale factors can be restrained to unity in LIST
16. In this case, they must all be freely refined in LIST 12.
\small\begin{verbatim}
\LIST12
FULL ........
CONTINUE ELEMENT SCALES
END
\LIST 16
SUM .0001 ELEMENT SCALES
END
\end{verbatim}\normalsize


\section{SORTING TWINNED STRUCTURE DATA  -  $\backslash$REORDER}



For a twinned structure, after the data have been merged,
 it is advisable to re-sort the reflections, placing observations
 that contain contributions from elements with the same indices
 adjacent in the new LIST 6.


\bigskip\Instruction{$\backslash$REORDER}




This directive initiates the re-sorting of reflections for
 a twinned structure. It is IMPERATIVE that the previous command
 has put the reflections on the disc. This is automatic if input
 is via a $\backslash$LIST 6 command (section \ref{LIST06}) or you can use 
 the $\backslash$LIST 6
 READ TYPE=TWIN command.


\bigskip\Directive{STORE MEDIUM=}
 This directive determines the output medium of the new LIST 6.


\bigskip\Keyword{MEDIUM}
 This parameter selects the output medium of the new LIST 6.
 The allowed values for this parameter are :
\small\begin{verbatim}
      M/T
      DISC   -  DEFAULT VALUE.
\end{verbatim}\normalsize


 The default output medium is usually to disk.
\small\begin{verbatim}
/REORDER
END
\end{verbatim}\normalsize




\bigskip\Instruction{Twins - backward compatability}


\small\begin{verbatim}
Note that the key /FOT/ can be given in the initial data reduction if the
crystal is also marked as twinned in LIST 13 (section \end{verbatim}\normalsize\ref{LIST13}), and 
the observed intensity
input as /FOT/. This is preserved for backwards compatibility.





\bigskip\Instruction{Twins - Worked Example}


      The data were provided by Simon Parsons, for a TLQS twin, where
the bulk of the data is from only one component. For reciprocal lattice
layers with h=3n, there is overlap from the second twin component. The
'elelent keys' are thus '12' for reflections with h=3n, otherwise '1'.




Sections of reflection file 'example.hkl'
\small\begin{verbatim}
  -6   0   0    2.16    1.08  12
  -6   0  -1   -0.47    0.93  12
  -6   0  -2   24.98    1.63  12
......
  -6  -2   0    1.64    0.95  12
  -6  -2  -1    8.40    1.06  12
  -6  -2  -2    3.33    1.18  12
  -5   5   1   10.61    1.22   1
  -5   5   2    0.75    0.96   1
........
  -4   0   3   -0.45    0.63   1
  -4   0   4    4.73    0.82   1
  -4   0   5   -0.78    0.71   1
  -4   0   6   48.40    1.69   1
  -4   0   7    0.12    0.68   1
  -4   0   8   -0.35    0.83   1
  -3  -7   0    7.68    1.24  12
  -3  -7  -1   13.11    1.45  12
  -3  -7  -2   13.89    1.36  12
.......
\end{verbatim}\normalsize


The data can be processed in the true space group. LIST 6 (reflection) 
input includes
the 'element keys'. After data reduction, the data is stored as
'TWINNED' by the call to LIST 6 which saves the data in the .DSC file. 
\small\begin{verbatim}
\  Input the cell parameters
\LIST 1
REAL 7.2847 9.74 15.231 90 94.386 90
END

\  Input the space group
\SPACEGROUP
SYMBOL p 21/n
END

\  Input the experimental data
\list 13
crystal  friedel = no twinned=no
cond wave=1.5418
end

\  Input the twin laws, including the identity matrix
\  which corresponds to the first component of the
\  twin, i.e. the one it was indexed on.
\list 25
read nele=2
matrix 1 0 0 0 1 0 0 0 1
matrix 1 0 0 0 -1 0 -.33333 0 -1
end

\  Input scattering factors (list 3) and cell contents 
\  (list 29) using the composition command:
\COMPOSITION
CONTENTS c  48  h  44  s  4  o  4  n  4
SCATT  CRSCP:SCATT
PROPER CRSCP:PROPERTIES
END

\  Specify how the SFLS calculations should be done:
\LIST 23
MINIMISE  F-SQ=no
modify  anomalous=yes
END

\  Input a whole model: scale parameter, twin element scales and
\  the atom parameters.
\list 5
read natom = 5 nelem=2
overal scale=.2
elem .5 .5
atom s    1         1.0000    0.0398    0.9390    0.3740    0.3888
atom n    2         1.0000    0.0617    0.6708    0.1939    0.3428
atom o    3         1.0000    0.0460    0.6967    0.4265    0.5265
atom c    4         1.0000    0.0416    0.9097    0.0426    0.2936
atom c    5         1.0000    0.0317    0.7467    0.2938    0.3989
end

\  Open a file on the device called 'HKLI'
\CLOSE HKLI
\OPEN HKLI example.hkl

\  Read data from that device into LIST 6 in the 
\  specified format and leave space for the specified
\  keys.
\list 6
READ   F'S=FSQ  NCOEF = 6  TYPE = FIXED CHECK = NO
INPUT H K L /FO/ SIGMA(/FO/) ELEMENT
FORMAT (3F4.0, 2F8.0,f4.0)
STORE NCOEF=7
OUTPUT INDICES /FO/ SIGMA(/FO/) RATIO/JCODE CORRECTIONS SERIAL ELEMENT
END

\  Remove systematic absences and move hkl indices by symmetry so that
\  they fall into a unique volume of reciprocal space:
\SYST
\  Sort the reflections:
\SORT
\  Merge adjacent reflections with the same indices:
\MERGE
END

\  Store the reflections and at the same time, guess the element
\  key using the twin laws in L25 to predict if overlap is likely.
\List 6
read type=twin
end

\  Compute the scale factor
\SFLS
SCALE
END

\  Set up the matrix of constraint (aka the refinement
\  directives):
\LIST 12
FULL FIRST(X'S, U[ISO]) UNTIL C(15)
equivalence element(1) element(2)
weight -1 element(2)
END

\  Carry out one cycle of least squares refinement:
\SFLS
REF
END
\end{verbatim}\normalsize





\bigskip\Instruction{Twinning - Mathematical aspects}




 In a twinned crystal, two or more separate components or  ELEMENTS
 contribute to the diffraction pattern,
 and the observed intensities may
 contain contributions from any one of the possible twin component
 In addition, the amount of each twin component
 present in a specified unit of volume is not restricted, and in
 general will vary between different samples of the same
 material.



 The expression for an observed intensity in such a case
 is given by :
\small\begin{verbatim}
 It = v1*I1 + v2*I2 + . . + vn*In
\end{verbatim}\normalsize


 Where  \emph{It}  is the total observed intensity to which \emph{N}
 components contribute,  \emph{Ii}  is the intensity of component  \emph{i} ,
 and  \emph{vi}  is the amount of component  \emph{i}  present in a given
 volume.
 The  \emph{vi}  are known as the 'component scale factors', and
 are conventionally taken to be
 the amount of the given component present in a unit volume of the
 crystal, so that :
\small\begin{verbatim}
 SUM(vi) = 1      over all the components.
\end{verbatim}\normalsize


 When a set of reflection data is handled for a twinned crystal,
 it is thus necessary to know which of the possible components
 contribute to the current reflection, and
 to be able to generate the indices
 of each of the components from a set
 of indices given in a standard reference system.
 If the indices of an component in its own reference system are
 given by the vector
  Hc  and those in the standard system by  H , the necessary
 interconversion is given by :
\small\begin{verbatim}
 Hc = R.H
\end{verbatim}\normalsize


 R  is a rotation matrix that describes the transformation
 of the indices.
 (The generation of the various sets of indices can be thought of
 as a rotation centred on the origin).
 The indices  Hc  are of necessity integers, but the components of
  H  may in general take any value.


The interconversion of atomic coordinates between the various
 reference systems in a twinned crystal can also be expressed
 in terms of  R  :
\small\begin{verbatim}
 Hc[T].Xc = H[T].X      for any component.
\end{verbatim}\normalsize


 Where  X  is the coordinate vector for any atom in the standard
 reference system,  Xc  is the coordinate vector for the
 same atom in the reference system for one of the components and
  H[T]  indicates  H  transposed.
 The above expression may be rewritten as :
\small\begin{verbatim}
 H[T].R[T].Q.X = H[T].X
\end{verbatim}\normalsize


 Where  Q  is the matrix that converts the atomic coordinates.
 Therefore :
\small\begin{verbatim}
 R[T].Q = I
\end{verbatim}\normalsize


 Where  I  is the unit matrix.
 The matrix  Q  is thus given by:
\small\begin{verbatim}
 Q = R[TI]
\end{verbatim}\normalsize


 Where  R[TI]  indicates  R  transposed and inverted.
 The coordinates therefore transform as :
\small\begin{verbatim}
 Xc = R[TI].X
\end{verbatim}\normalsize




Before any reflections can be processed, the matrices  R  must
 be provided.
 These are given in LIST 25, which must contain one matrix for
 each possible component.
 (If the standard system is chosen as that of component 1, for example,
 the first  R  matrix will be the unit matrix, which must be given
 as it is not assumed).







\chapter{Matrix Calculations}
\label{matrixcalc} \index{Matrix Calculator}\section{The Basic Matrix Calculator $\backslash$MATRIX}


\small\begin{verbatim}
 \MATRIX 
 A      
 B      
 MM     
 TM     
 MT     
 TT     
 TRANS  
 INV    
 EIG    
 ACC2A  
 ACC2B  
 EXECUTE
 END

 \MATRIX
 A 1 2 3 4 5 6 7 8 9
 B 4 5 6 7 8 9 10 11 12
 MM
 END

\end{verbatim}\normalsize




CRYSTALS contains a simple calculator for processing 3x3 matrices. Two 
matrices, A and B, can be input and operated on. The result is output to
the screen and left in an accumulator.  It can be transfered to A or B 
for further operations.  There is currently no interface to any stored 
crystallographic information.






\bigskip\Instruction{$\backslash$MATRIX}










\bigskip\Directive{A}


Followed by the nine values for the matrix A by rows





\bigskip\Directive{B}


Followed by the nine values for the matrix B by rows





\bigskip\Directive{MM}


Accumulator = AxB





\bigskip\Directive{TM}


Accumulator = A'xB





\bigskip\Directive{MT}


Accumulator = AxB'





\bigskip\Directive{TT}


Accumulator = A'xB'





\bigskip\Directive{TRANS}


Accumulator = A'





\bigskip\Directive{INV}


Accumulator = A-1





\bigskip\Directive{EIG}


Accumulator = Eigenvectors of A





\bigskip\Directive{ACC2A}


Matrix A = Accumulator





\bigskip\Directive{ACC2B}


Matrix B = Accumulator





\bigskip\Directive{EXECUTE}


Forces execution of the last directive








\chapter{Obsolete Commands}
\index{Obsolete commands}\section{Obsolete Commands}



The following Commands were available in earlier versions of CRYSTALS. 
 They are retained for compatibility reasons, but have been suppressed 
 or superceeded  by new commands.



\small\begin{verbatim}

 Least squares best planes                      MOLAX
 Thermal displacement parameter analysis        ANISO
 Principal atomic displacement directions       AXES
 Structure factors for a group of trial models  TRIAL

\end{verbatim}\normalsize


\index{MOLAX}
\index{Best lines and planes}
\section{Least squares best planes and lines  -  $\backslash$MOLAX}


\small\begin{verbatim}
 \MOLAX INPUTLIST=
 EXECUTE
 PUNCH
 ATOMS  W(1)  SPECIFICATION(1)  W(2)  SPECIFICATION(2) .
 PLOT
 PLANE
 LINE
 ANGLE  NP(1)  AND  NP(2)
 EVALUATE  ATOM SPECIFICATIONS . . . .
 REPLACE ATOM SPECIFICATIONS . . .
 SAVE
 QUIT
 END
\end{verbatim}\normalsize



\small\begin{verbatim}
 \MOLAX
 ATOM FIRST UNTIL LAST
 PLANE
 SAVE
 END
\end{verbatim}\normalsize




MOLAX is used for computing the principal axes of inertia
 through groups of atoms using  the routines described in
 Computing Methods in Crystallography, edited by J. S. Rollett,
 Pergamon Press, 1965, p67-68.
 It can be used to compute best lines and planes, and produce simple line
 printer plots of the atoms.


The  best  plane for a series of  N
 atoms whose positions have varying reliability, such that they can
 be assigned weights, w(1), w(2), . . . w(n), is defined as
 that for which
 the sum of the squares of the distances (in angstroms) of the
 atoms from the plane, multiplied by the weights, w(i), of
 the atomic positions, is a minimum.
 Note that the normal to the 'worst plane' is the 'best line',
 and if masses are used for weights, then the calculation gives the
 principal inertial axes.


The atomic positions are taken from LIST 5, possibly modified by
 symmetry information, to compute
 inertial axes, deviations of atoms from  the planes or lines, and the
 angles between normals to these planes or axes. Shape indices (Mingos M.P.
 and Rohl A.L. J Chem Soc Dalton Trans (1991) 3419) are computed.


Each time a line or plane
 is computed, the direction cosines of the relevent axis are stored as AXIS
 number 'n'. The angles between these axes can be computed.
 Three geometry indices are also
 computed. The geometry is best described by the index closest to unity.
 (Mingos,D.P.M \& Rohl,A.L., J.Chwm.Soc. Dalton Trans (1991) pp 3419 - 3425)


Immediate execution of a directive can be forced by issuing an EXECUTE
 directive.





\bigskip\Instruction{$\backslash$MOLAX INPUTLIST=}




\bigskip\Keyword{INPUTLIST=}

\small\begin{verbatim}
      5   -  Default value
      10
\end{verbatim}\normalsize







\bigskip\Directive{EXECUTE}


This forces the execution of preceding directives.





\bigskip\Directive{PUNCH}


This directive causes the orthogonal coordinates of the atoms of any plane or
 line computed in following tasks to be output to the 'punch' file.





\bigskip\Directive{ATOMS  W(1)  SPECIFICATION(1)  W(2)  SPECIFICATION(2) .}


This specifies atoms to be used in the calculation of the
 best plane.
  W(1)  Is the weight assigned to the atoms
 contained in the first atom specification,  W(2)  is the weight
 assigned to the second group of atoms, and so on.
 If  W(1)  is omitted, a default value of 1 is used,
 but any other  W(I)  term applies to all the atoms following it,
 until another  W  is found or the end of the directive is
 encountered.
 At least one  ATOM  directive must precede each  PLANE  or PLOT directive.
 An ATOM directive will over-rule an immediately preceding ATOM directive. If an
 input line is not long enough for the full atom list, use CONTINUE.





\bigskip\Directive{PUNCH}


This directive causes the orthogonal coordinates of the atoms of any plane or
 line computed or EVALUATED in the current task to be output to the 'punch'
 file.





\bigskip\Directive{PLOT}


This directive, (or  PLANE or LINE)
 must follow immediately after an  ATOM  directive and
 causes the calculation of inertial axes.
 Details of the computation are suppressed on the Monitor,
 but a line drawing projected onto the best plane is produced.
  MOLAX  Can thus be used as a means of displaying some or all
 of the atoms in a structure.





\bigskip\Directive{PLANE}


This directive, (or  LINE or PLOT)
 must follow immediately after an  ATOM  directive and
 causes the calculation of a least squares best plane.





\bigskip\Directive{LINE}


This directive, (or  PLANE or PLOT)
 must follow immediately after an  ATOM  directive and
 causes the calculation of a least squares best line.





\bigskip\Directive{ANGLE  NP(1)  AND  NP(2)}


If present, thus directive must follow at least
 two  ATOMS/PLANE (ATOMS/LINE, ATOMS/PLOT) directive sequences.
 It causes the program
 to calculate the angle between the axes with  serial numbers
   NP(1)  and  NP(2) .
 The   AND  must be present.





\bigskip\Directive{EVALUATE  ATOM SPECIFICATIONS . . . .}


If present, this directive
 must appear after a  PLANE, LINE or PLOT directive,
 and causes the co-ordinates of the atoms specified
 to be calculated and printed with respect to the least squares axial system.





\bigskip\Directive{REPLACE ATOM SPECIFICATIONS . . .}


if present, this directive
 must appear after a  PLANE, LINE or PLOT directive,
 and causes the co-ordinates of the atoms specified to be modified so that
 they lie on the previously defined plane. The LIST 5 in core is immediately
 updated, so that the new coordinates will be used for any subsequent
 computation. A LIST 5 is only written to the disc on a satisfactory exit from
 MOLAX.





\bigskip\Directive{SAVE}


This directive causes the latest plane defining matrix and
 vector to be stored in LIST 20. A LIST 20 is only written to the disc on
 a satisfactory exit from MOLAX.





\bigskip\Directive{QUIT}


This directive abandons the calculation without modifying the disc LISTs.
\small\begin{verbatim}
 \
 \ these instructions define a plane
 \ involving n(1),n(2),c(1),c(2) and n(3) and
 \ prints the co-ordinates of all the atoms with
 \ respect to this plane.  the positions of the
 \ nitrogen atoms have double weight
 \
 \MOLAX
 ATOMS 2 N(1) UNTIL N(3)  1 C(1) C(2)
 PLANE
 EVALUATE ALL
 \
 \ this set of directives also calculates another plane,
 \ printing only the co-ordinates of c(5) with respect to
 \ the second plane.  the angle between the two planes
 \ is then calculated
 \
 ATOMS C(1) S(1) N(1)
 PLANE
 EVALUATE C(5)
 ANGLE 1 AND 2
 END
\end{verbatim}\normalsize





\index{ANISO}
\index{Analysis of thermal displacement parameters}
\section{Thermal displacement parameter analysis  -  $\backslash$ANISO}


\small\begin{verbatim}
 \ANISO INPUTLIST
 EXECUTE
 ATOMS   WEIGHT ATOM SPECIFICATIONS
 CENTRE   X=, Y=, Z=
 REJECT   NV=
 LIMITS   VALUE=   RATIO=
 TLS
 EVALUATE ATOM SPECIFICATIONS
 REPLACE ATOM SPECIFICATIONS . . .
 SAVE
 QUIT
 AXES
 DISTANCES  DL=   AL=
 ANGLES  AL=
 END
\end{verbatim}\normalsize



\small\begin{verbatim}
 \ANISO
 ATOM C(1) UNTIL C(6)
 TLS
 SAVE
 END
\end{verbatim}\normalsize




This routine calculates the overall rigid-body
 motion tensors T, L, S (Shoemaker and Trueblood, Acta Cryst. B24,
 63, 1968) by a least-squares fit to the individual anisotropic
 temperature factor components, together with librational corrections
 to bond lengths and angles.


Shoemaker and Trueblood's conventions and reductions
 are followed throughout; in particular, the trace of S, which is
 indeterminant, is set to zero. The program therefore determines
 20 overall tensor components - the upper triangles of T and L
 together with the whole of S apart from S(33).


Even when the trace-of-S singularity has been removed, however,
 the nature of the rigid body problem is such that ill-conditioned
 and singular normal matrices are much more common than in
 structure refinement and the program therefore proceeds via
 the eigenvalues and eigenvectors of the normal matrix.  In most
 cases the largest and smallest eigenvalues are output for
 inspection, but if the ratio of these quantities is less than
 the LIMITing RATIO, a full eigenvalue/vector listing is produced. Further,
 if any eigenvalue is itself less than the LIMITing VALUE, the corresponding
 parameter combination is set to zero, thus removing the near-
 singularity. These actions can be modified by the use of the
   LIMIT  and  REJECT  directives described below.





\bigskip\Instruction{$\backslash$ANISO INPUTLIST}




\bigskip\Keyword{INPUTLIST}

\small\begin{verbatim}
      5   -  Default value
      10
\end{verbatim}\normalsize







\bigskip\Directive{EXECUTE}


This causes immediate execution of the previous directive, otherwise
 directives are executed on input of a new directive (or END).





\bigskip\Directive{ATOMS   WEIGHT ATOM SPECIFICATIONS}


This parameter specifies the set of atoms to be used for the following
 calculation.


WEIGHT. The default weight of 1.0 is used for all atoms except those
 following a WEIGHT value. Any decimal number on the ATOM directive 
 is taken as a weight and applied to any following atoms. 
 A subsequent atom directive over rules all previous atom directives.
 If the full atom specification cannot be got on one directive, use CONTINUE.
 The atom specifications are in the usual form with symmetry
 operators and  UNTIL  sequences permitted. An ATOM directive resets the CENTRE to
 its default value, 0,0,0.





\bigskip\Directive{CENTRE   X=, Y=, Z=}


This directive specifies the centre of libration,
 in crystal fractions, to be used in the original derivation of
 the overall motion tensors. The program derives and uses a unique
 origin at a later stage in the calculations. This directive
 is optional, the default centre being (0,0,0).
 If a centre of (0,0,0) is given or set by default, the program computes
 and uses the mean position of the given atoms, INCLUDING any which are
 isotropic, even though these are not used to compute TLS. The stored CENTRE
 is updated during TLS, and a second TLS computation may be performed using
 this new value as CENTRE. This may help stabilise certain forms of
 ill-conditioning.





\bigskip\Directive{REJECT   NV=}


Overrides normal action and sets the parameter combination
 corresponding to eigenvector number nv to zero.
 Eigenvectors are numbered in ascending order of their eigenvalues,
 so that nv
 is in the range 1 to 20 inclusive and will usually have been obtained
 from a full eigenvalue/vector listing produced in
 a previous run.





\bigskip\Directive{LIMITS   VALUE=   RATIO= }


If an eigenvalue is less than  VALUE  or its size is less than
  RATIO * (the next bigger), it is eliminated from the analysis.
 VALUE is currently .000001 and RATIO .01 .





\bigskip\Directive{TLS }


This causes the TLS calculation to be initiated. It MUST have been preceded
 by an ATOM directive.





\bigskip\Directive{EVALUATE ATOM SPECIFICATIONS }


This may be used after a successfull TLS calculation to list Ucalcs for
 the specified atoms. The atom list is not modified.





\bigskip\Directive{REPLACE ATOM SPECIFICATIONS . . .}


If present, his directive
 must appear after a TLS directive,
 and causes the co-ordinates of the atoms specified to be modified so that
 they have U's defined by the current T, L, and S matrices.
 The LIST 5 in core is immediately
 updated, so that the new coordinates will be used for any subsequent
 computation if a new ATOM directive is issued.
 The updated LIST 5 is only written to the disc on a satisfactory exit from
 ANISO.





\bigskip\Directive{SAVE}


This directive is optional. If it follows a TLS directive, it
 causes the latest L matrix and CENTRE  to be stored in LIST 20. If it
 follows an AXES directive, the direction cosines and centre if the ellipse FOR
 THE LAST ATOM are stored in LIST 20.
 A LIST 20 is only written to the disc on
 a satisfactory exit from ANISO.





\bigskip\Directive{QUIT}


This directive abandons the calculation without modifying the disc LISTs.





\bigskip\Directive{AXES}


This directive (like $\backslash$AXES) computes the principal axis lengths
 and directions for the atoms specified on a preceding ATOM directive.





\bigskip\Directive{DISTANCES  DL=   AL=}


This directive calculates all interatomic distances less than
 DL angstroms with librational corrections. If this directive is omitted,
 no distances are calculated; if DL is absent, a default value of 1.8 is
 inserted. If AL is present, angles between atoms separated by less than AL
 angstroms are computed.





\bigskip\Directive{ANGLES  AL=}


This directive calculates angles between all bonds less than
 AL angstroms. If this directive is omitted, no angles are calculated;
 if AL is absent, a default value of 1.8 is inserted.



 \emph{***********************} WARNING *************************





The directive DISTANCE may only be
 followed by ATOM, EXECUTE, or END.

\small\begin{verbatim}
 \ANISO
 ATOMS O(12) UNTIL LAST
 AXES
 TLS
 DISTANCES
 END
\end{verbatim}\normalsize








\index{AXES}
\index{Principal atomic displacement directions}
\section{Principal atomic displacement directions  -  $\backslash$AXES}


\small\begin{verbatim}
 \AXES INPUTLIST=
 END

\AXES
 END
\end{verbatim}\normalsize




This routine calculates the magnitudes and directions of the principal axes
 of the atomic dispacement ellipsoid of an anisotropic atom.
 Atoms which are isotropic are ignored.
 Atoms with a negative principal axis generate a warning.
 The output gives the mean square displacement in angstroms squared along each
 of the principal axes, together with the direction cosines with respect
 to the orthogonalized axes and with respect to the
 real cell axes.


This routine can also be called from $\backslash$ANISO to get the axes of specified
 atoms only.





\bigskip\Instruction{$\backslash$AXES INPUTLIST=}




This command initiates the routine for calculating the
 principal atomic vibration directions, and requires no other directives.


\bigskip\Keyword{INPUTLIST=}

\small\begin{verbatim}
      5   -  Default value
      10
\end{verbatim}\normalsize


 The default value is 5.






\section{Structure factors for a group of trial models  -  $\backslash$TRIAL}



 This procedure is currently unsupported. It is kept in the code 
 because it offers an opportunity for a new programmer to experiment with
 improved 'COST' functions.


At some stage during a structure determination, the orientation
 of a group of atoms may be known, but not their position in the
 unit cell.
 The routine described in this section provides a rapid method of
 calculating structure factors for a group of atoms at a series of
 points that fall on a grid in the unit cell.
 The algorithm used is similar to that employed in the slant fourier,
 (see the section of the user guide on 'Fourier routines') and is as
 follows :


The  A  part of the structure factor for the reflection with indices
 given by the vector  H  may be written as :
\small\begin{verbatim}
       A(H) = SUM[ G.SUM[ COS2PI(H'.S.X + H'.T) ] ]
\end{verbatim}\normalsize


 With a similar expression for the  B  part.
 ( G  Is the required form factor, modified by the temperature factor
 expression).
 Conventionally, the inner sum runs over the various symmetry operators
 that define the space group, and the outer sum runs over the number of
 atoms in the asymmetric unit.
 However, if the summation order is changed, it is possible to
 accumulate sums for all the atoms for each symmetry position :
\small\begin{verbatim}
       P(H,S) = SUM[ G.COS2PI(H'.S.X + H'.T) ]
\end{verbatim}\normalsize


 With a similar expression for  Q(H,S)  for the  B  part.
 It is now possible to use a recurrence relationship for  P
 and  Q  to give :

\small\begin{verbatim}
       P(H,S,2) = P(H,S,1)*2*COS2PI(H'.S.DX) - P(H,S,0)
 and
       Q(H,S,2) = Q(H,S,1)*2*COS2PI(H'.S.DX) - Q(H,S,0)
\end{verbatim}\normalsize


 P(H,S,0)  Is the original value of  P  for the symmetry position  S
 for the reflection given by  H .
  P(H,S,1)  Is the corresponding value of  P  after a vector  DX  has
 been added to each set of coordinates, and  P(H,S,2)  is the
 corresponding term after a vector  2*DX  has been added.
 Similar relationships hold for the  Q  terms.
 After the initial eight cosine and sine terms have been calculated,
 it is possible to calculate structure factors very rapidly
 as the group of atoms is moved about the unit cell,
 using the relationships given above.


Apart from an array to hold each section through the unit cell,
 it is necessary to store the eight cosine and sine terms, together
 with the three step vector cosines, for each reflection for each
 symmetry position.
 Because this imposes certain storage limitations, it is necessary
 to restrict the number of reflections that are used.
 In practice it is only the large reflections that must agree, and so
 the user is required to input a minimum Fo value, below which
 reflections are not used.
 The function that is displayed for each grid point is given by :
\small\begin{verbatim}
       SCALE*SUM[ Fo*Fc ]
\end{verbatim}\normalsize


 Accordingly, the largest value printed represents the most likely
 solution.
 The  SCALE  term may be calculated by the program to give numbers in
 a reasonable range, or input by the user.
 The time for each calculation is proportional to the number of
 reflections used, the number of symmetry operators in LIST 2,
 and the number of grid points calculated.
 (A calculation in a non-centro space group takes twice as long as a
 calculation in the corresponding centro space group).
 The atoms to be moved around are taken directly from LIST 5.


\bigskip\Instruction{$\backslash$TRIAL}




This is the command which initiates the routine to calculate
 structure factors for a group of trial models.


\bigskip\Directive{MAP Fo-MIN SCALE MIN-RHO}


This directive determines which reflections will be used in the
 calculations and how the map will be printed.


\bigskip\Keyword{Fo-MIN}
 This parameter is the minimum value of Fo that a reflection must
 have if it is to be used (this number must be on the scale of Fo).
 If this parameter is omitted, a value of zero is assumed.


\bigskip\Keyword{SCALE}
 If  SCALE  is equal to zero, its default value, the program will
 choose a scale factor that places all the numbers on
 a reasonable scale for printing.
 If this parameter is greater than zero, the sum of Fo*Fc
 is multiplied by  SCALE  before it is printed.
 (The scale factor computed by the program is dependent upon
 the origin chosen for the group of atoms, so that
 successive maps with different origins will be on different
 scales, unless this parameter is specified for all the maps
 after the first).


\bigskip\Keyword{MIN-RHO}
 This parameter is a cut-off value, such that all numbers less than
  MIN-RHO  are printed as zero. If this parameter is absent, a
 default value of zero is assumed, which means that all the points
 are printed.


\bigskip\Directive{DISPLACEMENT DELTA-X DELTA-Y DELTA-Z}


This directive defines a vector which is added to each set of
 coordinates in LIST 5 before the structure factor calculation
 starts.  DELTA-X ,  DELTA-Y  And  DELTA-Z  thus correspond
 to an initial origin shift for the group in LIST 5.


\bigskip\Keyword{DELTA-X}
 The shift along the x-direction.


\bigskip\Keyword{DELTA-Y}
 The shift along the y-direction.


\bigskip\Keyword{DELTA-Z}
 The shift along the z-direction.


The default values for these parameters are zero, indicating no
 initial origin shift before the structure factor calculation.


\bigskip\Directive{DOWN NUMBER X-COMPONENT Y-COMPONENT Z-COMPONENT}


This directive specifies the printing down the page.


\bigskip\Keyword{NUMBER}
 The number of points to be printed down the page, for which there is no
 default value.


\bigskip\Keyword{X-COMPONENT Y-COMPONENT Z-COMPONENT}
 There are no default values for these parameters, which specify
  the fractional coordinate shift vector. The vector moves the group so
 that :
\small\begin{verbatim}
      X1 = X0 + X-COMPONENT
      Y1 = Y0 + Y-COMPONENT
      Z1 = Z0 + Z-COMPONENT
\end{verbatim}\normalsize


 Where  1  and  0  define successive points down the page.


\bigskip\Directive{ACROSS NUMBER X-COMPONENT Y-COMPONENT Z-COMPONENT}


These are the corresponding values across the page.


\bigskip\Keyword{NUMBER}
 The number of points to be printed across the page, for which there is
 no default value.


\bigskip\Keyword{X-COMPONENT Y-COMPONENT Z-COMPONENT}
 There are no default values for these parameters, which specify the
 fractional coordinate shift vector.


\bigskip\Directive{THROUGH NUMBER X-COMPONENT Y-COMPONENT Z-COMPONENT}


These are the values that define the change from section to section.


\bigskip\Keyword{NUMBER}
 The number of sections to be printed, for which there is no default value.


\bigskip\Keyword{X-COMPONENT Y-COMPONENT Z-COMPONENT}
 There are no default values for these parameters, which specify the
 fractional coordinate shift vector.


These shift vectors allow any change of position for the group
 to be plotted out.

\small\begin{verbatim}
 \TITLE MOVE 2 SULPHURS AROUND
 \LIST 5
 READ NATOM=2
 ATOM S 1 X=0.00 0.15 0.37
 ATOM S 2 X=0.13 0.05 0.24
 \ call '\trial' with a min. fO of 250
 \TRIAL
 MAP Fo-MIN=250
 \ initial origin shift
 DISPLACEMENT 0 0 -0.3
 \ plot half of y down the page
 DOWN 26 0 0.02 0
 \ plot half of x across the page
 ACROSS 26 0.02 0 0
 \ plot half of z up the page negatively
 THROUGH 51 0 0 -0.01
 \FINISH
\end{verbatim}\normalsize









\chapter{Secret Lists}
\index{Secret LISTS}\section{Secret LISTS}








These are LISTS used by programmers to transfer data from one utility
to another.


The user can sometimes access them via a SCRIPT or the GUI.







\index{Miscelaneous}
\index{List 39}
\section{Storage of intermediate results and the status of a refinement -  LIST 39}
\label{LIST39} 




There are two INTEGER records.
\small\begin{verbatim}
 INFO holds flags indication what steps of the analysis have been completed.
 OVER holds flags indicating which OVERALL parameters are in LIST 12 are 
 being refined, indicated by a non-zero entry.
OVER 1 Scale Du(iso) Ou(iso) Polarity Enantiopole Extinction 
\end{verbatim}\normalsize


There are three REAL records



FLAC(0) holds information about the Absolute Structure
\small\begin{verbatim}
Slope of the npp for the refinement
Slope of the npp for the Bijvoet pairs
Hole-in-one value and su
Difference or Quotient value and su
Hooft y value and su
Histogram value and su
\end{verbatim}\normalsize


FLAC(1) Holds the numbers of reflections used for various calculations:
\small\begin{verbatim}
No of Friedel Pairs found
No of Friedel Pairs after applying filters 1,2 & 5
No of Friedel Pairs used for the normal probability plot
Not used
No of Friedel Pairs used for the Parsons method, applying filters 1,2,3 & 5
Not used
No of Friedel Pairs used for the Parsons method, applying filters 1,2,3 & 5
Not used
No of Friedel Pairs used for the histogram method, applying filters 1,2,3,4 & 5
Not used
\end{verbatim}\normalsize


HOOF holds the Hooft probabilities
\small\begin{verbatim}
Hooft(y)    P2_true P2_false    P3_true P3_twin P3_false

\end{verbatim}\normalsize


SFLS holfd info about the refinement process.
\small\begin{verbatim}
SFLS 0  1or2.  2 indicates CRYSTALS thinks refinement has converged.


\end{verbatim}\normalsize


\chapter{Crystals Manual}


 \small\begin{verbatim}\end{verbatim}\normalsize



\printindex
\end{document}
