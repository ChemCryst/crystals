\documentclass[10pt,a4paper]{report}
\usepackage[centertags]{amsmath}
\usepackage{amsfonts}
\usepackage{amssymb}
\usepackage{amsthm}
\usepackage{makeidx}
\usepackage{newlfont}
\usepackage{fancyhdr}
\usepackage[bookmarks,colorlinks,plainpages,backref]{hyperref}
\pagestyle{fancy}
\hfuzz2pt % Don't bother to report over-full boxes if over-edge is < 2pt
\makeindex
\addtolength{\headwidth}{80pt}
\addtolength{\textwidth}{80pt}
\addtolength{\oddsidemargin}{-30pt}
\begin{document}
\lhead{\slshape \rightmark}
\chead{}
\rhead{\thepage}
\lfoot{Fri Jul  5  2024}
\cfoot{}
\rfoot{}
\renewcommand{\headrulewidth}{0pt}
\renewcommand{\footrulewidth}{0pt}
\fancypagestyle{plain}{%
\fancyhf{}
\fancyfoot[L]{Fri Jul  5  2024}
\fancyfoot[C]{\thepage}
\renewcommand{\headrulewidth}{0pt}
\renewcommand{\footrulewidth}{0pt}}
\newcommand{\Instruction}[1]{{\bf #1}}
\newcommand{\Directive}[1]{{\bf \emph{#1}}}
\newcommand{\Keyword}[1]{\emph{#1}}
\sloppy
\title{Crystals User Guide}
\author{Chemical Crystallography Laboratory, Oxford}
\date{Fri Jul  5  2024}
\maketitle
\tableofcontents
\chapter{The Crystals User Guide}
\small\begin{verbatim}*************************************************************************This document is still under revision*************************************************************************\end{verbatim}\normalsize



The CRYSTALS package is being developed to meet the need of both experienced and new crystallographers. Because it is  a very large system, the range of options available  may daunt newcomers.  For such users,  CRYSTALS is distributed with a worked example, on-line HELP files (which may of course be printed out) and this Guide. 

The new $\backslash$SCRIPT instructions are prototypes for an interogative interface, in which the user is offered short menus of commands, and prompted for data. THESE ARE STILL ONLY PROTOTYPES   to what we hope will be a computer aided teaching system, but even in their present form they may help  beginners get going.

There is a myth that CRYSTALS is more difficult to use than SHELX. We hope the following example will correct this misunderstanding. This example does not, of course, show any of the advanced features of CRYSTALS nor the easy links between CRYSTALS  and direct Methods programs.



\bigskip\Instruction{Emulation of SHELX}



\bigskip\Directive{A Worked example}

CRYSTALS will not emulate SHELX completely - in particular there is no direct methods code, but for most other SHELX operations, there is a good  correspondence plus many other options. The input to CRYSTALS is longer, but the directive names should make it easier to remember what must be input. Once the basic data has been loaded, many operations are very quickly requested.



In these examples, comments have been added to the data file in lower case.



\bigskip\Directive{SHELX example}

This example is based on a SHELX manual example. The space group is Pbcn with four molecules of C13,H8,O,I2 per cell. Systematic absences are rejected, equivalent reflections averaged, and reflections less than 2.5 sigma(F) suppressed. The structure is refined by 4 cycles of full matrix least squares with anisotropic I and weight 'optimisation'. C7 and O1 lie on special positions with crystallographic site occupancy of .5. Since the chemical occupancy is unity, no special action is needed in CRYSTALS. C1, C3 and C5  are used to define a regular hexagon which is refined as a rigid group. Bond lengths, angles and esd's are  computed, followed be a weighted difference map, peak search and molecule assembly. The atoms and peaks are plotted on the line printer, and 2 copies of the structure factors are printed. Site occupation factors are 1 and U[iso] is .05 for the phenyl group.\small\begin{verbatim} TITL 3,3'-DIIODOBENZOPHENONE CELL 0.71069 4.05 11.74 24.80 90 90 90 SYMM .5-X, .5-Y, .5+Z SYMM .5+X, .5-Y, -Z SYMM -X, Y, .5-Z                    scattering factors built into program SFAC C H O SFAC I 20.1472 4.347 18.9949 .3814 7.5138 27.766 2.2735 =     66.8776 4.0712 -.726 1.812 7246 1.5 (MOKALPHA) UNIT 52 32 4 8 HKLF 4                                minimum F/sigma(F) is 2.5 MERGE -2 2.5 L.S. -4 WGHT -.001 2 FVAR .879 1.0              fix the I occupation number, and refine the                           anisotropic temperature factors ANIS 1 I 4 .342 .158 .044 11 .054                            generate a phenyl nucleus and                                      refine it as a group AFIX 76 C1 1 .009 .074 .197                                             a dummy atom C2 1 0 0 0 C3 1 -.128 -.082 .139                                             a dummy atom C4 1 0 0 0 C5 1 .156 .018 .103                                             a dummy atom C6 1 0 0 0                                            refine freely AFIX 0                          fix next two atoms at x=0,z=.25                        with fixed occupation number of .5 C7 1 10 .136 10.25 10.5 .056 O1 3 10 .239 10.25 10.5 .071 BOND .5                           specify details of Fourier map FMAP 2 1 13 50 0.3                         specify volume of map to compute GRID -5 -2 -1 5 2 1                            assemble molecule and plot it PLAN 15 LIST 1 -2 END\end{verbatim}\normalsize



\bigskip\Directive{CRYSTALS example}

The corresponding CRYSTALS input file. The reflections are in the file REFLECTIONS.DAT\small\begin{verbatim} \TITLE 3,3'-DIIODOBENZOPHENONE \ load the basic crystallographic data.  \ This is only done once. \QUICKSTART CELL 4.05 11.74 24.80 SPACEGROUP P b c n CONTENT C 13 H 8 O I 2 DATA 1.5418 I FILE CRDIR:REFLECTIONS.DAT FORMAT (3F4, 2X, 2F8.2) END \ set the minimum I/sigma(I) to be used.  \ This can be reset at any time \LIST 28 MINIMA RATIO=2.5 \LIST 5 OVERALL SCALE=.879 READ NATOM=6 \ Input the known model.  \ Some of the 'keyword =' are redundant ATOM I 1 U[ISO]=.054 X=.342 Y=.158 Z=.044  ATOM O 1 U[ISO]= .071  0 .239 0.25  ATOM C 7 U[ISO]= .056 0. .136 0.25  ATOM C 1 X= .009 .074 .197 ATOM C 3 X= -.128 -.082 .139 ATOM C 5 X= .156 .018 .103 \ generate the phenyl nucleus \REGULARISE REPLACE GROUP 6 OLD C(1) C(2) C(3) C(4) C(5) C(6) PHENYL \LIST 12 FULL I(1,X'S,U'S) C(7,X'S) O(1,Y) O(1,U[ISO]) UNTIL C(6) GROUP C(1) UNTIL C(6) \ generate the special position restraints \SPECIAL END \SFLS REFINE REFINE REFINE \LIST 4 \ automatic re-weighting before the final cycle SCHEME 10 NPARAM=4 \SFLS REFINE \DIST E.S.D YES \ compute final structure factors for the latest model \SFLS CALC \FOURIER MAP TYPE=DIFF REFLECT WEIGHT=SIM PEAKS HEIGHT=.3 NPEAK=50  \PEAKS \ assemble a molecule \COLLECT \ plot all the atoms and peaks onto their best plane \MOLAX ATOM FIRST UNTIL LAST PLOT \ analysis of variance \ANALYSE \ reflection listing \REFLECTIONS \FINISH\end{verbatim}\normalsize



\bigskip\Instruction{Facilities available}





CRYSTALS contains routines for performing most parts of a structure analysis with the exceptions of diffractometer data pre-processing, analytical  absorption corrections and direct methods. Excellent programs for direct methods are available from  groups specialising in those areas, and initial data processing tend to be rather site specific. 

The code has been written to be efficient in the maths and data handling, and robust and informative in the user interaction. There are no intrinsic limits on the numbers of atoms, reflections or refined parameters.  Only one array need be redimensioned to increase the size of matrix handled. The current limit is 512 variables in a single block. However, such a task would probably be ill-considered. CRYSTALS has been used to refine large molecules, such as Insulin.



Plain language keyed free format input, unified command syntax  for batch and interactive modes, parameter defaults, data checking, plain text error messages, 'data subroutine' libraries, data  libaries, data-base management and an optional interogative user  interface.



Extensive documentation, including a manual, this Guide, user modifiable HELP files, interactive querying for instructions  and keywords, examples and frequently asked questions.



Simple syntax for refering to individual atoms, and to groups of atoms, with or without the application of symmetry.



Data reduction of photographic and diffractometer data, including empirical absorption corrections. Application of decay curves, rejection of systematic absences, merging of equivalent reflections with special treatment of outliers, processing of both X-ray and neutron diffraction data.



Structural parameter editor for modifying individual parameters, atoms or groups of atoms. Modifications available include:  addition, subtraction, multiplication, division of parameters; application of matrices to coodinates; renaming, deletion, reordering generation of atoms; application of symmetry operators; selection by type or parameter value; interconversion between U[aniso] and U[equiv]



Structural analysis and model building. Best line and plane through groups of atoms, projection of other atoms onto these planes, forcing planarity onto parts of structure. TLS analysis of groups of atoms, forcing U[aniso] for groups to conform to rigid body motion. Comparison between a group of atoms and another group, a geometric figure or structure from the literature. Model a group of atoms onto another group, figure or structure. Geometric hydrogen and other atom placing.



Reflections to be skipped or included are selected individually or by tests. Testable parameters include Fo, Fc, sigma, I/sigma(I), phase, sin theta, index, zone, etc. The test conditions can be changed at any  time without loss of data.



Restrained refinement. Distances and angles restrained to numeric values or their mean values; components of U[aniso] along bonds; planarity of groups; sums and averages of parameters; shift limiting restraints; and user defined FORTRAN-like expressions given as data.



Highly flexible least-squares process. User can define up to 256 blocks per cycle. Blocks can be as small as one parameter per block, one atom per block, or full matrix, or any intermediate combination of parameters. Any atom parameter shifts (X's, U's, Occ) can be equivalenced, linked,  made to ride. Parameters can be redefined as sums and diferences. Atoms can be refined as rigid groups. Shift multipliers can be applied, the current matrix can be re-used, structure factors need only be calculated for partial structures. Data for up to 8 twin components can be refined. Refinable parameters include overall scale, overall Uiso, Dummy Uiso, Larson extinction, Rogers polarity, Flack enantiopole, batch scales, layer scales, twin components, x, y, z, Occ, Uiso, Uaniso.



15 weighting schemes are available, including statistical, unit, Hughes Cruickshank and Dunitz schemes. Weighting optimisation via a Chebychev polynomial, checking for outliers by the method of Tukey and Prince, can be used to to get 'correct' parameter esds. Analysis of residuals by Fo, sintheta, index, class, parity.



Fourier maps are computed with any number of points at any resolution along x, y, z. Map types are Fo, Fc, Fo-Fc, 2Fo-Fc, Fo**2, Fc**2. Maps can be weighted, including Sim weighting. Peak search gives indication  of peak quality. Automatic Fourier refinement, peak rejection  and  molecule assembly.  Lineprinter and VDU plots of atoms and peaks.



Comprehensive distance and angle calculation, with selection by atom  type, interatomic distance, connectivity, inter molecular, intramolecular. Symmetry included. Esds from the full variance-covariance matrix. Torsion angles, planes, lines, TLS (with bond correction), principal axes of  Uaniso.



Publication listings of atomic parameters, distances, angles, torsion angles, reflections.



\bigskip\Instruction{A strategy for structure analysis}





An over all scheme for a well behaved structure analaysis is:\small\begin{verbatim} 1. Select the best crystal you can find. This is your primary data source, and determines the quality and ease of the rest of the procedure. Complex mathematics is a poor substitute  for good experimental practices.  2. Collect the best data you can. Collect some (at least 10%) equivalent reflections for an indication of the data quality. Collect systematic absences for glide planes  and screw axes. Collect systematic absences for centred cells out to theta = theta(max)/2.  3. Preprocess the reflection data with some local software to  get a file of h, k, l, I, sigma(I), and setting angles if  these are needed for some geometry dependent absorption  correction.  4. Use CRYSTALS to do data reduction, and submit a Direct  Methods job with MULTAN, SHELXS, or SIR.  5. If Direct Methods fail, check the data, space group,  crystal density, data reduction process. If these are all  OK, use CRYSTALS or SHELXS to compute a Patterson  map.  Seek help.  6. Input the trial structure to CRYSTALS and perform Fourier  refinement. IT IS IMPORTANT to realize that though least  squares refinement is often used early in a refinement, there is no guarantee that it will converge, and in fact the results may actually be mis-leading. Proceed to refinement with care.  7. Compare the final results critically with your expectations. An unusual effect in part of the structure, even if of no  direct interest for the current investigation, may be  symptomatic of a faulty refinement.\end{verbatim}\normalsize



\bigskip\Instruction{Over-view of CRYSTALS}





The CRYSTALS system can be used in many ways. The program can be used in batch and interactively - the best solution probably being a combination of both modes. Each CRYSTALS job may be made independent of preceeding jobs by inputing all numeric data from card image files at the start of each job, or the analysis may be carried out as a cumulative process, with the results of each stage  being held in a binary data file. Data is inserted into this file by reading it from a card image file, from the terminal, or as a result of a succesful calculation. Data in this file can always be listed to  a terminal or printer, or output as a card image file for archiving and communication with other programs.

Commands to CRYSTALS may include some numeric data as part of the command, but in general they require that the data has been input previously. The crystallographic routines check for the existence of the necessary data before they start computations, and generate error messages if the data is not available. Batch jobs  consisting of multiple commands, which fail  through lack of some resource (time, disk space),  may usually be  restarted after the last succesful command. 

The numeric data for CRYSTALS is collected into groups of related  parameters, called LISTS. The data is both stored on the disk and represented to the user as LISTS. Thus LIST 5 contains all the refinable parameters, so that the model under  refinement is often refered to as 'LIST 5'. In their character representatons, these LISTS may either be typed directly into a running CRYSTALS system, or put into character files using some method provided by the computers operating system. 

The following table, which can be displayed online from a running CRYSTALS program by typing $\backslash$MANUAL LISTS, gives the function of most lists.\small\begin{verbatim} Lists marked * cannot be input by the User  List  Volume   Type of data Number  Name            1    VOL2    Cell parameters   2    VOL2    Unit cell symmetry   3    VOL2    Atomic scattering factors   4    VOL5    Weighting parameters   5    VOL4    Atomic parameters   6    VOL3    Reflection data  10*   VOL6(4) Peak coordinates from Fourier  11    VOL5    Least squares matrix  12    VOL5    Refinement directives  13    VOL2    Crystal and collection data  14    VOL6    Fourier directives  16    VOL5    General Restraint instructions  17    VOL5    Special Restraint instructions  22    VOL5    Refinement directives in internal format  23    VOL5    Structure factor calculation control list  24*   VOL5    Least squares shift list  25    VOL5    Twin component operators  26*   VOL5    Constraints in internal format  27    VOL3    Diffractometer scales  28    VOL5    Reflection condition list  29    VOL2    Contents of asymmetric unit                 and elemental properties  31    VOL2    Cell parameter E.S.D.'s  33*   VOL5    Internal - Refinement control\end{verbatim}\normalsize



This table, available via $\backslash$MANUAL COMMANDS, lists the commands currently implemented in CRYSTALS.\small\begin{verbatim}              Available INSTRUCTIONS  Introduction to the system  \FINISH       \ text     \TITLE        \COMMANDS      \HIDE         \SHOW      \LOG          \USE           \PAUSE        \TYPE      \MANUAL       \DISK       \PURGE        \RELEASE   \SUMMARY      \FOREIGN  Initial data input  \SPACEGROUP  \COMPOSITION \QUICKSTART  Reflection handling  \HKLI         \LP        \SYSTEMATIC   \SORT          \MERGE        \REORDER   \REFLECTIONS  \DIFABS  Atomic and structural parameters  \EDIT         \CONVERT   \HYDROGENS    \REGULARISE  Structure factors and refinement  \SFLS         \LIST 22   \LIST 26      \SPECIAL       \CHECK        \TRIAL     \ANALYSE      \WEIGHT        \CLEAR  Fourier maps  \FOURIER      \SLANT     \PEAKS        \COLLECT       \REGROUP  Analysis of results  \AXES         \DISTANCES \MOLAX        \TORSION       \PARAMETERS   \ANISO\end{verbatim}\normalsize



\bigskip\Instruction{The SCRIPT processor}



 

By default, interactive running of CRYSTALS starts up in 'SCRIPT' mode, in which the user is offered short menues of possible actions, and prompted for required data. The action of the program, including the  questions posed, the defaults offered and the actions taken, is under the joint control of the user and of data files called SCRIPTS. However, they are only data, and the users are encouraged to modify and extend them. Changing these files CANNOT corrupt the CRYSTALS program, though they can of course be made to corrupt data.

The master script, CONTROL, calls other scripts in a tree like way, and is perhaps useful for beginners. However, if the user knows the name and function of a sub-script, he can enter it directly from command level with the command '$\backslash$SCRIPT scriptname'. This list (June 1988) shows what scripts are available.\small\begin{verbatim} CONTROL        CRYSHELP       DATRED         DEVELOP         DIFABS         DIFFOUR        DIRECT         EDEMAP          EDLIST1        EDLIST29       EDPEAKS        ENDLSQ          EXECUTE        FINISH         FOURDEVEL      FOURREF         GENERAL        HELP           INCAD4         INFORN          INHKLI         INITIALLSQ     INLIST1        INLIST12        INLIST13       INLIST14       INLIST2        INLIST23        INLIST29       INLIST3        INLIST5        INPUT           INSPACE        LISTS          LOOK           MITHRIL         MODIFY         MULTAN         NEWMODEL       NEWSTRUCTURE    PATTERSON      QUESTIONS      REDFOR         REFINE          RENAME         SHELXS         SIR            SOLVE           SPAWN          STARTUP        TYPE           USE             WEIGHTS        \end{verbatim}\normalsize



\bigskip\Instruction{Running CRYSTALS}



 

CRYSTALS can be run in a mixed mode. In this mode, the user initiates an interactive job and then via the USE instruction directs CRYSTALS to take input from the nominated file. If the file ends with $\backslash$USE CONTROL, control is returned to the interactive terminal, where the user may issue the next instruction, which indeed could be another $\backslash$USE.  USE files may call other files, to a depth of 5, though any file already being USEd may not be called from another in-use USE file (the data stack may not be used recursively). Note that the file being USEd need not be a complete LIST or INSTRUCTION block. It can for example be a single line of data, such as a scattering factor to be included into a LIST 3 from a library. The USE instruction is also available in batch mode.\section{EXAMPLE}


The user has files of the type 'atom.cu' containing the scattering factor
 for the given atom with the anomalous component for Cu radiation.

\small\begin{verbatim}
      C.CU contains:
      
 SCATT C 0 0 1.930 12.72 1.878 28.65 1.57 .59 .37 65.03 0.25
\end{verbatim}\normalsize

 


  The CRYSTALS commands are:



\small\begin{verbatim} 
 \LIST 3
 READ NSCAT=3
 \USE H.CU
 \USE C.CU
 SCATT F 0 0 3.304 11.265 3.017 4.665 1.357 .338 .836 27.99
 END
\end{verbatim}\normalsize




The user has in his file store four files: his DISK file, his initial
 data (LISTINPUT), his compressed reflections (PLANES), and his current
 parameters (ATOMS). His aim is to compute structure factors.


\bigskip\Instruction{OPTION 1}




 Use a system facility (such as an editor) to append PLANES and
 ATOMS to LISTINPUT, and add a final instruction to compute structure factors.
 Use the resulting file (BIG) as the control file for a batch job.


\bigskip\Instruction{OPTION 2}




Initiate an on-line job with BIG as the control file. In an on-line
 job the user is computing in real time, but the course of the calculation
 is pre-determined by the control file.


\bigskip\Instruction{OPTION 3}




Initiate an interactive job and issue the instruction $\backslash$USE BIG.
 An interactive job is one in which control returns to the
 users terminal at the completion of each task


\bigskip\Instruction{OPTION 4}




Issue 4 sequential batch jobs being respectively controlled by
 LISTINPUT, PLANES, ATOMS and a file requesting the structure factors.


\bigskip\Instruction{OPTION 5}




Initiate an interactive job issuing sequentially $\backslash$USE LISTINPUT,
 $\backslash$USE PLANES, $\backslash$USE ATOMS and then the instruction to compute structure factors.


\bigskip\Instruction{OPTION 6}




Initiate an interactive job and type in all the information from a
 plain-language listing!


 


The last option is normally the least attractive. However, in the quite
 possible situation where the reflections are available in a machine-readable
 form and the atomic coordinates and other data are available in
 plain-language, the user might well introduce his reflections via a
 $\backslash$USE command and type the other data directly into the system.
 Such a procedure has an advantage
 over preparing the data file independently (using an editor for example) since
 CRYSTALS will monitor the input. However, since only syntactically correct
 information is transfered to the DISK, typing errors lead to the loss
 of the current item of data. CRYSTALS provides a mechanism  for retrieving
 the partially correct data through the LOG file. Unless the user specifically
 inhibits it, all data input to CRYSTALS is copied verbatim to a LOG file.
 Because this is an exact copy of instructions presented to the system, it
 can itself be used as a control file for a subsequent job. If it is error 
 free, it can be used unmodified, otherwise a service editor may be used to
 correct it. The basic data file for a structure can thus be assembled
 by running an interactive job, hooking in via $\backslash$USE previously archived
 data such as scattering factors and space group information, and typing
 in data specific to the compound. The LOG file produced can then be edited
 to remove the $\backslash$USE lines leaving only those lines either read from files
 or input directly. This file would probably end $\backslash$USE LAST so that it could
 itself be USEd later. 



\section{The disk file}


 


During a given run of CRYSTALS, the program MUST be linked to a 
 direct access file (DISK) holding the 
 data-base for the current structure. This file provides the mechanism
 for data transfer between different stages of the task, one stage of
 which may well be the input of the data itself. On sites where users
 are allocated sufficient file store to preserve their DISK file between
 jobs, the DISK also  provides a mechanism for communicating between
 jobs.
 In fact, if the DISK is preserved, CRYSTALS makes no distinction between
 successive tasks run in the same job and the same tasks
 run in successive jobs.
 Instructions exist in CRYSTALS for 'punching' out files containing reflections
 in various formats, including a compressed format, and the structure 
 coordinates.
 In most systems, the 'punched' files will remain in the filestore as character
 (ASCII) files  in a format suitable for direct re-input to a 
 subsequent job. If therefore, the user keeps a file of his initial data, such
 as cell parameters and scattering factors, and terminates each job by 
 outputting his current parameters as a 'punched' file, he is in a
 position to create a scratch DISK file for the 
 duration of a job, and then delete it at the end. Even if the user is able to
 preserve his DISK between jobs, it is prudent to punch out the parameters
 from time to time and keep these in a safe place, together with the initial
 data and the compressed reflections.
\section{The output files}


CRYSTALS plain-language output to a number of devices. Depending on the
 installation and mode of use, one or more of these devices may be a VDU.
 In the list below, the default file extension is given in parenthesis.


\bigskip\Instruction{PRINTER  (.LIS)}




This is the main output stream for the system, and contains detailed
 accounts of the calculation performed.


\bigskip\Instruction{MONITOR  (.MON)}




This is an abbreviated output, suitable for directing to the VDU in online
 or interactive modes. For batch mode jobs, it provides a summary of the
 computation performad.


\bigskip\Instruction{PUNCH    (.PCH)}




Use for ouputting LISTs in a format suitable for later re-input to CRYSTALS.
 These files can be used for archiving or communication with other systems.
 (This is a FORTRAN card image file, so column 1 may not be displayed on
 some operating systems - it is there though)


\bigskip\Instruction{LOG      (.LOG)}




A verbatim copy of the input presented to CRYSTALS. It can be used as 
 input for a subsequent job, and is thus a mechanism for rescuing partially
 correct user input.


\bigskip\Instruction{ERRORS   (.ERR)}




This only contains operating system error information.





The $\backslash$USE instruction permits the user to feed in input from several
 different files. The $\backslash$RELEASE instruction permits the user to disconnect
 the current PRINTER,PUNCH or MONITOR files from CRYSTALS (new ones are
 automatically connected). The $\backslash$TYPE instruction permits the user to
 examine any file available to him, including a previously RELEASEd file,
 so that he may examine the output from some earlier part of a task without
 exiting from CRYSTALS. A long batch job can RELEASE its MONITOR file
 after, say, each round of refinement so that the user can see how the job
 is progressing. 
\section{The work files}


 


CRYSTALS uses a number of work files in the course of most calculations. Of
 particular interest are the sequential files. Their use in initial reflection processing is explained in
 the Manual. The system manager will provide a
 suitable JCL procedure 
 or program environment to assign these files to scratch files. The user
 should however be aware of the existence of these files on systems where
 file store quotas operate. Similar watch should be kept on output files if
 these are attributed to the user while waiting to be listed.


In general, however, on a well regulated system the user should normally
 only be concerned with his primary source of reflection data, his initial
 crystallographic data, his DISK file and those files he creates
 for controlling individual jobs.
\section{The HELP facility }


The help facility in CRYSTALS is neither a computer-aided learning
 facility nor a substitute for the manuals, but rather an aide-memoire
 for the interactive user.


The instruction $\backslash$HELP HELP lists those facilities for which some 
 descriptive help is provided in the form of hints or tips. 


The instruction $\backslash$MANUAL INDEX explains where definitions
 of LISTS and INSTRUCTIONS can be found in the manual. The manual is
 indexed by Volume, Chapter, Section and Page number.


If he can remember which INSTRUCTION he needs
 to use, the $\backslash$COMMAND facility will tell
 the user  what DIRECTIVES, KEYWORDS and
 PARAMETERS are available for that INSTRUCTION.
 Where appropriate, it will also show default values.
 In addition, it also gives the values of some data-base and system parameters
 mainly of use to programmers.


The 'query' facility enables the user to interrogate the data-base.
 Lines beginning with '?' are not interpreted as data, but generate brief
 output from the database.
 The user curious about the workings of the 
 input processing routines should read the volume  DATABASE in the manual.  



\section{The manual}


 


The manual is extensive, but its completeness means it is the best place to look for detailed instructions and more unusual commands. The manual  occasionally refers to 'default' values. These values are set
 at the discretion
 of the system manager, who may well wish to change them for specfic reasons. 
 The file COMMANDS should be available to show the values currently installed,
 and the user can always use the $\backslash$COMMAND facility to obtain unequivocal
 values.
\section{SYSTEM instructions}


The manual defines the main system INSTRUCTIONS:




\small\begin{verbatim}
 \FINISH \TITLE     \ comment   \SET     \OPEN      
 \HELP   \USE       \PAUSE      \RELEASE \TYPE    
 \MANUAL    \$
\end{verbatim}\normalsize




The instruction 'SET MONITOR OFF'
 might well be the first instruction in a file
 to be USEd to recreate a DISK file, 
 and 'SET MONITOR ON' the penultimate. They would inhibit the listing of
 information the user knows to be valid, and restore it before returning
 to the main control stream. Comment lines,
 which begin $\backslash$$<$space$>$, can be inserted anywhere in the control channel
 data except in the body of a LIST 6 (reflections), and should be 
 included in all files that are to be  archived.


The introduction also defines the LIST input and output instructions:



\small\begin{verbatim}
 \LIST     \PRINT     \PUNCH     \SUMMARY
\end{verbatim}\normalsize




The $\backslash$LIST instruction is used to introduce a discrete group of data, called
 a LIST, into CRYSTALS, which validates it and then stores it on the DISK.
 Lists can be introduced to the DISK in any order
 and whenever the user likes,
 provided that the lists required for a calculation are on the DISK when that
 calculation is asked for. The program checks that the necessary lists are
 present before the calculation is started, and warns the user of omissions.
 All lists stored on the DISK can be PRINTed, though the format is rarely
 useful for publication purposes (for that, see RESULTS). These listings do
 however show the value associated with every element of the LIST, including
 those taken by default or modified by a calculation. 
 $\backslash$SUMMARY produces a summary of the required LIST.
 The $\backslash$PUNCH instruction
 produces character files suitable for archiving or re-input to CRYSTALS.


The DISK house-keeping can be altered by the instructions:-



\small\begin{verbatim}
 \DISK     \PURGE
\end{verbatim}\normalsize




When the DISK is created,  special code is run to allocate space in the
 file to hold two indices, the DISK index and the CURRENT index.
 An entry is put into the DISK index for each LIST subsequently added to the
 DISK. In general, when a new LIST is input or generated by calculation
 or modification, this new list is 
 put after the last  existing list on the DISK, and its address entered
 into the DISK index. This index will now contain two or more entries for
 lists of the given type, and so that subsequent calculations will know
 which one to use, the address of the most recent is inserted into the CURRENT
 index. Thus the CURRENT index is of fixed length, with one slot for each 
 possible list type, and the DISK index is of variable length with a slot
 for each list present on the DISK. The DISK index itself consists of fixed
 length blocks. In the event that a block becomes full, a new block is chained
 to the old one. This new block will be after the last list which was written.
 The indexing of the DISK, and hence the addressing of the lists, may become
 undefined if the last DISK index block extends beyond the physical end of the 
 file. Neither the DISK nor the CURRENT index is updated until a list is 
 safely written to the DISK, so that in the event of a system failure or
 a crash of the program, no evidence remains of the incomplete list, i.e. 
 the integrity of the DISK is preserved. Exceptions to this are during
 access to the reflection list (LIST 6) and the normal matrix (LIST 11).
 Because these lists are so large, the system will either try to update or
 overwrite them. If a job fails during an update, the list cannot be accessed
 again until a job has been run to complete the update. For example, if
 a job fails during a round of least squares, there may be some Fc values
 present for the current atomic parameters, and some left from a previous
 calculation. The user will not be able to access LIST 6 for, say, a FOURIER
 synthesis until he has recomputed structure factors for all the reflections.


Because new lists are usually added to the end of the DISK, the user has
 on his DISK a short term archive and record of the calculations he has done.
 In order to return to some previous state of the calculation, he merely has
 to enter the address of the list he wishes to use into the CURRENT index.
 To simplify this, each list is given a SERIAL number when it is written to the
 DISK. The serial number for LIST 5 (parameters) is usually
 displayed whenever the list is written to the
 DISK, and the user can RESET a LIST to a previous SERIAL 
 if it is prefered to the current list. Other lists can be monitored by
 the instructions $\backslash$SET WATCH and $\backslash$SET LIST.
 Eventually there may be so many versions of lists on the DISK
 that it risks becomming full. The user can PURGE his DISK of old lists, only
 the current or spacially marked
 ones remaining. If there are some particularly important early
 versions, the user can RETAIN them during the purge. A purge rewrites
 the required files at the beginning of the DISK, releasing space for new ones.
 It does not physically shorten the file. During a purge the DISK is in a very
 delicate state, possibly with some of the vital information only existing
 in the computer memory. A machine or job failure during a purge could well
 produce a DISK file that is unusable. The system may during certain types
 of machine or calculation failure mark lists as being in error, or alter
 their write/overwrite ststus. Such lists cannot normally be accessed for
 calculations, but if the user is convinced he understands why the flags were
 set, he can if he wishes reset them with MARK or USAGE.  Thus the status
 of LIST 6 may reasonably be reset if a failure occurs during the calculation
 of structure factors, since the Fo values will still be valid, and new Fc
 values can be recomputed. It would be quite unreasonable to reset the status
 of a LIST 6 marked as in error during data reduction. The EXTEND directive
 in $\backslash$DISK allows the user to control the growth of the disk file.


 


\bigskip\Instruction{The environment}




 


In addition to being familiar  with the organisation of CRYSTALS, the user
 will profit from being familiar with his operating environment. In particular,
 a fluency with the system editor will simplify many tasks. Editors with
 means of repeating blocks of instructions may prove particularly useful
 for processing large files, such as reflection lists, and should be mastered.
 The facilities in CRYSTALS for modifying atomic parameters are very powerfull,
 but the user should always be alert for ways of achieving particular results
 by using system editors if that would be more efficient.  It is important
 to see CRYSTALS as only part of the whole system available to the user.
 On systems offering hierarchic filenames, much of the management and control
 of files for different users and structures can be automatically achieved
 by intelligent use of system services. It is the system manager's task to
 provide the JCL procedures needed to profit from the system services,
 and the users task to understand them.



\chapter{Getting Started}




The CRYSTALS system contains no routines for initial processing of the  user's raw diffractometer data. Some other program must be used to prepare a file containing at least h,k,l and I for each reflection. If there is a serial number or X-ray exposure time associated with each reflection, the reflections need not be  on a common scale, and a list of scale factors can be input to CRYSTALS (LIST 27) to perform the scaling as a function of serial or time. If reflections fall into a number of distinct batches each on its own absolute scale (from two crystals for example) a batch serial number may be associated with each reflection, and batch scale factors refined later during least-squares. There is no mechanism provided for extracting scale factors from intersecting data sets (Hamilton, Rollet, Sparks, Acta Cryst (1965) 18, 129). If absorption corrections are going to be applied, the user can either let CRYSTALS compute the reflection setting angles, or if a non-standard geometry was used he can input the actual observed angles. If the user wishes to use the standard deviations of his reflection in some future calculation, they should be provided in the input data and keys set to store them on the DISK.

If Patterson methods are not the most suitable for solving the structure, the user must use some other program (for example SIR, SHELXS, MULTAN).  The instructions $\backslash$PUNCH 6 and $\backslash$FOREIGN enable structure factors to  be output from CRYSTALS after data reduction in a variety of fixed and variable formats, some suitable for direct input to other programs.

 The compressed format provides a suitable method for economical archiving   of the reflections. This file could perhaps be appended to a file containing the other basic data for the structure, thus providing a source for rebuilding the DISK file should that be necessary. From time to time the user could  append his latest coordinates to this file . 

\section{INITIALISATION file}


On program startup, CRYSTALS tries to obey an initialisation file,
  CRYSINIT.DAT, if it exists. This file may contain any valid CRYSTALS
 commands. Typically it contains 'system' commands like $\backslash$SET $<$parameters$>$.
 It may also contain instructions to perform disk house keeping, or pass
 control to other files. In particular, the command $\backslash$USE 
 CRDIR:CRYSINIT.DAT will pass control to a CRYSINIT file in the current 
 subdirectory, so that different subdirectories can have different
 initialisation files. The commands $\backslash$USE CONTROL passes control to the 
 terminal or batch control file without obeying futher CRYSINIT commands,
 and can be used to terminate the SCRIPT mode of working. A command like
 $\backslash$USE CRDIR:INITIAL.DAT causes the file INITIAL to be processed, and so
 provides a mechanism for loading the disk file on systems where it cannot
 be preserved between jobs.



\section{The basic data file}


 


The basic data file contains those data which are unlikely to change 
 during the course of the refinement. If subsequent events indicate that
 the original data were in error (for example, space group ambiguity) the
 new data could be appended to this file so that a record is kept of the
 initial tries. Details of most of the LISTS are found in VOL2,  INITIAL.




 The basic data file should contain:-
\small\begin{verbatim}
 TITLE   A line giving the compound name, source, formula, 
          date etc.
 LIST 1  The cell parameters. 
 LIST 2  The space group information. This is more 
          conveniently input with the SPACEGROUP command.
 LIST 3  The scattering factors. This list may contain 
          coefficients for atom types not in the current 
          structure, but since every atom type entry occupies 
          space in the users DISK, it would be unwise to use
          a single LIST 3 containing more than a dozen atom 
          types, unless the structure contained about that 
          many different atom types.
 LIST 4  The weighting scheme. The default is unit weights.
          Statistical weights are also available, but they
          are rarely suitable for the least-squares refinement 
          without extensive modification (Huml, Computing in 
          Crystallography, (1980), 12.15,  Indian Academy of 
          Sciences, Bangalore). 
 LIST 13 Details of the data collection procedure.
 LIST 14 The Fourier map limits. Remember the unique volume for
          a Patterson synthesis may be different to that for a 
          Fourier. Both are usually set with the SPACEGROUP 
          command.
 LIST 23 Mainly concerned with details for structure factors 
          and least-squares, but worth setting up here in case 
          it's forgotten later.
 LIST 29 Elemental properties, such as radii, weight.
\end{verbatim}\normalsize




The command $\backslash$SPACEGROUP can be used to generate operators for most
 spacegroups in most settings and will also generate the Fourier limits,
 and $\backslash$COMPOSITION will generate LISTS 3 and 29
 from standard data files provided with the program.

\section{Migration from other program systems}





If the basic data processing has been done in some other program system,
 the command $\backslash$QUICKSTART provides a convenient way to transfer the data to
 CRYSTALS. If intensity
 data, i.e. I not F or F**2, are given, data reduction for four circle
 geometry is automatically performed.
  This is an example taken from VOL2 of the manual.

\small\begin{verbatim}
 \QUICKSTART
 SPACEGROUP P 21 21 2
 FORMULA C 6 H 4 N O 2 CL
 FILE CRDIR:REFLECT.DAT
 FORMAT (3F3.0, 2X, 2F8.2)
 DATA 1.5418
 CELL 10.2 12.56 4.1
 END
\end{verbatim}\normalsize





\section{The SCRIPT 'NEWSTRUCTURE'}





At sites where some on-line working is permitted, the SCRIPT NEWSTRUCTURE
 (which call the scripts STARTUP and DATRED, and hence many other sub-scripts)
 will guide the user into getting his initial data into the 
 CRYSTALS data base.



\section{Data reduction}





Intensity data can be introduced to CRYSTALS either by embedding it in a 
 control file, or from an auxilliary file read by the instruction $\backslash$LIST 6 or
 $\backslash$HKLI. This is the only time in CRYSTALS that users data are read in
 from an auxilliary device. If the data are in a compact form,
 say lines with four reflections
 per line, and there are not too many reflections, it would be best to enter them directly into the command line. If the reflections are in an extended form
 with, say, setting angles and other information for each
 reflection, it is probably more efficient to read them from the auxilliary 
 file. The user has the option of FIXED or FREE format
 reads. With the exception of CRYSTALS compressed data, FIXED format reads
 are always more efficient. Remember that a figure of -512 or less for
 the first value on a reflection line terminates reflection input.


The section REFLECTIONS in the manual describes the options available
 and explains why work files are used at this stage. The user has already
 been cautioned (above) about the work files, he should also remember that
 sorting many thousand records can be time consuming. At the end of data
 reduction the user must copy his data from the work file onto his DISK
 (using READ TYPE=COPY in LIST 6)
 otherwise all it will contain is information pointing to the (possibly
 deleted!) work file. Some installations may provide means to preserve
 the work files, but this should only be necessary when processing large
 data sets. The reflection handling procedures in CRYSTALS are powerful
 but for many purposes a simple strategy will suffice. If multi-batch
 data is being processed, e.g. from several crystals, the user will need
 to change the data keys.
 REFLECTIONS bears reading several times.


The instruction $\backslash$SYSTEMATIC should always be used, since it may provide
 confirmation of the space group, and moves reflections, by the permitted
 symmetry operators, into a standard segment of reciprocal space. This
 is a prerequisite for $\backslash$SORT and $\backslash$MERGE, which should also usually be applied
 since the computation of Fourier maps and structure factors is speeded
 up if the reflections are in a standard order. The user should not MERGE
 his reflections or apply Friedel's law if subsequent computations are geometry
 (e.g. analytic absorption) or polarity (e.g. anomalous scattering) dependent,
 or if he has multi-batch data and intends to refine batch scale factors.


A user doing many crystals structures might prepare himself a standard
 data reduction file which he could USE for all structures.



\section{EXAMPLE 1}





Intensity data are available on a common scale, 4 reflections per line, having been collected on a Weissenberg diffractometer, and are to
 be embedded in the input stream. The INSTRUCTONS, DIRECTIVES and keywords
 are written out more or less in full, though the experienced user might
 wish to omit or abreviate them.

\small\begin{verbatim} 
 \LIST 13
 DIFFRACTION GEOMETRY=EQIUINCLINATION
 CONDITIONS WAVELENGTH=1.5418
 AXIS 1 0 0
 \LIST 6
 READ NCOEF=4 TYPE=FIXED F'S=NO NGROUP=4 UNIT=DATAFILE
 FORMAT (4(3F4.0,F8.0))
 END
 1   0   0    1234   2   0   0   24600   
 4   0   0    5000   6   0   0    100
 (etc)
 -512 
 \ a dummy reflection to end the list
 \LP
 \SYST
 \SORT
 \MERGE
 \LIST 6
 READ TYPE=COPY
 \FINISH
\end{verbatim}\normalsize


\section{EXAMPLE 2}


The reflection data output from a CAD4 are in a separate file produced 
 by a pre-processor and in a format defined in the manual. No geometry
 dependent calculations are being done, so the default null orientation matrix
 can be used.

\small\begin{verbatim}
 \LIST 13
 DIFF GEOM=CAD4
 MATRIX
 COND
 \ the standard conditions on this site are 
 \ monochromatic Mo radiation
 \HKLI
 \LP
 \SYST
 \SORT
 \MERGE
 WEIGHT SCHEME=2 NPARAM=6
 \ this scheme filters out outliers
 PARAM .05 3.0 1.0 2.0 .01 .0001
 REFLECTIONS LIST=MEDIUM
 \ contributors more than 3 sigma from the 
 \ mean are listed with a *
 \LIST 6
 READ TYPE=COPY
 \FINISH
\end{verbatim}\normalsize




The keyword RATIO should be used in LIST 28 to limit the minimum ratio of
 I/sigma(I) used in calculations.



\section{EXAMPLE 3}


The data are in a free format, and the user wishes to carry weights 
 based on sigma(Fo)
 through into least-squares. The reflections are embedded in the control data.

\small\begin{verbatim}
 \HKLI
 READ 5 FREE UNIT=DATAFILE F'S=YES
 INPUT H K L /FO/ SIGMA(/FO/)
 STORE 7
 OUTPUT INDICES /FO/ WEIGHT /FC/ BATCH/PHASE 
 CONT   SIGMA(/FO/) RATIO
 \ note that a slot is left for the weight later
 END
 1 0 0 123 10
 2  0 0 24600 130
 3  1 0 15420  70
 -512 0 0 0 0
 \LP
 \SYST
 \SORT
 \MERGE
 \LIST 4
 SCHEME 7
 \ to generate weights from sigma
 \WEIGHT
 \ this generates the weights from sigma, and stores 
 \ them in the weight slot. NOTE  that after the COPY, 
 \ the sigma slot does not exist by default, so that a 
 \ reWEIGHTing with scheme 7 will loose the weights.
 \LIST 6
 READ TYPE=COPY
 END
 \LIST 4
 SCHEME 5
 \ this ensures that we dont loose the weights by an 
 \ accidental reweight.
 END
 \USE LAST
\end{verbatim}\normalsize




In this example the weights are calculated and applied while the data is
 on the scratch files. The LIST 4 could equally well have been applied after
 the LIST 6, i.e. at some much later stage of the process, if we had preserved
 the sigmas.


The instruction $\backslash$HKLI enables the user to apply some simple absorption 
 corrections. All are only approximate, but do prove to be cost effective 
 for routine work. The PHI and THETA corrections involve interpolation. The 
 interval between  the points should reflect the rate at which the curve
 is changing. This is particularly important for the PHI correction (NORTH,
 PHILLIPS, MATHEWS, Acta Cryst (1968) A24, 351) when the sample is plate-like.
 A plan for preparing a phi curve is to observe the intensity of a reflection
 whose scattering vector is close to the diffractometer phi axis at a number
 of phi values (say, at 10 degree intervals). For the calculation to be even
 approximately valid, the I vs phi curve should have two-fold symmetry, so
 that the two halves of the curve (0-180 and 180-360) can be superimposed
 and a smooth curve drawn through them. Values of I and phi are taken from this
 curve for a range rather less than 180 degrees and input to CRYSTALS.
 The user should note that he must give PHI 
 Eulerian values, and not psi or phi-Kappa values.


The 'tube' correction, originally devised for cases where the sample was
 in a capilliary tube inclined at a substantial angle to the phi axis,
 (Watkin, J. Appl. Cryst. (1975) 8, 491),
 can be used to provide an approximate correction for needle-like crystals
 inclined to the phi axis. In this case the parameter 'MU' should be something
 like 0.8*mu*r, where r is the needle diameter. A[max] in this case would be
 the correction for a ray passing along the axis of needle. The 'plate'
 correction is similar, with 'MU' being mu*t, where t is the plate 
 thickness. A[max] is rather less than the correction for a ray passing along 
 a diameter of the plate.


The theta correction is suitable for any correction which is a smooth
 function of theta only, for example the spherical or cylindrical corrections
 tabulated in International Tables. Expect to see your temperature factors
 increase if you apply this correction!


If the user has measured standard reflections from time to time, he can 
 compute decay or stability corrections. These are introduced to CRYSTALS as 
 a LIST 27. Each reflection must have an associated sequence number.
 This can either be a serial number or a figure derived from, say, the X-ray
 exposure time. 
 The scale factors in LIST 27 are associated with the the sequence
 number of the first of each batch of standard reflections. The scale factor
 for general reflections is then obtained by interpolation in this table.
 To avoid interpolation across scale discontinuities, two different scale
 factors may be given with the same sequence number. If the reflection sequence
 numbers do not rise continuously, (for example data from two experiments)
 interpolation will continue safely if scale factors are present in LIST 27
 for the two reflections at the discontinuity. The scale factor scalenumbers
 are not used by CRYSTALS. Remember that there is a maximum value for
 the sequence numbers (try (X-ray time in secs)*0.1). 




\section{EXAMPLE 4}
\small\begin{verbatim}
 \LIST 27
 READ NSCALE=10
 SCALE 1 1.00 1.00 2
 SCALE 2 1.01 1.00 30
 SCALE 3 1.00 1.00 60
 \ reflection sequence 60 is the first with 
 \ the larger scale factor
 SCALE 4 3.00 3.00 60
 SCALE 5 3.01 3.01 90
 SCALE 6 3.02 3.02 120
 SCALE 7 3.03 3.03 147 
 \ reflection sequence 147 is the last in 
 \ the old experiment
 \ reflection sequence 3 is the first in the new
 SCALE 8 3.03 3.03 3
 SCALE 9 3.02 3.03 50
 SCALE 10 3.03 3.03 100
 END
\end{verbatim}\normalsize


\chapter{Atoms And Peaks, Parameters, And Parameter Values.}


For the user, it is probably the highly stylised and powerful methods for refering to atoms and structural parameters which provide the  greatest conveniences in CRYSTALS. Atoms can be refered to either by a simple symbolic string, or by a more complex string which includes symmetry information. Parameters can  be refered to either explicitly or implicitly.

\section{ATOMS and PEAKS}


Atoms are identified by their TYPE (up to four alphanumeric characters
 beginning with a letter) and their SERIAL number.
 Atoms with the same type are distinguished from one another by having
 different serial numbers. However, at no stage is a check made to ensure
 that there is not more than one atom in LIST 5 with the same type and
 serial number. If a routine is searching for an atom with a given type
 and serial number, the first atom so specified will always be taken, and any
 subsequent atoms with the same type and serial number will be ignored.


The atom 
 TYPE is used as the key when looking for scattering factors in LIST 3.
 The atom type PEAK is used by the system to identify peaks taken from a 
 map peak search. Users wishing to compute structure factors on peaks lists
 must either have a scattering factor for an atom type PEAK in their LIST 3,
 or use $\backslash$EDIT to change PEAKS into some other atom type.




\small\begin{verbatim}
 EXAMPLES
        C(14)   PB(2)    PEAK(64)   Ca(12)
\end{verbatim}\normalsize




Sometimes it is necessary to inform CRYSTALS that it must use an atom
 equivalent by space group symmetry to one actually occuring in LIST 5
 (for example, in defining a constraint), or CRYSTALS may wish inform the 
 user that an output atom is an equivalent (for example, in the distance
 angles output). On these occasions, the extended atom definition 
 (which conveys the symmetry information) is used. The specification is


TYPE(SERIAL,S,L,TX,TY,TZ)


TYPE and SERIAL were defined above, and have no default values. The other
 terms are optional, taking default values if omitted. Terms that are omitted
 must be indicated by their delimiting ",", except that the last given term may
 be delimited by a right parenthesis and the other default term delimiters be
 omitted.  Default values are given after the examples.




\small\begin{verbatim}
 EXAMPLES
 C(14,-1,2,1,0,1)
 ZN(1,-2,,1,,1)   equivalent to       ZN(1,-2,1,1,0,1)
 CU(3,-1)         equivalent to	     CU(3,-1,1,0,0,0)
\end{verbatim}\normalsize




S specifies the symmetry matrix from
 LIST 2 which should be used. If S is negative, the atom coordinates from LIST
 5 are negated before the operator is applied. The default value for S is one
 corresponding to the first symmetry operator given in LIST 2. THIS NEED NOT
 BE THE UNIT MATRIX, though it will be if the user takes care to make his
 first  operator in LIST 2 the identity matrix.


L specifies the non-primative lattice translation to be applied AFTER the S
 operation. The translations for the various centrings are listed in  
 VOL4. The default value for L is 1, corresponding to no translation.


TX, TY, and TZ are whole cell translation parallel to the cell edges, and 
 are applied after S and L. Their default values are zero


Thus if the user makes the unit matrix his first operator in LIST 2,
 atom specifications of the form


C(2) FE(16)


will refer to the atoms at the actual positions given in LIST 5. Two special
 'atoms' are permitted in CRYSTALS to enable users to handle LIST 5s in which
 they are not sure of the atom order. These are FIRST and LAST, refering to
 the first and last atoms in the list. A  SERIAL cannot be given for these 
 atoms, but S, L and the Ts can.


Groups of atoms can be referenced by a composite atom specification, the
 UNTIL sequence.


TYPE1(SERIAL1,S,L,TX,TY,TZ) UNTIL TYPE2(SERIAL2)


In this definition, all atoms between and including TYPE1(SERIAL1) and
 TYPE2(SERIAL2) in LIST 5 are referenced. Clearly, a sensibly ordered 
 LIST 5 will enable the user to profit from this construction.
 If any operators 
 are specified for the first atom, these operators are applied to all the atoms
 in the sequence. The atoms FIRST and LAST may be used in an UNTIL sequence.



\section{Parameters}


Parameters may be given either implicitly or explicitly.
\section{IMPLICIT parameters}


In an implicit reference, all parameters 
 of a given type are invoked merely by giving the parameter name.

\section{EXAMPLE 1}
\small\begin{verbatim}
 \LIST 12
 FULL X Y Z U'S
 END
\end{verbatim}\normalsize




This instruction, which will be discussed more fully under Least-Squares,
 causes full matrix refinement to be carried out on the x, y, z and all the
 anisotropic temperature factors of all the atoms.



\section{EXPLICIT parameters}


In an explicit reference to atomic or overall parameters,
 the individual parameters must be individually defined, though the UNTIL
 sequence, discussed above, provides a method for referencing groups of atoms.


The CRYSTALS system allows the user to associate  parameter names 
 with parameter values. These names are used in symbolic references to the
 parameter itself. 
 Thus, to refer to the X coordinate of carbon(1), the user would type


C(1,X)


Multiple parameter references are permitted in some instances, for example in
 defining matrix blocks for least-squares. e.g.


CL(17,X,Z) indicating that the operation will be applied to both X and Z of the Cl atom with serial number 17.



\section{Parameter values}





Values corresponding to parameters are stored 
 either in LIST 5 or in LIST 10. LIST 5
 is either input by the user, or may be generated by system operations
 on an existing LIST 5 or LIST 10. LIST 10 has a similar format to LIST 5
 and is produced either by a peak search on a Fourier map or by an operation on
 an existing LIST 10. Default values are provided by the system for some 
 parameters.





The parameters stored for an atom are:




\small\begin{verbatim}
 TYPE - Described above under ATOMS and PEAKS.
 SERIAL - Described above under ATOMS and PEAKS
 OCC - The chemical site occupation factor, defaults to 1.0 
      (excluding symmetry effects).
 U[ISO]  - The isotropic temperature factor, defaults to 0.05.
 X -  The fractional x coordinate.
 Y -  The fractional y coordinate.
 Z -  The fractional z coordinate.
 U[11] | 
 U[22] |
 U[33] | The components of the atomic displacement parameters.
 U[23] |
 U[13] |
 U[12] |
\end{verbatim}\normalsize




The temperature factors are defined by:

\small\begin{verbatim}
 T[ISO] = EXP(-8*PI*PI*U[ISO]*S**2) 
 where S = sin(theta)/lambda
\end{verbatim}\normalsize




and

\small\begin{verbatim}
 T[ANISO] = EXP(-2*PI*PI*(h*h*as*as*U[11]
             +k*k*bs*bs*U[22]+l*l*cs*cs*U[33]
             +2.0*k*l*bs*cs*U[23]+2.0*h*l*as*cs*U[13]
             +2.0*h*k*as*bs*U[12])).
\end{verbatim}\normalsize




where 'xs' are the reciprocal cell parameters
 and h, k and l are the Miller indices of the current reflection.


An atom is considered to be isotropic when  abs(U[ISO])$>$0.00005.
 If an atom is isotropic and one or more of the anisotropic temperature
 factors is to be refined, the anisotropic terms are calculated from
 U[ISO]  and then  U[ISO]  is set to zero. The conversion in the
 other direction is also made, except that  U[ISO]  assumes its default
 setting. $\backslash$EDIT can be used to generate U[EQUIV] from U[ANISO]. The
 minimum (most negative) permitted value for U[ISO] or U[ii] is set in
 LIST 23.
\section{CAUTION}


Beware the U[ISO] default when inputting a LIST 5!
 It should be noted that if a set of anisotropic atoms are input with no
 U[ISO]  key or  U[ISO]  data, then the default value of 0.05 will be
 inserted. This implies that all such atoms are
 isotropic, so that the anisotropic temperature factors will be treated as
 zero, and the calculation will proceed for isotropic atoms.


Although it is possible to input an overall temperature factor as one
 of the overall parameters, it is not possible to use it under all
 circumstances. In fact, the structure factor routines always take the
 temperature factor of an individual atom as the value or values stored
 for that atom. If the overall temperature factor is to be refined, the
 system will ensure that the current value of the overall temperature
 factor is inserted for the temperature factor of all the atoms. When the
 new parameters are computed after the solution of the normal equations,
 this substitution is again made, so that all the atoms have the same
 overall isotropic temperature factor. However, if the overall temperature
 factor is not refined, or no refinement is done, the individual temperature
 factor for each atom will be used, and the overall temperature factor ignored.


The dummy overall temperature factor is in no way related to the overall
 temperature factor, and its use is explained in the section on refinement.




\section{EXAMPLE 2}
\small\begin{verbatim}
 \LIST 5
 READ NATOM=2
 ATOM  PB 1 U[ISO]=0 X=.89 .78 .97 .11 .22 .33 .23 .13 .12
 ATOM  C  2          X=.45 .56 .46
\end{verbatim}\normalsize




The overall structural parameters are:

\small\begin{verbatim}
 SCALE    - The overall scale factor. Defaults to 1.0
 DU[ISO]  - The dummy overall temperature factor. 
           See the section on L.S.
 OU[ISO]  - The overall isotropic temperature factor. 
           Default is .05
 POLARITY - The Rogers polarity parameter. Default is 1.
 ENANTIO  - The Flack polarity parameter. Default is zero.
 EXTPARAM - The Larson extinction parameter. Default is zero.
\end{verbatim}\normalsize




The remaining parameters are:

\small\begin{verbatim}
 LAYERS - The layer scale factors.
 BATCH - The batch scale factors.
\end{verbatim}\normalsize


\section{LIST 5}





LIST 5 holds the parameter values, and the user should go to the Manual
  to find the full specifications. It can best be regarded as a table
 in which the column headings define the types of parameter, and the row
 labels the types of entity to which the paramertes belong. Certain entries
 in the table are mandatory, e.g. parameter type SERIAL for entity type
 ATOM, and others are optional, e.g. U[ISO], for which there is already
 a default value entered.


 
 As an aid to beginners, some examples are given here.




\section{EXAMPLE 3}
\small\begin{verbatim}
 \LIST 5\
 \ no overall parameters. DU[ISO],OU[ISO] and SCALE 
 \ all have default values
 \ 5 atoms
 READ NATOM = 5
 ATOM C 1 X=0.094,Y=0.343,Z=0.890
 ATOM C 2 1 0.05 0.149 0.411 0.651 0 0 0 0 0 0
 ATOM C 3 X=0.050 0.406 0.648
 ATOM TYPE=C,SERIAL=4,OCC=1,U[ISO]=0,
 CONT X=0.027,Y=0.384,Z=0.725,
 CONT U[11]=0.075,U[22]=0.048,U[33]=.069,
 CONT U[23]=-.007,U[13]=.043,U[12]=-.001
 ATOM C 5 U[ISO]=0.0 .108,.365,.815,.074 
 CONT .051 .065 -.015 .048 -.014
 END
\end{verbatim}\normalsize



\section{EXAMPLE 4}
\small\begin{verbatim}
 \LIST 5
 READ NATOM = 2, NLAYER =   6
 \ the zone axis is (H00)
 INDEX 1 0 0 1
 \ there are 6 layers, 0 to MOD(5)
 \ layer 3 (H=2) had the X-rays turned down!
 LAYERS 1 1 1.3 1 1 1
 OVERALL     20.821314   EXTPARAM=21.2
 ATOM C 1  U[ISO]=0. X=-0.231   0.085   0.066 
 CONT .038 .043 .041 .001 -.006 .003
 ATOM H 73           X=-0.443   0.231   0.219
 END
 \LIST 23
 \ note that layer scales are not applied unless
 \ set in LIST 23
 MODIFY EXTINCTION=YES LAYERSCALES=YES ANOMALOUS=YES
 \LIST 12
 \ remember that 'FULL' implies the overall scale factor.
 FULL LAYER(3)
 END
 \SFLS
 REFINE
 END
\end{verbatim}\normalsize




In this example we have some parameters defined by position, (SCALE,
 TYPE, SERIAL, Y, Z, and the temperature factors), and some by keyword
 (U[ISO], EXTPARAM and X). Although only layer 3, h=2, is anomalous,
 layer scales must be given for all layers. The layer scale factors and
 the extinction parameter are
 not applied during the computation of structure factors unless LIST 23
 requests them.


Batch scale factors are handled in the same way as layer scales, except
 that the BATCH parameter must be set for each reflection in LIST 6.
 Remember that the BATCH parameter will be undefined if reflections 
 from different BATCHES have been merged together.



\section{Modifying LIST 5}





As mentioned elsewhere, LIST 5 can be output from the binary database
 as a character file. The instruction $\backslash$PUNCH 5 produces a file in a form
 suitable for re-input to CRYSTALS. '$\backslash$PUNCH 5 D' produces a file in the
 SHELX format. These files can be edited using the manufacturer's editor,
 and such a procedure might be appropriate in an environment where interactive
 processing is restricted. In general however, the user is advised to use the
 $\backslash$EDIT instruction in CRYSTALS. This is a purpose-built editor for manipulating
 parameters. Like all editors, it seems to get better as the user becomes more
 familiar with the facilities.
 Before editing a LIST 5, the user will usually wish to know something about
 its contents. Information can be extracted in several ways.


\bigskip\Instruction{1. What is the order of the atoms in LIST 5?}




This must be known in order to use the UNTIL construct effectively. Possible
 solutions are:-

\small\begin{verbatim}
   \SUMMARY LIST 5 LOW
   END
\end{verbatim}\normalsize




This produces an ordered listing of the atom types and serial numbers
 on the terminal and output listing.

\small\begin{verbatim}
   \MOLAX
   ATOM FIRST UNTIL LAST
   END
\end{verbatim}\normalsize




This produces an ordered list of the atom types, serial numbers and 
 'best plane' coordinates onthe terminal and output listing.



\bigskip\Instruction{2.  What are the atomic coordinates?}


\small\begin{verbatim}
   \DISPLAY HIGH
   END
\end{verbatim}\normalsize

This displays LIST 5 on the terminal.

\small\begin{verbatim}
   \EDIT
   LIST HIGH
   END
\end{verbatim}\normalsize




This produces an ordered listing of the atom types and serial numbers
 and parameters in an abbreviated form on the terminal.

\small\begin{verbatim}
    \PRINT 5
    END
\end{verbatim}\normalsize




This produces a full listing of LIST 5 in the line printer file.



\bigskip\Instruction{3. What is the connectivity?}


\small\begin{verbatim}
   \DISTANCE
   OUTPUT MONITOR=DISTANCE
   END
\end{verbatim}\normalsize




This produces a  listing on the terminal and in the printer file.

\small\begin{verbatim}
    \MOLAX
    ATOM FIRST UNTIL LAST
    PLOT
    END
\end{verbatim}\normalsize




This produces a labelled line diagram on the terminal and the printer file.



\chapter{Fourier And Patterson Functions}


Fourier summations are computed by the instruction $\backslash$FOURIER with an appropriate keyword. Patterson functions can be computed directly from the observed structure amplitudes, Fourier summations require that phase angles have been computed previously. This may be done by issuing the instruction $\backslash$SFLS with the directive SCALE or CALCULATE. Remember, if you have been doing Least-Squares refinement, that after a new parameter list has been generated the structure factors will need recomputing if they are to reflect these new parameters.

If the user has not used the SPACEGROUP command, he is required to choose for himself that fraction of the unit cell which must be computed, and enter this information as a LIST 14. Where there are alternative choices possible, computational efficiency is the  main factor to be concidered if the map is not going to be printed.  If the X-ray intensity data are to the same resolution along all axes, the  user should, if the space-group permits, choose an asymmetric volume in  the form of a thin 'slab', and use an ORIENT directive to compute sections parallel to the slab face to minimise computation time. For example, in an orthorhombic  cell 16x11x24, a volume 1/2 x 1/4 x 1 computed in 3min 27secs, and a volume 1 x 1/2 x 1/4 computed in 1min 40secs.

When the map is to be printed, then convenience in handling the final  map is the main consideration. For example, if the structure has an essentially planar fragment which needs looking at, the sections should be chosen  approximately parallel to the plane. Also, though CRYSTALS will compute maps with any number of points in each direction, and will split the sections up onto several sheets of paper if they are too wide to fit directly, the output is clearly more manageable if the user can arrange the width to fit onto a single page. The fraction of the cell chosen should be such that it contains at  least an asymmetric unit, with a small additional border if the user  is explicitly defining the volume and grid intervals. Failure to include this border may lead to the peak-search routine finding (during the  least-squares peak fitting) very large or small peaks several Angstroms beyond the edge of the volume computed. Such spurious peaks are easily found by inspection, but could be troublesome if some sort of automatic interpretation of the map was being attempted. When the system is allowed to chose its own sampling interval, a border is  automatically included, and these spurious peaks seldom occur.

Parameters on the $\backslash$FOURIER instruction enable the figure fields to be scaled and formatted, and the user should well consider what precision he requires in the output map. If he is looking for features that can be represented to two integer places, and there are other previosly characterised features requiring three places, a format permitting printing printing of signed two digit integers will be adequate if rho-max is set to 99.

The default map type is an F-OBS map. The peak search routines automatically reject all but one peak  from sets of peaks related by a symmetry element. Additionally, if an atoms list, LIST 5, is present, the program evaluates the electron density at atomic sites. The message 'NOT FOUND IN THE MAP' inevitably means that the user has used an inappropriate LIST 14, and has not computed a complete asymmetric volume (though he may have computed some part twice!).

\section{EXAMPLE 1}
\small\begin{verbatim}
 \FOURIER
 MAP PRINT=YES
 LAYOUT NLINE=1 NCHAR=3 MARGIN=2 NSPACE=2 MIN-RHO=0 MAX-RHO=99
 END
\end{verbatim}\normalsize


\section{Patterson maps}


Patterson functions are computed by the instruction FOURIER
 with the directive MAP TYPE = FO-PATTERSON.  CRYSTALS 
 automatically modifies the symmetry information taken from  LIST 2 
 into that appropriate for a Patterson map, i.e. includes a centre of symmetry
 and drops the translational components. In some space groups this may lead
 to an origin shift, so that the user will have to 
 input appropriate values in  LIST 14. 


The modified LIST 2 is used for the implied distance-angle calculation called
 by FOURIER to detect degenerate peaks, but it is not written back to the DISK,
 which retains the original LIST 2. If the user wishes to explicitly
 compute distances and angles on the Patterson peaks list, he must include
 the parameter SYMMETRY=PATTERSON in the distance calculation.
\section{Sharpened Patterson maps}





Any of the weighting schemes described for LIST 4 can be applied to
 the Patterson coefficients. Unit weights can be  modified using  Scheme 13
 to sharpen the map. The formula of Dunitz (Acta Cryst(1973), B29, 589) 
 should be used with P(1)=1 and P(2)=10 to avoid over-sharpening. The weight
 for a reflection at theta=30 will be 50 for Mo radiation. Do not forget
 you will probably wish to restore the original weighting scheme.

\small\begin{verbatim}
 \LIST 4
 \ Use the default scheme, which produces unit weights.
 END
 \WEIGHT
 \ To apply unit weights
 \LIST 4
 SCHEME NUMBER=13 NPARAM=2
 PARAM 1 10
 END
 \WEIGHT
 \ To apply weights for sharpening
 END
\end{verbatim}\normalsize


\section{Fourier refinement}


CRYSTALS provides a mechanism for performing elementary Fourier refinement.
 Such a procedure is very cost-effective during the initial development of
 poorly phased structures, such as those based on E-maps or heavy-atom phased 
 maps. At that stage, non-linearities degrade the convergence of simple
 least-squares, though the use of constraints may substantially improve the
 situation (see the section on Least-squares). Fourier refinement has a
 good range of convergence and is relatively insensitive to errors in the
 model, though it does require the atomic sites to be resolved.

\small\begin{verbatim}
 \SFLS
 \ we will compute phases and put Fo on an 
 \ approximately absolute scale
 SCALE
 END
 \FOURIER
 \ set the reject limit very small so that peaks 
 \ which lie close to input atomic sites are not 
 \ rejected
 PEAKS HEIGHT=10 NPEAK=16 REJECT=.0005
 END
 \PEAKS
 \ hang on to all the peaks again
 SELECT REJECT=.0005
 \ this does the refinement.
 \ peaks within .15 of an input atomic site are 
 \ identified with that atom.
 \ The positional coordinates of the atom are computed 
 \ from the corresponding peak coordinates
 REFINE DISTANCE=.15
 END
\end{verbatim}\normalsize




In addition to maps computed with /Fo/, /Fc/ or /Fo-Fc/ as coefficients,
 maps can be based on /2*Fo-Fc/. Such maps are commonly used in protein
 crystals structure analysis, and are useful in small structure analyses
 when there is disorder. The reduced contribution from /Fc/ means that
 a diminished image of the 'known' structure is retained in the map, and
 serves as a guide through those features due to /Fo/.
\section{Assembling a single molecule}


It is rare that the user can use LIST 14 to choose a fraction of his cell 
 so that the resulting map will contain a discrete and connected molecule.
 In general, the peaks taken from the map must have symmetry operators applied 
 to them to generate a unified molecule. This can often be done automatically
 for organic molecules using the instruction $\backslash$REGROUP. By using $\backslash$REGROUP with
 different collection ranges, it may be possible to construct a very acceptable
 molecule.

\small\begin{verbatim}
 \REGROUP
 \ build up the main frame of non-hydrogens first
 SELECT MOVE=1.6
 END
 \REGROUP
 \ now re-move the hydrogens, important if the 
 \ molecule is crowded or distorted
 SELECT MOVE=1.1
 END
\end{verbatim}\normalsize




The REGROUP instruction reorders (and optionally renumbers) LIST 5.
 $\backslash$COLLECT, which computes connectivity based on radii taken from LIST 29,
 leaves the order of LIST 5 unchanged, but applies any necessary symmetry
 to the coordinates. COLLECT with the parameter TYPE=PEAK is useful for
 moving new found peaks as close as possible to existing atoms.


While the user can often get some idea of the shape of his structure by
 looking at peaks lists and inter-atomic distances and angles, simple line
 diagrams may also be useful. In CRYSTALS, these are obtained via the
 molecular axes (q.v.) calculation $\backslash$MOLAX. This instruction enables a
 plot
 to be made onto the best plane through a specified list of atoms or peaks.
 The list may include all the items in LIST 5, only selected items, and
 symmetry operators may be introduced if the user thinks that this may
 produce a more interpretable figure. The example below is based on trans-1,4 
 dimethyl cyclohexane, which lies on a centre of symmetry and
 has 4 unique carbon atoms and 8 hydrogen atoms. 

\small\begin{verbatim}
 \MOLAX
 \ use the whole of LIST 5, together with its 
 \ image in the centre at 0,0,0
 ATOM FIRST UNTIL LAST FIRST(-1) UNTIL LAST
 PLOT
 EXEC
 \ now just look at the methyl group
 ATOM C(4) H(41) H(42) H(43)
 PLOT
 EXEC
 END
\end{verbatim}\normalsize






 The output from these calculation is:
 
\small\begin{verbatim} ATOM FIRST UNTIL LAST FIRST(-1) UNTIL LAST
 
 
                   H32                     H22
                      .                    .
                       .                 ..
                       H+13..........CH*1
                        ..            ..
 H43                  ..                .                  H42
   ..                .                   ..                .
     .             ..                      .             ..
     H+14........+1*                       H+11.......C.4
      .           ..                       ...         H41
    ..              .                     .               .
   .                 ..                  .                 ..
 H42                   .               ..                   H43
                       H+12..........C.3
                       ...            H31
                      ..                .
                     .                   .
                   H22                   H32
 \end{verbatim}\normalsize

 

\small\begin{verbatim}
 ATOM C(4) H(41) H(42) H(43)                
                                                
                           H43                  
                             ..                 
                               ..               
                                C.4.......H42   
                                .
                               .
                            H41
\end{verbatim}\normalsize

 Note that atom types are abbreviated to their
 initial letter only, and serials are represented
 by only the last two digits. The symbols + and * 
 mark character overlap.  



\chapter{Generalised Fourier Sections}


Generalised Fourier sections, that is sections not parallel to a cell face, are computed by a special program ($\backslash$SLANT), rather than by interpolation into  the normal map. Such maps are useful for detailed examination of special problems, such as disorder, poor resolution  or large thermal motion.  Features identified in the map can be converted back to crystal coordinates for inclusion in the model. In spite of special computing techniques, the generalised Fourier is about 10 times as slow as the normal Fourier, for which the Beevers-Lipson method is used.

The position, orientation and extent of the map must be specified in a  consistent coodinate system. This is most conveniently found from a MOLAX calculation on atoms or points in the required plane. The required position is then the centroid in orthogonal angstrom, and the matrix is the transformation from crystals fractions w.r.t. this centroid to best plane co-ordinates in Angstrom, and the extent is defined by the number of steps and the step lengths parallel to XP and YP. There are no formatting parameters available  for the instruction $\backslash$SLANT.

The editor $\backslash$EDIT can be used to apply the rotation and translation needed to transform points back from map to cell coordinates.\chapter{Regularisation}


As explained above (Part 4), Fourier techniques provide a powerful way  for initiating the refinement of a structure. When the geometry of fragments are well known from previous analyses, for example phenyl groups, the initial refinement can be speeded even futher by forcing the observed fragment to adopt an idealised geometry. This process, common in protein crystallography, is known as regularisation, and can be used not only to tidy up a poor geometry, but also to postulate sites for missing atoms.

\section{EXAMPLE 1}





Regularising a cyclopentadienyl ring. The idealised coordinates are from
 the literature.

\small\begin{verbatim}
 \REGULARISE REPLACE
 GROUP 5
 OLD C(1) C(2) C(3) C(4) C(5)
 ATOM 0 0 0
 ATOM .846  1.165 0
 ATOM .846 -1.165 0
 ATOM 2.216  0.72 0
 ATOM 2.216 -0.72 0
\end{verbatim}\normalsize




Completing and regularising a phenyl ring. C(3) is very poorly placed,
 and so will first be deleted, and C(6) has not been found at all. 
 New atoms will be generated for them.

\small\begin{verbatim}
 \EDIT
 DELETE C(3)
 \REGULARISE REPLACE
 OLD C(1) C(2) C(3) C(4) C(5) C(6)
 PHENYL
 END
\end{verbatim}\normalsize





p
 BE SURE THAT THE GIVEN ATOMS ARE ALL IN THE SAME MOLECULAR 
 FRAGMENT, OTHERWISE YOU WILL NEED TO GIVE SYMMETRY OPERATORS.



\chapter{Hydrogen Placing}


A special case of regularisation is the computation of the position of hydrogen atoms in an organic structure. Unless there is likely to be  some ambiguity over the nature of the bonding, there is generally little to be gained in searching for hydrogen atoms in difference Fourier maps. Instead, their coordinates can (and should) be computed geometrically. CRYSTALS has routines for placing sp1, sp2, sp3 and phenyl hydrogens, taking proper account of symmetry, bond lengths and angles.\section{EXAMPLE 1}


Place the hydrogen atoms on carbons C(2) to C(6) of a phenyl group and
 the two hydrogens on C(31) of an aliphatic chain.

\small\begin{verbatim}
 \HYDROGEN
 \ note that the pivot atom C(1) must be given
 PHENYL C(1) UNTIL C(6)
 \ two H atoms on an sp3 C(31) atom
 H23 C(30) C(31) C(32)
\end{verbatim}\normalsize




Although the default bond length is suitable for placing hydrogen atoms
 as seen with X-rays, it can be changed to that suitable for neutrons,
 and $\backslash$HYDROGENS can even be used to place non-hydrogen atoms, such as
 the oxygens in borate groups.
\chapter{Refinement}


The refinement of  crystal structures is  complex and rarely a well behaved process, and only occasionally can it  be performed reliably without some human supervision. In CRYSTALS we have tried to provide a structure that will reduce the amount of supervision necessary under normal circumstances, and still be very flexible for use in difficult or unusual cases.

Before discussing the facilities available in CRYSTALS, some of the aims and factors relevent to refinement need considering.\section{Aims of Refinement}


The overall aim is to find an explanation of some observable phenomena
 in terms of a model which is intellectually acceptable. Inevitably, the
 thresholds for acceptability will change from observer to observer,
 and from time to time. However, certain norms exist for the models,
 and it is deviations from these norms that increase our understanding
 of the physical process. The steps in arriving at a suitable model are:





1. Deciding what the observations are, and what errors are associated
  with them.




Postulating a model that will enable the observations to be simulated.




Deciding if the differences between the actual and the simulated 
 observations are acceptably small. If they are, then we understand
 the phenomena, otherwise we must try to modify the model.





Step 1 poses the problem of resolving the 'observation' from the 'model'.
 In X-ray crystallography, are the observations 'I' or 'F'? If an absorption
 correction is applied, is the correction part of the observation, or of the
 model? The answer to these questions really depends on what additional
 questions the model must answer.



\section{Factors Influencing Refinement}


The refinement process cannot be made to yield more information than
 that present in the data (though the data may include items other than
 the diffraction observations). Deficiencies in the data may include:


\bigskip\Instruction{Systematic Errors}




These really are short-commings in the model. If the model being refined
 contains parameters highly correlated with the systematic errors, then the
 model will be prejudiced in a way depending upon the form of the systematic
 error. For example, failure to correct the diffraction data for the theta
 dependent component of the absorption effect will systematically reduce
 the atomic temperature factors.


\bigskip\Instruction{Random Errors}




These are inevitable in any experimental observation. They should be
 minimised by careful design of the experiment (e.g. choosing appropriate
 crystal size,  counting times and radiation). Often reducing random errors
 is something that must be purchased (by increasing the time spent in
 performing the experiment), or is in conflict with reducing systematic errors
 (large crystals giving good counting statistics may introduce non-random
 errors).


\bigskip\Instruction{Shortage of Data}




If some of the data is unobservable in a way that is not
 highly correlated with 
 the model, then the result is that the random errors in the data lead to
 random errors in the model. Increasing the amount of data, or its quality,
 will improve the model. Note that reducing the number of parameters in 
 the model implies increasing the amount of data, since those parameters
 not refined will be given (either explicitly or implicitly) values that
 are themselves observations of high precision. Their accuracy will depend
 upon the approximation made in simplifying the model. The user should
 always apply Occams Razor (The Principal of Minimum Assumption) to any
 new model.


\bigskip\Instruction{Lack of Resolution}




If some of the data is not observable in a systematic way, this may
 mean that it will not adequately define some parameters in the model. The
 most common example is the   link between the angular range of the diffraction
 data and the detail that can be resolved in the structure. Lack of resoluton
 in the data usually shows itself as large e.s.d.s in the refined parameters,
 though these can also arise from a very inadequate model. For example,
 an anisotropic temperature factor may fail to give a satisfactory
 representation for an atom that is disordered over two widely separated sites.



\section{Restraints and Constraints}
 The term 'Restraint' has come to mean the introduction of non-diffraction
 data into the refinement process in such a way that it guides the refinement
 towards some goal, rather than forcing it to that goal. Then, if there is
 an intrinsic conflict between the diffraction data and the restraining data,
 the user can be given useful warnings. A 'Constraint' imposes a restriction
 that cannot be violated on the process. If, for example, we have a structure
 in which both Na and K atoms can lie on the same site, then we can apply
 the constraint (via LIST 12) that the sum of their occupancies will be 1.0
 exactly, or we can apply the restraint, (via LIST 16) that the sum will be
 approximately 1.0, and thus admit the possibility that not all equivalent sites
 are 
 occupied by metal atoms. Note that if the total occupancy was less than unity
 and yet LIST 12 had been used to impose a constraint, this erroneous
 assumption could not have revealed itself explicitly via the refined
 occupancies, but would probably have lead to incorrect values for temperature
 factors, scale factors, or extinction parameters. The LIST 16 approach would
 have revealed, through the sum of occupancies being far from unity, the real
 nature of the problem if the diffraction data contained the information
 needed to resolve these parameters. 


Generally restraints, also called 'soft constraints', are used to test the
 hypothesis that the diffraction data will yield parameters not
 incompatible with those taken from some other source. For example, the
 restraint may request the refinement to verify that the diffraction data
 is not incompatible with a certain bond having a length of 1.39(1)A.
 The user should realise that just because a restraint is satisfied, he has
 no guarantee that no  other hypothesis would fit the data just as  well. 


Restraints may be used to help speed up the convergence of a refinement.
 If the restraints are imposed with large weights (i.e. small e.s.d.s)
 in the initial stages, they will force the refinement rapidly towards the 
 preconceived structure, possibly helping to keep it from false minima.
 As convergence approaches, the weights can be reduced, or the restraints 
 removed altogether. 



\section{Notes on Weights}


The reflection weights for least squares must reflect errors
 in BOTH Fo and the model since we are using (Fo-Fc) as the argument
 in the LS procedure. The most serious approximations in the model 
 are usually made in:
\small\begin{verbatim}
 1. Absorption corrections.
 2. The form of the temperature factors.
 3. Atomic scattering factors.
 4. Extinction corrections.
\end{verbatim}\normalsize




Errors in the diffraction data should ideally follow Poisson statistics.
 However, it has been our experience that except in the case of painstaking
 work on very high quality crystals, other errors are present. These generally
 seem to be approximately correlated with the reflection intensity, and can
 be represented by an expression like:
\small\begin{verbatim}
 sigma(F) = a*sigma(I) + b*I + c*I**2
\end{verbatim}\normalsize




Since values for the constants a, b and c are arrived at by inspired guesses,
 we now prefer to represent sigma(F) by a smooth function in Fo. The function
 chosen is a Chebychev polynomial with the minimum number of coefficients
 needed to make w(Fo-Fc)**2 constant as a function of Fo. This procedure
 only yields weights appropriate for the current model, and so is inappropriate
 for intermediate stages of the analysis. 
 Futher, introduction of a model dependent weighting scheme too early in a
 structure analysis may lead to important features being concealed.
 Until all the parameters that need to
 be refined have been identified, unit weights or the scheme due to Hughes are
 used. A valid weighting of the reflections makes w(Fo-Fc)**2 constant for all
 rational samplings of the data. Changes of weighting scheme usually have
 the most dramatic effects on thermal and extinction parameters, which should
 be re-refined whenever the scheme is modified.


The user should not be unduly surprised to find his normal R factor
 increasing after the application of weights (he is not trying to minimise
 'R'), and will generally find Rw higher than R (Rw uses delta squared).
 If Rw is much higher than R, then the model may be very inadequate in 
 some aspects, or there may be a few 'rogue' reflectins in the data. These
 should be found, considered, and possibly rejected.



\section{A Strategy}


Currently, the only effective way for completing the refinement of a
 crystal structure at atomic resolution is by the method of least-squares
 in which we minimise w(Fo-Fc)**2 summed over the diffraction data. In
 restrained refinement we also simultaneously minimise w(Pt-Pc) ( where
 Pt and Pc are the theoretical and calculated values for some structural
 parameter) summed over the required parameters. 
 In both cases 'w' is the weight appropriate for the term in the summation.
 Fc and Pc are usually
 non-linear functions of the atomic parameters. The functions are thus expanded
 as a Taylor series, and the second and higher order terms neglected. The
 zero'th order terms (Fc and Pc) depend upon the current model, and the
 neglected terms only really become insignificant as the process approaches
 convergence. e.g.

\small\begin{verbatim}
 Fc(i) + [x(k).d(Fc(i))/d(p(k))] = Fo(i) 
 Pc(i) + [x(k).d(Pc(i))/d(p(k))] = Po(i) 
\end{verbatim}\normalsize




where the terms k in [] are summed over the parameters being varied and we
 solve for the shifts x(k) in parameter p(k). The terms Fc(i) and Pc(i) are 
 effectively observations for the current model, and can be moved to the 
 right hand side.


Because of the approximations made, the process need not be
 convergent, but this becomes more likely the better the starting model.
 Fc is more sensitive to reasonable errors in some types of parameter
 (e.g. x,y,z) than in others (e.g. U's). At the outset of refinement therefore
 the least sensitive parameters can be given reasonable values (e.g. 0.05
 for U[iso] of a carbon atom) which is not refined until the more volatile
 parameters have stabilised. In this example, the
 refinement is heavily constrained with
 the constraint that U[iso] = 0.05. Once the model for the geometry has begun
 to stabilise, the thermal model can be refined, though here again it is
 generally wise to constrain the temperature factors to be isotropic initially.
 Because all the parameters are correlated, it makes no sense to refine these
 intermediate models to convergence. An r.m.s.(shift/e.s.d.) of about 1 to 3
 is adequate before the model can be relaxed (though any individual
 anomalies should be investigated).


As convergence approaches, the gradient terms (d(F)/d(p)) change only slowly
 compared with Fc, and it is thus appropriate in large structures only to 
 recompute Fo-Fc (the Right Hand Side) every cycle, and re-use
 the matrix of derivatives for alternate cycles, thus saving considerable
 machine time.


Throughout this section, the importance of the starting model has been 
 emphasised. The user should now see why Fourier refinement and Regularisation
 were discussed in detail earlier. Non-linear least-squares is a powerful
 method for acheiving a satisfactory result only when it is given a good
 starting model. While it can give some indications as to redundant parameters
 (by refining them to absurd values or attributing them large e.s.d.s),
 it cannot introduce new parameters. The final test of a 'good' structure
 is not a low 'R' factor or minimisation function - it is more likely to be
 an intuitive assesment based on the relationship between the current model and
 similar structures, and a critical examination of difference Fourier
 syntheses.
\section{A Practical Scheme}


The following list is a flow diagram for a typical
 crystal structure refinement assuming that there are no problems with
 disorder, pseudo-symmetry, twinning etc.

\small\begin{verbatim}
 1.  Get a trial structure (Patterson or direct methods)
 2.  Calculate structure factors and a Scale Factor.
 3.  Do an Fo map and Fourier refinement.
 4.  Regularise if you can. Use REGULARIZE or MOLAX
 5.  If you have found new atoms, go to 2.
 6.  Refine the positional parameters.
 7.  Refine isotropic temperature factors.
 8.  Locate or compute hydrogen atoms.
 9.  Refine positions again.
 10. Refine aniso temperature factors.
 11. Locate or compute any missing hydrogen atoms.
 12. Enter the final stages of refinement
  a. If a small structure, FULL matrix.
  b. If a medium structure, LARGE BLOCK approximation.
     If refinement is stable and correlation coefficients  
     are small, then use DIAGONAL approximation
  c. If a large structure, DIAGONAL approximation or 
     CASCADE refinement.
 13. Look at the agreement (variance) analysis (\ANALYSE).
     If <w(Fo-Fc)**2> not constant, change weights to SCHEME 1 
     with a parameter P(1) = F(min<w delsq>).
 14. Do 1 cycle of refinement of temperature factors plus 
     a CALCULATION.
 15. Look at the agreement analysis again, and decide if you 
    need a Chebychev weighting scheme.
 16. Refine to convergence, re-using matrix on alternate cycles.
 17. If there are unusual features (distances, angles, planarity, 
     Uaniso) use REGULARIZE, MOLAX, ANISO to correct the feature, 
     and use restrained refinement to test the hypothesis that 
     the 'corrected' model is compatible with the X-ray data.
\end{verbatim}\normalsize

 
\section{Structure Factor Control List - LIST 23 }


The overall conditions governing the calculation of structure factors
 and monitoring the refinement process are generally kept constant throughout
 a refinement, and so have been collected together into the Structure
 Factor Control List, LIST 23, which should be input with the other initial
 data. The directive MODIFY consists of a series of switches for causing the
 application (or not) of data contained in other lists, thus removing the
 need to re-input those lists as circumstances change. Note that, for example,
 though LAYER scale factors or an EXTINCTION parameter may be present
 in LIST 5, they will only be included in the calculation of Fc if the switches
 are set in LIST 23.


The MINIMISE directive controls whether or not the restraints are to be
 added into the minimisation function (i.e. a LIST 16 may be present on
 the DISK but will not be actioned unless LIST 23 requests it), and whether
 F or F**2 is to be used for the diffraction data. Note that if you request
 F**2, you will probably need to concider changing the weighting scheme.


The directives ALLCYCLES and INTERCYCLE control the convergence-divergence
 parameters.
\section{Defining the Matrix - LIST 12}


In PART 5 we introduced the definitions for parameter names and showed
 how these were associated with parameter values via LIST 5.  For least -
 squares we need to set out a table showing the relationship between
 least-squares parameter names, least-squares parameter shifts, and 
 the physical parameters. This table is held as LIST 22, which the 
 interested user may care to print (for a small matrix!). For those
 cases where there is one least-squares parameter per physical parameter
 and the refinement is full matrix, this table is relatively trivial. For
 refinements in which physical parameters are EQUIVALENCed or RIDING,
 or where the parameters are divided into several matrix blocks, it
 would be unreasonably complex for the user to have to construct this table for
 himself. The information which CRYSTALS needs to construct this table is
 presented in a symbolic form as a LIST 12.


LIST 12 consists of directives that define matrix blocks (FULL, BLOCK,
 DIAGONAL) and directives that modify the contents of the current block
 (RIDE, EQUIVALENCE, FIX, WEIGHT, PLUS). The directive CONTINUE merely extends
 the directive begun on the previous line.



\section{Simple examples}


For many simple structures a simple full, large-block or atom-block
 diagonal matrix refinement may be sufficient. Parameters implicitly
 included by a preceding FULL, BLOCK or DIAGONAL card may be removed from the
 refinement by the FIX directive.


\bigskip\Instruction{Rules for parameters}




In the examples, 'parameters' may be either EXPLICIT (e.g. C(1,X)
 C(3,X,Y,Z) C(2,X'S) ) or IMPLICIT (e.g. X  Y U'S ). Parameters defined
 by an 'UNTIL' sequence (e.g. C(1,U[11],U[22]) UNTIL C(6) )
 are regarded as implicit for LIST 12 processing.


No parameter may be given implicitly twice, nor given explicitly twice,
 in the same LIST 12.


\bigskip\Instruction{Rules for FULL, BLOCK, DIAGONAL, FIX and PLUS}




LIST 12 must begin with FULL, BLOCK or DIAGONAL.


LIST 12 must not contain any two of FULL, BLOCK or DIAGONAL.


LIST 12 must not contain more than 1 FULL directive. 





Parameters on a BLOCK card must not appear on another BLOCK or DIAGONAL card.


Parameters on a DIAGONAL card must not appear on another DIAGONAL or
 BLOCK card.


Parameters on a PLUS card must not appear on another PLUS, FULL, BLOCK or
 DIAGONAL card.
\small\begin{verbatim}
 \LIST 12            This defines a single matrix block          
 FULL parameters     containing all the specified parameters 
 END                 PLUS the overall scale factor.
 
 \LIST 12            This defines two independant matrix blcks,
 BLOCK parameters(1) one containing parameters(1), the other
 BLOCK parameters(2) containing parameters(2) and (3), i.e. the
 PLUS  parameters(3) PLUS directive is acting rather like CONTINUE.
 END                 The overall scale factor is not implied, and
                    must be specified explicitly in the appropriate
                    block if required. Note that the same parameter
                    must not occur more than once.
\end{verbatim}\normalsize



\section{EXAMPLES}
\small\begin{verbatim}
 \LIST 12            Refine the positions and anisotropic 
 FULL X'S U'S        temperature factors of all atoms in a single
 END                 matrix block with the overall scale factor.
 
 \LIST 12            As above, only now don't refine the y 
 FULL X'S U'S        coordinate of the lead atom (polar axis? 
 FIX PB(1,Y)         see also restraints).
 END
 
 \LIST 12              Refine the positions for all the carbon 
 BLOCK C(1,X,Y,Z) UNTIL LAST                         one block
 BLOCK C(1,U'S) UNTIL C(10) SCALE          and the anisotropic 
 PLUS  H(11,U[ISO]) UNTIL LAST           t.f.s for the carbons 
 CONT UNTIL LAST      together with an isotropic t.f.s for the
 END            hydrogen atom and the scale factor in another. 
 
 Note that PLUS could be replaced by CONTINUE.
\end{verbatim}\normalsize


\section{Advanced examples}


In some situations a simple one-to-one relationship between structural and
 least squares parameters is not suitable. For example, the x and y coordinates
 of an atom on the special position X,X,1/6 must move synchronously, that is,
 be represented by a single least squares parameter. Similarly, an atom on
 X,-X,11/12 has the x and y coordinates linked, but here the shift to y is in
 the opposite sense of that for x.


Sometimes the relationship between parameters is not due to space group
 symmetry, but to some other physical requirement. In the example given above
 (Restraints),
 where either Na or K can occupy the same site, we may have site occupancy
 disorder such that the total occupancy for the ions is fixed, but the ratio
 is to be determined. In this case, the shifts in the occupation
 factors for the two ions are equivalenced to a single least squares parameter,
 but made to move in opposite senses.


Linking of physical parameters like this provides a powerfull way for
 dealing with certain types of instability in the refinement, and for reducing
 the cost by reducing the number of parameters refined. For example, the
 hydrogen atoms on a carbon can be shifted synchronously with it, thus 
 preserving the local geometry (but watch second neighbour relationships).
 This procedure is sometimes known as a 'riding' refinement. Another use
 for riding parameters is to give a poorly defined residue a single overall
 anisotropic temperature factor (not as good as proper TLS of course, but
 perhaps more realistic than individual atomic temperature factors).


For EQUIVALENCED LINKED or RIDING parameters, the partial derivative for each
 parameter is computed, multiplied by the WEIGHT if requested, and added into
 the normal equations. After solution of the equations, the resultant shifts
 are multiplied by the same weights and then applied to the corresponding 
 structural parameters.


\bigskip\Instruction{Rules for FIX, EQUIVALENCE, LINK, COMBINE, GROUP, RIDE, WEIGHT}




Parameters on a FIX, EQUIVALENCE,  LINK, COMBINE, GROUP,
 RIDE or WEIGHT card are added to or modify
 the action of parameters on the last previous FULL, BLOCK or DIAGONAL card.


Parameters given explicitly over ride those given implicitly.


Parameters on an EQUIVALNCE, LINK, COMBINE, GROUP
  or RIDE card must not occur on another
 EQUIVALENCE, LINK, COMBINE, GROUP,  RIDE or a PLUS card.


Parameters on a FIX card may not appear on another FIX or a WEIGHT card.


Parameters on a WEIGHT card may not appear on another WEIGHT or a FIX card.


Parameters on an EQUIVALENCE, LINK, COMBINE, GROUP,
  or RIDE card need not have been  given on a
 preceeding FULL, BLOCK or DIAGONAL card.


Parameters on an EQUIVALENCE, LINK, COMBINE, GROUP,
  or RIDE card will modify the action of
 parameters given on a preceeding FULL, BLOCK or DIAGONAL card.


Parameters on a WEIGHT card should also occur on an EQUIVALENCE, LINK,
 COMBINE, GROUP,  or
 RIDE card.


Parameters on a RIDE card must be either explicit definitions or UNTIL
 sequences. There must be the same number of parameters in each argument
 list.


Parameters on a LINK card must be either explicit definitions or UNTIL
 sequences. There must be the same number of parameters in each argument
 list.


Parameters on a COMBINE card must be either explicit definitions or UNTIL
 sequences. There must be the same number of parameters in each argument
 list.


Only atom names can be given on a GROUP card



\section{EXAMPLES}


This example is also given in the MANUAL. The structure contains C,N and O
 in the space group P6122, with N at x,-x,11/12, O at x,x,1/6 and C isotropic.
 The refinement will be by full matrix. The following 3 LIST 12s all lead to
 the same matrix. The example in the manual has only 9 directives.
\small\begin{verbatim}10
 \LIST 12                \LIST 12                  \LIST 12
 FULL C(1,X'S)           FULL X'S                  FULL X'S U'S
                                                  FIX C(1,U'S)
 PLUS C(1,U[ISO])        PLUS C(1,U[ISO])          PLUS C(1,U[ISO])
 PLUS N(1,U[33],U[12])   PLUS N(1,U[33],U[12])
 EQUIV N(1,X) N(1,Y)     EQUIV N(1,X) N(1,Y)       EQUIV N(1,X) N(1,Y)
 WEIGHT -1. N(1,Y)       WEIGHT -1. N(1,Y)         WEIGHT -1. N(1,Y)
                        FIX N(1,Z)                FIX N(1,Z)
 EQUIV N(1,U[11],U[22])  EQUIV N(1,U[11],U[22])    EQUIV N(1,U[11],U[22])
 EQUIV N(1,U[23],U[13])  EQUIV N(1,U[23],U[13])    EQUIV N(1,U[23],U[13])
 PLUS O(1,U[33],U[12])   PLUS O(1,U[33],U[12])
 EQUIV O(1,X) O(1,Y)     EQUIV O(1,X) O(1,Y)       EQUIV O(1,X) O(1,Y)
                        FIX O(1,Z)                FIX O(1,Z)
 EQUIV O(1,U[11],U[22])  EQUIV O(1,U[11],U[22])    EQUIV O(1,U[11],U[22])
 EQUIV O(1,U[23],U[13])  EQUIV O(1,U[23],U[13])    EQUIV O(1,U[23],U[13])
 WEIGHT -1. O(1,U[13])   WEIGHT -1. O(1,U[13])     WEIGHT -1. O(1,U[13])
 END                     END                       END 
\end{verbatim}\normalsize




The next example is for H(11) H(12) and H(13) riding upon C(1). Note
 that the hydrogen atoms are all given the same temperature factor shifts.
 They would also probably start with the same temperature factor values,
 though for the isotropic refinement of a phenyl group (for example)
 the investigator may have some a priori reason for starting the atoms off
 with slightly differing values. We are not concerned here with the rest of
 the structure.

\small\begin{verbatim}
 \LIST 12
 FULL (parameters)                    The x, y and z coordinates for 
 RIDE C(1,X'S) H(11,X'S) UNTIL H(13)  the four atoms are handled by
 EQUIV H(11,U[ISO]) UNTIL H(13)       3 ls parameters. The 3 U(iso)s
 END                                  are handled by 1 parameter.
\end{verbatim}\normalsize


\section{Matrix blocking schemes}


There are no fixed rules for deciding how or even whether a given structure
 should be refined by full or block approximations to the matrix. The choice
 will change from institution to institution and from time to time.


Broadly, full matrix methods are the most certain to converge. They are
 also likely to require less cycles of refinement than a block matrix method.
 However, the time per cycle will always be more than for a block matrix
 method, and there will be increased storage requirements. Thus, small
 structures should be refined using the full matrix, and large ones using a
 block approximation. On computing machinery where time is paid for whether 
 the machine is idle or not (perhaps the users own machine) it makes sense
 to use the largest approximations to the full matrix compatible with other
 users of the machine and its efficiency of operation. At some stage, for
 large structures, a substantial number of the off-diagonal elements of the
 matrix will be negligable, and the cost of accumulating and paging these
 may not be justified. Such large structures need a special strategy.


For medium large structures, it is probably convenient to break the matrix
 into a small number of largish blocks. Experience has shown what parameters
 can be expected to be correlated: these MUST be included in the same matrix
 block.


\bigskip\Instruction{Correlated Parameters}




There is no definative way for predicting which parameters will be highly
 correlated, but experience has shown that the groups of parameters in the
 following table should normally be refined together.

\small\begin{verbatim}
 - bonded atoms
 - non-orthogonal coordinates
 - scale factors, extinction parameters and 
   temperature factors.
 - molecules or fragments related by a pseudo
   symmetry operator.
\end{verbatim}\normalsize


 The program always lists those parameters from the current blocking
 structure which are highly correlated. These parameters must be refined
 together from time to time to preserve their proper interdependance.
 If this results in matrix blocks that
 are too large,then the blocking scheme should be changed between cycles
 to ensure that all significant cross terms appear at some stage 
 ('cascade refinement').
 For example
 if there are N atoms to be refined in 2 blocks, the following LIST 12s might
 be suitable. 




\small\begin{verbatim}
 \LIST 12
 BLOCK ATOM(1,X'S) UNTIL ATOM(1n/2)
 BLOCK ATOM(1n/2+1,X'S) UNTIL ATOM(n)
 END
 
 \LIST 12
 BLOCK ATOM(1,X'S) UNTIL ATOM(1n/4)
 PLUS  ATOM(3n/4+1,X'S) UNTIL LAST
 BLOCK ATOM(1n/4+1,X'S) UNTIL ATOM(3n/4)
 END
\end{verbatim}\normalsize




\bigskip\Instruction{Singularities}




 High correlation in itself does not necessarily lead to invalid
 parameter estimates. For example, the effect of bond length restraints is
 to deliberately increase the off-diagonal terms between the coordinates
 of the atoms involved. If some parameters are so highly correlated that
 the ill-conditioning of the matrix makes it (almost) singular, the user is
 left with few solutions. Essentially his problem is that he has an 
 inapropriate model, or that his data are not resolving or defining his model.
 The first approach is to ascertain whether more or different
 data would cure the problem, for example collecting data to a higher theta
 value, at a lower temperature, or by adding restraints. If no solution
 of this type is possible, then the approximate relationship implied
 between the highly correlated parameters should be formalised, that is
 a new model is required. For example, if the temperature factor and 
 occupation factor for an atom fail to refine properly, the user may choose
 to define one and refine the other, and perhaps later reverse the roles.
 He should be aware however that he is not performing a free refinement of
 both parameters, and that in the final cycle he is making the statement
 that he 'knows' the correct value for the unrefined parameter. Separating
 correlated parameters into different refinement cycles (or into different
 matrix blocks in the same cycle - the effect is almost mathemtically
 equivalent) is an extremely hazardous way of avoiding singularities, and
 implies effectively uncontrolled assumptions on the part of the user. 
 Reparameterisation may help with pseudosymmetry problems, using the COMBINE
 instruction to combine the approximately equivalent parameters.



\section{Special positions and floating origins}





Atoms on special positions can be handled either through constraint 
 or through restraints. The latter method can be applied automatically, and
 is to be prefered if LIST 12 is already complex because of GROUPed, LINKed
 RIDING or COMBINed parameters.


\bigskip\Instruction{Constraints}




Constraining atomic parameters to remain in fixed positions or have 
 certain relationships between them is done in LIST 12 with the directives
 FIX, EQUIVALENCE and WEIGHT. See VOL5 for examples. The special
 atomic coordinates
 must have correct values or relationships before starting the refinement


\bigskip\Instruction{Restraints}




Restraining atomic parameters to remain in fixed positions or have 
 certain relationships between them is done in LIST 17 using the RESTRAIN
 directive. LIST 12 must be set up to let the special parameters refine, 
 but their refinement is controlled by the restraints. The instruction
 $\backslash$SPECIAL APPLY automatically generates the correct restraints and warns of
 non-unity site occupancies. The special atomic parameters need not be
 exactly correct before refinement starts - the restraint will correct them.


 


\bigskip\Instruction{Floating Origins}




These can be fixed by not refining the appropriate coordinates of a heavy
 atom, using FIX in LIST 12. However, a much better solution is to restrain
 the centroid of the structure, using SUM in either LIST 16 or LIST 17



\section{Tips for least squares}



 1) Look at the output.



 2) If refinement is going well, consider using the DIAGONAL approximation.



 3) If refinement is slow to converge, or gives poor molecular parameters,
   or temperature factors are anomalous, use a large block approximation to 
   identify correlated parameters.



 4) Watch out for high correlations or physically implausible parameters.



 5) Do NOT 'FIX' an unexpected singularity by putting the related parameters 
   into different matrix blocks. Make a better guess at the parameter values, 
   or use restraints.



 6) If your data dont define a parameter, dont try to refine it.



 7) Consider using EQUIVALENCE to reduce the number of refined parameters.



 8) Look at mean shifts and 'reversals'. At convergence, shifts should be 
   small, and reversals ca. 50\%. If reversals are $<$50\% and shifts are large, 
   structure is still converging. If reversals are large and shifts are 
   large, structure is not converging. If the model is the best you can think 
   of, use geometric restraints and the LIMIT restraint to stabilise 
   divergence or oscillations.



 9)  Think. Use LIST 22 and CHECK online before issuing batch jobs. Use AXES
    and DISTANCES to verify molecular reasonableness.



 10) Remember that the mathematical 'best solution' depends on the maths you 
    use, and may not be the intellectually best solution.



 11) Get good data. If you know the data are bad, consider a partial refinement
    only.



\chapter{Results}


Increasingly, the aim of a crystal structure analysis is not to produce a detailed description of a single structure, but either to determine the gross structure of a compound, or obtain geometric details of a series of compounds.  Few crystallographers now have the time to make friends with all their structures. None the less, since most published structures now find their way into computer readable data banks, from which they may be retrieved by programs with less insight than human researchers, it has become more vital than ever that structures are processed in a well defined and documented way, and that the results are published without additional errors. CRYSTALS does watch what the user is doing, and may alert him to real or potential problems, but eventually the user is responsible for his own work. The following list gives some of the points to be checked, and the facilities available for doing this.

\section{Difference electron density}


This provides a broad and largely unbiassed view in an intuitatively 
 interpretable form  of the differences between the observed and calculated
 structure factors. An entirely featureless map indicates that the model
 can simulate the observations well, though other models may do almost as
 well, and there may be errors in the data that it would be preferable that
 the model didn't emulate (e.g. temperature factors concealing the need for
 an absorption correction). In any case, since most refinements will be 
 conducted with some sort of weighting of the observations, it is unlikely
 that the map will be without features of any kind. What is important is that
 any features observed can be explained.



\section{Analysis of differences}


For a formally valid least squares refinement, the average value of
 w(Fo-Fc)**2 must be constant for any systematic sectioning of the data. In 
 CRYSTALS the instruction $\backslash$ANALYSE provides an analysis  by sectioning as
 a function of h, k, l, parity, class, /Fo/ and sign(theta). The entries in
 the column $<$w(Fo-Fc)**2$>$  should be approximately constant 
 throughout the table. If this condition is  not satisfied,
 then systematic variations in $<$(Fo-Fc)$>$
 as a function of one or more of the sectionings may throw light on serious
 failures in the model. For example, if Fo is less than Fc for strong
 reflections and also for low angle reflections, then an extinction correction
 may be necessary. Systematic variations as a function of index may indicate
 the need for an absorption correction or an anisotropic extinction correction.
\section{Physical reasonableness}


A refinement which converges to physically unreasonable parameter values
 cannot be regarded as satisfactory. Positional parameters usually need
 translating into molecular parameters before their significance becomes
 apparent. Several translations available in CRYSTALS.


\bigskip\Instruction{$\backslash$DISTANCES}




This instruction computes inter and intra atomic distances and angles. All
 necessary symmetry operators are automatically applied to ensure that values
 within the specified limits are generated. The user should check that there
 are no inexplicably short inter molecular contacts, as well as checking the
 ususal bond lengths and angles. The program will also compute e.s.d.s
 from the full covarience matrix (including symmetry effects). The information
 for this is taken from the least squares matrix, and the program will not
 permit this information to be applied to any other LIST 5 than that produced
 be the corresponding round refinement. It is thus fairly important to give
 your atoms systematic names and serial numbers, and to get the list into
 a convenient order AS SOON AS POSSIBLE. Even just changing the serial of
 an atom will inhibit the linking of the LIST 5 and the matrix.


\bigskip\Instruction{$\backslash$MOLAX}




This procedure computes the principal axes of inertia of a group of atoms.
 If they have unit weights or weights derived from their standard deviations,
 then the shortest axis is parallel to the normal to the best plane, and
 the longest is parallel to the best line. 


\bigskip\Instruction{$\backslash$TORSION}




This procedure computes the torsion angles for the specified atoms. e.s.d.s
 are not currently computed.


\bigskip\Instruction{$\backslash$CHECK}




This procedure compares the current values of parameters with those 
 requested in the restraint definitions. Current values differing
 significantly  from those required should be carefully reappraised, since
 they indicate a conflict between the diffraction data and the hypotheses.
\section{Thermal parameters}


\bigskip\Instruction{$\backslash$AXES}




The six components of the anisotropic temperature factor can be transformed
 to define a three dimensional ellipsoid representing the harmonic motion of
 the atom. If free refinement of the U's leads to an ellipsoid with a negative
 volume, then the transformation becomes meaningless, usually indicating that
 there are grave deficiencies in the model or the data. For data collected
 at very low temperatures, the volume may go marginally negative, and
 restraints on the U's should stabilise the situation. The procedure AXES
 is called automatically after every round of refinement, to give warning
 of non-positive definite U's. Refinement will however continue even if
 some atoms are unsatisfactory. The result is generally a rapidly diverging
 process with rapidly increasing R factors.


\bigskip\Instruction{$\backslash$ANISO}




This procedure tries to fit the atomic temperature factors to a rigid body
 composite temperature factor. This is partitioned into three parts, T
 representing the rectilinear vibration of the body, L representing its
 torsional libration, and S representing the coordination between these two
 parts. There are 20 independant terms to be evaluated in TLS for an
 unsymmetrical fragment not lying on a symmetry operator, so that a rigid
 body of this sort cannot be defined for less than 4 atoms. For a body
 of six atoms, the calculation might just be meaningfull, though there are
 problems with the conditioning of the matrix if the atoms lie close to a conic
 section. Under these conditions the maths cannot distinguish two correlated
 motions in T space from a corresponding libration in L space. The user
 should be aware that good agreement between U(obs) and U(calc) does not mean
 that the values of T, L and S have any physical significance.
\section{Computer Graphics}


CRYSTALS comes with the graphics program Cameron for visualising structures.
\chapter{Conclusion}


CRYSTALS contains many more options than have been mentioned in this Guide. We know of no easy way to pass on knowledge about the system. It is big, and it is powerful, and for the users who have come to terms with the disk and parameter handling facilities, it is an enjoyable system to work with. When you begin to use it, use keywords frequently and in full. This will help you to remember the syntax of the instructions. Use your imagination in trying to break the system, and so learn its useful limits.



\printindex
\end{document}
