\documentclass[10pt,a4paper]{report}
\usepackage[centertags]{amsmath}
\usepackage{amsfonts}
\usepackage{amssymb}
\usepackage{amsthm}
\usepackage{makeidx}
\usepackage{newlfont}
\usepackage{fancyhdr}
\usepackage[bookmarks,colorlinks,plainpages,backref]{hyperref}
\pagestyle{fancy}
\hfuzz2pt % Don't bother to report over-full boxes if over-edge is < 2pt
\makeindex
\addtolength{\headwidth}{80pt}
\addtolength{\textwidth}{80pt}
\addtolength{\oddsidemargin}{-30pt}
\begin{document}
\lhead{\slshape \rightmark}
\chead{}
\rhead{\thepage}
\lfoot{Fri Jul  5  2024}
\cfoot{}
\rfoot{}
\renewcommand{\headrulewidth}{0pt}
\renewcommand{\footrulewidth}{0pt}
\fancypagestyle{plain}{%
\fancyhf{}
\fancyfoot[L]{Fri Jul  5  2024}
\fancyfoot[C]{\thepage}
\renewcommand{\headrulewidth}{0pt}
\renewcommand{\footrulewidth}{0pt}}
\newcommand{\Instruction}[1]{{\bf #1}}
\newcommand{\Directive}[1]{{\bf \emph{#1}}}
\newcommand{\Keyword}[1]{\emph{#1}}
\sloppy
\title{Getting Started}
\author{Chemical Crystallography Laboratory, Oxford}
\date{Fri Jul  5  2024}
\maketitle
\tableofcontents
\chapter{Data Organisation}
It's always a good idea to keep your data organised, and it will alsohelp for the purposes of running CRYSTALS.

1. Organise your computer. Create a master folder tohold all your structures and keep each structure in a sub-folder of this:e.g. \emph{c:$\backslash$strs$\backslash$compound1$\backslash$}

2. Copy the crystallographic data in to the sub-folder. CRYSTALS acceptsdata files (usuall cif files) from most modern machines, and SHELX format .INS/RES and .HKL files.

{\bf Take} {\bf care:} {\bf If} {\bf you} {\bf open} {\bf a} {\bf data} {\bf file} {\bf with} {\bf WORD,}{\bf you} {\bf may} {\bf accidentally} {\bf save} {\bf it} {\bf as} {\bf a} {\bf WORD} {\bf document.}Use notepad, EDIT, or some other \emph{ASCII} text editor.

3. Click the CRYSTALS icon, and use the folder browser to find thefolder containing your data. Click OK.

4. The CRYSTALS window is divided into several areas:

 On the right: {\bf The} {\bf model} {\bf window.} This will display the model once the structure is solved.

 Above the model: {\bf The} {\bf model} {\bf toolbar,} lets you change the appearanceand functioning of the model window.

 At the very top: {\bf The} {\bf menu} {\bf bar.} The pull down menus representthe principal stages of structure analysis, working from left to right.

 Below that: {\bf The} {\bf main} {\bf toolbar,} provides quick access to somecommon options that should also be in the menus.

 On the upper left:{\bf The} {\bf text} {\bf output} {\bf window.} Thisdisplays the results of calculations, or asks the user questions.It has a large scroll-back capability.

 Below that: {\bf The} {\bf information} {\bf tabs.} This panel can be hidden away,as can the toolbars, by clicking the little triangles in the bar thatdivides them from their neighbouring elements.

 And below that: {\bf The} {\bf text} {\bf input} {\bf line.} This is below theoutput window. The cursor automatically moves to this area when you type.The 'arrow' keys enable you to recover previous input (like the DOSKEYutility). If you are a new user, you shouldn't worry too much aboutthe text input line.

 At the very bottom: {\bf A} {\bf status} {\bf bar.} Shows progress during longcalculations, and status as CRYSTALS is doing different things.

5. CRYSTALS creates many files:

 {\bf A} {\bf binary} {\bf database} for each structure,always named CRFILEV2.DSC. {\bf NEVER} {\bf try} {\bf to} {\bf edit} {\bf this} {\bf file.} Thisfile enables CRYSTALS jobs to be restarted from where they were leftoff. All other files may be opened with a text editor e.g.notepad, wordpad, or edit.

 Double clicking on the .DSC file will start CRYSTALS in that folder, usingthat .DSC file.

 {\bf Text} {\bf files} bfile.l\emph{nn} and bfile.m\emph{nn.} The number \emph{nn}is automatically incremented with each run of the program. These files cangenerally be deleted without looking at them unless the analysis begins to gowrong. They provide post-mortem information.The currently in use files may be opened quickly using the "Files" tab on theinformation panel.

 {\bf PUBLISH.*} files. These are final listings for preparing papers.\chapter{Using The Guide}


 Once you have started CRYSTALS in a given folder, nothing much happens.To make things happen, try clicking the "GUIDE" button in thetoolbar. The GUIDE assesses where a structure analysis has got to, andwill provide a list of possible options in a pull-down menu. Choose therequired option and click OK.



An explanation of the screen layout can be found on the menu bar under:

{\bf Appearance} / {\bf Screen} {\bf Layout} 

\chapter{Importing Shelx Data}


 It is best to import the .INS or .RES file first. You can importa file with or without atoms depending on where the data is from - CRYSTALSwill just grab all the information that it can.

 From the X-ray data menu, choose "Import SHELX file". You will needto know the space group symbol since this information is notpresent in the SHELX file format.

 Type in the file name, or find it using "Browse", click OK and the filewill be imported. (You will be asked for the spacegroup).

 Next import the reflection file. Choose "X-ray data"-$>$"Import reflections".The default settings are for a SHELX hkl file, so just type or browsefor the required file, and click OK.



A description of the relationship between CRYSTALS and SHELX commands, including examples of their use, can be found on the menu bar under:

{\bf Help} / {\bf Migrating} {\bf from} {\bf SHELX} 

\printindex
\end{document}
